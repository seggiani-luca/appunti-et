
\documentclass[a4paper,11pt]{article}
\usepackage[a4paper, margin=8em]{geometry}

% usa i pacchetti per la scrittura in italiano
\usepackage[french,italian]{babel}
\usepackage[T1]{fontenc}
\usepackage[utf8]{inputenc}
\frenchspacing 

% usa i pacchetti per la formattazione matematica
\usepackage{amsmath, amssymb, amsthm, amsfonts}

% usa altri pacchetti
\usepackage{gensymb}
\usepackage{hyperref}
\usepackage{standalone}

% imposta il titolo
\title{Appunti Elettrotecnica}
\author{Luca Seggiani}
\date{2024}

% imposta lo stile
% usa helvetica
\usepackage[scaled]{helvet}
% usa palatino
\usepackage{palatino}
% usa un font monospazio guardabile
\usepackage{lmodern}

\renewcommand{\rmdefault}{ppl}
\renewcommand{\sfdefault}{phv}
\renewcommand{\ttdefault}{lmtt}

% disponi il titolo
\makeatletter
\renewcommand{\maketitle} {
	\begin{center} 
		\begin{minipage}[t]{.8\textwidth}
			\textsf{\huge\bfseries \@title} 
		\end{minipage}%
		\begin{minipage}[t]{.2\textwidth}
			\raggedleft \vspace{-1.65em}
			\textsf{\small \@author} \vfill
			\textsf{\small \@date}
		\end{minipage}
		\par
	\end{center}

	\thispagestyle{empty}
	\pagestyle{fancy}
}
\makeatother

% disponi teoremi
\usepackage{tcolorbox}
\newtcolorbox[auto counter, number within=section]{theorem}[2][]{%
	colback=blue!10, 
	colframe=blue!40!black, 
	sharp corners=northwest,
	fonttitle=\sffamily\bfseries, 
	title=~\thetcbcounter: #2, 
	#1
}

% disponi definizioni
\newtcolorbox[auto counter, number within=section]{definition}[2][]{%
	colback=red!10,
	colframe=red!40!black,
	sharp corners=northwest,
	fonttitle=\sffamily\bfseries,
	title=~\thetcbcounter: #2,
	#1
}

% U.D.M
\newcommand{\amp}{\ensuremath{\mathrm{A}}}
\newcommand{\volt}{\ensuremath{\mathrm{V}}}
\newcommand{\meter}{\ensuremath{\mathrm{m}}}
\newcommand{\second}{\ensuremath{\mathrm{s}}}
\newcommand{\farad}{\ensuremath{\mathrm{F}}}
\newcommand{\henry}{\ensuremath{\mathrm{H}}}
\newcommand{\siemens}{\ensuremath{\mathrm{S}}}

% circuiti
\usepackage{circuitikz}
\usetikzlibrary{babel}

% disegni
\usepackage{pgfplots}
\pgfplotsset{width=10cm,compat=1.9}

% disponi codice
\usepackage{listings}
\usepackage[table]{xcolor}

\lstdefinestyle{codestyle}{
		backgroundcolor=\color{black!5}, 
		commentstyle=\color{codegreen},
		keywordstyle=\bfseries\color{magenta},
		numberstyle=\sffamily\tiny\color{black!60},
		stringstyle=\color{green!50!black},
		basicstyle=\ttfamily\footnotesize,
		breakatwhitespace=false,         
		breaklines=true,                 
		captionpos=b,                    
		keepspaces=true,                 
		numbers=left,                    
		numbersep=5pt,                  
		showspaces=false,                
		showstringspaces=false,
		showtabs=false,                  
		tabsize=2
}

\lstdefinestyle{shellstyle}{
		backgroundcolor=\color{black!5}, 
		basicstyle=\ttfamily\footnotesize\color{black}, 
		commentstyle=\color{black}, 
		keywordstyle=\color{black},
		numberstyle=\color{black!5},
		stringstyle=\color{black}, 
		showspaces=false,
		showstringspaces=false, 
		showtabs=false, 
		tabsize=2, 
		numbers=none, 
		breaklines=true
}

\lstdefinelanguage{javascript}{
	keywords={typeof, new, true, false, catch, function, return, null, catch, switch, var, if, in, while, do, else, case, break},
	keywordstyle=\color{blue}\bfseries,
	ndkeywords={class, export, boolean, throw, implements, import, this},
	ndkeywordstyle=\color{darkgray}\bfseries,
	identifierstyle=\color{black},
	sensitive=false,
	comment=[l]{//},
	morecomment=[s]{/*}{*/},
	commentstyle=\color{purple}\ttfamily,
	stringstyle=\color{red}\ttfamily,
	morestring=[b]',
	morestring=[b]"
}

% disponi sezioni
\usepackage{titlesec}

\titleformat{\section}
	{\sffamily\Large\bfseries} 
	{\thesection}{1em}{} 
\titleformat{\subsection}
	{\sffamily\large\bfseries}   
	{\thesubsection}{1em}{} 
\titleformat{\subsubsection}
	{\sffamily\normalsize\bfseries} 
	{\thesubsubsection}{1em}{}

% disponi alberi
\usepackage{forest}

\forestset{
	rectstyle/.style={
		for tree={rectangle,draw,font=\large\sffamily}
	},
	roundstyle/.style={
		for tree={circle,draw,font=\large}
	}
}

% disponi algoritmi
\usepackage{algorithm}
\usepackage{algorithmic}
\makeatletter
\renewcommand{\ALG@name}{Algoritmo}
\makeatother

% disponi numeri di pagina
\usepackage{fancyhdr}
\fancyhf{} 
\fancyfoot[L]{\sffamily{\thepage}}

\makeatletter
\fancyhead[L]{\raisebox{1ex}[0pt][0pt]{\sffamily{\@title \ \@date}}} 
\fancyhead[R]{\raisebox{1ex}[0pt][0pt]{\sffamily{\@author}}}
\makeatother

\begin{document}
% sezione (data)
\section{Lezione del 23-10-24}

% stili pagina
\thispagestyle{empty}
\pagestyle{fancy}

% testo
\subsection{Circuiti RC a regime sinusoidale}
Veniamo quindi a studiare circuiti dove i generatori non sono in regime costante, ma \textbf{sinusoidali}.
Un circuito si dice in \textbf{regime periodico sinusoidale} se tutte le correnti $i(t)$ e tutte le tensioni $v(t)$ possono essere espresse come funzioni sinusoidali con la stessa pulsazione.
Abbiamo quindi 3 proprietà:
Le grandezze devono essere descritte da funzioni \textbf{periodiche}, ergo:
		$$ 
		f(t + T) = f(t) \quad \forall t
		$$
		dove $T$ si dice \textbf{periodo}.
		Notiamo che periodico non significa sinusoidale, in quanto altri tipi di onde (quadre, a dente di sega, ecc...) sono ugualmente periodiche.
	In particolare noi tratteremo funzioni in forma:
	$$ 
		f(t) = F \cdot \sin(\omega t + \phi)	
	$$
	dove $F$ è l'\textbf{ampiezza} dell'oscillazione, $\omega$ la \textbf{pulsazione} e $\phi$ \textbf{fase}.

\begin{center}
\begin{tikzpicture}
    \begin{axis}[
        xlabel={$t$},
        ylabel={$f(t)$},
        domain=-10:10, % set the x range you want
				samples=100,
        grid=major, % add a grid
				ytick={-2, 2},
				yticklabels={$-F$, $F$},
				xtick={0, 7},
				xticklabels={$\phi$, $\phi + T$},
				width=15cm,
				height=7cm
    ]
    \addplot[
        blue,
        thick
    ] {2 * sin(50*x)}; 
    \end{axis}
\end{tikzpicture}
\end{center}

	Abbiamo che esiste una relazione fra pulsazione $\omega$ e periodo $T$, e che si può introdurre la \textbf{frequenza} $f$:
	$$
	\omega = \frac{2\pi}{T} = 2\pi f \quad f = \frac{1}{T}
	$$
	Si ha che il periodo si misura in secondi, la frequenza in Herz ($\mathrm{Hz}$), e la pulsazione in $\mathrm{rad}/\mathrm{s}$.

\subsection{Fasori}
Si ha che per lo studio delle equazioni differenziali date da sistemi a regime sinusoidale, torna utile passare ad un dominio \textbf{fasoriale} delle grandezze in questione.

Prendiamo una funzione sinusoidale in forma:
$$
x(t) = X_M \cdot \sin (\omega t + \phi)
$$
e trasformiamolo nella forma esponenziale:
$$
x(t) = x_m \cdot e^{j(\omega t + \phi)}
$$
dove $j$ rappresenta la \textbf{costante immaginaria}.

Chiamiamo $x(t)$ \textbf{vettore rotante}.
Si ha che il vettore rotante ha distanza dall'origine pari all'ampiezza dell'oscillazione, e si muove di moto rotante sul piano di Gauss con velocità angolare pari alla pulsazione dell'oscillazione, con relativa fase.

Chiamiamo il valore assunto da $x(t)$ in un dato momento $t_0 = 0$ \textbf{fasore} (da \textit{phase vector}), e lo indichiamo come $\dot{X}$.
Chiaramente, il vettore rotante assume valori complesso e il fasore stesso è un numero complesso:

\begin{center}
\begin{tikzpicture}
  \begin{axis}[
    axis lines = center,
    xlabel={$Re$},
    ylabel={$Im$},
    xmin=-5.4, xmax=5.4,
    ymin=-5.4, ymax=5.4,
    grid=minor,
  ]
    % Draw the phasor (example: magnitude=1, angle=45 degrees)
    \addplot[thick,->,red] coordinates {(0,0) (0.707*5,0.707*5)};
    
    % Optional: Add a label for the phasor
		\node at (axis cs: 0.707*5, 0.707*5) [anchor=west] {$\dot{I}$};

		\node at(axis cs: 0.5,0.3) [anchor=west] {$\phi$};
  \end{axis}
\end{tikzpicture}
\end{center}

Nell'esempio, posto $i(t) = 5 \cdot \sin (1000 t + \frac{\pi}{4})A$, avremo il fasore a $t_0 = 0$:
$$
\dot{I} = 5 \cdot e^{j \frac{\pi}{4}}
$$

Dal grafico vediamo poi che il modulo del vettore coincide col valore della corrente o della tensione che rappresenta, e che l'angolo che forma con l'asse delle ascisse rappresenta la fase $\phi$.

Visto che le pulsazioni sono comuni fra fasori diversi (circuiti a regime sinusoidale), abbiamo che le uniche informazioni che conserviamo nel diagramma dei fasori sono l'ampiezza e la fase.

\subsubsection{Proprietà dei fasori}
\begin{enumerate}
	\item \textbf{Addizione e sottrazione:} dati due fasori $\dot{I_1}$ e $\dot{I_2}$ si ha che:
		$$
		\dot{I_1} + \dot{I_2} = I_1 \cdot e^{j\phi_1} + I_2 \cdot e^{j\phi_2} = I_1 \cdot \cos(\phi_1) + j \cdot I_1 \cdot \sin(\phi_1) + I_2 \cdot \cos(\phi_2) + j \cdot I_2 \cdot \sin(\phi_2)
		$$
		$$
		= \left( I_1 \cdot \cos(\phi_1) \pm I_2 \cdot \cos(\phi_2) \right) + j (I_1 \cdot \sin(\phi_1) \pm I_2 \cdot \sin(\phi_2))
		$$
		dove $I_1$ e $I_2$ sono i moduli rispettivamente di $\dot{I_1}$ e $\dot{I_2}$.
		Abbiamo che si sommano separatamente parti reali e immaginarie, ergo si si va calcolare banalmente la \textbf{somma vettoriale} nel piano di Gauss dei due fasori.
	\item \textbf{Derivata:} definiamo la derivata di $x(t)$ rispetto al tempo come:
		$$
		y(t) = \frac{d\left(x(t)\right)}{dt} = \mathrm{Im}\left\{ \dot{X} e^{j \omega t} \right\} = \frac{d\left( \mathrm{Im}\left\{\dot{X}e^{j\omega t}\right\} \right)}{dt} = \mathrm{Im}\left\{ \frac{d\left( \dot{X} e^{j\omega t} \right)}{dt} \right\} 
		$$
		$$
		= \mathrm{Im}\left\{ \dot{X} \cdot \frac{d \left( e^{j \omega t} \right)}{dt} \right\} = \mathrm{Im} \{ \dot{X} e^{j \omega t} \cdot j \omega \} = \mathrm{Im} \{ j\omega \cdot \dot{X} \cdot e^{j\omega t} \}
		$$
		da cui:
		$$
		\dot{Y} = j \omega \dot{X}
		$$
	\item \textbf{Integrale:} definiamo l'integrale di un fasore come:
		$$
		y(t) = \int x(t) dt = \mathrm{Im}\left\{ \hat{Y} e^{j\omega t} \right\} = \int \mathrm{Im} \left\{ \dot{X} e^{j\omega t} \right\} dt = \mathrm{Im} \left\{ \int \dot{X} e^{j \omega t} dt \right\} 
		$$
		$$
		= \mathrm{Im} \left\{ \dot{X} \cdot \int e^{j \omega t} dt \right\} = \mathrm{Im} \left\{ \dot{X} \cdot \frac{e^{j \omega t}}{j\omega} \right\} = \mathrm{Im}\left\{ \frac{\dot{X}}{j\omega} e^{j \omega t} \right\}
		$$
		da cui:
		$$
		\dot{Y} = \frac{\dot{X}}{j\omega}
		$$
\end{enumerate}

\subsection{Dipoli in regime sinusoidale}
Vediamo come si comportano i dipoli visti finora in regime sinusoidale.

\subsubsection{Resistori}
Poniamo di avere un resistore in regime sinsuoidale.
Avevamo che:
$$
v_R = R i_R
$$
In quanto a fasori, avremo la stesssa legge:
$$
\dot{V}_R = R \dot{I}_R
$$

Ergo non cambia niente rispetto a quanto avevamo in regime costante.
Sul diagramma dei fasori, avremmo che i fasori tensione e corrente saranno fra di loro paralleli:
\begin{center}
\begin{tikzpicture}
  \begin{axis}[
    axis lines = center,
    xlabel={$Re$},
    ylabel={$Im$},
    xmin=-5.4, xmax=5.4,
    ymin=-5.4, ymax=5.4,
		xtick={0},
		ytick={0},
		grid=minor,
  ]
    % Draw the phasor (example: magnitude=1, angle=45 degrees)
    \addplot[thick,->,blue] coordinates {(0,0) (0.707*5,0.707*5)};
    \addplot[thick,->,red] coordinates {(0,0) (0.707*3,0.707*3)};
    
    % Optional: Add a label for the phasor
		\node at (axis cs: 0.707*5, 0.707*5) [anchor=south] {$\dot{V}$};
		\node at (axis cs: 0.707*3, 0.707*3) [anchor=south] {$\dot{I}$};
  \end{axis}
\end{tikzpicture}
\end{center}

\subsubsection{Induttori}
Un induttore è governato dalla legge:
$$
v_L(t) = L \frac{d \, i_L(t)}{dt}
$$
Sui fasori si avrà, applicando la legge di derivazione dei fasori:
$$
\dot{V}_L = L j \omega \dot{I}_L = j \omega L \cdot \dot{I}_L
$$

Notiamo come il legame differenziale della prima legge diventa algebrico nella seconda, e anzi si riconduce ad una forma che ricorda la legge di Ohm.
Sul diagramma dei fasori, avremo che i fasori tensione e corrente formano fra di loro un angolo di $\frac{\pi}{2}$:
\begin{center}
\begin{tikzpicture}
  \begin{axis}[
    axis lines = center,
    xlabel={$Re$},
    ylabel={$Im$},
    xmin=-5.4, xmax=5.4,
    ymin=-5.4, ymax=5.4,
		xtick={0},
		ytick={0},
		grid=minor,
  ]
    % Draw the phasor (example: magnitude=1, angle=45 degrees)
    \addplot[thick,->,blue] coordinates {(0,0) (0.707*5,0.707*5)};
    \addplot[thick,->,red] coordinates {(0,0) (-0.707*3,0.707*3)};
    
    % Optional: Add a label for the phasor
		\node at (axis cs: 0.707*5, 0.707*5) [anchor=west] {$\dot{V}$};
		\node at (axis cs: -0.707*3, 0.707*3) [anchor=east] {$\dot{I}$};
  \end{axis}
\end{tikzpicture}
\end{center}

\subsubsection{Condensatori}
Un condensatore è governato dalla legge:
$$
v_C(t) = \frac{1}{C} \int i_C(t) dt
$$
Sui fasori si avrà, applicando la legge di integrazione dei fasori:
$$
\dot{V}_C = \frac{1}{C} \frac{\dot{I}_C}{j\omega} = \frac{\dot{I}_C}{j \omega C}
$$

Come prima, ci riportiamo in una forma lineare.
Sul diagramma dei fasori, avremo che i fasori tensione e corrente formano fra di loro un angolo di $-\frac{\pi}{2}$:
\begin{center}
\begin{tikzpicture}
  \begin{axis}[
    axis lines = center,
    xlabel={$Re$},
    ylabel={$Im$},
    xmin=-5.4, xmax=5.4,
    ymin=-5.4, ymax=5.4,
		xtick={0},
		ytick={0},
		grid=minor,
  ]
    % Draw the phasor (example: magnitude=1, angle=45 degrees)
    \addplot[thick,->,blue] coordinates {(0,0) (0.707*5,0.707*5)};
    \addplot[thick,->,red] coordinates {(0,0) (0.707*3,-0.707*3)};
    
    % Optional: Add a label for the phasor
		\node at (axis cs: 0.707*5, 0.707*5) [anchor=west] {$\dot{V}$};
		\node at (axis cs: 0.707*3, -0.707*3) [anchor=west] {$\dot{I}$};
  \end{axis}
\end{tikzpicture}
\end{center}

\subsection{Analisi circuitale coi fasori}
La definizione dei fasori ci permette di studiare circuiti in regime sinusoidali attraverso gli stessi strumenti che abbiamo studiato finora, a patto di dover risolvere un sistema di equazioni complesse (che possiamo sempre dividere in parte reale e immaginaria).

\TODO % esempio appena puoi

\end{document}
