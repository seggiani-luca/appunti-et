
\documentclass[a4paper,11pt]{article}
\usepackage[a4paper, margin=8em]{geometry}

% usa i pacchetti per la scrittura in italiano
\usepackage[french,italian]{babel}
\usepackage[T1]{fontenc}
\usepackage[utf8]{inputenc}
\frenchspacing 

% usa i pacchetti per la formattazione matematica
\usepackage{amsmath, amssymb, amsthm, amsfonts}

% usa altri pacchetti
\usepackage{gensymb}
\usepackage{hyperref}
\usepackage{standalone}

% imposta il titolo
\title{Appunti Elettrotecnica}
\author{Luca Seggiani}
\date{2024}

% imposta lo stile
% usa helvetica
\usepackage[scaled]{helvet}
% usa palatino
\usepackage{palatino}
% usa un font monospazio guardabile
\usepackage{lmodern}

\renewcommand{\rmdefault}{ppl}
\renewcommand{\sfdefault}{phv}
\renewcommand{\ttdefault}{lmtt}

% disponi il titolo
\makeatletter
\renewcommand{\maketitle} {
	\begin{center} 
		\begin{minipage}[t]{.8\textwidth}
			\textsf{\huge\bfseries \@title} 
		\end{minipage}%
		\begin{minipage}[t]{.2\textwidth}
			\raggedleft \vspace{-1.65em}
			\textsf{\small \@author} \vfill
			\textsf{\small \@date}
		\end{minipage}
		\par
	\end{center}

	\thispagestyle{empty}
	\pagestyle{fancy}
}
\makeatother

% disponi teoremi
\usepackage{tcolorbox}
\newtcolorbox[auto counter, number within=section]{theorem}[2][]{%
	colback=blue!10, 
	colframe=blue!40!black, 
	sharp corners=northwest,
	fonttitle=\sffamily\bfseries, 
	title=~\thetcbcounter: #2, 
	#1
}

% disponi definizioni
\newtcolorbox[auto counter, number within=section]{definition}[2][]{%
	colback=red!10,
	colframe=red!40!black,
	sharp corners=northwest,
	fonttitle=\sffamily\bfseries,
	title=~\thetcbcounter: #2,
	#1
}

% U.D.M
\newcommand{\amp}{\ensuremath{\mathrm{A}}}
\newcommand{\volt}{\ensuremath{\mathrm{V}}}
\newcommand{\meter}{\ensuremath{\mathrm{m}}}
\newcommand{\second}{\ensuremath{\mathrm{s}}}
\newcommand{\farad}{\ensuremath{\mathrm{F}}}
\newcommand{\henry}{\ensuremath{\mathrm{H}}}
\newcommand{\siemens}{\ensuremath{\mathrm{S}}}

% circuiti
\usepackage{circuitikz}
\usetikzlibrary{babel}

% disegni
\usepackage{pgfplots}
\pgfplotsset{width=10cm,compat=1.9}

% disponi codice
\usepackage{listings}
\usepackage[table]{xcolor}

\lstdefinestyle{codestyle}{
		backgroundcolor=\color{black!5}, 
		commentstyle=\color{codegreen},
		keywordstyle=\bfseries\color{magenta},
		numberstyle=\sffamily\tiny\color{black!60},
		stringstyle=\color{green!50!black},
		basicstyle=\ttfamily\footnotesize,
		breakatwhitespace=false,         
		breaklines=true,                 
		captionpos=b,                    
		keepspaces=true,                 
		numbers=left,                    
		numbersep=5pt,                  
		showspaces=false,                
		showstringspaces=false,
		showtabs=false,                  
		tabsize=2
}

\lstdefinestyle{shellstyle}{
		backgroundcolor=\color{black!5}, 
		basicstyle=\ttfamily\footnotesize\color{black}, 
		commentstyle=\color{black}, 
		keywordstyle=\color{black},
		numberstyle=\color{black!5},
		stringstyle=\color{black}, 
		showspaces=false,
		showstringspaces=false, 
		showtabs=false, 
		tabsize=2, 
		numbers=none, 
		breaklines=true
}

\lstdefinelanguage{javascript}{
	keywords={typeof, new, true, false, catch, function, return, null, catch, switch, var, if, in, while, do, else, case, break},
	keywordstyle=\color{blue}\bfseries,
	ndkeywords={class, export, boolean, throw, implements, import, this},
	ndkeywordstyle=\color{darkgray}\bfseries,
	identifierstyle=\color{black},
	sensitive=false,
	comment=[l]{//},
	morecomment=[s]{/*}{*/},
	commentstyle=\color{purple}\ttfamily,
	stringstyle=\color{red}\ttfamily,
	morestring=[b]',
	morestring=[b]"
}

% disponi sezioni
\usepackage{titlesec}

\titleformat{\section}
	{\sffamily\Large\bfseries} 
	{\thesection}{1em}{} 
\titleformat{\subsection}
	{\sffamily\large\bfseries}   
	{\thesubsection}{1em}{} 
\titleformat{\subsubsection}
	{\sffamily\normalsize\bfseries} 
	{\thesubsubsection}{1em}{}

% disponi alberi
\usepackage{forest}

\forestset{
	rectstyle/.style={
		for tree={rectangle,draw,font=\large\sffamily}
	},
	roundstyle/.style={
		for tree={circle,draw,font=\large}
	}
}

% disponi algoritmi
\usepackage{algorithm}
\usepackage{algorithmic}
\makeatletter
\renewcommand{\ALG@name}{Algoritmo}
\makeatother

% disponi numeri di pagina
\usepackage{fancyhdr}
\fancyhf{} 
\fancyfoot[L]{\sffamily{\thepage}}

\makeatletter
\fancyhead[L]{\raisebox{1ex}[0pt][0pt]{\sffamily{\@title \ \@date}}} 
\fancyhead[R]{\raisebox{1ex}[0pt][0pt]{\sffamily{\@author}}}
\makeatother

\begin{document}
% sezione (data)
\section{Lezione del 17-10-24}

% stili pagina
\thispagestyle{empty}
\pagestyle{fancy}

% testo
\subsection{Condensatori}
Introduciamo un nuovo bipolo: il \textbf{condensatore} o \textit{capacitore}. 
Si indica come:

\begin{center}
	\begin{circuitikz}
		\draw (0,0) to [ C ] (2, 0); 
	\end{circuitikz}
\end{center}
ed è costituito da due armature di materiale conduttore, inframezzate da un \textbf{dielettrico}.

Il verso di percorrenza nei condensatori, come nei resistori, è irrilevante.
La loro funzione è quella di accumulare energia, secondo la legge:
$$
q(t) = C \cdot v(t)
$$
dove $C$ è la \textbf{capacità}, misurata in Farad ($\mathrm{F}$).

Nota la superficie delle armature e la distanza fra di esse, si può calcolare la capacità come:
$$
C = \varepsilon \cdot \frac{s}{d}
$$
Nel caso di dielettrici, si indica con $\varepsilon_0$ la costante dielettrica introdotta e si scrive:
$$
C = \varepsilon \cdot \varepsilon_0 \cdot \frac{s}{d}
$$

Si ricorda che il Farad è un'unita di misura molto grande, e solitamente si usano i sottomultipli:
\begin{table}[H]
	\center \rowcolors{2}{white}{black!10}
	\begin{tabular} { c | c }
		\bfseries Simbolo & \bfseries Ordine \\
		\hline 
		$ \mathrm{mF} $ & $10^{-3}$ \\
		$ \mathrm{\mu F} $ & $10^{-6}$ \\
		$ \mathrm{nF} $ & $10^{-9}$ \\
		$ \mathrm{pF} $ & $10^{-12}$ \\
	\end{tabular}
\end{table}

Diciamo che il condensatore ideale è:
\begin{itemize}
	\item \textbf{Lineare:} dalla legge $ q(t) = C \cdot v(t)$;
	\item \textbf{Tempo-invariante:} trascurando cambi di temperatura, si comportano come i resistori;
	\item \textbf{Con memoria:} visto che legano tensione a carica, dobbiamo prendere la corrente come derivata della carica:
		$$
		i(t) = \frac{dq(t)}{dt} = \frac{d}{dt}(C \cdot v(t)) = C \frac{dv(t)}{dt}
		$$
		Possiamo quindi integrare:
		$$
		v(t) = \int_{-\infty}^{t} \frac{1}{C} \cdot i(\tau) \, d\tau = \frac{1}{C} \int_{-\infty}^{t} i(\tau) \, d\tau = \frac{1}{C} \left[ \int_{-\infty}^{t_0} i(\tau) + \int_{t_0}^{t} i(\tau) \right] 
		$$
		$$
		= v(t_0) + \frac{1}{C} \int_{t_0}^{t} i(\tau) \, d\tau
		$$

		Abbiamo quindi che la tensione sul condensatore dipende dalla tensione iniziale $v(t_0)$ e dalle correnti precedeneti $i(t')$ a $t' < t$, ergo è un componente con memoria.
	\item \textbf{Passivo:} anche qui possiamo definire $p(t)$ e derivare:
		$$
		p(t) = v_C(t)i_C(t) = v_C(t) \cdot C\frac{dv_C(t)}{dt}
		$$
		da cui si ottiene che $p(t)$ ha qualsiasi segno.
		Vediamo quindi l'energia:
		$$
		w(t) = \int_{-\infty}^t p(\tau) \, d\tau = \int_{-\infty}^t C v_C(\tau) \frac{dv_C(\tau)}{d\tau} \, d\tau 
		$$
		$$
		= C \int_{-\infty}^t v_C(\tau) dv_C(\tau) = C \left[ \frac{1}{2} v_C^2 (\tau) \right]_{-\infty}^{t} 
		$$
		da cui si ha, risolvendo:
		$$
		w(t) = \frac{1}{2} C v_C^2(t) - \frac{1}{2} C v_C^2 (-\infty)
		$$
		
		Assumendo $v_C^2 (-\infty) = 0$, cioè condensatore inizialmente scarico, si ha $w(t) \geq 0$, ergo è un componente passivo.
		Solo nel caso in cui parte da carico il condensatore può (temporaneamente) erogare energia.
\end{itemize}

\subsection{Induttori}
Introduciamo un nuovo bipolo: l'\textbf{induttore} o \textit{induttanza}. 
Si indica come:

\begin{center}
	\begin{circuitikz}
		\draw (0,0) to [ inductor ] (2, 0); 
	\end{circuitikz}
\end{center}
ed è costituito da spire di materiale ferromagnetico avvolte attorno a un dielettrico.

La loro funzione è ancora quella di accumulare energia, secondo la legge:
$$
\phi(t) = L \cdot i_L (t)
$$
dove $L$ è l'\textbf{induttanza}, misurata in Henry ($\mathrm{H}$), e $\phi$ è il \textbf{flusso magnetico}, misurata in Weber ($\mathrm{Wb}$).
L'induttanza dipende dalla geometria dell'induttore, e ad esempio in un solenoide di $N$ spire di superficie $s$ su una lunghezza $l$ è:
$$ 
L = \mu \cdot \frac{S}{l} \cdot N^2 = \mu_0 \cdot \mu_r \frac{S}{l} N^2
$$

Vediamo quindi le proprietà:
\begin{itemize}
	\item \textbf{Lineare:} sempre per la legge $\phi(t) = L \cdot i_L (t)$;
	\item \textbf{Tempo-invariante:} il flusso interno dipende solo dalla corrente;
	\item \textbf{Con memoria:} possamo derivare la legge:
		$$ 
			v_L (t) = \frac{d \phi(t)}{dt} = \frac{d (L i_L(t))}{dt} = L\frac{d i_L(t)}{dt}
		$$
		e ricavare e derivare la corrente $i_L(t)$:
		$$
		i_L(t) = \frac{1}{L} \int_{-\infty}^t v_L (\tau) \, d\tau = i_L(t_0) + \frac{1}{L} \int_{t_0}^{t} v_L(\tau) \, d\tau
		$$
		da dove si ricava che, come il condensatore, l'induttore ha memoria di uno stato iniziale a $t_0$.
	\item \textbf{Passivo:} ritroviamo la potenza:
		$$
			p(t) = v_C(t) i_C(t) = L \frac{d i_C(t)}{dt} \cdot i_L(t)
		$$
		da cui si ottiene che $p(t)$ ha qualsiasi segno. Vediamo quindi l'energia:
		$$
		w(t) = \int_{-\infty}^{t} p(\tau) \, d \tau = \int_{-\infty}^{t} L \frac{d i_L(\tau)}{d\tau} \cdot i_L(\tau) d\tau = L \int_{-\infty}^t i_L(\tau) d i_L(\tau)
		$$
		da cui si ha:
		$$
		w(t) = \frac{1}{2} L i_L^2(t) - \frac{1}{2} L i_L^2 (-\infty)
		$$
		Come prima, assumendo $i_L^2(\infty) = 0$, cioè induttore inizialmente scarico, si ha $w(t) \geq 0$, e che l'induttore è un componente passivo (salvo parta da carico).
\end{itemize}

\subsubsection{Induttori mutuamente accoppiati}
Si possono avere più induttanze (prendiamo 2 per semplicità) accoppiate fra di loro attraverso l'effetto del magnetico generato da entrambe sulle reciproche spire, cioè:

\begin{center}
	\begin{circuitikz}
		\draw (0,0) to[ short, i=$i_1$] (1,0)
			to[ inductor , l=$L_1$] (1,-2)
			-- (0, -2);

		\draw (4,0) to[ short, i_=$i_2$] (3,0)
			to[ inductor , l_=$L_2$] (3,-2)
			-- (4, -2);

			\draw (0.9,-0.6) node {$\scriptscriptstyle\bullet$};
			\draw (2.9,-0.6) node {$\scriptscriptstyle\bullet$};

			\draw (0,0) node[anchor=east] {$A$};
			\draw (0,-2) node[anchor=east] {$B$};

			\draw (0,-0.5) node[anchor=east] {$+$};
			\draw (0,-1.5) node[anchor=east] {$-$};

			\draw (4,0) node[anchor=west] {$C$};
			\draw (4,-2) node[anchor=west] {$D$};

			\draw (4,-0.5) node[anchor=west] {$+$};
			\draw (4,-1.5) node[anchor=west] {$-$};

			\draw (0,-1) node[anchor=east] {$v_1$};
			\draw (4,-1) node[anchor=west] {$v_2$};
	\end{circuitikz}
\end{center}

Dove il flusso su una e l'altra induttanza è espresso come:
\[
	\begin{cases}
			
  \phi_1(t) = \phi_{1.1} \pm \phi_{1.2} = L_1 \cdot i_1 (t) \pm M i_2 (t) \\
  \phi_2(t) = \phi_{1.2} \pm \phi_{1.1} = L_2 \cdot i_2 (t) \pm M i_1 (t) 
	\end{cases}
\]

Qui $M$ prende il nome di \textbf{coefficiente di mutua induzione}.
Anche questo coefficiente dipende dalla geometria della configurazione degli induttori.

Chiamiamo quindi la componente $L_i \cdot i_i$ \textbf{caduta di auto}, per \textit{caduta di auto induttanza}, e la componente $M \cdot i_i$ \textbf{caduta di mutua}, per \textit{caduta di mutua induttanza}.
Conviene fare una riflessione sui segni delle cadute di auto e di mutua: 
\begin{itemize}
	\item Prendendo riferimenti associati, cioè percorrendo le induttanze nella direzione della corrente propria, si ha che le cadute di auto sono positive. In caso contrario, si prendono come negative;
	\item Per i segni delle cadute di mutua si usa la regola dei \textbf{contrassegni}:
		\begin{itemize}
			\item Se la corrente entra al terminale contrassegnato di un induttore e induce una forza elettromotrice $\mathcal{E}$ positiva al terminale contrassegnato dell'altro induttore, si ha che $M>0$;
			\item Altrimenti, se la corrente entra al terminale contrassegnato di un induttore e induce una forza elettromotrice $\mathcal{E}$ negativa al terminale contrassegnato dell'altro induttore, si ha che $M<0$;
		\end{itemize}
	ergo, scelti riferimenti associati per le cadute di auto, si ha che se le correnti raggiungono il contrassegno della propria induttanza entrambe entrando o uscendo dall'induttore, le cadute di mutua sono positive.
	Altrimenti, se le correnti raggiungono il contrassegno della propria induttanza una entrando e una uscendo dall'induttore, o viceversa, le cadute di mutua sono negative.
\end{itemize}

Si ha quindi, derivando:
\[
	\begin{cases}
		v_1(t) = L_1 \frac{d i_1(t)}{dt} \pm M \frac{d i_2(t)}{dt} \\ 
		v_2(t) = L_2 \frac{d i_2(t)}{dt} \pm M \frac{d i_1(t)}{dt}
	\end{cases}
\]

Gli induttori mutuamente accoppiati vengono detti \textbf{quadripoli} o \textbf{doppi bipoli}, in quanto hanno effettivamente 4 poli.

Si calcola la potenza semplicemente sommando le potenze sulle singole induttanze:
$$
p(t) = v_1(t)i_1(t) + v_2(t)i_2(t)  
$$
$$
= \left( L_1 \frac{d i_1(t)}{dt} \pm M \frac{d i_2 (t)}{dt} \right) \cdot i_2(t) +  \left( L_2 \frac{d i_2(t)}{dt} \pm M \frac{d i_1 (t)}{dt} \right) \cdot i_1(t)
$$
$$
= L_1 i_1(t) \frac{d i_1(t)}{dt} + L_2 i_2 \frac{d i_2(t)}{dt} \pm M \left( i_1(t) \frac{d i_2(t)}{dt} + \frac{i_2 d i_1(t)}{dt} \right)
$$
E l'energia integrando la potenza:
$$
w(t) = \int_{-\infty}^t p(\tau) d\tau 
= L_1 \int_{-\infty}^t i_1(\tau) \frac{d i_1 (\tau)}{d\tau}d\tau + L_2 \int_{-\infty}^t i_2(\tau) \frac{d i_2 (\tau)}{d\tau}d\tau 
$$
$$
\pm M \int_{-\infty}^t \left[ i_1(\tau) \frac{d i_2(\tau)}{d \tau} + i_2(\tau) \frac{d i_1(\tau)}{d \tau} \right] d\tau
$$
$$
= L_1 \cdot \frac{1}{2} i_1^2(t) + L_2 \cdot \frac{1}{2} i_2^2(t) \pm M \int_{-\infty}^t \frac{d(i_1(\tau) \cdot i_2(\tau))}{d\tau}d\tau
$$
$$
= L_1 \cdot \frac{1}{2} i_1^2(t) + L_2 \cdot \frac{1}{2} i_2^2(t) \pm M i_1(t)i_2(t) 
$$

Si può dimostrare che $ M \leq \sqrt{L_1 L_2}$, e nel caso $M = \sqrt{L_1 L_2}$ si parla di \textbf{accoppiamento ideale}. Nel caso peggiore si ha quindi:
$$
w = \frac{1}{2} L_1 i_1^2(t) + \frac{1}{2} L_2 i_2^2(t) - \sqrt{L_1 L_2} i_1(t) i_2(t) = \frac{1}{2}\left( \sqrt{L_1} i_1(t) - \sqrt{L_2} i_2(t) \right)^2 \geq 0
$$
ergo, salvo caricamenti iniziali, le induttanze mutuamente accoppiate sono componenti \textbf{passivi}.

\subsection{Circuiti con impedenze a regime costante}
Vediamo il comportamento dei condensatori e degli induttori in circuiti a regime costante, cioè come quelli che abbiamo studiato finora.
\begin{itemize}
	\item \textbf{Condensatori:} il condensatore ha legge $q(t) = C \cdot v_C(t) \Rightarrow i_C(t) = C \frac{dv_C(t)}{dt}$, ergo se siamo in continua, $i_C(t) = C \cdot 0 = 0$ e il condensatore si comporta come un \textbf{aperto};
	\item \textbf{Induttori:} l'induttore ha legge $\phi(t) = L \cdot i_L(t) \Rightarrow V_L(t) = L \frac{di_L(t)}{dt}$. ergo se siamo in continua, $v_L (t) = L \cdot 0 = 0$, e l'induttore si comporta come un \textbf{cortocircuito}.
\end{itemize}
Abbiamo quindi che non si possono apprezzare gli effetti di questi nuovi bipoli finché si studiano circuiti a regime stazionario.
Infatti vedremo più nel dettaglio quali sono le loro caratteristiche quando parleremo di circuiti in regime sinusoidale.
\end{document}
