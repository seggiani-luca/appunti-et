
\documentclass[a4paper,11pt]{article}
\usepackage[a4paper, margin=8em]{geometry}

% usa i pacchetti per la scrittura in italiano
\usepackage[french,italian]{babel}
\usepackage[T1]{fontenc}
\usepackage[utf8]{inputenc}
\frenchspacing 

% usa i pacchetti per la formattazione matematica
\usepackage{amsmath, amssymb, amsthm, amsfonts}

% usa altri pacchetti
\usepackage{gensymb}
\usepackage{hyperref}
\usepackage{standalone}

% imposta il titolo
\title{Appunti Elettrotecnica}
\author{Luca Seggiani}
\date{2024}

% imposta lo stile
% usa helvetica
\usepackage[scaled]{helvet}
% usa palatino
\usepackage{palatino}
% usa un font monospazio guardabile
\usepackage{lmodern}

\renewcommand{\rmdefault}{ppl}
\renewcommand{\sfdefault}{phv}
\renewcommand{\ttdefault}{lmtt}

% disponi il titolo
\makeatletter
\renewcommand{\maketitle} {
	\begin{center} 
		\begin{minipage}[t]{.8\textwidth}
			\textsf{\huge\bfseries \@title} 
		\end{minipage}%
		\begin{minipage}[t]{.2\textwidth}
			\raggedleft \vspace{-1.65em}
			\textsf{\small \@author} \vfill
			\textsf{\small \@date}
		\end{minipage}
		\par
	\end{center}

	\thispagestyle{empty}
	\pagestyle{fancy}
}
\makeatother

% disponi teoremi
\usepackage{tcolorbox}
\newtcolorbox[auto counter, number within=section]{theorem}[2][]{%
	colback=blue!10, 
	colframe=blue!40!black, 
	sharp corners=northwest,
	fonttitle=\sffamily\bfseries, 
	title=~\thetcbcounter: #2, 
	#1
}

% disponi definizioni
\newtcolorbox[auto counter, number within=section]{definition}[2][]{%
	colback=red!10,
	colframe=red!40!black,
	sharp corners=northwest,
	fonttitle=\sffamily\bfseries,
	title=~\thetcbcounter: #2,
	#1
}

% U.D.M
\newcommand{\amp}{\ensuremath{\mathrm{A}}}
\newcommand{\volt}{\ensuremath{\mathrm{V}}}
\newcommand{\meter}{\ensuremath{\mathrm{m}}}
\newcommand{\second}{\ensuremath{\mathrm{s}}}
\newcommand{\farad}{\ensuremath{\mathrm{F}}}
\newcommand{\henry}{\ensuremath{\mathrm{H}}}
\newcommand{\siemens}{\ensuremath{\mathrm{S}}}

% circuiti
\usepackage{circuitikz}
\usetikzlibrary{babel}

% disegni
\usepackage{pgfplots}
\pgfplotsset{width=10cm,compat=1.9}

% disponi codice
\usepackage{listings}
\usepackage[table]{xcolor}

\lstdefinestyle{codestyle}{
		backgroundcolor=\color{black!5}, 
		commentstyle=\color{codegreen},
		keywordstyle=\bfseries\color{magenta},
		numberstyle=\sffamily\tiny\color{black!60},
		stringstyle=\color{green!50!black},
		basicstyle=\ttfamily\footnotesize,
		breakatwhitespace=false,         
		breaklines=true,                 
		captionpos=b,                    
		keepspaces=true,                 
		numbers=left,                    
		numbersep=5pt,                  
		showspaces=false,                
		showstringspaces=false,
		showtabs=false,                  
		tabsize=2
}

\lstdefinestyle{shellstyle}{
		backgroundcolor=\color{black!5}, 
		basicstyle=\ttfamily\footnotesize\color{black}, 
		commentstyle=\color{black}, 
		keywordstyle=\color{black},
		numberstyle=\color{black!5},
		stringstyle=\color{black}, 
		showspaces=false,
		showstringspaces=false, 
		showtabs=false, 
		tabsize=2, 
		numbers=none, 
		breaklines=true
}

\lstdefinelanguage{javascript}{
	keywords={typeof, new, true, false, catch, function, return, null, catch, switch, var, if, in, while, do, else, case, break},
	keywordstyle=\color{blue}\bfseries,
	ndkeywords={class, export, boolean, throw, implements, import, this},
	ndkeywordstyle=\color{darkgray}\bfseries,
	identifierstyle=\color{black},
	sensitive=false,
	comment=[l]{//},
	morecomment=[s]{/*}{*/},
	commentstyle=\color{purple}\ttfamily,
	stringstyle=\color{red}\ttfamily,
	morestring=[b]',
	morestring=[b]"
}

% disponi sezioni
\usepackage{titlesec}

\titleformat{\section}
	{\sffamily\Large\bfseries} 
	{\thesection}{1em}{} 
\titleformat{\subsection}
	{\sffamily\large\bfseries}   
	{\thesubsection}{1em}{} 
\titleformat{\subsubsection}
	{\sffamily\normalsize\bfseries} 
	{\thesubsubsection}{1em}{}

% disponi alberi
\usepackage{forest}

\forestset{
	rectstyle/.style={
		for tree={rectangle,draw,font=\large\sffamily}
	},
	roundstyle/.style={
		for tree={circle,draw,font=\large}
	}
}

% disponi algoritmi
\usepackage{algorithm}
\usepackage{algorithmic}
\makeatletter
\renewcommand{\ALG@name}{Algoritmo}
\makeatother

% disponi numeri di pagina
\usepackage{fancyhdr}
\fancyhf{} 
\fancyfoot[L]{\sffamily{\thepage}}

\makeatletter
\fancyhead[L]{\raisebox{1ex}[0pt][0pt]{\sffamily{\@title \ \@date}}} 
\fancyhead[R]{\raisebox{1ex}[0pt][0pt]{\sffamily{\@author}}}
\makeatother

\begin{document}
% sezione (data)
\section{Lezione del 14-11-24}

% stili pagina
\thispagestyle{empty}
\pagestyle{fancy}

% testo
\subsection{Legge di Ohm per circuiti a due porte}
Abbiamo visto come, su circuiti a due porte, \textit{tensione} e \textit{corrente} sono \textbf{vettori} e \textit{impedenza} e \textit{ammettenza} sono \textbf{matrici}, ergo valgono le equazioni:
\[
	\begin{cases}
		\dot{V} = \overline{Z} \dot{I} \\ 
		\dot{I} = \overline{Y} \dot{V}
	\end{cases}
\]
con $\overline{Y} = \overline{Z}^{-1}$. 

\subsection{Sintesi a parametri ibridi}
Finora abbiamo scelto le variabili indipendenti come entrambe le tensioni o entrambe le correnti. 
Nessuno ci nega però di scegliere come variabili indipendenti una tensione e una corrente.
Chiamiamo la parametrizzazione che otteniamo da questa scelta \textbf{sintesi a parametri ibridi}, o \textbf{a parametri h}.

La forma generale di una sintesi a parametri ibridi è:
\[
	\begin{cases}
		\dot{V}_1 = \overline{h_{11}} \dot{I}_1 + \overline{h_{12}} \dot{V}_2 \\ 
		\dot{I}_2 = \overline{h_{21}} \dot{I}_1 + \overline{h_{22}} \dot{V}_2
	\end{cases}
\]
o come matrice:
$$
\begin{pmatrix}
	\dot{V}_1 \\ \dot{I}_2
\end{pmatrix}
= \overline{h}
\begin{pmatrix}
	\dot{I}_1 \\ \dot{V}_2
\end{pmatrix}
$$

Le componenti di $\overline{h}$ si ricavano quindi come:
\[
	\begin{cases}
		\overline{h_{11}} = \frac{\dot{V}_1}{\dot{I}_1} \Big|_{\dot{V}_2 = 0} \\
		\overline{h_{12}} = \frac{\dot{V}_1}{\dot{V}_2} \Big|_{\dot{I}_1 = 0} \\
		\overline{h_{21}} = \frac{\dot{I}_2}{\dot{I}_1} \Big|_{\dot{V}_2 = 0} \\
		\overline{h_{22}} = \frac{\dot{I}_2}{\dot{V}_2} \Big|_{\dot{I}_1 = 0}
	\end{cases}
\]

Questo tipo di sintesi ha un significato interessante sui vari parametri:
\begin{itemize}
	\item $\overline{h_{11}}$: impedenza in entrata;
	\item $\overline{h_{12}}$: inverso dell'amplificazione di tensione;
	\item $\overline{h_{21}}$: amplificazione di corrente;
	\item $\overline{h_{22}}$: inverso dell impedenza in uscita.
\end{itemize}

Il circuito equivalente che possiamo formare da una sintesi a parametri ibridi è il seguente: 

\begin{center}
	\begin{circuitikz}
		\draw (-4, 1) -- (-3, 1) 
			to [ european resistor, l=$\overline{h_{11}}$, i=$I_1$] (-1, 1)
			to [ controlled voltage source, v<=$\overline{h_{12}} \dot{V}_2$ ] (-1, -1) 
			to [ short, i=$I_1$ ] (-3, -1)	
			-- (-4, -1);
			
		\draw (-4.6, 1) node[anchor=west] {$+$};
		\draw (-4.6, 0) node[anchor=west] {$V_1$};
		\draw (-4.6, -1) node[anchor=west] {$-$};

		\draw (6,1) to [ short, i=$I_2$] (5, 1) 
			to [ short] (3, 1)
			to [ controlled current source, cI_>=$\overline{h_{21}} \dot{I}_2$ ] (3, -1) 
			to [ short] (5, -1)
			-- (6, -1);
	
		\draw (6.6, 1) node[anchor=east] {$+$};
		\draw (6.6, 0) node[anchor=east] {$V_2$};
		\draw (6.6, -1) node[anchor=east] {$-$};
		
		\draw (4, 1) to [ european resistor, l=$\overline{h_{22}}$] (4, -1);

		\node[rectangle, draw, minimum width = 8.5cm, minimum height = 4cm] (a) at (1,0) {};
	\end{circuitikz}
\end{center}

# riguarda tutti i circuiti sono sbagliati

\subsubsection{Condizioni di reciprocità}
Vediamo quindi quando ci troviamo in condizioni di reciprocità.
Si ha dalle formule della rappresentazione:
\[
	\begin{cases}
		\dot{V}_1 = \overline{h_{11}} \dot{I}_1 + \overline{h_{12}} \dot{V}_2 \\ 
		\dot{I}_2 = \overline{h_{21}} \dot{I}_1 + \overline{h_{22}} \dot{V}_2
	\end{cases}
\]

Dalla seconda equazione possiamo scrivere:
$$
\overline{h_{22}} \dot{V}_2 = - \overline{h_{21}} \dot{I_1} + \dot{I_2} \Rightarrow \dot{V}_2 = -\frac{\overline{h_{21}}}{\overline{h_{22}}} \dot{I}_1 + \frac{1}{\overline{h_{22}}} \dot{I}_2 
$$
dove $\frac{\overline{h_{21}}}{\overline{h_{22}}} = \overline{z_{21}}$ e $\frac{1}{\overline{h_{22}}} = \overline{z_{22}}$ sono effettivamente i parametri Z del circuito.
Analogamente, dalla prima equazione possiamo scrivere:
$$
\dot{V}_1 = \overline{h_{11}} \dot{I_1} + \overline{h_{12}} \left(  -\frac{\overline{h_{21}}}{\overline{h_{22}}} \dot{I}_1 + \frac{1}{\overline{h_{22}}} \dot{I}_2  \right) \Rightarrow \dot{V}_1 = \left( \overline{h_{11}} - \frac{\overline{h_{12}}\overline{h_{21}}}{\overline{h_{22}}} \right) \dot{I_1} + \frac{\overline{h_{12}}}{\overline{h_{22}}} \dot{I}_2 
$$
dove ancora una volta $\overline{h_{11}} - \frac{\overline{h_{12}}\overline{h_{21}}}{\overline{h_{22}}} = z_{11}$ e $\frac{\overline{h_{12}}}{\overline{h_{22}}} = z_{12}$ sono i parametri Z del circuito. # risvolgi tutti sti calcoli

Dalle equazioni si ha che il circuito è \textit{reciproco} quando la matrice dei parametri h è \textbf{antisimmetrica}, cioè negata attorno alla diagonale # rivedi la definizione di matrice antisimmetrica.



\end{document}
