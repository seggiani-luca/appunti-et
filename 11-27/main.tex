
\documentclass[a4paper,11pt]{article}
\usepackage[a4paper, margin=8em]{geometry}

% usa i pacchetti per la scrittura in italiano
\usepackage[french,italian]{babel}
\usepackage[T1]{fontenc}
\usepackage[utf8]{inputenc}
\frenchspacing 

% usa i pacchetti per la formattazione matematica
\usepackage{amsmath, amssymb, amsthm, amsfonts}

% usa altri pacchetti
\usepackage{gensymb}
\usepackage{hyperref}
\usepackage{standalone}

% imposta il titolo
\title{Appunti Elettrotecnica}
\author{Luca Seggiani}
\date{2024}

% imposta lo stile
% usa helvetica
\usepackage[scaled]{helvet}
% usa palatino
\usepackage{palatino}
% usa un font monospazio guardabile
\usepackage{lmodern}

\renewcommand{\rmdefault}{ppl}
\renewcommand{\sfdefault}{phv}
\renewcommand{\ttdefault}{lmtt}

% disponi il titolo
\makeatletter
\renewcommand{\maketitle} {
	\begin{center} 
		\begin{minipage}[t]{.8\textwidth}
			\textsf{\huge\bfseries \@title} 
		\end{minipage}%
		\begin{minipage}[t]{.2\textwidth}
			\raggedleft \vspace{-1.65em}
			\textsf{\small \@author} \vfill
			\textsf{\small \@date}
		\end{minipage}
		\par
	\end{center}

	\thispagestyle{empty}
	\pagestyle{fancy}
}
\makeatother

% disponi teoremi
\usepackage{tcolorbox}
\newtcolorbox[auto counter, number within=section]{theorem}[2][]{%
	colback=blue!10, 
	colframe=blue!40!black, 
	sharp corners=northwest,
	fonttitle=\sffamily\bfseries, 
	title=~\thetcbcounter: #2, 
	#1
}

% disponi definizioni
\newtcolorbox[auto counter, number within=section]{definition}[2][]{%
	colback=red!10,
	colframe=red!40!black,
	sharp corners=northwest,
	fonttitle=\sffamily\bfseries,
	title=~\thetcbcounter: #2,
	#1
}

% U.D.M
\newcommand{\amp}{\ensuremath{\mathrm{A}}}
\newcommand{\volt}{\ensuremath{\mathrm{V}}}
\newcommand{\meter}{\ensuremath{\mathrm{m}}}
\newcommand{\second}{\ensuremath{\mathrm{s}}}
\newcommand{\farad}{\ensuremath{\mathrm{F}}}
\newcommand{\henry}{\ensuremath{\mathrm{H}}}
\newcommand{\siemens}{\ensuremath{\mathrm{S}}}

% circuiti
\usepackage{circuitikz}
\usetikzlibrary{babel}

% disegni
\usepackage{pgfplots}
\pgfplotsset{width=10cm,compat=1.9}

% disponi codice
\usepackage{listings}
\usepackage[table]{xcolor}

\lstdefinestyle{codestyle}{
		backgroundcolor=\color{black!5}, 
		commentstyle=\color{codegreen},
		keywordstyle=\bfseries\color{magenta},
		numberstyle=\sffamily\tiny\color{black!60},
		stringstyle=\color{green!50!black},
		basicstyle=\ttfamily\footnotesize,
		breakatwhitespace=false,         
		breaklines=true,                 
		captionpos=b,                    
		keepspaces=true,                 
		numbers=left,                    
		numbersep=5pt,                  
		showspaces=false,                
		showstringspaces=false,
		showtabs=false,                  
		tabsize=2
}

\lstdefinestyle{shellstyle}{
		backgroundcolor=\color{black!5}, 
		basicstyle=\ttfamily\footnotesize\color{black}, 
		commentstyle=\color{black}, 
		keywordstyle=\color{black},
		numberstyle=\color{black!5},
		stringstyle=\color{black}, 
		showspaces=false,
		showstringspaces=false, 
		showtabs=false, 
		tabsize=2, 
		numbers=none, 
		breaklines=true
}

\lstdefinelanguage{javascript}{
	keywords={typeof, new, true, false, catch, function, return, null, catch, switch, var, if, in, while, do, else, case, break},
	keywordstyle=\color{blue}\bfseries,
	ndkeywords={class, export, boolean, throw, implements, import, this},
	ndkeywordstyle=\color{darkgray}\bfseries,
	identifierstyle=\color{black},
	sensitive=false,
	comment=[l]{//},
	morecomment=[s]{/*}{*/},
	commentstyle=\color{purple}\ttfamily,
	stringstyle=\color{red}\ttfamily,
	morestring=[b]',
	morestring=[b]"
}

% disponi sezioni
\usepackage{titlesec}

\titleformat{\section}
	{\sffamily\Large\bfseries} 
	{\thesection}{1em}{} 
\titleformat{\subsection}
	{\sffamily\large\bfseries}   
	{\thesubsection}{1em}{} 
\titleformat{\subsubsection}
	{\sffamily\normalsize\bfseries} 
	{\thesubsubsection}{1em}{}

% disponi alberi
\usepackage{forest}

\forestset{
	rectstyle/.style={
		for tree={rectangle,draw,font=\large\sffamily}
	},
	roundstyle/.style={
		for tree={circle,draw,font=\large}
	}
}

% disponi algoritmi
\usepackage{algorithm}
\usepackage{algorithmic}
\makeatletter
\renewcommand{\ALG@name}{Algoritmo}
\makeatother

% disponi numeri di pagina
\usepackage{fancyhdr}
\fancyhf{} 
\fancyfoot[L]{\sffamily{\thepage}}

\makeatletter
\fancyhead[L]{\raisebox{1ex}[0pt][0pt]{\sffamily{\@title \ \@date}}} 
\fancyhead[R]{\raisebox{1ex}[0pt][0pt]{\sffamily{\@author}}}
\makeatother

\begin{document}
% sezione (data)
\section{Lezione del 27-11-24}

% stili pagina
\thispagestyle{empty}
\pagestyle{fancy}

% testo
\subsection{Antitrasformata con soluzioni complesse}
Esistono eccezioni al processo visto di antitrasformazione visto finora.
Prendiamo ad esempio la forma rapporto di polinomi:
$$
I(s) = \frac{s + 2}{s^2 + 9}
$$
Notiamo come i poli saranno numeri complessi.

Si avrà quindi:
$$
= \frac{s + 2}{(s+3j)(s-3j)} = \frac{A_1}{s+3j} + \frac{A_2}{s-3j}
$$

Applicando il teorema dei residui per trovare $A_1$ e $A_2$ si ottiene:
$$
A_1 = \lim_{s\rightarrow -3j} (s + 3j) \frac{s + 2}{(s+3j)(s-3j)} = \frac{2-3j}{-6j} = \frac{1}{2} + \frac{1}{3}j
$$
$$
A_2 = \lim_{s\rightarrow 3j} (s - 3j) \frac{s + 2}{(s+3j)(s-3j)} = \frac{2+3j}{6j} = \frac{1}{2} - \frac{1}{3}j
$$
da cui l'espressione in funzione di $t$:
$$
i(t) = \left( \left( \frac{1}{2} + \frac{1}{3}j \right) e^{-3j t} + \left( \frac{1}{2} - \frac{1}{3}j \right) e^{3jt} \right) u(t)
$$
dove notiamo che anche i coefficienti, oltre che gli esponenti, risultano numeri complessi.
Prese le radici complesse nell'ordine $+$, $-$, si ha che il primo esponenziale ha argomento $\omega$ negativo.

Abbiamo quindi che incontreremo generalmente soluzioni in forma:
$$
i(t) = \left( M + jN \right) e^{-(\sigma + j \omega)t} + \left( M - jN \right) e^{-(\sigma - j \omega)t}
$$
da cui:
$$
= M e^{-\sigma t}e^{-j \omega t} + j N e^{-\sigma t}e^{-j \omega t} + M e^{-\sigma t}e^{j \omega t} - jN e^{-\sigma t}e^{j \omega t}
$$
$$
= M \left( e^{- \sigma t}e^{-j \omega t} + e^{- \sigma t}e^{j \omega t} \right) + j N  \left( e^{-\sigma t}e^{-j \omega t} - e^{-\sigma t}e^{j \omega t} \right)
$$
$$
= M e^{- \sigma t} \left( e^{-j \omega t} + e^{j \omega t} \right) + j N e^{-\sigma t} \left( e^{-j \omega t} - e^{j \omega t} \right)
$$
visto che:
\[
	\begin{cases}
		e^{j \omega t} = \cos(\omega t) + i \sin(\omega t) \\ 	
		e^{-j \omega t} = \cos(\omega t) - i \sin(\omega t) \\ 	
	\end{cases}
\]
si ha:
$$
i(t) = M e^{-\sigma t} ( \cos(\omega t) + i \sin(\omega t) + \cos(\omega t) - i \sin(\omega t) ) 
$$
$$
+ j N e^{-\sigma t} ( \cos(\omega t) - i \sin(\omega t) - \cos(\omega t) - i \sin(\omega t) )
$$
$$
= 2 M e^{-\sigma t} \cos{\omega t} - 2 N e^{-\sigma t} \sin(\omega t)
$$

Vogliamo riportare questa forma nella più concisa $ke^{-\sigma t} \sin(\omega t + \alpha)$. Usiamo allora le formule di traduzione in forma sinusoidale, di cui una dimostrazione nel caso cosinusoidale si trova a \url{https://github.com/seggiani-luca/appunti-fis/blob/main/master/master.pdf}:
\[
	c_1 \sin{\omega t} + c_2 \cos{\omega t}
	\Leftrightarrow
	k \sin(\omega t + \alpha)
\]
con:
\[
	\begin{cases}
		k = \sqrt{c_1^2 + c_2^2} \\ 
		\alpha = \tan^{-1}(\frac{c1}{c2})
	\end{cases}
\]

da cui diciamo:
$$
i(t) = 2 M e^{-\sigma t} \cos{\omega t} + 2 N e^{-\sigma t} \sin(\omega t)
$$
$$
= 2 e^{-\sigma t} \left( M \cos{\omega t} + N \sin{\omega t} \right)
= ke^{-\sigma t} \sin(\omega t + \alpha)
$$
con:
\[
	\begin{cases}
		k = 2\sqrt{M^2 + N^2} \\ 
		\alpha = \tan^{-1}\left(\frac{M}{N}\right)
	\end{cases}
\]

\par\smallskip 

Possiamo quindi riportare l'esempio precedente in questa forma.
Si avrà:
$$
i(t) = \left( \left( \frac{1}{2} + \frac{1}{3}j \right) e^{-3j t} + \left( \frac{1}{2} - \frac{1}{3}j \right) e^{3jt} \right) u(t)
$$
$$
= \left( 2M \cos(\omega t) + 2N \sin(\omega t) \right) \Big|_{M = \frac{1}{2}, \, N =\frac{1}{3}} = \cos(3t) + \frac{2}{3}\sin(3t)
$$
o la forma più compatta $k \sin(\omega t + \alpha)u(t)$, con:
\[
	\begin{cases}
		k = 2\sqrt{\left( \frac{1}{3} \right)^2 + \left( \frac{1}{3} \right)^2} \approx 1.201 \\ 
		\alpha = \tan^{-1} \left( \frac{M}{N} \right) \approx 0.983
	\end{cases}
\]

\end{document}
