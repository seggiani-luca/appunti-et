
\documentclass[a4paper,11pt]{article}
\usepackage[a4paper, margin=8em]{geometry}

% usa i pacchetti per la scrittura in italiano
\usepackage[french,italian]{babel}
\usepackage[T1]{fontenc}
\usepackage[utf8]{inputenc}
\frenchspacing 

% usa i pacchetti per la formattazione matematica
\usepackage{amsmath, amssymb, amsthm, amsfonts}

% usa altri pacchetti
\usepackage{gensymb}
\usepackage{hyperref}
\usepackage{standalone}

% imposta il titolo
\title{Appunti Elettrotecnica}
\author{Luca Seggiani}
\date{2024}

% imposta lo stile
% usa helvetica
\usepackage[scaled]{helvet}
% usa palatino
\usepackage{palatino}
% usa un font monospazio guardabile
\usepackage{lmodern}

\renewcommand{\rmdefault}{ppl}
\renewcommand{\sfdefault}{phv}
\renewcommand{\ttdefault}{lmtt}

% disponi il titolo
\makeatletter
\renewcommand{\maketitle} {
	\begin{center} 
		\begin{minipage}[t]{.8\textwidth}
			\textsf{\huge\bfseries \@title} 
		\end{minipage}%
		\begin{minipage}[t]{.2\textwidth}
			\raggedleft \vspace{-1.65em}
			\textsf{\small \@author} \vfill
			\textsf{\small \@date}
		\end{minipage}
		\par
	\end{center}

	\thispagestyle{empty}
	\pagestyle{fancy}
}
\makeatother

% disponi teoremi
\usepackage{tcolorbox}
\newtcolorbox[auto counter, number within=section]{theorem}[2][]{%
	colback=blue!10, 
	colframe=blue!40!black, 
	sharp corners=northwest,
	fonttitle=\sffamily\bfseries, 
	title=~\thetcbcounter: #2, 
	#1
}

% disponi definizioni
\newtcolorbox[auto counter, number within=section]{definition}[2][]{%
	colback=red!10,
	colframe=red!40!black,
	sharp corners=northwest,
	fonttitle=\sffamily\bfseries,
	title=~\thetcbcounter: #2,
	#1
}

% U.D.M
\newcommand{\amp}{\ensuremath{\mathrm{A}}}
\newcommand{\volt}{\ensuremath{\mathrm{V}}}
\newcommand{\meter}{\ensuremath{\mathrm{m}}}
\newcommand{\second}{\ensuremath{\mathrm{s}}}
\newcommand{\farad}{\ensuremath{\mathrm{F}}}
\newcommand{\henry}{\ensuremath{\mathrm{H}}}
\newcommand{\siemens}{\ensuremath{\mathrm{S}}}

% circuiti
\usepackage{circuitikz}
\usetikzlibrary{babel}

% disegni
\usepackage{pgfplots}
\pgfplotsset{width=10cm,compat=1.9}

% disponi codice
\usepackage{listings}
\usepackage[table]{xcolor}

\lstdefinestyle{codestyle}{
		backgroundcolor=\color{black!5}, 
		commentstyle=\color{codegreen},
		keywordstyle=\bfseries\color{magenta},
		numberstyle=\sffamily\tiny\color{black!60},
		stringstyle=\color{green!50!black},
		basicstyle=\ttfamily\footnotesize,
		breakatwhitespace=false,         
		breaklines=true,                 
		captionpos=b,                    
		keepspaces=true,                 
		numbers=left,                    
		numbersep=5pt,                  
		showspaces=false,                
		showstringspaces=false,
		showtabs=false,                  
		tabsize=2
}

\lstdefinestyle{shellstyle}{
		backgroundcolor=\color{black!5}, 
		basicstyle=\ttfamily\footnotesize\color{black}, 
		commentstyle=\color{black}, 
		keywordstyle=\color{black},
		numberstyle=\color{black!5},
		stringstyle=\color{black}, 
		showspaces=false,
		showstringspaces=false, 
		showtabs=false, 
		tabsize=2, 
		numbers=none, 
		breaklines=true
}

\lstdefinelanguage{javascript}{
	keywords={typeof, new, true, false, catch, function, return, null, catch, switch, var, if, in, while, do, else, case, break},
	keywordstyle=\color{blue}\bfseries,
	ndkeywords={class, export, boolean, throw, implements, import, this},
	ndkeywordstyle=\color{darkgray}\bfseries,
	identifierstyle=\color{black},
	sensitive=false,
	comment=[l]{//},
	morecomment=[s]{/*}{*/},
	commentstyle=\color{purple}\ttfamily,
	stringstyle=\color{red}\ttfamily,
	morestring=[b]',
	morestring=[b]"
}

% disponi sezioni
\usepackage{titlesec}

\titleformat{\section}
	{\sffamily\Large\bfseries} 
	{\thesection}{1em}{} 
\titleformat{\subsection}
	{\sffamily\large\bfseries}   
	{\thesubsection}{1em}{} 
\titleformat{\subsubsection}
	{\sffamily\normalsize\bfseries} 
	{\thesubsubsection}{1em}{}

% disponi alberi
\usepackage{forest}

\forestset{
	rectstyle/.style={
		for tree={rectangle,draw,font=\large\sffamily}
	},
	roundstyle/.style={
		for tree={circle,draw,font=\large}
	}
}

% disponi algoritmi
\usepackage{algorithm}
\usepackage{algorithmic}
\makeatletter
\renewcommand{\ALG@name}{Algoritmo}
\makeatother

% disponi numeri di pagina
\usepackage{fancyhdr}
\fancyhf{} 
\fancyfoot[L]{\sffamily{\thepage}}

\makeatletter
\fancyhead[L]{\raisebox{1ex}[0pt][0pt]{\sffamily{\@title \ \@date}}} 
\fancyhead[R]{\raisebox{1ex}[0pt][0pt]{\sffamily{\@author}}}
\makeatother

\begin{document}
% sezione (data)
\section{Lezione del 24-10-24}

% stili pagina
\thispagestyle{empty}
\pagestyle{fancy}

% testo
\subsubsection{Mutua induttanza nel dominio fasoriale}
Avevamo le formule per la mutua induttanza:
\[
	\begin{cases}
		v_1(t) = L_1 \frac{d i_1(t)}{dt} \pm M \frac{d i_2(t)}{dt} \\
		v_2(t) = L_2 \frac{d i_2(t)}{dt} \pm M \frac{d i_1(t)}{dt}	
	\end{cases}
\]
nel dominio tempo.
Portandoci nel dominio fasoriale, abbiamo:
\[
	\begin{cases}
		\dot{V}_1 = j\omega L_1 \dot{I}_1	\pm j\omega M \dot{I}_2 \\
		\dot{V}_2 = j\omega L_2 \dot{I}_2	\pm j\omega M \dot{I}_1
	\end{cases}
\]

Vedremo poi metodi specifici di risoluzione applicabili su circuiti in corrente alternata con mutue induttanze.

\subsection{Circuito RLC}
Poniamo di avere un circuito con un resistore, un induttore e un capacitore.
\begin{center}
	\begin{circuitikz}
		\draw (0,0) to [ resistor, l=$R$] (2,0) to [ inductor, l=$L$] (4,0) to [ capacitor, l=$C$ ] (6,0);
	\end{circuitikz}
\end{center}

Avremo le cadute di potenziale $\dot{V}_R$, $\dot{V}_L$ e $\dot{V}_C$ sui singoli componenti, da cui:
$$
\dot{V} = \dot{V}_R + \dot{V}_L + \dot{V}_C = R \dot{I} + j\omega L \dot{I} + \frac{1}{j \omega c} \dot{I} = \left( R + j\omega l+ \frac{1}{j\omega C} \right) \dot{I}
$$
Portando in forma cartesiana, si ha:
$$
= (R + j\omega L - \frac{j}{\omega C}) \dot{I} = \left( R + j \left( \omega L - \frac{1}{\omega C} \right) \right) \dot{I}
$$
che possiamo riscrivere come:
$$
\dot{V} = \overline{Z} \dot{I}, \quad Z = R + j \left( \omega L - \frac{1}{\omega C} \right)
$$

\subsection{Impedenza}
Il numero $Z$, un complesso, è chiamato \textbf{impedenza} del circuito.
Abbiamo che l'equazione $\dot{V} = \overline{Z} \dot{I}$ è un'equazione complessa, ergo che potremmo riscrivere in forma esponenziale come:
$$
\Rightarrow V_M \cdot e^{j \phi_v} = Z \cdot e^{j \phi_z} I_M e^{j \phi_i}
$$
e quindi ridurre al sistema:
\[
	\begin{cases}
		V_M = Z I_M \\ 
		\phi_v = \phi_z + \phi_i
	\end{cases}
\]

Spesso si indica $\phi_z$ come semplicemente $\phi$, cioè la \textbf{fase dell'impedenza}, definita quindi come:
$$
\phi = \phi_v - \phi_i
$$

Possiamo scrivere l'impedenza come:
$$
\bar{Z} = Z \cdot e^{j \phi} = R + j X, \quad z = |\bar{z}|
$$
dove $R$ corrisponde alla \textbf{resistenza} che già conosciamo, mentre $X$ viene detta \textbf{reattanza}.

Vediamo le reattanze dei bipoli studiati finora:
\begin{itemize}
	\item \textbf{Resistenza:} da $\dot{V} = R \dot{I}$ abbiamo $X_R = 0$, cioè reattanza nulla (e chiaramente $R_R = R$);
	\item \textbf{Induttore:} da $\dot{V} = j \omega L \dot{I}$, ricaviamo:
		\[
			\begin{cases}
				R_L = 0	\\ 
				X_L = \omega L
			\end{cases}
		\]
	\item \textbf{Condensatore:} da $\dot{V} = \frac{1}{j \omega C}$, ricaviamo:
		\[
			\begin{cases}
				R_C = 0	\\ 
				X_C = -\frac{1}{\omega C}
			\end{cases}
		\]
		dove l'ultima $X_C$ si è ricavata da $-\frac{j}{\omega C}$, già usato sopra nella forma cartesiana dell'RLC in serie.
\end{itemize}

Avremo che, riguardo all impededenza, avremo solitamente i valori di fase:
$$
\bar{Z} = R \cdot j X = z e^{j \phi}:
	\begin{cases}
		\phi = \frac{\pi}{2} \\ 
		\phi = 0 \\ 
		\phi = -\frac{pi}{2}
	\end{cases}
$$

Classifichiamo questi valori.
\begin{itemize}
	\item $\phi = \frac{\pi}{2}$, si dice che l'impedenza è \textbf{induttiva};
	\item $0 < \phi < \frac{\pi}{2}$, si dice che l'impedenza è \textbf{ohmico-induttiva};
	\item $\phi = 0$, si dice che l'impedenza è \textbf{resistiva};
	\item $-\frac{\pi}{2} < \phi < 0$, si dice che l'impedenza è \textbf{ohmico-capacitiva}; 
	\item $\phi = -\frac{\pi}{2}$, si dice che l'impedenza è \textbf{capacitiva};
\end{itemize}

\subsubsection{Ammettenza}
Possiamo definire l'opposto dell'impedenza:
$$
\bar{Y} = \frac{1}{\hat{Z}}
$$

Chiamiamo $Y$ \textbf{ammettenza}.
Possiamo dividere anche l'ammettenza in componenti cartesiane, cioè:
$$
\bar{Y} = G + j B
$$
dove $G$ è la \textbf{conduttanza}, e $B$ viene detta \textbf{suscettanza}.

Possiamo eliminare il complesso al denominatore come:
$$
\bar{Y} = \frac{1}{\hat{Z}} = \frac{1}{R + jX} = \frac{R - jX}{R^2 + X^2} = \frac{R}{R^2 + X^2} + j \left(-\frac{X}{R^2 + X^2}\right)
$$
da cui:
\[
	\begin{cases}
		G = \frac{R}{R^2 + X^2} \\ 	
		B = -\frac{X}{R^2 + X^2} \\ 	
	\end{cases}
\]

\subsubsection{Unità di misura}
Abbiamo che, essendo quantità omegenee (le abbiamo sommate fra di loro senza problemi), l'impedenza $Z$, la reattanza $X$ e la resistenza $R$ si misurano in Ohm, mentre l'ammettenza $Y$, la suscettanza $B$ e la conduttanza $G$ si misurano in Siemens.

\subsubsection{Rappresentazione grafica dell'impedenza}
Abbiamo che il vettore di $Z$ rappresentato sul piano di Argand-Gauss rappresenta in componenti $R$ e $X$ (com'è ovvio), e che l'angolo che forma con l'asse delle x rappresenta $\phi$.

Inoltre, si  ha che l'ammettenza corrispondente è un vettore con modulo $\frac{1}{Z}$ e angolo $-\phi$, da:
$$
\bar{Y} = \frac{1}{\bar{Z}} = \frac{1}{Ze^{j\phi}} = \frac{1}{Z} e^{-j\phi}
$$

\begin{center}
\begin{tikzpicture}
  \begin{axis}[
    axis lines = center,
    xlabel={$R$},
    ylabel={$X$},
    xmin=-5.4, xmax=5.4,
    ymin=-5.4, ymax=5.4,
    grid=minor,
  ]
    % Draw the phasor (example: magnitude=1, angle=45 degrees)
    \addplot[thick,->,black] coordinates {(0,0) (0.707*5,0.707*5)};
    
    % Optional: Add a label for the phasor
		\node at (axis cs: 0.707*5, 0.707*5) [anchor=west] {$Z$};

		\node at(axis cs: 0.5,0.3) [anchor=west] {$\phi$};
  \end{axis}
\end{tikzpicture}
\end{center}

Notiamo il significato dei quadranti: 
\begin{itemize}
	\item \textbf{Primo quadrante:} $R > 0$, $X > 0$ abbiamo che sia la resistenza che la reattanza sono positive, quindi si parla di impedenza ohmico-induttiva.
	\item \textbf{Secondo quadrante:} $R < 0$, $X > 0$ abbiamo che la resistenza è negativa e la reattanza è positiva, quindi si parla di impedenza ohmico-induttiva negativa.
	\item \textbf{Terzo quadrante:} $R < 0$, $X < 0$ abbiamo che sia la resistenza che la reattanza sono negative, quindi si parla di impedenza ohmico-capacitiva negativa.
	\item \textbf{Quarto quadrante:} $R < 0$, $X > 0$ abbiamo che la resistenza è positiva e la reattanza è negativa, quindi si parla di impedenza ohmico-capacitiva.
\end{itemize}

In parallelo a quanto detto prima sull'angolo $\phi$.
Ai vettori paralleli agli assi corrispondono, come ci si aspetterebbe, abbiamo l'impedenza puramente induttiva ($R = 0$, $X > 0$), puramente capacitiva ($R = 0$, $X < 0$) e puramente resistiva positiva ($R > 0$, $X = 0$) e negativa ($R < 0$, $X = 0$).

\end{document}
