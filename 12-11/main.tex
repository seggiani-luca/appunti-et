
\documentclass[a4paper,11pt]{article}
\usepackage[a4paper, margin=8em]{geometry}

% usa i pacchetti per la scrittura in italiano
\usepackage[french,italian]{babel}
\usepackage[T1]{fontenc}
\usepackage[utf8]{inputenc}
\frenchspacing 

% usa i pacchetti per la formattazione matematica
\usepackage{amsmath, amssymb, amsthm, amsfonts}

% usa altri pacchetti
\usepackage{gensymb}
\usepackage{hyperref}
\usepackage{standalone}

% imposta il titolo
\title{Appunti Elettrotecnica}
\author{Luca Seggiani}
\date{2024}

% imposta lo stile
% usa helvetica
\usepackage[scaled]{helvet}
% usa palatino
\usepackage{palatino}
% usa un font monospazio guardabile
\usepackage{lmodern}

\renewcommand{\rmdefault}{ppl}
\renewcommand{\sfdefault}{phv}
\renewcommand{\ttdefault}{lmtt}

% disponi il titolo
\makeatletter
\renewcommand{\maketitle} {
	\begin{center} 
		\begin{minipage}[t]{.8\textwidth}
			\textsf{\huge\bfseries \@title} 
		\end{minipage}%
		\begin{minipage}[t]{.2\textwidth}
			\raggedleft \vspace{-1.65em}
			\textsf{\small \@author} \vfill
			\textsf{\small \@date}
		\end{minipage}
		\par
	\end{center}

	\thispagestyle{empty}
	\pagestyle{fancy}
}
\makeatother

% disponi teoremi
\usepackage{tcolorbox}
\newtcolorbox[auto counter, number within=section]{theorem}[2][]{%
	colback=blue!10, 
	colframe=blue!40!black, 
	sharp corners=northwest,
	fonttitle=\sffamily\bfseries, 
	title=~\thetcbcounter: #2, 
	#1
}

% disponi definizioni
\newtcolorbox[auto counter, number within=section]{definition}[2][]{%
	colback=red!10,
	colframe=red!40!black,
	sharp corners=northwest,
	fonttitle=\sffamily\bfseries,
	title=~\thetcbcounter: #2,
	#1
}

% U.D.M
\newcommand{\amp}{\ensuremath{\mathrm{A}}}
\newcommand{\volt}{\ensuremath{\mathrm{V}}}
\newcommand{\meter}{\ensuremath{\mathrm{m}}}
\newcommand{\second}{\ensuremath{\mathrm{s}}}
\newcommand{\farad}{\ensuremath{\mathrm{F}}}
\newcommand{\henry}{\ensuremath{\mathrm{H}}}
\newcommand{\siemens}{\ensuremath{\mathrm{S}}}

% circuiti
\usepackage{circuitikz}
\usetikzlibrary{babel}

% disegni
\usepackage{pgfplots}
\pgfplotsset{width=10cm,compat=1.9}

% disponi codice
\usepackage{listings}
\usepackage[table]{xcolor}

\lstdefinestyle{codestyle}{
		backgroundcolor=\color{black!5}, 
		commentstyle=\color{codegreen},
		keywordstyle=\bfseries\color{magenta},
		numberstyle=\sffamily\tiny\color{black!60},
		stringstyle=\color{green!50!black},
		basicstyle=\ttfamily\footnotesize,
		breakatwhitespace=false,         
		breaklines=true,                 
		captionpos=b,                    
		keepspaces=true,                 
		numbers=left,                    
		numbersep=5pt,                  
		showspaces=false,                
		showstringspaces=false,
		showtabs=false,                  
		tabsize=2
}

\lstdefinestyle{shellstyle}{
		backgroundcolor=\color{black!5}, 
		basicstyle=\ttfamily\footnotesize\color{black}, 
		commentstyle=\color{black}, 
		keywordstyle=\color{black},
		numberstyle=\color{black!5},
		stringstyle=\color{black}, 
		showspaces=false,
		showstringspaces=false, 
		showtabs=false, 
		tabsize=2, 
		numbers=none, 
		breaklines=true
}

\lstdefinelanguage{javascript}{
	keywords={typeof, new, true, false, catch, function, return, null, catch, switch, var, if, in, while, do, else, case, break},
	keywordstyle=\color{blue}\bfseries,
	ndkeywords={class, export, boolean, throw, implements, import, this},
	ndkeywordstyle=\color{darkgray}\bfseries,
	identifierstyle=\color{black},
	sensitive=false,
	comment=[l]{//},
	morecomment=[s]{/*}{*/},
	commentstyle=\color{purple}\ttfamily,
	stringstyle=\color{red}\ttfamily,
	morestring=[b]',
	morestring=[b]"
}

% disponi sezioni
\usepackage{titlesec}

\titleformat{\section}
	{\sffamily\Large\bfseries} 
	{\thesection}{1em}{} 
\titleformat{\subsection}
	{\sffamily\large\bfseries}   
	{\thesubsection}{1em}{} 
\titleformat{\subsubsection}
	{\sffamily\normalsize\bfseries} 
	{\thesubsubsection}{1em}{}

% disponi alberi
\usepackage{forest}

\forestset{
	rectstyle/.style={
		for tree={rectangle,draw,font=\large\sffamily}
	},
	roundstyle/.style={
		for tree={circle,draw,font=\large}
	}
}

% disponi algoritmi
\usepackage{algorithm}
\usepackage{algorithmic}
\makeatletter
\renewcommand{\ALG@name}{Algoritmo}
\makeatother

% disponi numeri di pagina
\usepackage{fancyhdr}
\fancyhf{} 
\fancyfoot[L]{\sffamily{\thepage}}

\makeatletter
\fancyhead[L]{\raisebox{1ex}[0pt][0pt]{\sffamily{\@title \ \@date}}} 
\fancyhead[R]{\raisebox{1ex}[0pt][0pt]{\sffamily{\@author}}}
\makeatother

\begin{document}
% sezione (data)
\section{Lezione del 11-12-24}

% stili pagina
\thispagestyle{empty}
\pagestyle{fancy}

% testo
\subsubsection{Trasformatore reale con ciclo di isteresi}
Nel caso dei trasformatori reali, non ammessa la terza ipotesi considerata prima, quindi $\mu_{fe}$ non costante nel tempo e il materiale caratterizzato da un ciclo di isteresi, si noterà un effetto non modellizabile matematicamente ma comunque sperimentabile nella realtà.

Dal trasformatore, considerato \textbf{a vuoto} (quindi senza carico sul secondario):

\begin{center}
	\begin{circuitikz}
		\draw (0,0) 
			-- (3,0)
			-- (3,-3)
			-- (0,-3)
			-- (0,0);
		\draw (0.5,-0.5) 
			-- (2.5,-0.5)
			-- (2.5,-2.5)
			-- (0.5,-2.5)
			-- (0.5,-0.5);
 
		\ctikzset{bipoles/thickness=1}
		\ctikzset{bipoles/cuteinductor/width=1.13}
		\ctikzset{bipoles/cuteinductor/height=0.9}
		
		\ctikzset{bipoles/cuteinductor/coils = 6}
		
		\draw (-1, -0.7)
			to [short, i=$i$] (0, -0.7) 
			to [inductor] (0, -2.3)
			-- (-1, -2.3);

		\draw (-1,-0.7) node[anchor=east] {$+$};
		\draw (-1,-2.3) node[anchor=east] {$-$};

		\draw (-1,-1.5) node[anchor=east] {$E_1$};
		\draw (-0.2,-1.5) node[anchor=east] {$N_1$};
	\end{circuitikz}
\end{center}

ricaveremo il circuito magnetico:

\begin{center}
	\begin{circuitikz}
		\draw (0,0) to [resistor, R=$\mathcal{R}_2$] (3, 0)
		to [resistor, R=$\mathcal{R}_3$] (3, -3)
		to [resistor, R=$\mathcal{R}_4$] (0, -3)
		to [voltage source, v=$Ni$] (0, -1.5)
		to [resistor, R=$\mathcal{R}_1$] (0, 0);
	\end{circuitikz}
\end{center}

Potremo ricavare, dalla legge di Ohm:
$$
N_1 I_0 = \mathcal{R}(\Phi_0) \Phi_0
$$
notando che la riluttanza era espressa come $\mathcal{R} = \frac{l}{\mu \mu_r S}$, e che $\mu_r$ è funzione della magnetizzazione $B(H)$ e quindi del flusso $\Phi_0$:
$$
\mu_r(\Phi_0) \Rightarrow \mathcal{R}(\Phi_0) = \frac{l}{\mu \mu_r(\Phi_0) S}
$$

Considerando lo stesso trasformatore, ma stavolta con un impedenza di carico $\overline{Z}_C$:

\begin{center}
	\begin{circuitikz}
		\draw (0,0) 
			-- (3,0)
			-- (3,-3)
			-- (0,-3)
			-- (0,0);
		\draw (0.5,-0.5) 
			-- (2.5,-0.5)
			-- (2.5,-2.5)
			-- (0.5,-2.5)
			-- (0.5,-0.5);
 
		\ctikzset{bipoles/thickness=1}
		\ctikzset{bipoles/cuteinductor/width=1.13}
		\ctikzset{bipoles/cuteinductor/height=0.9}
		
		\ctikzset{bipoles/cuteinductor/coils = 6}
		
		\draw (-1, -0.7)
			to [short, i=$i_1$] (0, -0.7) 
			to [inductor] (0, -2.3)
			-- (-1, -2.3);

		\ctikzset{bipoles/cuteinductor/coils = 4}

		\draw (5, -2.3)
			-- (3, -2.3) 
			to [inductor] (3, -0.7)
			to [short, i<=$i_2$] (5, -0.7)
			to [ european resistor, l=$Z_C$] (5, -2.3);

		\draw (-1,-0.7) node[anchor=east] {$+$};
		\draw (-1,-2.3) node[anchor=east] {$-$};
		\draw (-1,-1.5) node[anchor=east] {$E_1$};
		\draw (-0.2,-1.5) node[anchor=east] {$N_1$};
		\draw (3.2,-1.5) node[anchor=west] {$N_2$};

	\end{circuitikz}
\end{center}
avremo il circuito magentico:
\begin{center}
	\begin{circuitikz}
		\draw (0,0) to [resistor, R=$\mathcal{R}_2$] (3, 0)
		to [voltage source, v=$N_2 i_2$] (3, -1.5)
		to [resistor, R=$\mathcal{R}_3$] (3, -3)
		to [resistor, R=$\mathcal{R}_4$] (0, -3)
		to [voltage source, v=$N_1 i_1$] (0, -1.5)
		to [resistor, R=$\mathcal{R}_1$] (0, 0);
	\end{circuitikz}
\end{center}
da cui ricaveremo invece:
$$
N_1 I_1 + N_2 I_2 = \mathcal{R}(\Phi) \Phi
$$

Quello che si misura sperimentalmente è che $\Phi_0$ a vuoto è $\approx \Phi$ preso un carico, cioè che in generale il flusso $\Phi$ \textit{non dipende dal carico}.
Sarà quindi:
$$
N_1 I_0 = N_1 I_1 + N_2 I_2, \quad I_1 = \frac{N_2}{N_1} I_2 + I_0 = -\frac{1}{n} I_2 + I_0
$$
dove ricordiamo sempre $I_0$ è la corrente sul primo avvolgimento \textit{a vuoto}.

Possiamo quindi migliorare il nostro modello di trasformatore reale introducendo un'altra maglia, dove scorrerà la corrente $I_{10}$, e su cui si troveranno una resistenza $R_m$ e un'induttanza $L_m$ in parallelo, dette \textbf{resistenza di magnetizzazione} e \textbf{impedenza di magnetizzazione} e che dipendono dal ciclo di isteresi del materiale, e dalle \textit{correnti parassite} che vi si formano a regime periodico:

\begin{center}
	\begin{circuitikz}
		\draw (-6,0) -- (-3.5, 0)
		to[ short, i=$i_1$] (-3,0)
			to [ resistor, l=$R_{1d}$] (-1,0)
			to [ inductor, l=$L_{1d}$] (1,0)
			to [ inductor , l=$N_1$] (1,-2)
			-- (-6, -2);
		
		\draw (3, 0) to [ resistor, l=$R_{2d}$] (5,0)
			to [ inductor, l=$L_{2d}$] (7,0) 
			to[ short, i<=$i_2$] (8,0);
			
			\draw (3,0) to[ inductor , l_=$N_2$] (3,-2)
			-- (8, -2);

		\draw (-5.5, 0) to [ resistor, l=$R_m$] (-5.5, -2);
		\draw (-4, 0) to [ inductor, l=$L_m$] (-4, -2);

			\draw (0.9,-0.6) node {$\scriptscriptstyle\bullet$};
			\draw (2.9,-0.6) node {$\scriptscriptstyle\bullet$};


			\draw (-6,-0.5) node[anchor=east] {$+$};
			\draw (-6,-1.5) node[anchor=east] {$-$};

			\draw (8,-0.5) node[anchor=west] {$+$};
			\draw (8,-1.5) node[anchor=west] {$-$};

			\draw (-6,-1) node[anchor=east] {$V_1$};
			\draw (8,-1) node[anchor=west] {$V_2$};

			\draw (0.7,-0.5) node[anchor=east] {$+$};
			\draw (0.7,-1.5) node[anchor=east] {$-$};

			\draw (3.3,-0.5) node[anchor=west] {$+$};
			\draw (3.3,-1.5) node[anchor=west] {$-$};

			\draw (0.7,-1) node[anchor=east] {$E_1$};
			\draw (3.3,-1) node[anchor=west] {$E_2$};

	\end{circuitikz}
\end{center}

L'impedenza di magnetizzazione, in particolare, serve a modellizzare il fatto che le correnti sugli avvolgimenti spesso non sono in fase fra di loro, sempre per esperienza sperimentale derivante dal ciclo di isteresi e dalle correnti parassite.

\subsubsection{Calcolo dei parametri di un trasformatore reale}
Si ha, visto il fatto che i parametri che modellizzano un trasformatore sono effettivamente empirici, che il loro calcolo si fa per prove.
Nello specifico si fa:
\begin{enumerate}
	\item Una \textbf{prova a vuoto}, quindi prendendo il ramo a destra (il secondario) come aperto:

\begin{center}
	\begin{circuitikz}
		\draw (-6,0) -- (-3.5, 0)
		to[ short, i=$i_1$] (-3,0)
			to [ resistor, l=$R_{1d}$] (-1,0)
			to [ inductor, l=$L_{1d}$] (1,0)
			to [ inductor , l=$N_1$] (1,-2)
			-- (-6, -2);
		
		\draw (3, 0) to [ resistor, l=$R_{2d}$] (5,0)
			to [ inductor, l=$L_{2d}$] (7,0); 
			
			\draw (3,0) to[ inductor , l_=$N_2$] (3,-2)
			-- (7, -2);

		\draw (-5.5, 0) to [ resistor, l=$R_m$] (-5.5, -2);
		\draw (-4, 0) to [ inductor, l=$L_m$] (-4, -2);

			\draw (0.9,-0.6) node {$\scriptstyle\bullet$};
			\draw (2.9,-0.6) node {$\scriptstyle\bullet$};

			\draw (7,) node {$\times$};
			\draw (7,-2) node {$\times$};

			\draw (-6,-0.5) node[anchor=east] {$+$};
			\draw (-6,-1.5) node[anchor=east] {$-$};

			\draw (-6,-1) node[anchor=east] {$V_1$};

			\draw (0.7,-0.5) node[anchor=east] {$+$};
			\draw (0.7,-1.5) node[anchor=east] {$-$};

			\draw (3.3,-0.5) node[anchor=west] {$+$};
			\draw (3.3,-1.5) node[anchor=west] {$-$};

			\draw (0.7,-1) node[anchor=east] {$E_1$};
			\draw (3.3,-1) node[anchor=west] {$E_2$};

	\end{circuitikz}
\end{center}

	In questo caso è immediato che la corrente sugli induttori del mutuo accoppiamento è nulla, e assumendo che la caduta di potenziale su $R_{1d}$ e $L_{1d}$ è trascurabile avremo che l'unica corrente rimasta è quella sul ramo di magnetizzazione, cioè quello che contiene $R_m$ e $L_m$.
		Avremo quindi la possibilità di misurare $V_0$, $I_0$ e $P_0$, cioè tensione, corrente e potenza dissipata.

	Da queste si ricaverà:
	$$
	P_0 = R_m I_r^2 = R_m \left( \frac{V_0}{R_m} \right)^2 = \frac{V_0^2}{R_m} = G_m V_0^2
	$$
	dove $G_m$ è la \textbf{conduttanza di magnetizzazione} $\frac{1}{R_m}$.
	Si ha quindi:
	$$
	G_m = \frac{P_0}{V_0^2}
	$$
	Per quanto riguarda l'impedenza, invece, prendiamo l'\textbf{ammettenza di magnetizzazione}:
	$$
	Y_m = \frac{I_0}{V_0} 
	$$
	Notiamo che non ci è possibile calcolare le grandezze così come sono, cioè in fasori, ma solo come moduli, cioè non si può calcolare:
	$$
	\overline{Y}_m = \frac{\dot{I}_0}{\dot{V}_0} = G_m + j B_m
	$$
	con $G_m$ la conduttanza di prima, e $B_m$ \textbf{suscettanza di magnetizzazione}, ma solo:
	$$
	|\overline{Y}_m| = Y_m
	$$
	notiamo però di aver già trovato la conduttanza $G_m$, e quindi:
	$$
	B_m = \sqrt{Y_m^2 - G_m^2}
	$$
	$$
	\overline{Y}_m = \frac{1}{R_m} + \frac{1}{j \omega L_m} = \frac{1}{R_m} - \frac{j}{\omega L_m}
	$$
	$$
	\overline{Y}_m = G + j B \implies G = \frac{1}{R_m}, \quad B_m = - \frac{1}{\omega L_m}
	$$

\item Una \textbf{prova in cortocircuito}, qunidi prendendo il secondario come chiuso.
	In questo caso possiamo trascurare il ramo di magnetizzazione, in quanto abbiamo già calcolato resistenza e impedenza di magnetizzazione.
	
\begin{center}
	\begin{circuitikz}
		\draw (-6,0) -- (-3.5, 0)
		to[ short, i=$i_1$] (-3,0)
			to [ resistor, l=$R_{1d}$] (-1,0)
			to [ inductor, l=$L_{1d}$] (1,0)
			to [ inductor , l=$N_1$] (1,-2)
			-- (-6, -2);
		
		\draw (3, 0) to [ resistor, l=$R_{2d}$] (5,0)
			to [ inductor, l=$L_{2d}$] (7,0) 
			to [ short] (7,0)
			to [ short] (7,-2);
			
			\draw (3,0) to[ inductor , l_=$N_2$] (3,-2)
			-- (7, -2);

		\draw (-5.5, 0) to [ resistor, l=$R_m$] (-5.5, -2);
		\draw (-4, 0) to [ inductor, l=$L_m$] (-4, -2);

			\draw (0.9,-0.6) node {$\scriptscriptstyle\bullet$};
			\draw (2.9,-0.6) node {$\scriptscriptstyle\bullet$};


			\draw (-6,-0.5) node[anchor=east] {$+$};
			\draw (-6,-1.5) node[anchor=east] {$-$};

			\draw (-6,-1) node[anchor=east] {$V_1$};

			\draw (0.7,-0.5) node[anchor=east] {$+$};
			\draw (0.7,-1.5) node[anchor=east] {$-$};

			\draw (3.3,-0.5) node[anchor=west] {$+$};
			\draw (3.3,-1.5) node[anchor=west] {$-$};

			\draw (0.7,-1) node[anchor=east] {$E_1$};
			\draw (3.3,-1) node[anchor=west] {$E_2$};

	\end{circuitikz}
\end{center}

	A questo punto riportiamo, con le formule dell'adattamento di impedenza, l'impedenza al secondario sul ramo del primario, quindi introduciamo la resistenza $n^2 R_{2d}$ e l'induttanza $n^2 X_{1d}$:

\begin{center}
	\begin{circuitikz}
		\draw (-6,0) -- (-3.5, 0)
		to[ short, i=$i_1$] (-3,0)
			to [ resistor, l=$R_{1d}$] (-1,0)
			to [ inductor, l=$L_{1d}$] (1,0)
			to [ resistor, l=$n^2 R_{2d}$] (3, 0)
			to [ inductor, l=$n^2 L_{2d}$] (5, 0)
			to [ inductor , l_=$N_1$] (5,-2)
			-- (-6, -2);
		
			
		\draw (7, 0) -- (6,0) 
			to[ inductor , l_=$N_2$] (6,-2)
			-- (7, -2)
			-- (7, 0);

		\draw (-5.5, 0) to [ resistor, l=$R_m$] (-5.5, -2);
		\draw (-4, 0) to [ inductor, l=$L_m$] (-4, -2);

			\draw (4.9,-0.6) node {$\scriptscriptstyle\bullet$};
			\draw (5.9,-0.6) node {$\scriptscriptstyle\bullet$};

			\draw (-6,-0.5) node[anchor=east] {$+$};
			\draw (-6,-1.5) node[anchor=east] {$-$};

			\draw (-6,-1) node[anchor=east] {$V_1$};

			\draw (4.3,-0.5) node[anchor=east] {$+$};
			\draw (4.3,-1.5) node[anchor=east] {$-$};

			\draw (6.3,-0.5) node[anchor=west] {$+$};
			\draw (6.3,-1.5) node[anchor=west] {$-$};

			\draw (4.3,-1) node[anchor=east] {$E_1$};
			\draw (6.3,-1) node[anchor=west] {$E_2$};

	\end{circuitikz}
\end{center}

	Troveremo quindi la resistenza e l'induttanza complessive $R_{cc}$ e $X_{cc}$, semplicemente come le somme delle resistenze delle induttanze sul ramo. Attraverso le stesse misure di prima, potremo poi dire:
	$$
	P_{cc} = R_{cc} I_{cc}^2 \implies R_{cc} = \frac{P_{cc}}{I^2_{cc}}
	$$
	$$
	\frac{V_{cc}}{I_{cc}} = Z_{cc}
	$$
	$$
	X_{cc} = \sqrt{Z_{cc} ^ 2 - R_{cc}^2}
	$$

	notiamo che non possiamo mai separare l'impedenza sul primario e sul secondario, ma questo non ci interessa se vogliamo calcolare la potenza complessivamente dissipata sul trasformatore.
	Allo stesso modo, non potremo calcolare i flussi magnetici dispersi sul primario o sul secondario, ma solo i flussi dispersi complessivamente. 
\end{enumerate}

\subsection{Macchine elettriche} 
Vediamo quindi le \textbf{macchine elettriche} vere e proprie, che in questo caso intendiamo come tali perchè comportano effettivamente movimento meccanico.
Le macchine elettriche comportano trasferimento di \textbf{potenza}, fra \textbf{elettrica} e \textbf{meccanica}, detto processo di \textbf{conversione elettromeccanica}.
Una macchina che trasforma potenza elettrica in potenza meccanica si dice \textbf{motore}, mentre una macchina che trasforma potenza meccanica in potenza elettrica si dice \textbf{generatore}.

\subsubsection{Trasduttore a barretta mobile}
Consideriamo un circuito formato da una resistenza $R$ e un generatore $v$, con un lato aperto su cui si trova una barretta mobile di lunghezza $l$ e massa $m$.
Il circuito si trova inoltre in un campo magnetico entrante nella pagina.

La corrente nel circuito sarà:
$$
i = \frac{dq}{dt}
$$
che possiamo moltiplicare da entrambi i lati per $dl$ distanza infinitesima sulla barretta:
$$
dl \, i = \frac{dq}{dt} dl = dq \cdot u \Rightarrow dF = dq \, u \cdot B = dl \, i \cdot  B
$$
con $u$ velocità della singola carica sulla bobina mobile. Integrando a destra e a sinistra si ha:
$$
F = B \int i \, dl = Bli  
$$
Chiamiamo quest'ultimo risultato \textbf{legge} $\boldsymbol{Bli}$: $F = Bli$.

Per ricavare la differenza di potenziale $e$ ai capi della bobina possiamo sfruttare la legge di Faraday, che ci fornisce la forza elettromotrice (o più propriamente \textit{controelettromotrice}) $\varepsilon$:
$$
\varepsilon = -\frac{d \Phi(\vec{B})}{dt}
$$
Il flusso è dato dall'integrale di superficie fra il campo e l'area delimitata dalla spira che la barretta mobile va a formare col resto del circuito,
Si avrà che questa vale $S = lx$ dove $x$ è la posizione orizzontale della barretta rispetto al resto del circuito.
Sarà quindi:
$$
\Phi(\vec{B}) = \int_S \vec{B} dS = BS \Rightarrow \varepsilon = \frac{d BL x}{dt} = - Bl \cdot \frac{d}{dt}x = -Bl u
$$
che corrisponde, considerata la direzione del campo, a una differenza di potenziale $e = -\varepsilon$.
Chiamiamo questo risultato \textbf{legge} $\boldsymbol{Blu}$: $e = Bl u$.

\subsubsection{Considerazioni di potenza}
Possiamo notare come la potenza elettrica dissipata equivale alla potenza meccanica.
Prendiamo il caso $V = 0$:
\begin{itemize}
	\item \textbf{\textsf{Potenza elettrica}} \\ 
		Si applica la formula della potenza dissipata su un resistore:
		$$
		P_{el} = i^2 R = \frac{V^2}{R} = \frac{(Blu)^2}{R} = \frac{B^2 l^2 u^2}{R}
		$$
	\item \textbf{\textsf{Potenza meccanica}} \\ 
		Si ha che la potenza meccanica vale $P = F u$, ergo:
		$$
		P_{me} = Fv = Bli \cdot u 
		$$
		con $i = \frac{Blu}{R}$ questo significa:
		$$
		P_{me} = Blu \cdot \frac{Blu}{R} = \frac{B^2 l^2 u^2}{R}
		$$
\end{itemize}

Quindi $P_{me} - P_{el} = 0$, cioè la potenza meccanica introdotta nel sistema è uguale alla potenza elettrica dissipata dal resistore per effetto Joule.
Reintroducendo il generatore $V$, si avrà che la potenza non è più solo quella dissipata dal resistore, bensì quella erogata dal generatore stesso. 

\subsubsection{Velocità della barretta}
Possiamo quindi prendere la spira formata come un unico circuito, su cui applicheremo la legge di Ohm:
$$
V - e = i R, \quad i = \frac{V - e}{R}
$$

Il circuito studiato si comporterà quindi, a seconda del segno di $i$, come:
\begin{itemize}
	\item $V > e \Leftrightarrow i > 0$: \textbf{motore};
	\item $V = e \Leftrightarrow i = 0$: caso ideale;
	\item $v < e \Leftrightarrow i < 0$: \textbf{generatore}.
\end{itemize}

Possiamo scrivere le equazioni che descrivono il sistema introducendo la \textbf{forza resistente} $F_r$.
Si avrà, dal secondo principio della dinamica:
$$
F - F_r = M\frac{du}{dt}
$$
mentre dall'applicazione già vista della legge di Ohm:
$$
V - e - Ri = 0
$$
da cui il sistema:
\[
	\begin{cases}
		F - F_r - m \frac{du}{dt} = 0 \quad \text{equazione meccanica} \\
		V - e - Ri = 0 \quad \quad \ \text{equazione elettrica} 
	\end{cases}
\]

Consideriamo il caso di equilibrio, cioè a velocità costante ($\frac{du}{dt} = 0$) in cui:
\begin{itemize}
	\item Non ci sono forze resistenti, quindi $F_r = 0$;
	\item La corrente nel circito è nulla, quindi $i = 0$.
\end{itemize}
In questo caso si avrà, dalla legge Blu $e = Blu$, e dall'equazione elettrica, posto $i=0$:
$$
V - e - Ri = V - Blu = 0 \Rightarrow V = Blu
$$

Prendiamo allora il caso reale, dove:
\begin{itemize}
	\item Esistono forze resistenti, quindi $F_r \neq 0$;
	\item La corrente nel circuito è $i = \frac{V - e}{R}$.
\end{itemize}
Se la situazione considerata è sempre in equilibrio, quindi a velocità costante, si avrà dall'equazione meccanica $F - F_r = 0 \Rightarrow F = F_r$, e dalla legge Bli:
$$
F_r = Bli = Bl \left( \frac{V - e}{R} \right)
$$

Notiamo che si può prendere:
\begin{itemize}
	\item $V = B l u_0$ da quanto dimostrato prima nel caso a forze resistenti nulle;
	\item $e = B l u$ dalla legge Blu.
\end{itemize}

Sarà allora:
$$
i = \frac{V - e}{R} = \frac{B l u_0 - B l u}{R}
$$
e quindi:
$$
\Rightarrow F_r = B l \frac{B l v_0 - Bl v}{R} = \frac{B^2l^2}{R}(u_0 - u)
$$
e:
$$
v_0 - v = \frac{R F_r}{B^2 l^2}, \quad v = v_0 - \frac{R F_r}{B^2 l^2}
$$

Questo significherà che la velocità $u$ raggiunta quando $F_r \neq 0$ sarà minore della velocità a vuoto $u_0$.

\par\smallskip

La macchina studiata è una macchina \textbf{lineare}: vedremo che nella pratica si parlerà di macchine \textbf{rotanti}, da cui potremo ricavare le seguenti corrispondenze fra grandezze:

\begin{table}[h!]
	\center \rowcolors{2}{white}{black!10}
	\begin{tabular} { c | c }
		\bfseries Macchine lineari & \bfseries Macchine rotanti \\ 
		\hline
		$BL$ & $\Phi$ \\ 
		$F$ & $C$ (coppia) \\ 
		$u$ & $\Omega$ (velocità angolare) \\ 
	\end{tabular}
\end{table}

\end{document}
