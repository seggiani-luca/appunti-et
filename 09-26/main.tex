
\documentclass[a4paper,11pt]{article}
\usepackage[a4paper, margin=8em]{geometry}

% usa i pacchetti per la scrittura in italiano
\usepackage[french,italian]{babel}
\usepackage[T1]{fontenc}
\usepackage[utf8]{inputenc}
\frenchspacing 

% usa i pacchetti per la formattazione matematica
\usepackage{amsmath, amssymb, amsthm, amsfonts}

% usa altri pacchetti
\usepackage{gensymb}
\usepackage{hyperref}
\usepackage{standalone}

% imposta il titolo
\title{Appunti Elettrotecnica}
\author{Luca Seggiani}
\date{2024}

% imposta lo stile
% usa helvetica
\usepackage[scaled]{helvet}
% usa palatino
\usepackage{palatino}
% usa un font monospazio guardabile
\usepackage{lmodern}

\renewcommand{\rmdefault}{ppl}
\renewcommand{\sfdefault}{phv}
\renewcommand{\ttdefault}{lmtt}

% disponi il titolo
\makeatletter
\renewcommand{\maketitle} {
	\begin{center} 
		\begin{minipage}[t]{.8\textwidth}
			\textsf{\huge\bfseries \@title} 
		\end{minipage}%
		\begin{minipage}[t]{.2\textwidth}
			\raggedleft \vspace{-1.65em}
			\textsf{\small \@author} \vfill
			\textsf{\small \@date}
		\end{minipage}
		\par
	\end{center}

	\thispagestyle{empty}
	\pagestyle{fancy}
}
\makeatother

% U.D.M
\newcommand{\amp}{\ensuremath{\mathrm{A}}}
\newcommand{\volt}{\ensuremath{\mathrm{V}}}
\newcommand{\meter}{\ensuremath{\mathrm{m}}}
\newcommand{\second}{\ensuremath{\mathrm{s}}}
\newcommand{\farad}{\ensuremath{\mathrm{F}}}
\newcommand{\henry}{\ensuremath{\mathrm{H}}}
\newcommand{\siemens}{\ensuremath{\mathrm{S}}}

% disponi teoremi
\usepackage{tcolorbox}
\newtcolorbox[auto counter, number within=section]{theorem}[2][]{%
	colback=blue!10, 
	colframe=blue!40!black, 
	sharp corners=northwest,
	fonttitle=\sffamily\bfseries, 
	title=Teorema~\thetcbcounter: #2, 
	#1
}

% disponi definizioni
\newtcolorbox[auto counter, number within=section]{definition}[2][]{%
	colback=red!10,
	colframe=red!40!black,
	sharp corners=northwest,
	fonttitle=\sffamily\bfseries,
	title=Definizione~\thetcbcounter: #2,
	#1
}

% disegni
\usepackage{pgfplots}
\pgfplotsset{width=10cm,compat=1.9}

% disponi codice
\usepackage{listings}
\usepackage[table]{xcolor}

\lstdefinestyle{codestyle}{
		backgroundcolor=\color{black!5}, 
		commentstyle=\color{codegreen},
		keywordstyle=\bfseries\color{magenta},
		numberstyle=\sffamily\tiny\color{black!60},
		stringstyle=\color{green!50!black},
		basicstyle=\ttfamily\footnotesize,
		breakatwhitespace=false,         
		breaklines=true,                 
		captionpos=b,                    
		keepspaces=true,                 
		numbers=left,                    
		numbersep=5pt,                  
		showspaces=false,                
		showstringspaces=false,
		showtabs=false,                  
		tabsize=2
}

\lstdefinestyle{shellstyle}{
		backgroundcolor=\color{black!5}, 
		basicstyle=\ttfamily\footnotesize\color{black}, 
		commentstyle=\color{black}, 
		keywordstyle=\color{black},
		numberstyle=\color{black!5},
		stringstyle=\color{black}, 
		showspaces=false,
		showstringspaces=false, 
		showtabs=false, 
		tabsize=2, 
		numbers=none, 
		breaklines=true
}

\lstdefinelanguage{javascript}{
	keywords={typeof, new, true, false, catch, function, return, null, catch, switch, var, if, in, while, do, else, case, break},
	keywordstyle=\color{blue}\bfseries,
	ndkeywords={class, export, boolean, throw, implements, import, this},
	ndkeywordstyle=\color{darkgray}\bfseries,
	identifierstyle=\color{black},
	sensitive=false,
	comment=[l]{//},
	morecomment=[s]{/*}{*/},
	commentstyle=\color{purple}\ttfamily,
	stringstyle=\color{red}\ttfamily,
	morestring=[b]',
	morestring=[b]"
}

% disponi sezioni
\usepackage{titlesec}

\titleformat{\section}
	{\sffamily\Large\bfseries} 
	{\thesection}{1em}{} 
\titleformat{\subsection}
	{\sffamily\large\bfseries}   
	{\thesubsection}{1em}{} 
\titleformat{\subsubsection}
	{\sffamily\normalsize\bfseries} 
	{\thesubsubsection}{1em}{}

% disponi alberi
\usepackage{forest}

\forestset{
	rectstyle/.style={
		for tree={rectangle,draw,font=\large\sffamily}
	},
	roundstyle/.style={
		for tree={circle,draw,font=\large}
	}
}

% disponi algoritmi
\usepackage{algorithm}
\usepackage{algorithmic}
\makeatletter
\renewcommand{\ALG@name}{Algoritmo}
\makeatother

% disponi numeri di pagina
\usepackage{fancyhdr}
\fancyhf{} 
\fancyfoot[L]{\sffamily{\thepage}}

\makeatletter
\fancyhead[L]{\raisebox{1ex}[0pt][0pt]{\sffamily{\@title \ \@date}}} 
\fancyhead[R]{\raisebox{1ex}[0pt][0pt]{\sffamily{\@author}}}
\makeatother

% circuiti
\usepackage{circuitikz}
\usetikzlibrary{babel}

\begin{document}
% sezione (data)
\section{Lezione del 26-09-24}

% stili pagina
\thispagestyle{empty}
\pagestyle{fancy}

% testo
\subsection{Dipolo elettrico}
Abbiamo introdotto i componenti circuitali come \textbf{dipoli elettrici}.
In particolare, diciamo che un dipolo elettrico è un componente, con una certa differenza di potenziale $V_{AB}$ ai suoi capi e una corrente $i_{AB}(t)$ che vi scorre all'interno, tale per cui si può definire una funzione del tipo:
$$
V_{AB} = f(i_{AB}(t))
$$

Possiamo individuare alcune caratteristiche importanti dei dipoli:
\begin{itemize}
	\item \textbf{Linearità:} un dipolo si dice lineare se la funzione che lega voltaggio e corrente è lineare.
		Tutti i dipoli che studieremo sono lineari (resistenze, capacitori, ecc...).
		Esistono però svariati dipoli che hanno risposte non lineari ai voltaggi/correnti a cui vengono sottoposti (diodi (risposte diverse a direzioni diverse della corrente), amplificatori operazionali, ecc...).
	\item \textbf{Tempo invarianza:} un dipolo si dice tempo invariante quando le sue caratteristiche non variano nel tempo.
	\item \textbf{Memoria:} un dipolo si dice dotato di memoria quando i suoi valori di corrente e tensione attuali dipendono da valori di corrente e tensioni ad un'istante $t$ precedente.
		I dipoli dotati di memoria presentano solitamente \textit{cicli di isteresi}.
	\item \textbf{Passività/attività:} si dice \textbf{passivo} un dipolo che dissipa potenza, e \textbf{attivo} un dipolo che la eroga.
		Più propriamente, si ha che un dipolo e passivo quando l'energia su di esso, presa un riferimento associato, è $\geq 0$.
\end{itemize}

\subsection{Resistori}
Un resistore è un componente circuitale caratterizzato dalla legge di Ohm ($ J = \sigma E$), e quindi formato da un materiale \textit{ohmico} che ha risposta lineare in densità di corrente alle variazioni del campo.
Si indica come:

\begin{center}
\begin{circuitikz}
\draw (0,0) to[ resistor ] (2,0); 
\end{circuitikz}
\end{center}

\begin{theorem}{Prima legge di Ohm}	
Il voltaggio è legato alla corrente, in un resistore, secondo la relazione:
$$V_R(t) = R \ i_R(t)$$
\end{theorem}

dove R prende il nome di \textbf{resistenza}, misurata in Ohm [$\ohm$], definita come:
$$
R = \frac{V}{i}
$$

\subsubsection{Resistenza e resistività}
Conosciamo la legge di Ohm sui materiali ohmici riportata prima.
Da questa legge si ricava:
\begin{theorem}{Seconda legge di Ohm}
	La resistenza di un filo di lunghezza $l$ e sezione $s$ è data da:
	$$
		R = \rho \frac{l}{s}
	$$
\end{theorem}
dove $\rho$ prende il nome di \textbf{resistività}, misurata in Ohm per metro [$\ohm \cdot \meter$].

Questo significa che la resistenza cresce con il crescere della lunghezza, e diminuisce con il crescere della sezione.

In verità questa non sono le uniche caratteristiche che influenzano la resistenza: un apporto significativo è dato anche dalla \textbf{temperatura}, alla quale la resistenza ha proporzionalità quasi lineare, ma che noi ignoreremo.

\subsubsection{Conduttanza e conducibilità}
Conviene definire altre due unità di misura: l'inverso della resistenza, detta \textbf{conduttanza}, che si misura in Siemens [$\ohm^{-1} = \siemens$], o in \textbf{mho} [$\mho = \ohm^{-1}$]:
$$
G = \frac{1}{R}
$$
e l'inverso della resistività, detta \textbf{conducibilità}, che si misura in [$\ohm^{-1} \cdot \meter^{-1}$]:
$$
\sigma = \frac{1}{\rho}
$$

\par\smallskip

I resistori sono inoltre:
\begin{itemize}
	\item Tempo invarianti (a patto di trascurare la temperatura);
	\item Senza memoria;
	\item Passivi (dissipano potenza per \textbf{effetto Joule}).
		Ciò si può dimostrare calcolando la potenza dalla prima legge di Ohm:
		$$
		p(t) = v_{AB}(t) \cdot i_{AB}(t) = R \ i_{AB}^2(t) \geq 0
		$$
		e calcolando l'energia come integrale:
		$$
		w(t) = \int_{-\infty}^{t} p(t)dt \Rightarrow w(t) > 0
		$$
\end{itemize}

\subsubsection{Circuiti aperti/chiusi}
Le resistenza, sopratutto nei loro casi limite, aiutano a modellizzare varie parti di un circuito:
\begin{itemize}
	\item \textbf{Cortocircuito:} indicato da una resistenza nulla, ergo:
		$$
			V_{AB}(t) = 0 \Leftrightarrow R = 0
		$$
		
		Modellizza il filo ideale, ergo ciò che per noi è un ramo.
	\item \textbf{Circuito aperto:} indicato da una corrente nulla, ergo:
		$$
			i_{AB} = 0 \Leftrightarrow R = +\infty
		$$

		Modellizza interruzioni nel circuito: si può dimostrare che la corrente attraverso un'interruzione in un circuito è nulla sfruttando la prima legge di Kirchoff: una linea chiusa che comprende il nodo finale di un'interruzione avrà un ramo entrante e 0 uscenti, ovvero corrente entrante nulla.
\end{itemize}

\subsubsection{Resistenze in serie}
Poniamo di avere una configurazione di resistenze del tipo:

\begin{center}
\begin{circuitikz}
    \draw (0,0) node[left] {$A$} 
        to[R, l=$R_1$] (2,0) 
        to[R, l=$R_2$] (4,0) 
        to[short] (5,0)
        node[anchor=west,xshift=0.15cm] {\dots} (6,0) 
        to[short] (7,0)
        to[R, l=$R_n$] (9,0) node[right] {$B$};

    \draw[decorate,decoration={brace,amplitude=10pt,mirror}] (0,-1) -- (9,-1)
        node[midway,below=10pt]{$n$ resistori};
\end{circuitikz}
\end{center}

Vogliamo calcolare una resistenza $R_{eq}$ che valga quando la resistenza cumulativa di tutte e $n$ le resistenze.
Abbiamo allora che la corrente lungo ogni resistenza $i(t)$ è costante, mentre ogni resistenza contribuisce al potenziale $V_{AB}$ con una certa caduta di potenziale $V_1(t), V_2(t), ..., V_n(T)$.
Si applica quindi la prima legge di Ohm:
$$
V_{AB} = V_1{t} + V_2{t} + ... + V_n{t} = R_1 \cdot i(t) + R_2 \cdot i(t) + ... + R_n \cdot i(t) = i(t) \cdot \left( R_1 + R_2 + ... + R_n \right)
$$
quindi, da $V_{AB} = i(t) \ R_{eq}$ si ha:

\begin{theorem}{Resistenze in serie}	
$$
R_{eq} = R_1 + R_2 + ... + R_n
$$
\end{theorem}

\subsubsection{Resistenze in parallelo}
Poniamo di avere le resistenze in parallelo invece che in serie:
\begin{center}
\begin{circuitikz}
    \draw (2,0) 
				to[short] (0,0) node[left] {$A$};
    
    \draw (2,-2) 
				to[short] (0,-2) node[left] {$B$};
		
		\draw (2,0 )to[R, l=$R_1$] (2,-2);
		\draw (4,0 )to[R, l=$R_2$] (4,-2);
		
		
    \draw (4,0) node[anchor=west,xshift=0.7cm] {\dots} (6,0);
    \draw (4,-2) node[anchor=west,xshift=0.7cm] {\dots} (6,-2); 
		
		\draw (8,0 )to[R, l=$R_n$] (8,-2);
        
    % Draw horizontal lines at the top and bottom
    \draw (0,0) -- (4,0);
    \draw (0,-2) -- (4,-2);

    \draw (6,0) -- (8,0);
    \draw (6,-2) -- (8,-2);

    % Bracket below the resistors
    \draw[decorate,decoration={brace,amplitude=10pt,mirror}] (0,-3) -- (8,-3)
        node[midway,below=10pt]{$n$ resistori};
\end{circuitikz}
\end{center}

Vogliamo ancora calcolare una resistenza $R_{eq}$ che valga quando la resistenza cumulativa di tutte e $n$ le resistenze.
Qui abbiamo che la differenza di potenziale lungo ogni resistenza $V(t)$ costante.
Si applica ancora la prima legge di Ohm:
$$
i = i_1(t) + i_2(t) + ... + i_n(t) = \frac{V_{AB}(t)}{R_1} + \frac{V_{AB}(t)}{R_2} + ... + \frac{V_{AB}(t)}{R_n}
$$
conviene raccogliere e passare alle conduttanze:
$$
G_{eq} = V_{AB}(t)(G_1 + G_2 + ... + G_n) = G_{eq} \cdot V_{AB}(t)
$$
Ora, se $G = \frac{1}{R}$:
$$
R_{eq} = G_{eq}^{-1} = \left( \frac{1}{R_1} + \frac{1}{R_2} + ... + \frac{1}{R_n} \right)^{-1} 
$$
quindi, si ha:

\begin{theorem}{Resistenze in parallelo}
$$
R_{eq} = \left( \frac{1}{R_1} + \frac{1}{R_2} + ... + \frac{1}{R_n} \right)^{-1}
$$
\end{theorem}

\end{document}
