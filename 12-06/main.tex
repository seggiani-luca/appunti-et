\documentclass[a4paper,11pt]{article}
\usepackage[a4paper, margin=8em]{geometry}

% usa i pacchetti per la scrittura in italiano
\usepackage[french,italian]{babel}
\usepackage[T1]{fontenc}
\usepackage[utf8]{inputenc}
\frenchspacing 

% usa i pacchetti per la formattazione matematica
\usepackage{amsmath, amssymb, amsthm, amsfonts}

% usa altri pacchetti
\usepackage{gensymb}
\usepackage{hyperref}
\usepackage{standalone}

% imposta il titolo
\title{Appunti Elettrotecnica}
\author{Luca Seggiani}
\date{2024}

% imposta lo stile
% usa helvetica
\usepackage[scaled]{helvet}
% usa palatino
\usepackage{palatino}
% usa un font monospazio guardabile
\usepackage{lmodern}

\renewcommand{\rmdefault}{ppl}
\renewcommand{\sfdefault}{phv}
\renewcommand{\ttdefault}{lmtt}

% disponi il titolo
\makeatletter
\renewcommand{\maketitle} {
	\begin{center} 
		\begin{minipage}[t]{.8\textwidth}
			\textsf{\huge\bfseries \@title} 
		\end{minipage}%
		\begin{minipage}[t]{.2\textwidth}
			\raggedleft \vspace{-1.65em}
			\textsf{\small \@author} \vfill
			\textsf{\small \@date}
		\end{minipage}
		\par
	\end{center}

	\thispagestyle{empty}
	\pagestyle{fancy}
}
\makeatother

% disponi teoremi
\usepackage{tcolorbox}
\newtcolorbox[auto counter, number within=section]{theorem}[2][]{%
	colback=blue!10, 
	colframe=blue!40!black, 
	sharp corners=northwest,
	fonttitle=\sffamily\bfseries, 
	title=~\thetcbcounter: #2, 
	#1
}

% disponi definizioni
\newtcolorbox[auto counter, number within=section]{definition}[2][]{%
	colback=red!10,
	colframe=red!40!black,
	sharp corners=northwest,
	fonttitle=\sffamily\bfseries,
	title=~\thetcbcounter: #2,
	#1
}

% U.D.M
\newcommand{\amp}{\ensuremath{\mathrm{A}}}
\newcommand{\volt}{\ensuremath{\mathrm{V}}}
\newcommand{\meter}{\ensuremath{\mathrm{m}}}
\newcommand{\second}{\ensuremath{\mathrm{s}}}
\newcommand{\farad}{\ensuremath{\mathrm{F}}}
\newcommand{\henry}{\ensuremath{\mathrm{H}}}
\newcommand{\siemens}{\ensuremath{\mathrm{S}}}

% circuiti
\usepackage{circuitikz}
\usetikzlibrary{babel}

% disegni
\usepackage{pgfplots}
\pgfplotsset{width=10cm,compat=1.9}

% disponi codice
\usepackage{listings}
\usepackage[table]{xcolor}

\lstdefinestyle{codestyle}{
		backgroundcolor=\color{black!5}, 
		commentstyle=\color{codegreen},
		keywordstyle=\bfseries\color{magenta},
		numberstyle=\sffamily\tiny\color{black!60},
		stringstyle=\color{green!50!black},
		basicstyle=\ttfamily\footnotesize,
		breakatwhitespace=false,         
		breaklines=true,                 
		captionpos=b,                    
		keepspaces=true,                 
		numbers=left,                    
		numbersep=5pt,                  
		showspaces=false,                
		showstringspaces=false,
		showtabs=false,                  
		tabsize=2
}

\lstdefinestyle{shellstyle}{
		backgroundcolor=\color{black!5}, 
		basicstyle=\ttfamily\footnotesize\color{black}, 
		commentstyle=\color{black}, 
		keywordstyle=\color{black},
		numberstyle=\color{black!5},
		stringstyle=\color{black}, 
		showspaces=false,
		showstringspaces=false, 
		showtabs=false, 
		tabsize=2, 
		numbers=none, 
		breaklines=true
}

\lstdefinelanguage{javascript}{
	keywords={typeof, new, true, false, catch, function, return, null, catch, switch, var, if, in, while, do, else, case, break},
	keywordstyle=\color{blue}\bfseries,
	ndkeywords={class, export, boolean, throw, implements, import, this},
	ndkeywordstyle=\color{darkgray}\bfseries,
	identifierstyle=\color{black},
	sensitive=false,
	comment=[l]{//},
	morecomment=[s]{/*}{*/},
	commentstyle=\color{purple}\ttfamily,
	stringstyle=\color{red}\ttfamily,
	morestring=[b]',
	morestring=[b]"
}

% disponi sezioni
\usepackage{titlesec}

\titleformat{\section}
	{\sffamily\Large\bfseries} 
	{\thesection}{1em}{} 
\titleformat{\subsection}
	{\sffamily\large\bfseries}   
	{\thesubsection}{1em}{} 
\titleformat{\subsubsection}
	{\sffamily\normalsize\bfseries} 
	{\thesubsubsection}{1em}{}

% disponi alberi
\usepackage{forest}

\forestset{
	rectstyle/.style={
		for tree={rectangle,draw,font=\large\sffamily}
	},
	roundstyle/.style={
		for tree={circle,draw,font=\large}
	}
}

% disponi algoritmi
\usepackage{algorithm}
\usepackage{algorithmic}
\makeatletter
\renewcommand{\ALG@name}{Algoritmo}
\makeatother

% disponi numeri di pagina
\usepackage{fancyhdr}
\fancyhf{} 
\fancyfoot[L]{\sffamily{\thepage}}

\makeatletter
\fancyhead[L]{\raisebox{1ex}[0pt][0pt]{\sffamily{\@title \ \@date}}} 
\fancyhead[R]{\raisebox{1ex}[0pt][0pt]{\sffamily{\@author}}}
\makeatother

\begin{document}
% sezione (data)
\section{Lezione del 06-12-24}

% stili pagina
\thispagestyle{empty}
\pagestyle{fancy}

% testo
\subsection{Macchine elettriche}
Nei circuiti visti finora scorrono \textbf{correnti elettriche}.
Una corrente elettrica genera inevitabilmente \textbf{campi magnetici}.
A volte questi campi magnetici sono effetti indesiderati che vogliamo schermare attraverso appositi meccanismi per evitare l'interferenza fra i componenti.
Altre volte questi campi possono esserci utili a compiere lavoro, come nel caso delle \textbf{macchine elettriche}.

\subsubsection{Campo magnetico}
Introduciamo le grandezze $H$, detta \textbf{intensità di campo magnetico} e misurata in $\frac{\mathrm{A}}{\mathrm{m}}$, e $B$, detta \textbf{induzione magnetica} e misurata in Tesla ($\mathrm{T}$).
Intensità e induzione magnetica sono legati da:
$$
B = \mu H = \mu \mu_r H
$$
dove $\mu_r$ è la \textbf{permeabilità magnetica di un certo materiale} e $\mu$ è la permeabilità elettrica \textbf{del vuoto}.

Introduciamo poi il \textbf{flusso magnetico} $\Phi$, misurato in Weber ($\mathrm{Wb}$), e definito come l'integrale di superficie dell'induzione magnetica:
$$
\Phi = \int_S B \, ds = BS
$$

Da questa definizione, a volte, l'induzione magnetica viene dettà \textbf{densità di flusso magnetico}.

\subsubsection{Legge di Ampere}
La legge che ci aiuta a studiare i campi magnetici è la legge di Ampere:
\begin{theorem}{Legge di Ampere}
	La circuitazione  del capmo magnetico lungo una curva chiusa $\Phi$ è data dalla sommatoria delle correnti concatenate alla curva $i_j$:
	$$
	\int_\Phi H \, dl = \sum_{j=1}^m i_j(t)
	$$
\end{theorem}

\par\smallskip

# disegnino di un mezzo trasformatore

Un caso di esempio che possiamo studiare è quello di un avvolgimento di filo attorno a un corpo ferromagnetico, nel nostro caso un blocco perforato.
All'interno del campo magnetico, il materiale ferromagnetico si magnetizza nella direzione imposta dalle correnti che scorrono nell'avvolgimento di filo.

Un ipotesi comune che facciamo nello studio di sistemi che coinvolgono materiali ferromagnetici, e che la permeabilità magnetica associata $\mu_{fe}$ tenda a $+\infty$, cioè che non ci siano linee di campo magnetico che si chiudono al di fuori del materiale ferromagnetico stesso.

Prendiamo quindi una superficie chiusa $\gamma$ all'interno del blocco perforato.
Sarà che:
$$
\sum_{i=1}^A H_i \cdot l_i = N i = \sum_{i=1}^A \frac{B_i}{\mu \mu_r} \cdot l_i = N i = \sum_{i=1}^A \frac{\Phi_i}{\mu \mu s_i} l_i = Ni
$$
dove abbiamo potuto considerare il solo campo all'interno del tronco del blocco perforato per la semplificazione notata prima.

Notiamo quindi che il campo magnetico magnetico scorre effettivamente come una corrente all'interno dei tronchi del blocco cavo.
Potremo infatti definire la grandezza $\mathcal{R}$, detta \textbf{riluttanza}:
$$
\mathcal{R}_i = \frac{l_i}{\mu \mu_r s_i}
$$

E definire quanto segue; 
\begin{definition}{Tensione magnetica}
	Definiamo la tensione magnetica $Ni$ come segue:
	$$
	\sum_{i=1}^A \Phi_i R_i = N i
	$$
\end{definition}

Notiamo come questa relazione ha una forma simile alla legge di Ohm, dove si possono confrontare le seguenti grandezze:
\begin{table}[h!]
	\center \rowcolors{2}{white}{black!10}
	\begin{tabular} { c | c }
		\bfseries Campi elettrici & \bfseries Campi magnetici \\ 
		\hline
		$V$ & $N i$ \\ 
		$I$ & $\Phi$ \\ 
		$R$ & $\mathcal{R}$ \\ 
	\end{tabular}
\end{table}

La legge riportata sopra, che aveva forma simile alla legge di Ohm, è detta \textbf{legge di Hopkinson}.
Possiamo riportare la forma completa:
$$
\sum_{i=1}^n \Phi_i \mathcal{R}_i = \sum_{j=1}^m N_j i_j
$$

\subsection{Trasformatore}

# disegnino di un trasformatore

Collegando un'ulteriore spira al lato destro del blocco perforato cavo visto prima, otteniamo una macchina elettrica detta \textbf{trasformatore}.
Chiamiamo l'avvolgimento a sinistra \textbf{primario}, che assumiamo come \textit{alimentato}, e l'avvolgimento a destra \textbf{secondario}, che assumiamo come legato a un \textbf{carico} resistivo o induttivo.

Notiamo il circuito elettrico dovrà a trovarsi a regime non costante, in quanto in caso contrario l'induttore si comporterà come un cortocircuito e non ci sarà variazione di campo magnetico: prendiamo nel nosro caso un generatore di funzione sinusoidale.

Scelte le correnti entranti secondo la convenzione delle reti due porte, possiamo disegnare il circuito magnetico:

# circuito magnetico

dove si trattano gli avvolgimenti di filo come generatori di tensione $N_1 i_1$ e $N_2 i_2$ e i tronchi come riluttanze $\mathcal{R}_i$ con $i = 1, ..., 4$.
Potremo studiare questo circuito come qualsiasi altro:
$$
-N_1 I_1 + \mathcal{R}_1 \Phi_1 + \mathcal{R}_2 \Phi_2 + \mathcal{R}_3 \Phi_3 - N_2 I_2 + R_4 \Phi_4 = 0
$$
che non è altro che la legge di Hopkinson applicata all'unica maglia del circuito:
$$
\mathcal{R}_1 \Phi_1 + \mathcal{R}_2 \Phi_2 + \mathcal{R}_3 \Phi_3 + R_4 \Phi_4 = N_1 I_1 + N_2 I_2
$$

\subsubsection{Trasformatore ideale}
Nel trasformatore ideale facciamo 3 ipotesi semplificative:
\begin{enumerate}
	\item La resistenza $R$ degli avvolgimenti dovrà essere uguale a 0;
	\item $\mu_{fe}$ del materiale ferromagnetico tenderà a $+\infty$, quindi non ci sono flussi dispersi in aria: questo ci permette di ridurre i flussi $\Phi_i$ a un unico flusso $\Phi$;
	\item $\mu_{fe}$ è inoltre costante nel tempo.
\end{enumerate}

Possiamo quindi riportare l'equazione precedente, dalla (2), a:
$$
(\mathcal{R}_1 + \mathcal{R}_2 + \mathcal{R}_3 + \mathcal{R}_4) \Phi = N_1 I_1 + N_2 I_2
$$

Ricordando la definizione di riluttanza:
$$
\mathcal{R}_i = \frac{l_i}{\mu \mu_r s_i}
$$
e imponendo la sempre la (2), si avrà che $\mathcal{R} \rightarrow 0$, quindi:
$$
N_1 I_1 + N_2 I_2 = 0 \implies I_1 = -\frac{N_2}{N_1} I_2 = -\frac{1}{n} I_2
$$
dove $n$ viene detto \textbf{rapporto spire} $n = \frac{N_1}{N_2}$.

Reintroducendo le tensioni ai capi degli avvolgimenti $\dot{E}_1$ ed $\dot{E}_2$, abbiamo nel dominio fasoriale:
$$
\dot{E}_1 = j \omega \dot{\Phi}_1 N_1, \quad \dot{E}_2 = j \omega \dot{\Phi}_2 N_2
$$
da cui prendiamo il rapporto:
$$
\frac{\dot{E}_1}{\dot{E}_2} = \frac{j \omega \dot{\Phi}_1 N_1}{j \omega \dot{\Phi}_2 N_2} = \frac{N_1}{N_2} = n
$$

Abbiamo quindi le due relazioni per il trasformatore ideale:
\[
	\begin{cases}
		\dot{I} = - \frac{1}{n} \dot{I}_2 \\ 
		\dot{E}_1 = n \dot{E}_2
	\end{cases}
\]

Distinguiamo quindi 3 casi, notando che la convenzione si riferisce alle tensioni:
\begin{itemize}
	\item $n > 1$: trasformatore \textbf{riduttore} di tensione, \textit{elevatore} di corrente;
	\item $n < 1$: trasformatore \textbf{elevatore} di tensione, \textit{riduttore} di corrente;
	\item $n = 1$: \textbf{disaccoppiatore}.
\end{itemize}

\subsubsection{Potenza sui trasformatori ideali}

Facciamo le solite considerazioni di \textbf{potenza}.
Si ha riguardo alla potenza apparente che:
$$
S_1 = E_1 N_1 = n E_2 \left( -\frac{1}{n} I_2 \right) = -E_2 I_2 = S_2
$$

\subsubsection{Circuito equivalente del trasformatore ideale}

Un circuito equivalente a un trasformatore ideale è dato alle induttanze mutuamente accoppiate:

# disegno

dove si impone alle induttanze:
$$
L_1 = \mu \mu_r \frac{s}{l} N_1^2
$$
$$
L_2 = \mu \mu_r \frac{s}{l} N_2^2
$$

Con mutuo accoppiamento ideale si ha:
$$
M = \sqrt{L_1 L_2} = \mu \mu_r \frac{s}{l} N_1 N_2 
$$

Possiamo applicare le relazioni tensione corrente delle mutue induttanze:
\[
	\begin{cases}
		v_1(t) = L_1 \frac{d\,i_1(t)}{dt} + M \frac{d\, i_2(t)}{dt} \\ 	
		v_2(t) = M \frac{d\,i_1(t)}{dt} + L_2 \frac{d\, i_2(t)}{dt} \\ 	
	\end{cases}
	\implies
	\begin{cases}
		v_1(t) = L_1 \left( \frac{d\,i_1(t)}{dt} + \frac{M}{L_1} \frac{d\, i_2(t)}{dt} \right) \\ 	
		v_2(t) = M \left( \frac{d\,i_1(t)}{dt} + \frac{L_2}{M} \frac{d\, i_2(t)}{dt} \right) \\ 	
	\end{cases}
\]
da cui il rapporto:
$$
\frac{v_1(t)}{v_2(t)} = \frac{ L_1 \left( \frac{d\,i_1(t)}{dt} + \frac{M}{L_1} \frac{d\, i_2(t)}{dt} \right)}{ M \left( \frac{d\,i_1(t)}{dt} + \frac{L_2}{M} \frac{d\, i_2(t)}{dt} \right)} = \frac{L_1}{M} = \frac{N_1}{N_2} = n
$$

Abbiamo quindi dimostrato che il circuito è effettivamente equivalente a un trasformatore ideale.

\subsubsection{Disaccoppiatore}
Possiamo fare un esempio dell'utilità di un trasformatore con $n=1$ riprendendo il circuito equivalente di una rete due porte recipreca a parametri Z:

# rete due porte reciproca a parametri Z

Questa non era altro che la semplificazione della rete generale a parametri Z:

# rete due porte a parametri Z

Notiamo che il potenziale ai due nodi in basso non è identico nel caso del circuito generale, mentre lo abbiamo posto come tale nel circuito reciproco.
Questa è una semplificazione che potremmo non voler fare: possiamo permettere potenziali diversi introducendo appunto un disaccoppiatore:

# io penso si chiami disaccoppiatore poi che ne so, cmq disegno

\subsubsection{Adattamento di impedenza}

Poniamo il circuito magnetico dato da generatore di tensione che alimenta, attraverso un trasformatore, un carico $\overline{z_C}$:

# disegno

Si avrà che, preso il circuito come una rete due porte:
$$
\dot{E}_2 = - \overline{z_C} \dot{I}_2
$$
e dal circuito magnetico:
$$
\dot{E}_1 = n \dot{E}_2 \Rightarrow \dot{E}_2 = \frac{\dot{E}_1}{n}
$$
$$
\dot{I}_1 = -\frac{1}{n} \dot{I_2} \Rightarrow \dot{I}_2 = -n \dot{I}_1
$$
quindi sostituendo:
$$
\frac{\dot{E}_1}{n} = - \overline{z_C} (-n \dot{I_1}) \Rightarrow \dot{E}_1 = \overline{z_C} n^2 \dot{I}_1
$$

# troppo forte riguarda

\subsubsection{Trasformatore reale}
Riprendiamo il trasformatore reale.
Il circuito equivalente non può più essere formato prendendo due induttori in accoppiamento ideale.
Introduciamo quindi due resistenze $R_{1d}$ e $R_{2d}$ in entrata agli induttori in maniera tale da modellizzare la resistenza $R$ degli avvolgimenti (ipotesi 1), e altre due induttanze $L_{1d}$ e $L_{2d}$ per modellizzare i flussi dispersi in aria da un coefficiente $\mu_{fe}$ non ideale (cioè non tendente a $+\infty$, dall'ipotesi 2).

Notiamo che la permeabilità $\mu_{fe}$ non è nemmeno costante: il rapporto fra intensità di campo magnetico e induzinoe magnetica, che finora avevamo descritto come:
$$
B = \mu \mu_{fe} H
$$
presenta in verità delle non linearità date dall'\textbf{isteresi} del materiale.

# grafichino classico 

\end{document}
