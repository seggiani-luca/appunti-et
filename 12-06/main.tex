\documentclass[a4paper,11pt]{article}
\usepackage[a4paper, margin=8em]{geometry}

% usa i pacchetti per la scrittura in italiano
\usepackage[french,italian]{babel}
\usepackage[T1]{fontenc}
\usepackage[utf8]{inputenc}
\frenchspacing 

% usa i pacchetti per la formattazione matematica
\usepackage{amsmath, amssymb, amsthm, amsfonts}

% usa altri pacchetti
\usepackage{gensymb}
\usepackage{hyperref}
\usepackage{standalone}

% imposta il titolo
\title{Appunti Elettrotecnica}
\author{Luca Seggiani}
\date{2024}

% imposta lo stile
% usa helvetica
\usepackage[scaled]{helvet}
% usa palatino
\usepackage{palatino}
% usa un font monospazio guardabile
\usepackage{lmodern}

\renewcommand{\rmdefault}{ppl}
\renewcommand{\sfdefault}{phv}
\renewcommand{\ttdefault}{lmtt}

% disponi il titolo
\makeatletter
\renewcommand{\maketitle} {
	\begin{center} 
		\begin{minipage}[t]{.8\textwidth}
			\textsf{\huge\bfseries \@title} 
		\end{minipage}%
		\begin{minipage}[t]{.2\textwidth}
			\raggedleft \vspace{-1.65em}
			\textsf{\small \@author} \vfill
			\textsf{\small \@date}
		\end{minipage}
		\par
	\end{center}

	\thispagestyle{empty}
	\pagestyle{fancy}
}
\makeatother

% disponi teoremi
\usepackage{tcolorbox}
\newtcolorbox[auto counter, number within=section]{theorem}[2][]{%
	colback=blue!10, 
	colframe=blue!40!black, 
	sharp corners=northwest,
	fonttitle=\sffamily\bfseries, 
	title=~\thetcbcounter: #2, 
	#1
}

% disponi definizioni
\newtcolorbox[auto counter, number within=section]{definition}[2][]{%
	colback=red!10,
	colframe=red!40!black,
	sharp corners=northwest,
	fonttitle=\sffamily\bfseries,
	title=~\thetcbcounter: #2,
	#1
}

% U.D.M
\newcommand{\amp}{\ensuremath{\mathrm{A}}}
\newcommand{\volt}{\ensuremath{\mathrm{V}}}
\newcommand{\meter}{\ensuremath{\mathrm{m}}}
\newcommand{\second}{\ensuremath{\mathrm{s}}}
\newcommand{\farad}{\ensuremath{\mathrm{F}}}
\newcommand{\henry}{\ensuremath{\mathrm{H}}}
\newcommand{\siemens}{\ensuremath{\mathrm{S}}}

% circuiti
\usepackage{circuitikz}
\usetikzlibrary{babel}

% disegni
\usepackage{pgfplots}
\pgfplotsset{width=10cm,compat=1.9}

% disponi codice
\usepackage{listings}
\usepackage[table]{xcolor}

\lstdefinestyle{codestyle}{
		backgroundcolor=\color{black!5}, 
		commentstyle=\color{codegreen},
		keywordstyle=\bfseries\color{magenta},
		numberstyle=\sffamily\tiny\color{black!60},
		stringstyle=\color{green!50!black},
		basicstyle=\ttfamily\footnotesize,
		breakatwhitespace=false,         
		breaklines=true,                 
		captionpos=b,                    
		keepspaces=true,                 
		numbers=left,                    
		numbersep=5pt,                  
		showspaces=false,                
		showstringspaces=false,
		showtabs=false,                  
		tabsize=2
}

\lstdefinestyle{shellstyle}{
		backgroundcolor=\color{black!5}, 
		basicstyle=\ttfamily\footnotesize\color{black}, 
		commentstyle=\color{black}, 
		keywordstyle=\color{black},
		numberstyle=\color{black!5},
		stringstyle=\color{black}, 
		showspaces=false,
		showstringspaces=false, 
		showtabs=false, 
		tabsize=2, 
		numbers=none, 
		breaklines=true
}

\lstdefinelanguage{javascript}{
	keywords={typeof, new, true, false, catch, function, return, null, catch, switch, var, if, in, while, do, else, case, break},
	keywordstyle=\color{blue}\bfseries,
	ndkeywords={class, export, boolean, throw, implements, import, this},
	ndkeywordstyle=\color{darkgray}\bfseries,
	identifierstyle=\color{black},
	sensitive=false,
	comment=[l]{//},
	morecomment=[s]{/*}{*/},
	commentstyle=\color{purple}\ttfamily,
	stringstyle=\color{red}\ttfamily,
	morestring=[b]',
	morestring=[b]"
}

% disponi sezioni
\usepackage{titlesec}

\titleformat{\section}
	{\sffamily\Large\bfseries} 
	{\thesection}{1em}{} 
\titleformat{\subsection}
	{\sffamily\large\bfseries}   
	{\thesubsection}{1em}{} 
\titleformat{\subsubsection}
	{\sffamily\normalsize\bfseries} 
	{\thesubsubsection}{1em}{}

% disponi alberi
\usepackage{forest}

\forestset{
	rectstyle/.style={
		for tree={rectangle,draw,font=\large\sffamily}
	},
	roundstyle/.style={
		for tree={circle,draw,font=\large}
	}
}

% disponi algoritmi
\usepackage{algorithm}
\usepackage{algorithmic}
\makeatletter
\renewcommand{\ALG@name}{Algoritmo}
\makeatother

% disponi numeri di pagina
\usepackage{fancyhdr}
\fancyhf{} 
\fancyfoot[L]{\sffamily{\thepage}}

\makeatletter
\fancyhead[L]{\raisebox{1ex}[0pt][0pt]{\sffamily{\@title \ \@date}}} 
\fancyhead[R]{\raisebox{1ex}[0pt][0pt]{\sffamily{\@author}}}
\makeatother

\begin{document}
% sezione (data)
\section{Lezione del 06-12-24}

% stili pagina
\thispagestyle{empty}
\pagestyle{fancy}

% testo
Nei circuiti visti finora scorrono \textbf{correnti elettriche}.
Una corrente elettrica genera inevitabilmente \textbf{campi magnetici}.
A volte questi campi magnetici sono effetti indesiderati che vogliamo schermare attraverso appositi meccanismi per evitare l'interferenza fra i componenti.
Altre volte questi campi possono esserci utili a compiere lavoro, come nel caso delle \textbf{macchine elettriche}.

\subsubsection{Campo magnetico}
Introduciamo le grandezze $\vec{H}$, detta \textbf{intensità di campo magnetico} e misurata in $\frac{\mathrm{A}}{\mathrm{m}}$, e $\vec{B}$, detta \textbf{induzione magnetica} e misurata in Tesla ($\mathrm{T}$).
Intensità e induzione magnetica sono legati da:
$$
\vec{B} = \mu \vec{H} = \mu_0 \mu_r \vec{H}
$$
dove $\mu_r$ è la \textbf{permeabilità magnetica} \textit{di un certo materiale} e $\mu_0$ è la \textbf{permeabilità magnetica del vuoto}.
In particolare, quindi, la prima equazione si riferisce all'induzione magnetica \textit{nel vuoto}, e la seconda all'induzione magnetica \textit{in un certo materiale}.

Introduciamo poi il \textbf{flusso magnetico} $\Phi$, o più propriamente $\Phi(\vec{B})$ (anche se lo indicheremo spesso solo come $\Phi$), misurato in Weber ($\mathrm{Wb}$), e definito come l'integrale di superficie dell'induzione magnetica:
$$
\Phi(\vec{B}) = \int_S \vec{B} \, ds = \vec{B} \cdot \hat{N} = BS
$$
dove l'ultima relazione vale se $\vec{B} \cdot \hat{N}$ con $N$ normale a $s$ è uguale a 1 (altrimenti compare il solito termine $\cos{\theta}$).

Da questa definizione, a volte, l'induzione magnetica viene dettà \textbf{densità di flusso magnetico}.

\subsubsection{Legge di Ampère}
La relazione matematica che ci aiuta a studiare i campi magnetici è la legge di Ampère:
\begin{theorem}{Legge di Ampère}
	La circuitazione  del campo magnetico lungo una curva chiusa $\gamma$ è data dalla sommatoria delle correnti concatenate alla curva $I_j$:
	$$
	\int_\gamma \vec{H} \, dl = \sum_{j=1}^n I_j(t)
	$$
\end{theorem}

Possiamo esprimere la legge di Ampère anche attraverso l'induzione magnetica, come:
$$
\int_\gamma \vec{B} \, dl = \mu_0 \sum_{j=1}^{n} I_j(t)
$$
nel vuoto e:
$$
\int_\gamma \vec{B} \, dl = \mu_0 \mu_r \sum_{j=1}^{n} I_j(t)
$$
in un materiale con permeabilità magnetica $\mu_r$. 

\par\smallskip

Un caso di esempio che possiamo studiare è quello di un avvolgimento di filo in $N$ spire attorno a un corpo ferromagnetico, nel nostro caso un blocco perforato:

\begin{center}
	\begin{circuitikz}
		\draw (0,0) 
			-- (3,0)
			-- (3,-3)
			-- (0,-3)
			-- (0,0);
		\draw (0.5,-0.5) 
			-- (2.5,-0.5)
			-- (2.5,-2.5)
			-- (0.5,-2.5)
			-- (0.5,-0.5);
 
		\ctikzset{bipoles/thickness=1}
		\ctikzset{bipoles/cuteinductor/width=1.13}
		\ctikzset{bipoles/cuteinductor/height=0.9}
		
		\ctikzset{bipoles/cuteinductor/coils = 6}
		
		\draw (-1, -0.7)
			to [short, i=$i$] (0, -0.7) 
			to [inductor] (0, -2.3)
			-- (-1, -2.3);

		\draw (-1,-0.7) node[anchor=east] {$+$};
		\draw (-1,-2.3) node[anchor=east] {$-$};

		\draw (-1,-1.5) node[anchor=east] {$E$};
		\draw (-0.2,-1.5) node[anchor=east] {$N$};
	\end{circuitikz}
\end{center}

Ogni sezione del blocco viene detta \textbf{tronco}, e assumiamo le superfici trasversali $S$ dei tronchi pressoché costanti fra tronchi e sulla lunghezza di un tronco.
Sottoposto a un campo magnetico, il materiale ferromagnetico si magnetizza nella direzione imposta dalle correnti che scorrono nell'avvolgimento di filo, e il blocco presenta effettivamente quello che possiamo modellizzare come un fenomeno di conduzione di $\vec{B}$.
In particolare, nello studio di sistemi che coinvolgono materiali ferromagnetici, facciamo due ipotesi:
\begin{enumerate}
	\item Il rapporto fra le permeabiltà magnetiche associate al materiale ferromagnetico $\mu_{fe}$ e all'aria circostante $\mu_{a}$ tende a $+\infty$ o comunque è $\mu_{fe} >> \mu_{a}$, cioè non ci sono (o sono molto limitate) linee di campo magnetico che si chiudono al di fuori del materiale ferromagnetico stesso;
	\item In ogni sezione del ferromagnetico si ha una distribuzione uniforme dell'induzione magnetica $\vec{B}$, cioè modulo, verso e direzione di $\vec{B}$ sono costanti su ogni superficie $S$ trasversale dei tronchi del ferromagnetico.
\end{enumerate}

Prendiamo quindi una superficie chiusa $\gamma$ all'interno del blocco perforato.
Le ipotesi scelte ci permetteranno di dire:
$$
\int_\gamma H \, dl = Ni
$$
che riportandosi al caso discreto dei 4 tronchi dell'esempio significa:
$$
\int_\gamma H \, dl = \sum_{i=1}^A H_i \cdot l_i = \sum_{i=1}^A \frac{\vec{B_i}}{\mu_0 \mu_r} \cdot l_i = \sum_{i=1}^A \frac{\Phi_i}{\mu_0 \mu_r S_i} \cdot l_i = Ni
$$

In particolare, l'ipotesi (1) ci permette di passare alla sommatoria in $\vec{B_i}$, cioè affermare che il campo magnetico è tutto interno al tronco, mentre l'ipotesi (2) ci permette invece di prendere il campo come $\vec{B_i} = \frac{\Phi_i}{S_i}$ su una superficie $S_i$ assunta costante.
Inoltre, l'ipotesi (1) ci permette di ridurre i flussi $\Phi_i$ all'unico flusso $\Phi$ che scorrerà in comune su tutti i tronchi del corpo ferromagnetico, mentre attraverso la (2), assunti tronchi di superficie identica, ridurremo le $S_i$ a un unica $S$ comune:
$$
Ni = \sum_{i=1}^A \frac{\Phi}{\mu_0 \mu_r S} \cdot l_i
$$

Notiamo quindi che il campo magnetico magnetico, o meglio il flusso magnetico $\Phi$, scorre effettivamente come una corrente all'interno dei tronchi del blocco cavo.
Potremo definire la grandezza $\mathcal{R}$, detta \textbf{riluttanza}:
$$
\mathcal{R}_i = \frac{l_i}{\mu \mu_r s_i}
$$
in generale:
$$
\mathcal{R} = \frac{l}{\mu \mu_r s}
$$

Diciamo poi che $N i$ rappresenta una \textbf{tensione magnetica}, da cui:
$$
\sum\limits_{i=1}^A \Phi_i \mathcal{R}_i = N i
$$

Le definizioni di riluttanza e tensione magnetica ci permettono quindi di realizzare, sempre sotto le precedenti ipotesi, una modellizzazione che prende il nome di \textbf{circuito magnetico}:
\begin{center}
	\begin{circuitikz}
		\draw (0,0) to [resistor, R=$\mathcal{R}_2$] (3, 0)
		to [resistor, R=$\mathcal{R}_3$] (3, -3)
		to [resistor, R=$\mathcal{R}_4$] (0, -3)
		to [voltage source, v=$Ni$] (0, -1.5)
		to [resistor, R=$\mathcal{R}_1$] (0, 0);
	\end{circuitikz}
\end{center}

La legge riportata sopra, che aveva forma simile alla legge di Ohm, è detta \textbf{legge di Hopkinson}.
Riportiamo la forma completa:
\begin{theorem}{Legge di Hopkinson}
	Su un circuito magnetico con $n$ avvolgimenti di $N_i$ spire dove scorre corrente $I_i$, e $m$ tronchi con riluttanza $\mathcal{R}_j$ su cui scorre un flusso $\Phi_j$, si ha che:
$$
\sum_{i=1}^n N_i i_i = \sum_{j=1}^m \Phi_j \mathcal{R}_j 
$$
\end{theorem}

Notiamo come questa relazione ha una forma simile alla legge di Ohm $V = IR$, dove si possono confrontare le seguenti grandezze:
\begin{table}[h!]
	\center \rowcolors{2}{white}{black!10}
	\begin{tabular} { c | c }
		\bfseries Campi elettrici & \bfseries Campi magnetici \\ 
		\hline
		$V$ & $N i$ \\ 
		$I$ & $\Phi$ \\ 
		$R$ & $\mathcal{R}$ \\ 
	\end{tabular}
\end{table}

\subsection{Trasformatore}
Collegando un'ulteriore spira al lato destro del blocco perforato cavo visto prima, otteniamo una macchina elettrica detta \textbf{trasformatore}:

\begin{center}
	\begin{circuitikz}
		\draw (0,0) 
			-- (3,0)
			-- (3,-3)
			-- (0,-3)
			-- (0,0);
		\draw (0.5,-0.5) 
			-- (2.5,-0.5)
			-- (2.5,-2.5)
			-- (0.5,-2.5)
			-- (0.5,-0.5);
 
		\ctikzset{bipoles/thickness=1}
		\ctikzset{bipoles/cuteinductor/width=1.13}
		\ctikzset{bipoles/cuteinductor/height=0.9}
		
		\ctikzset{bipoles/cuteinductor/coils = 6}
		
		\draw (-1, -0.7)
			to [short, i=$i_1$] (0, -0.7) 
			to [inductor] (0, -2.3)
			-- (-1, -2.3);

		\ctikzset{bipoles/cuteinductor/coils = 4}

		\draw (4, -2.3)
			-- (3, -2.3) 
			to [inductor] (3, -0.7)
			to [short, i<=$i_2$] (4, -0.7);

		\draw (-1,-0.7) node[anchor=east] {$+$};
		\draw (-1,-2.3) node[anchor=east] {$-$};

		\draw (4,-0.7) node[anchor=west] {$+$};
		\draw (4,-2.3) node[anchor=west] {$-$};

		\draw (-1,-1.5) node[anchor=east] {$E_1$};
		\draw (-0.2,-1.5) node[anchor=east] {$N_1$};
		\draw (4,-1.5) node[anchor=west] {$E_2$};
		\draw (3.2,-1.5) node[anchor=west] {$N_2$};

	\end{circuitikz}
\end{center}

Chiamiamo l'avvolgimento a sinistra \textbf{primario}, che assumiamo come \textit{alimentato}, e l'avvolgimento a destra \textbf{secondario}, che assumiamo come legato a un \textbf{carico}, solitamente resistivo, induttivo o ohmico-induttivo (carichi ohmico-capacitivi o puramente capacitivi sono più rari, in quanto solitamente le macchine elettriche lavorano con campi magnetici, quindi intrinsecamente induttivi).
Il primario avrà $N_1$ spire e il secondario $N_2$.
Notiamo il circuito elettrico dovrà a trovarsi a regime non costante, in quanto in caso contrario l'induttore si comporterà come un cortocircuito e non ci sarà variazione di campo magnetico: prendiamo, ad esempio, un generatore di funzione sinusoidale.

Possiamo disegnare il circuito magnetico:
\begin{center}
	\begin{circuitikz}
		\draw (0,0) to [resistor, R=$\mathcal{R}_2$] (3, 0)
		to [voltage source, v=$N_2 i_2$] (3, -1.5)
		to [resistor, R=$\mathcal{R}_3$] (3, -3)
		to [resistor, R=$\mathcal{R}_4$] (0, -3)
		to [voltage source, v=$N_1 i_1$] (0, -1.5)
		to [resistor, R=$\mathcal{R}_1$] (0, 0);
	\end{circuitikz}
\end{center}

dove si trattano gli avvolgimenti di filo come generatori di tensione $N_1 i_1$ e $N_2 i_2$ e i tronchi come riluttanze $\mathcal{R}_i$ con $i = 1, ..., 4$.
Potremo studiare questo circuito come qualsiasi altro:
$$
-N_1 I_1 + \mathcal{R}_1 \Phi_1 + \mathcal{R}_2 \Phi_2 + \mathcal{R}_3 \Phi_3 - N_2 I_2 + R_4 \Phi_4 = 0
$$
che non è altro che la legge di Hopkinson applicata all'unica maglia del circuito:
$$
\mathcal{R}_1 \Phi_1 + \mathcal{R}_2 \Phi_2 + \mathcal{R}_3 \Phi_3 + R_4 \Phi_4 = N_1 I_1 + N_2 I_2
$$

\subsubsection{Trasformatore ideale}
Nel trasformatore ideale facciamo 3 ipotesi semplificative:
\begin{enumerate}
	\item La resistenza $R$ degli avvolgimenti dovrà essere uguale a 0;
	\item Il rapporto fra le permeabiltà magnetiche associate al materiale ferromagnetico $\mu_{fe}$ e all'aria circostante $\mu_{a}$ tenderà a $+\infty$, quindi non ci saranno flussi dispersi in aria: questo ci permette di ridurre i flussi $\Phi_i$ a un unico flusso $\Phi$;
	\item $\mu_{fe}$ è costante nel tempo.
\end{enumerate}

Possiamo quindi riportare l'equazione precedente, dalla (2), a:
$$
(\mathcal{R}_1 + \mathcal{R}_2 + \mathcal{R}_3 + \mathcal{R}_4) \Phi = N_1 I_1 + N_2 I_2
$$

Ricordando la definizione di riluttanza:
$$
\mathcal{R}_i = \frac{l_i}{\mu \mu_r s_i}
$$
e imponendo sempre la (2), si avrà che $\mu_r \rightarrow +\infty$, cioè:
$$
\mathcal{R}_i = \lim_{\mu_r \rightarrow +\infty} \frac{l_i}{\mu \mu_r s_i} = 0
$$ 
e quindi:
$$
N_1 I_1 + N_2 I_2 = 0 \implies I_1 = -\frac{N_2}{N_1} I_2 = -\frac{1}{n} I_2
$$
dove $n$ viene detto \textbf{rapporto spire} $n = \frac{N_1}{N_2}$.

Reintroducendo le tensioni ai capi degli avvolgimenti $\dot{E}_1$ ed $\dot{E}_2$, abbiamo nel dominio fasoriale:
$$
\dot{E}_1 = j \omega \dot{\Phi}_1 N_1, \quad \dot{E}_2 = j \omega \dot{\Phi}_2 N_2
$$
da cui prendiamo il rapporto:
$$
\frac{\dot{E}_1}{\dot{E}_2} = \frac{j \omega \dot{\Phi}_1 N_1}{j \omega \dot{\Phi}_2 N_2} = \frac{N_1}{N_2} = n
$$

Abbiamo quindi le due relazioni per il trasformatore ideale:
\[
	\begin{cases}
		\dot{I}_1 = - \frac{1}{n} \dot{I}_2 \\ 
		\dot{E}_1 = n \dot{E}_2
	\end{cases}
\]

Distinguiamo quindi 3 casi, notando che la convenzione si riferisce alle tensioni:
\begin{itemize}
	\item $n > 1$: trasformatore \textbf{riduttore} di tensione, \textit{elevatore} di corrente (detto anche \textit{step-up});
	\item $n < 1$: trasformatore \textbf{elevatore} di tensione, \textit{riduttore} di corrente (detto anche \textit{step-down});
	\item $n = 1$: in questo caso il trasformatore si comporta da \textbf{isolatore galvanico} e viene detto \textbf{accoppiamento induttivo}.
\end{itemize}

\subsubsection{Potenza sui trasformatori ideali}

Facciamo le solite considerazioni di \textbf{potenza}.
Si ha, ad esempio riguardo alla potenza apparente, che posto quanto trovato prima ($\dot{I}_1 = -\frac{1}{n} \dot{I}_2$ e $\dot{E}_1 = n \dot{E}_2$):
$$
S_1 = E_1 N_1 = n E_2 \left( -\frac{1}{n} I_2 \right) = -E_2 I_2 = S_2
$$
cioè che la potenza ai due lati del trasformatore è identica.
Questo significa che il trasformatore ci permette di trasferire energia da un lato all'altro variando il rapporto fra corrente e voltaggio.
Assumendo condizioni ideali, quindi, si può applicare la legge di Ohm, ad esempio:
$$
V_2 = 2 \cdot V_1 \implies I_2 = \frac{I_1}{2}
$$
cioè si può trasmettere energia a voltaggio doppio con una corrente dimezzata.

Questo torna particolarmente utile nella trasmissione di energia elettrica nelle reti di distribuzione.
Per esigenze tecniche, le centrali eletriche producono energia a tensioni relativamente basse. 
Prese le linee di trasmissione come enormi resistori con $R$ determinata dalla seconda legge di Ohm ($R = \rho \frac{l}{A}$), e trascurando l'errore che otteniamo dall'applicazione del modello a \textit{lumped circuit} ad una linea di trasmissione (con estensione possibilmente anche di diverse centinaia di chilometri), abbiamo che su queste viene dissipata una potenza $P = I^2 R$.
Chiaramente, vorremo trasmettere energia a correnti il più basse possibili per risparmiare sulla potenza.
Un trasformatore \textit{step-up} alla centrale ci permette quindi di portare la corrente a tensioni molto alte, mentre un trasformatore \textit{step-up} al centro di distribuzione locale ci permette di riabbassare la tensione a livelli utili, appunto, alla distribuzione locale. 
La differenza di potenziale $\Delta V$ complessiva fra i due capi della linea risulterà quindi piccola, ergo sarà piccola la corrente $I$, e otterremo quanto desiderato in termini di potenza dissipata. 

\subsubsection{Circuito equivalente del trasformatore ideale}
Possiamo modellizzare il trasformatore ideale attraverso induttanze mutuamente accoppiate:

\begin{center}
	\begin{circuitikz}
		\draw (0,0) to[ short, i=$i_1$] (1,0)
			to[ inductor , l=$N_1$] (1,-2)
			-- (0, -2);

		\draw (4,0) to[ short, i_=$i_2$] (3,0)
			to[ inductor , l_=$N_2$] (3,-2)
			-- (4, -2);

			\draw (0.9,-0.6) node {$\scriptscriptstyle\bullet$};
			\draw (2.9,-0.6) node {$\scriptscriptstyle\bullet$};

			\draw (0,-0.5) node[anchor=east] {$+$};
			\draw (0,-1.5) node[anchor=east] {$-$};

			\draw (4,-0.5) node[anchor=west] {$+$};
			\draw (4,-1.5) node[anchor=west] {$-$};

			\draw (0,-1) node[anchor=east] {$E_1$};
			\draw (4,-1) node[anchor=west] {$E_2$};
	\end{circuitikz}
\end{center}

Si impone alle induttanze:
$$
L_1 = \frac{N_1^2}{\mathcal{R}_{eq}} = \mu \mu_r \frac{s}{l} N_1^2
$$
$$
L_2 = \frac{N_2^2}{\mathcal{R}_{eq}} = \mu \mu_r \frac{s}{l} N_2^2
$$
e, assunto mutuo accoppiamento ideale:
$$
M = \sqrt{L_1 L_2} = \mu \mu_r \frac{s}{l} N_1 N_2 
$$

Notiamo che nel caso ideale ($\mathcal{R}_{eq} \rightarrow 0$), si ha che $L_1, L_2, M \rightarrow + \infty$.

Vediamo allora di dimostrare che questa configurazione è effettivamente equivalente al trasformatore ideale.
Possiamo applicare le relazioni tensione corrente delle mutue induttanze:
$$
	\begin{cases}
		v_1(t) = L_1 \frac{d\,i_1(t)}{dt} + M \frac{d\, i_2(t)}{dt} \\ 	
		v_2(t) = M \frac{d\,i_1(t)}{dt} + L_2 \frac{d\, i_2(t)}{dt} \\ 	
	\end{cases}
$$
Operiamo quindi il raggruppamento:
$$
	\begin{cases}
		v_1(t) = L_1 \left( \frac{d\,i_1(t)}{dt} + \frac{M}{L_1} \frac{d\, i_2(t)}{dt} \right) \\ 	
		v_2(t) = M \left( \frac{d\,i_1(t)}{dt} + \frac{L_2}{M} \frac{d\, i_2(t)}{dt} \right) \\ 	
	\end{cases}
$$

Vediamo che:
$$
\frac{M}{L_1} = \frac{N_1 N_2}{\mathcal{R}_{eq}} \cdot \frac{\mathcal{R}_{eq}}{N_1^2} = \frac{N_2}{N_1}, \quad
\frac{L_2}{M} = \frac{N_2^2}{\mathcal{R}_{eq}} \cdot \frac{\mathcal{R}_{eq}}{N_1 N_2} = \frac{N_2}{N_1}
$$

Possiamo quindi semplificare, ricordando da sopra che  $\frac{M}{L_1} = \frac{N_2}{N_1}$ e quindi $\frac{L_1}{M_1} = \left( \frac{M}{L_1} \right)^{-1} = \frac{N_1}{N_1}$:
$$
\frac{v_1(t)}{v_2(t)} = \frac{ L_1 \left( \frac{d\,i_1(t)}{dt} + \frac{M}{L_1} \frac{d\, i_2(t)}{dt} \right)}{ M \left( \frac{d\,i_1(t)}{dt} + \frac{L_2}{M} \frac{d\, i_2(t)}{dt} \right)} = \frac{L_1}{M} = \frac{N_1}{N_2} = n
$$

Abbiamo quindi dimostrato che il circuito è effettivamente equivalente a un trasformatore ideale.

\subsubsection{Trasformtore isolatore}
Possiamo fare un esempio dell'utilità di un trasformatore con $n=1$ riprendendo il circuito equivalente di una rete due porte recipreca a parametri Z:

\begin{center}
	\begin{circuitikz}
		\draw (-4, 1) -- (-3, 1) 
			to [ european resistor, l=$\overline{z_a}$, i=$I_1$] (0, 1)
			to [ european resistor, l=$\overline{z_b}$] (0, -1) 
			to [ short, i=$I_1$ ] (-3, -1)	
			-- (-4, -1);
			
		\draw (-4.6, 1) node[anchor=west] {$+$};
		\draw (-4.6, 0) node[anchor=west] {$V_1$};
		\draw (-4.6, -1) node[anchor=west] {$-$};

		\draw (4, 1) -- (3, 1) 
			to [ european resistor, l_=$\overline{z_c}$, i_=$I_2$] (0, 1);

		\draw (0, -1) to [ short, i_=$I_2$ ] (3, -1)
			-- (4, -1);
	
		\draw (4.6, 1) node[anchor=east] {$+$};
		\draw (4.6, 0) node[anchor=east] {$V_2$};
		\draw (4.6, -1) node[anchor=east] {$-$};
		
		\node[rectangle, draw, minimum width = 6.5cm, minimum height = 4cm] (a) at (0,0) {};
	\end{circuitikz}
\end{center}

Questa non era altro che la semplificazione della rete generale a parametri Z:

\begin{center}
	\begin{circuitikz}
		\draw (-4, 1) -- (-3, 1) 
			to [ european resistor, l=$\overline{z_{11}}$, i=$I_1$] (-1, 1)
			to [ controlled voltage source, v<=$\overline{z_{12}} \dot{I}_2$ ] (-1, -1) 
			to [ short, i=$I_1$ ] (-3, -1)	
			-- (-4, -1);
			
		\draw (-4.6, 1) node[anchor=west] {$+$};
		\draw (-4.6, 0) node[anchor=west] {$V_1$};
		\draw (-4.6, -1) node[anchor=west] {$-$};

		\draw (6,1) -- (5, 1) 
			to [ european resistor, l_=$\overline{z_{12}}$, i_=$I_2$] (3, 1)
			to [ controlled voltage source, v_<=$\overline{z_{21}} \dot{I}_1$ ] (3, -1) 
			to [ short, i_=$I_2$ ] (5, -1)
			-- (6, -1);
	
		\draw (6.6, 1) node[anchor=east] {$+$};
		\draw (6.6, 0) node[anchor=east] {$V_2$};
		\draw (6.6, -1) node[anchor=east] {$-$};
		
		\node[rectangle, draw, minimum width = 8.5cm, minimum height = 4cm] (a) at (1,0) {};
	\end{circuitikz}
\end{center}

Notiamo che il potenziale ai due nodi in basso non è identico nel caso del circuito generale, mentre lo abbiamo posto come tale nel circuito reciproco.
Questa è una semplificazione che potremmo non voler fare.
Possiamo permettere potenziali diversi introducendo appunto un trasformatore a rapporto spire $n$ unitario:

\begin{center}
	\begin{circuitikz}
		\draw (-4, 1) -- (-3, 1) 
			to [ european resistor, l=$\overline{z_a}$, i=$I_1$] (0, 1)
			to [ european resistor, l=$\overline{z_b}$] (0, -1) 
			to [ short, i=$I_1$ ] (-3, -1)	
			-- (-4, -1);
			
		\draw (-4.6, 1) node[anchor=west] {$+$};
		\draw (-4.6, 0) node[anchor=west] {$V_1$};
		\draw (-4.6, -1) node[anchor=west] {$-$};

		\draw (3, 1) 
			to [ european resistor, l_=$\overline{z_c}$, i_=$I_2$] (0, 1);

		\draw (0, -1) to [ short, i_=$I_2$ ] (3, -1);
	
		\draw (7, 1) -- (5, 1);
		\draw (7, -1) -- (5, -1);

		\draw (3, 1) to [ inductor, l=$L_1$] (3, -1);
		\draw (5, 1) to [ inductor, l_=$L_2$] (5, -1);

		\draw (7.6, 1) node[anchor=east] {$+$};
		\draw (7.6, 0) node[anchor=east] {$V_2$};
		\draw (7.6, -1) node[anchor=east] {$-$};
	
		\draw (2.9,0.4) node {$\scriptscriptstyle\bullet$};
		\draw (4.9,0.4) node {$\scriptscriptstyle\bullet$};

		\node[rectangle, draw, minimum width = 8.5cm, minimum height = 4cm] (a) at (1.5,0) {};
	\end{circuitikz}
\end{center}

Assunto $L_1, L2, M \rightarrow +\infty$, cioè mutuo accoppiamento ideale.
Una modifica analoga si può operare all'equivalente reciproco a parametri Y, con gli stessi vantaggi.

\subsubsection{Adattamento di impedenza}

Poniamo il circuito magnetico considerato prima, dato da generatore di tensione che alimenta, attraverso un trasformatore, un carico $\overline{z_C}$:

\begin{center}
	\begin{circuitikz}
		\draw (0,0) to[ short, i=$i_1$] (1,0)
			to[ inductor , l=$N_1$] (1,-2)
			-- (0, -2);

		\draw (5,0) to[ short, i_=$i_2$] (3,0)
			to[ inductor , l_=$N_2$] (3,-2)
			-- (5, -2)
			to [ european resistor, l_=$Z_c$] (5, 0);

			\draw (0.9,-0.6) node {$\scriptscriptstyle\bullet$};
			\draw (2.9,-0.6) node {$\scriptscriptstyle\bullet$};

			\draw (0,-0.5) node[anchor=east] {$+$};
			\draw (0,-1.5) node[anchor=east] {$-$};

			\draw (0,-1) node[anchor=east] {$E_1$};
	\end{circuitikz}
\end{center}

Si avrà che, preso il circuito come una rete due porte:
$$
\dot{E}_2 = - \overline{z_C} \dot{I}_2
$$
e dal circuito magnetico:
$$
\dot{E}_1 = n \dot{E}_2 \Rightarrow \dot{E}_2 = \frac{\dot{E}_1}{n}
$$
$$
\dot{I}_1 = -\frac{1}{n} \dot{I_2} \Rightarrow \dot{I}_2 = -n \dot{I}_1
$$
quindi sostituendo:
$$
\frac{\dot{E}_1}{n} = - \overline{z_C} (-n \dot{I_1}) \Rightarrow \dot{E}_1 = \overline{z_C} n^2 \dot{I}_1
$$
cioè, l'impedenza di $\overline{z_C}$ su secondario viene "vista" dal primario come moltiplicata per un fattore pari al quadrato del rapporto spire:
$$
\overline{z_C}_1 = \overline{z_C}_2 \cdot n^2
$$

\subsubsection{Trasformatore reale}
Riprendiamo il trasformatore reale sollevando le ipotesi prese finora (resistenza degli avvolgimenti nulla, permeabilità $\rightarrow \infty$ e costante).
Il circuito equivalente non può più essere formato prendendo i soli due induttori in accoppiamento ideale.
Introduciamo quindi due resistenze $R_{1d}$ e $R_{2d}$ in entrata agli induttori in maniera tale da modellizzare la resistenza $R$ degli avvolgimenti (ipotesi 1), e altre due induttanze $L_{1d}$ e $L_{2d}$ per modellizzare i flussi dispersi in aria da un coefficiente $\mu_{fe}$ non ideale (cioè non tendente a $+\infty$, dall'ipotesi 2):

\begin{center}
	\begin{circuitikz}
		\draw (-4,0) to[ short, i=$i_1$] (-3,0)
			to [ resistor, l=$R_{1d}$] (-1,0)
			to [ inductor, l=$L_{1d}$] (1,0)
			to [ inductor , l=$N_1$] (1,-2)
			-- (-4, -2);
		
		\draw (3, 0) to [ resistor, l=$R_{2d}$] (5,0)
			to [ inductor, l=$L_{2d}$] (7,0) 
			to[ short, i<=$i_2$] (8,0);
			
			\draw (3,0) to[ inductor , l_=$N_2$] (3,-2)
			-- (8, -2);

			\draw (0.9,-0.6) node {$\scriptscriptstyle\bullet$};
			\draw (2.9,-0.6) node {$\scriptscriptstyle\bullet$};

			\draw (-4,-0.5) node[anchor=east] {$+$};
			\draw (-4,-1.5) node[anchor=east] {$-$};

			\draw (8,-0.5) node[anchor=west] {$+$};
			\draw (8,-1.5) node[anchor=west] {$-$};

			\draw (-4,-1) node[anchor=east] {$V_1$};
			\draw (8,-1) node[anchor=west] {$V_2$};

			\draw (0.7,-0.5) node[anchor=east] {$+$};
			\draw (0.7,-1.5) node[anchor=east] {$-$};

			\draw (3.3,-0.5) node[anchor=west] {$+$};
			\draw (3.3,-1.5) node[anchor=west] {$-$};

			\draw (0.7,-1) node[anchor=east] {$E_1$};
			\draw (3.3,-1) node[anchor=west] {$E_2$};
	\end{circuitikz}
\end{center}
dove si è distinto fra $V$ e $E$, rispettivamente la differenza di potenziale ai capi del trasformatore e la forza elettromtrice generata sui rispettivi avvolgimenti.

Notiamo che la permeabilità $\mu_{fe}$ non è nemmeno costante (come lo era dal'ipotesi 3): il rapporto fra intensità di campo magnetico e induzinoe magnetica, che finora avevamo descritto come:
$$
B = \mu \mu_{fe} H
$$
presenta in verità delle non linearità date dall'\textbf{isteresi} del materiale.
Si mostra un grafico della classica curva di isteresi, con l'intensità del campo magnetico $H$ alle ascisse e l'induzione magnetica corrispondente $B(H)$ alle ordinate, nelle due direzioni di magnetizzazione:

\begin{center}
\begin{tikzpicture}
    \begin{axis}[
        xlabel={$H$},
        ylabel={$B(H)$},
        domain=-10:10, % set the x range you want
				samples=100,
        grid=major, % add a grid
				ytick={-2, 0, 2},
				yticklabels={$-B_s$, $0$, $B_s$},
				xtick={-2.6, 0, 2.6},
				xticklabels={$-H_0$, $0$, $H_0$},
				ymin = -3, ymax = 3,
				width=15cm,
				height=7cm
    ]
   \addplot[
        blue,
        thick,
				domain=0:2.6
    ] {2 * tanh(2 * x) * (1 - exp(-x/0.4)^3)}; 

    \addplot[
        red,
        thick,
				domain=-2.6:2.6
    ] {2 * tanh(2 * (x - 1)};

    \addplot[
        red,
        thick,
				domain=-2.6:2.6
    ] {2 * tanh(2 * (x + 1)}; 
    \end{axis}
\end{tikzpicture}
\end{center}

Sul grafico chiamiamo $B_s$ la \textbf{magnetizzazione di saturazione}, cioè la magnetizzazione massima che il materiale è capace di ottenere, e con $H_0$ il \textbf{campo di saturazione}, cioè l'intensità del campo magnetico per cui il materiale raggiunge il punto di saturazione. 

\end{document}
