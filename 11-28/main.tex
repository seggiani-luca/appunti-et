
\documentclass[a4paper,11pt]{article}
\usepackage[a4paper, margin=8em]{geometry}

% usa i pacchetti per la scrittura in italiano
\usepackage[french,italian]{babel}
\usepackage[T1]{fontenc}
\usepackage[utf8]{inputenc}
\frenchspacing 

% usa i pacchetti per la formattazione matematica
\usepackage{amsmath, amssymb, amsthm, amsfonts}

% usa altri pacchetti
\usepackage{gensymb}
\usepackage{hyperref}
\usepackage{standalone}

% imposta il titolo
\title{Appunti Elettrotecnica}
\author{Luca Seggiani}
\date{2024}

% imposta lo stile
% usa helvetica
\usepackage[scaled]{helvet}
% usa palatino
\usepackage{palatino}
% usa un font monospazio guardabile
\usepackage{lmodern}

\renewcommand{\rmdefault}{ppl}
\renewcommand{\sfdefault}{phv}
\renewcommand{\ttdefault}{lmtt}

% disponi il titolo
\makeatletter
\renewcommand{\maketitle} {
	\begin{center} 
		\begin{minipage}[t]{.8\textwidth}
			\textsf{\huge\bfseries \@title} 
		\end{minipage}%
		\begin{minipage}[t]{.2\textwidth}
			\raggedleft \vspace{-1.65em}
			\textsf{\small \@author} \vfill
			\textsf{\small \@date}
		\end{minipage}
		\par
	\end{center}

	\thispagestyle{empty}
	\pagestyle{fancy}
}
\makeatother

% disponi teoremi
\usepackage{tcolorbox}
\newtcolorbox[auto counter, number within=section]{theorem}[2][]{%
	colback=blue!10, 
	colframe=blue!40!black, 
	sharp corners=northwest,
	fonttitle=\sffamily\bfseries, 
	title=~\thetcbcounter: #2, 
	#1
}

% disponi definizioni
\newtcolorbox[auto counter, number within=section]{definition}[2][]{%
	colback=red!10,
	colframe=red!40!black,
	sharp corners=northwest,
	fonttitle=\sffamily\bfseries,
	title=~\thetcbcounter: #2,
	#1
}

% U.D.M
\newcommand{\amp}{\ensuremath{\mathrm{A}}}
\newcommand{\volt}{\ensuremath{\mathrm{V}}}
\newcommand{\meter}{\ensuremath{\mathrm{m}}}
\newcommand{\second}{\ensuremath{\mathrm{s}}}
\newcommand{\farad}{\ensuremath{\mathrm{F}}}
\newcommand{\henry}{\ensuremath{\mathrm{H}}}
\newcommand{\siemens}{\ensuremath{\mathrm{S}}}

% circuiti
\usepackage{circuitikz}
\usetikzlibrary{babel}

% disegni
\usepackage{pgfplots}
\pgfplotsset{width=10cm,compat=1.9}

% disponi codice
\usepackage{listings}
\usepackage[table]{xcolor}

\lstdefinestyle{codestyle}{
		backgroundcolor=\color{black!5}, 
		commentstyle=\color{codegreen},
		keywordstyle=\bfseries\color{magenta},
		numberstyle=\sffamily\tiny\color{black!60},
		stringstyle=\color{green!50!black},
		basicstyle=\ttfamily\footnotesize,
		breakatwhitespace=false,         
		breaklines=true,                 
		captionpos=b,                    
		keepspaces=true,                 
		numbers=left,                    
		numbersep=5pt,                  
		showspaces=false,                
		showstringspaces=false,
		showtabs=false,                  
		tabsize=2
}

\lstdefinestyle{shellstyle}{
		backgroundcolor=\color{black!5}, 
		basicstyle=\ttfamily\footnotesize\color{black}, 
		commentstyle=\color{black}, 
		keywordstyle=\color{black},
		numberstyle=\color{black!5},
		stringstyle=\color{black}, 
		showspaces=false,
		showstringspaces=false, 
		showtabs=false, 
		tabsize=2, 
		numbers=none, 
		breaklines=true
}

\lstdefinelanguage{javascript}{
	keywords={typeof, new, true, false, catch, function, return, null, catch, switch, var, if, in, while, do, else, case, break},
	keywordstyle=\color{blue}\bfseries,
	ndkeywords={class, export, boolean, throw, implements, import, this},
	ndkeywordstyle=\color{darkgray}\bfseries,
	identifierstyle=\color{black},
	sensitive=false,
	comment=[l]{//},
	morecomment=[s]{/*}{*/},
	commentstyle=\color{purple}\ttfamily,
	stringstyle=\color{red}\ttfamily,
	morestring=[b]',
	morestring=[b]"
}

% disponi sezioni
\usepackage{titlesec}

\titleformat{\section}
	{\sffamily\Large\bfseries} 
	{\thesection}{1em}{} 
\titleformat{\subsection}
	{\sffamily\large\bfseries}   
	{\thesubsection}{1em}{} 
\titleformat{\subsubsection}
	{\sffamily\normalsize\bfseries} 
	{\thesubsubsection}{1em}{}

% disponi alberi
\usepackage{forest}

\forestset{
	rectstyle/.style={
		for tree={rectangle,draw,font=\large\sffamily}
	},
	roundstyle/.style={
		for tree={circle,draw,font=\large}
	}
}

% disponi algoritmi
\usepackage{algorithm}
\usepackage{algorithmic}
\makeatletter
\renewcommand{\ALG@name}{Algoritmo}
\makeatother

% disponi numeri di pagina
\usepackage{fancyhdr}
\fancyhf{} 
\fancyfoot[L]{\sffamily{\thepage}}

\makeatletter
\fancyhead[L]{\raisebox{1ex}[0pt][0pt]{\sffamily{\@title \ \@date}}} 
\fancyhead[R]{\raisebox{1ex}[0pt][0pt]{\sffamily{\@author}}}
\makeatother

\begin{document}
% sezione (data)
\section{Lezione del 28-11-24}

% stili pagina
\thispagestyle{empty}
\pagestyle{fancy}

% testo
\subsection{Antitrasformata con poli multipli}
Vediamo un'altra particolarità del processo di antitrasformazione, in particolare nel caso di \textbf{poli multipli}, cioè poli che compaiono con molteplicità $\geq 2$ al denominatore.
Prendiamo ad esempio la forma rapporto di polinomi:
$$
I(s) = \frac{s^2 + 2s + 3}{(s+1)^3}
$$
Avremo che, per esprimere i residui, non basterà il rapporto su $s+1$ al primo grado, ma servirà bensì:
$$
I(s) = \frac{A_1}{s + 1} + \frac{A_2}{(s+1)^2} + \frac{A_3}{(s+1)^2}
$$

Vorremo quindi portare le soluzioni con poli multipli nella forma generale:
$$
I(s) = \frac{A_1}{(s-p)} + \frac{A_2}{(s-p)^2} + ... + \frac{A_n}{(s-p)^n} = \sum_{i=1}^n \frac{A_i}{(s-p)^i}
$$
su cui si userà agevolmente la regola di antitrasformazione:
$$
\mathcal{L}^{-1} \left\{ \frac{A_i}{(s-p)^i} \right\} = \frac{A_i}{(i - 1)!} t^{i-1} e^{pt}
$$
da cui quindi:
$$
\mathcal{L}^{-1}(I(s)) = A_1 e^{pt} + \frac{A_2}{2} t e^{pt} + ... + \frac{A_n}{(n-1)!} t^{n-1} e^{pt} = \sum\limits_{i=1}^n  \frac{A_i}{(i-1)!} t^{i - 1} e^{pt}
$$ 

A questo punto basterà calcolare i residui, cosa che potremo fare applicando il \textbf{teorema dei residui ai poli multipli}:
$$
A_i = \lim_{s \rightarrow p} \frac{1}{(n - 1)!} \frac{\partial^{n-i} \, (s-p)^n \cdot I(s)}{\partial s^{n-i}}
$$

Riprendendo l'esempio:
$$
A_1 = \lim_{s \rightarrow -1} \frac{1}{2!} \frac{\partial^2 (s+1)^3 \frac{s^2+2s+3}{(s+1)^3} }{\partial s^2} = \lim_{s \rightarrow -1} \frac{1}{2} \frac{\partial^2}{\partial s^2} (s^2 + 2s + 3) = \lim_{s \rightarrow -1} \frac{1}{2} 2 = 1
$$
$$
A_2 = \lim_{s \rightarrow -1} \frac{1}{1!} \frac{\partial (s+1)^3 \frac{s^2+2s+3}{(s+1)^3} }{\partial s} = \lim_{s \rightarrow -1} 1 \cdot (2s + 2) = (2s + 2) \Big|_{s = -1} = 0
$$
$$
A_3 = \lim_{s \rightarrow -1} \frac{1}{0!} \frac{\partial^0 (s+1)^3 \frac{s^2+2s+3}{(s+1)^3} }{\partial s^0}
$$
che assumendo $\frac{\partial^0}{\partial s^0} = 1$ significa:
$$
A_3 = \lim_{s \rightarrow -1} = 1 \cdot (s^2 +2s +3) \Big|_{s=-1} = 2
$$

Si ha quindi il risultato finale:
$$
i(t) = \left( e^{-t} + t^2 e^{-t} \right) u(t)
$$


\end{document}
