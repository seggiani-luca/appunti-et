
\documentclass[a4paper,11pt]{article}
\usepackage[a4paper, margin=8em]{geometry}

% usa i pacchetti per la scrittura in italiano
\usepackage[french,italian]{babel}
\usepackage[T1]{fontenc}
\usepackage[utf8]{inputenc}
\frenchspacing 

% usa i pacchetti per la formattazione matematica
\usepackage{amsmath, amssymb, amsthm, amsfonts}

% usa altri pacchetti
\usepackage{gensymb}
\usepackage{hyperref}
\usepackage{standalone}

% imposta il titolo
\title{Appunti Elettrotecnica}
\author{Luca Seggiani}
\date{2024}

% imposta lo stile
% usa helvetica
\usepackage[scaled]{helvet}
% usa palatino
\usepackage{palatino}
% usa un font monospazio guardabile
\usepackage{lmodern}

\renewcommand{\rmdefault}{ppl}
\renewcommand{\sfdefault}{phv}
\renewcommand{\ttdefault}{lmtt}

% disponi il titolo
\makeatletter
\renewcommand{\maketitle} {
	\begin{center} 
		\begin{minipage}[t]{.8\textwidth}
			\textsf{\huge\bfseries \@title} 
		\end{minipage}%
		\begin{minipage}[t]{.2\textwidth}
			\raggedleft \vspace{-1.65em}
			\textsf{\small \@author} \vfill
			\textsf{\small \@date}
		\end{minipage}
		\par
	\end{center}

	\thispagestyle{empty}
	\pagestyle{fancy}
}
\makeatother

% disponi teoremi
\usepackage{tcolorbox}
\newtcolorbox[auto counter, number within=section]{theorem}[2][]{%
	colback=blue!10, 
	colframe=blue!40!black, 
	sharp corners=northwest,
	fonttitle=\sffamily\bfseries, 
	title=~\thetcbcounter: #2, 
	#1
}

% disponi definizioni
\newtcolorbox[auto counter, number within=section]{definition}[2][]{%
	colback=red!10,
	colframe=red!40!black,
	sharp corners=northwest,
	fonttitle=\sffamily\bfseries,
	title=~\thetcbcounter: #2,
	#1
}

% U.D.M
\newcommand{\amp}{\ensuremath{\mathrm{A}}}
\newcommand{\volt}{\ensuremath{\mathrm{V}}}
\newcommand{\meter}{\ensuremath{\mathrm{m}}}
\newcommand{\second}{\ensuremath{\mathrm{s}}}
\newcommand{\farad}{\ensuremath{\mathrm{F}}}
\newcommand{\henry}{\ensuremath{\mathrm{H}}}
\newcommand{\siemens}{\ensuremath{\mathrm{S}}}

% circuiti
\usepackage{circuitikz}
\usetikzlibrary{babel}

% disegni
\usepackage{pgfplots}
\pgfplotsset{width=10cm,compat=1.9}

% disponi codice
\usepackage{listings}
\usepackage[table]{xcolor}

\lstdefinestyle{codestyle}{
		backgroundcolor=\color{black!5}, 
		commentstyle=\color{codegreen},
		keywordstyle=\bfseries\color{magenta},
		numberstyle=\sffamily\tiny\color{black!60},
		stringstyle=\color{green!50!black},
		basicstyle=\ttfamily\footnotesize,
		breakatwhitespace=false,         
		breaklines=true,                 
		captionpos=b,                    
		keepspaces=true,                 
		numbers=left,                    
		numbersep=5pt,                  
		showspaces=false,                
		showstringspaces=false,
		showtabs=false,                  
		tabsize=2
}

\lstdefinestyle{shellstyle}{
		backgroundcolor=\color{black!5}, 
		basicstyle=\ttfamily\footnotesize\color{black}, 
		commentstyle=\color{black}, 
		keywordstyle=\color{black},
		numberstyle=\color{black!5},
		stringstyle=\color{black}, 
		showspaces=false,
		showstringspaces=false, 
		showtabs=false, 
		tabsize=2, 
		numbers=none, 
		breaklines=true
}

\lstdefinelanguage{javascript}{
	keywords={typeof, new, true, false, catch, function, return, null, catch, switch, var, if, in, while, do, else, case, break},
	keywordstyle=\color{blue}\bfseries,
	ndkeywords={class, export, boolean, throw, implements, import, this},
	ndkeywordstyle=\color{darkgray}\bfseries,
	identifierstyle=\color{black},
	sensitive=false,
	comment=[l]{//},
	morecomment=[s]{/*}{*/},
	commentstyle=\color{purple}\ttfamily,
	stringstyle=\color{red}\ttfamily,
	morestring=[b]',
	morestring=[b]"
}

% disponi sezioni
\usepackage{titlesec}

\titleformat{\section}
	{\sffamily\Large\bfseries} 
	{\thesection}{1em}{} 
\titleformat{\subsection}
	{\sffamily\large\bfseries}   
	{\thesubsection}{1em}{} 
\titleformat{\subsubsection}
	{\sffamily\normalsize\bfseries} 
	{\thesubsubsection}{1em}{}

% disponi alberi
\usepackage{forest}

\forestset{
	rectstyle/.style={
		for tree={rectangle,draw,font=\large\sffamily}
	},
	roundstyle/.style={
		for tree={circle,draw,font=\large}
	}
}

% disponi algoritmi
\usepackage{algorithm}
\usepackage{algorithmic}
\makeatletter
\renewcommand{\ALG@name}{Algoritmo}
\makeatother

% disponi numeri di pagina
\usepackage{fancyhdr}
\fancyhf{} 
\fancyfoot[L]{\sffamily{\thepage}}

\makeatletter
\fancyhead[L]{\raisebox{1ex}[0pt][0pt]{\sffamily{\@title \ \@date}}} 
\fancyhead[R]{\raisebox{1ex}[0pt][0pt]{\sffamily{\@author}}}
\makeatother

\begin{document}
% sezione (data)
\section{Lezione del 04-10-24}

% stili pagina
\thispagestyle{empty}
\pagestyle{fancy}

% testo
\subsection{Metodo del tableau}
Vediamo un metodo generale per la risoluzione completa di un circuito, detto metodo delle correnti di ramo, o \textit{del tablau}.
Di base, si hanno $r$ incognite per $r$ rami, ergo le correnti che passano nei rispettivi rami.
Applichiamo quindi il seguente algoritmo:

\begin{enumerate}
	\item Dare un nome a tutti i nodi ($A_1, A_2, ..., A_n$);
	\item Dare un nome a tutte le correnti di ramo ($I_1, I_2, ..., I_n$).
		Si noti che in questo passaggio il verso delle correnti è puramente arbitrario: il segno positivo o negativo della corrente trovata ci darà il verso rispetto al riferimento (associato o meno) scelto.
	\item Scrivere $n - 1$ equazioni per $n - 1$ nodi applicando la prima legge di Kirchoff.
		Si scelgono $n - 1$ equazioni perché, essendo la corrente vincolata dalla legge di Kirchoff, la $n$-esima equazione sarà linearmente dipendente alle altre, ergo ridondante in informazioni riguardo al circuito.
	\item Scrivere $r - n + 1$ equazioni con la seconda legge di Kirchoff.
		La prima maglia si sceglie a caso, mentre le successive maglie si scelgono cancellando un ramo della maglia scelta al passaggio scorso.
		Questa procedura è, ancora una volta, necessaria per evitare la dipendenza lineare delle equazioni trovate.
\end{enumerate}

Seguendo questi passaggi, si trova un sistema lineare in $r$ variabili con $r$ equazioni, che sappiamo essere determinato e risolvibile come:
$$ Ax = b \Rightarrow x = A^{-1} b $$

\par\smallskip

Prendiamo in esempio il seguente circuito:

\begin{center}
\begin{circuitikz}
	\draw (0,0)
		-- (0,1.5)
		to[R, l=$R_1$] (5,1.5)
		-- (5, 0);
	\draw (0,0)
		to[R, l=$R_2$] (2.5, 0)
		to[R, l=$R_3$] (5, 0);
	\draw (0,0)
		to[R, l=$R_4$] (0, -3);
	\draw (2.5,0)
		to[R, l=$R_5$] (2.5, -3);
	\draw (5,-3)
		to[ european voltage source, v=$E_2$] (5, 0);
	\draw (5, -3)
		to[R, l=$R_6$] (2.5, -3)
		to[ european voltage source, v=$E_1$] (0, -3);
\end{circuitikz}
\end{center}

Eseguiamo i passi in ordine.
\begin{enumerate}
	\item Innanzitutto, si danno i nomi $A$, $B$, $C$ e $D$ ai nodi del circuito:

\begin{center}
\begin{circuitikz}
	\draw (0,0)
		-- (0,1.5)
		to[R, l=$R_1$] (5,1.5)
		-- (5, 0);
	\draw (0,0)
		to[R, l=$R_2$] (2.5, 0)
		to[R, l=$R_3$] (5, 0);
	\draw (0,0)
		to[R, l=$R_4$] (0, -3);
	\draw (2.5,0)
		to[R, l=$R_5$] (2.5, -3);
	\draw (5,-3)
		to[ european voltage source, v=$E_2$] (5, 0);
	\draw (5, -3)
		to[R, l=$R_6$] (2.5, -3)
		to[ european voltage source, v=$E_1$] (0, -3);

	\draw (0,0) node[circ] {};
	\draw (0,0) node[left] {A};
	
	\draw (2.5,0) node[circ] {};
	\draw (2.5,0) node[above] {B};

	\draw (5,0) node[circ] {};
	\draw (5,0) node[right] {C};

	\draw (2.5,-3) node[circ] {};
	\draw (2.5,-3) node[below] {D};
\end{circuitikz}
\end{center}

	\item Quindi, si danno i nomi $I_1$, $I_2$, $I_3$, $I_4$, $I_5$ e $I_6$, e i versi di percorrenza alle correnti sui rispettivi rami:

\begin{center}
\begin{circuitikz}
	\draw (0,0)
		-- (0,1.5)
		to[R, l=$R_1$, i=$I_1$] (5,1.5)
		-- (5, 0);
	\draw (0,0)
		to[R, l=$R_2$, i_ = $I_2$] (2.5, 0);
		
	\draw (5, 0) to[R, l_=$R_3$, i = $I_3$] (2.5, 0);
	
	\draw (0,0)
		to[R, l=$R_4$, i = $I_4$] (0, -3);
	\draw (2.5,0)
		to[R, l=$R_5$, i = $I_5$] (2.5, -3);
	\draw (5,-3)
		to[ european voltage source, v=$E_2$] (5, 0);
	\draw (2., -3)
		to[R, l_=$R_6$, i = $I_6$] (5, -3);
	\draw (2.5, -3)
		to[ european voltage source, v=$E_1$] (0, -3);

	\draw (0,0) node[circ] {};
	\draw (0,0) node[left] {A};
	
	\draw (2.5,0) node[circ] {};
	\draw (2.5,0) node[above] {B};

	\draw (5,0) node[circ] {};
	\draw (5,0) node[right] {C};

	\draw (2.5,-3) node[circ] {};
	\draw (2.5,-3) node[below] {D};
\end{circuitikz}
\end{center}

\item A questo punto possiamo applicare la prima legge di Kirchoff su $n-1$ nodi, diciamo $A$, $B$ e $C$:
$$
A: I_1 + I_2 + I_4 = 0 
$$
$$
B: I_2 + I_3 = I_5 
$$
$$
C: I_1 + I_6 = I_3
$$
La legge di Kirchoff applicata al nodo $D$ qui sarebbe inutile, in quanto fornirebbe:
$$
D: I_5 + I_4 = I_6
$$
che potevamo già ottenere sostituendo la $C$ alla $B$:
$$
C: I_2 + I_3 = I_5 \Rightarrow I_2 + I_1 + I_6 = I_5  
$$
e a questo punto usando la $A$ come $I_1 + I_2 = -I_4$:
$$
\Rightarrow -I_4 + I_6 = I_5 \Rightarrow I_5 + I_4 = I_6
$$

\item Completiamo il sistema introducendo $r - n + 1 = 3$ equazioni ricavate applicando la seconda legge di Kirchoff in senso antiorario.
	Dovremo prendere i segni delle cadute di potenziale tenendo conto della direzione della corrente scelta su quel ramo.
	
	Prendiamo innanzitutto la maglia in basso a sinistra:
	$$ -E_1 + R_5 I_5 + R_2 I _2 - R_4 I_4 = 0 $$
	ed eliminiamo il ramo di corrente $I_4$.
	Quindi prendiamo la maglia formata dalla parte superiore e dalla parte in basso a destra del circuito:
	$$ E_2 + R_1 I_1 - R_2 I_2 - R_5 I_5 - R_6 I_6 = 0 $$
	ed eliminiamo il ramo di corrente $I_5$ o $I_6$ (è irrilevante per il prossimo passaggio), in quanto prenderemo come ultima la maglia in alto:
	$$ R_1 I_1 - R_2 I_2 + R_3 I_3 = 0 $$
\end{enumerate}

Conclusi questi passaggi, abbiamo ricavato il sistema lineare:
\[
	\begin{cases}
I_1 + I_2 + I_4 = 0 \\ 
I_2 + I_3 = I_5 \\
I_1 + I_6 = I_3 \\
-E_1 + R_5 I_5 + R_2 I _2 - R_4 I_4 = 0 \\
E_2 + R_1 I_1 - R_2 I_2 - R_5 I_5 - R_6 I_6 = 0 \\
R_1 I_1 - R_2 I_2 + R_3 I_3 = 0
	\end{cases}
\]

Possiamo ricavare le matrici $A$, $x$ e $b$ del sistema, ed esprimerlo come $Ax = b$:
\[
\underbrace{\begin{pmatrix}
		1 & 1 & 0 & 1 & 0 & 0 \\ 
		0 & 1 & 1 & 0 & -1 & 0 \\ 
		1 & 0 & -1 & 0 & 0 & 1 \\ 
		0 & R_2 & 0 & -R_4 & R_5 & 0 \\ 
		R_1 & -R_2 & 0 & 0 & -R_5 & -R_6 \\ 
		R_1 & -R_2 & R_3 & 0 & 0 & 0 \\ 
\end{pmatrix}}_{\mathbf{A}} 
\underbrace{\begin{pmatrix}
		I_1 \\
		I_2 \\
		I_3 \\
		I_4 \\
		I_5 \\
		I_6 \\
\end{pmatrix}}_{\mathbf{x}} 
= 
\underbrace{\begin{pmatrix}
		0 \\ 
		0 \\ 
		0 \\ 
		E_1 \\ 
		-E_2 \\ 
		0
\end{pmatrix}}_{\mathbf{b}}
\]

Il sistema si risolve con qualsiasi metodo di risoluzione di sistemi lineari.

\par\smallskip

Notiamo come il metodo del tableu ci permette di notare due fatti già visti sui circuiti.

Innanzitutto, abbiamo che fra i termini noti $b$ compaiono solo le correnti e i voltaggi dei generatori indipendenti.
Senza di questi, il sistema sarebbe di equazioni omogenee, e quindi ammetterebbe unica soluzione $0$.
Questa affermazione è equivalente a quella già vista: sono i generatori indipendenti a portare energia dentro il circuito.

Possiamo poi dire che il il metodo del tableau è compatibile col principio di sovrapposizione.
Presi separatamente i generatori, infatti, avremo $x', x'', ..., x^n$ soluzioni date da $b', b'', ..., b^n$ vettori $b$, ergo:
$$ x = x' + x'' + ... + x^n = A^{-1} b' + A^{-1} b'' + ... + A^{-1} b^n$$

Intuitivamente, la matrice $A$ rappresenta la risposta del circuito a diversi stimoli $b', b'', ..., b^n$.
Nell'esempio, questi sono i generatori indipendenti:

$$
b':
\begin{pmatrix}
		0 \\ 
		0 \\ 
		0 \\ 
		E_1 \\ 
		0 \\ 
		0
\end{pmatrix}, \quad 
b'':
\begin{pmatrix}
		0 \\ 
		0 \\ 
		0 \\ 
		0 \\ 
		-E_2 \\ 
		0
\end{pmatrix}
$$

\subsubsection{Circuiti con generatori di corrente}
Quando si applica il metodo del tableau ad un circuito con generatori di corrente, bisogna notare che questi riducono il numero di incognite (le correnti) del sistema, cioè portano il numero di correnti di ramo a $r - N_I$ dove $N_I$ è il numero di generatori di corrente.
In seguito, bisogna fare attenzione a tagliare preliminarmente i rami contenenti generatori di correnti prima di eseguire il punto 4) dell'algoritmo, visto che il potenziale su quei rami potrebbe essere qualsiasi. 

\par\smallskip 

Prendiamo in esempio il seguente circuito:

\begin{center}
\begin{circuitikz}
	\draw (0,0)
		-- (0,1.5)
		to[R, l=$R_3$] (5,1.5)
		-- (5, 0);
	\draw (0,0)
		to[ european current source, i = $I$] (2.5, 0)
		to[short] (5, 0);
	\draw (0,0)
		to[R, l=$R_1$] (0, -3);
	\draw (2.5,0)
		to[R, l=$R_2$] (2.5, -3);
	\draw (5,-3)
		to[ european voltage source, v=$E$] (5, 0);
	\draw (5, -3)
		to[short] (2.5, -3)
		to[short] (0, -3);
\end{circuitikz}
\end{center}

Si ricava che i nodi e le correnti sono le seguenti, facendo attenzione alla disposizione di $B$, che viene detto \textbf{macronodo}.
Un nodo del genere è utile in quanto il potenziale di ogni suo punto è effettivamente equivalente.

\begin{center}
\begin{circuitikz}
	\draw (0,0)
		-- (0,1.5)
		to[R, l=$R_3$, i = $I_3$] (5,1.5)
		-- (5, 0);
	\draw (0,0)
		to[ european current source, i = $I$] (2.5, 0)
		to[short] (5, 0);
	\draw (0,0)
		to[R, l=$R_1$, i = $I_1$] (0, -3);
	\draw (2.5,0)
		to[R, l=$R_2$, i = $I_2$] (2.5, -3);
	\draw (5,-3)
		to[ european voltage source, v=$E$, i = $I_4$] (5, 0);
	\draw (5, -3)
		to[short] (2.5, -3)
		to[short] (0, -3);

	\draw (0,0) node[circ] {};
	\draw (0,0) node[left] {A};
	
	\draw (2.5,0) node[circ] {};
	\draw (3.75,0) node[above] {B};

	\draw (5,0) node[circ] {};

	\draw[line width=1.25mm] (2.5, 0) -- (5,0);

	\draw (2.5,-3) node[circ] {};
	\draw (2.5,-3) node[below] {C};

\end{circuitikz}
\end{center}

Calcoliamo quindi la prima legge di Kirchoff sui nodi $A$ e $B$:
$$
A: I + I_1 + I_3 = 0
$$
$$
B: I + I_3 = I_2 + I_4
$$

Adesso calcoliamo la seconda legge di Kirchoff sulle maglie.
Abbiamo detto che dobbiamo ignorare i rami contenenti generatori di corrente.
Questo è vero perché provando, ad esempio, il ramo in basso a sinistra, avremmo:
$$
-V_I - R_1 I_1 + R_2 I_2 = 0
$$
dove $V_I$ è la differenza di potenziale sul generatore di corrente, che sappiamo può essere qualsiasi.

Usiamo quindi altre due maglie: quella data dalle maglie in alto e in basso a sinistra, e quella in basso a destra:
$$
R_3 I_3 - R_1 I_1 + R_2 I_2 = 0
$$
$$
-E + R_2 I_2 = 0
$$

Ricaviamo il sistema:
\[
	\begin{cases}
I + I_1 + I_3 = 0 \\ 
I + I_3 = I_2 + I_4 \\ 
R_3 I_3 - R_1 I_1 + R_2 I_2 = 0 \\
-E + R_2 I_2 = 0
	\end{cases}
\]

ovvero:

\[
\begin{pmatrix}
		1 & 0 & 1 & 0 \\ 
		0 & -1 & 1 & -1 \\ 
		-R_1 & R_2 & R_3 & 0 \\ 
		0 & R_2 & 0 & 0
\end{pmatrix} 
\begin{pmatrix}
		I_1 \\
		I_2 \\
		I_3 \\
		I_4 \\
\end{pmatrix} 
= 
\begin{pmatrix}
		-I \\ 
		-I \\ 
		0 \\ 
		E \\ 
\end{pmatrix}
\]

\subsubsection{Circuiti con generatori pilotati}
I generatori pilotati possono esprimere nuove incognite, in quanto la grandezza pilota è una corrente o una tensione ricavata su qualche ramo del circuito.
Inoltre, bisogna comunque tagliare fuori dal circuito o meno la sorgente in base al fatto che essa sia di voltaggio o di corrente, come avevamo già visto per i generatori indipendenti.

\par\smallskip

Prendiamo in esempio il seguente circuito:

\begin{center}
	\begin{circuitikz}[scale=1.2]
    \draw (2,-2) 
				-- (0,-2) 
				to[R, l=$R_1$] (0,1)
				-- (2, 1);	
		\draw (2,1)
			to [R, l = $R_2$] (4, 1)
			to [R, l = $R_3$] (6, 1)
			to [R, l = $R_4$] (8, 1)
			to [ european voltage source, v=$E_2$] (8, -2)
			-- (0, -2);

		\draw (2, -2)
			to [ european voltage source, v_=$E_1$] (2, 1);
		\draw (4, 1)
			to [ european current source, i=$J_1$] (4, -2);
		\draw (6, -2)
			to [ european controlled current source, i=$\alpha I_4$] (6, 1);
\end{circuitikz}
\end{center}

dove compare una sorgente pilotata in corrente di corrente.
Per quanto riguarda le maglie, la considereremo come una qualsiasi sorgente indipendente di corrente.
La grandezza pilota, invece, è una corrente ($I_4$) comune: fosse stata un voltaggio, avremmo dovuto esplicitarla con una variabile ed un equazione a sé (ad esempio $v_{CD} = R_4 I_4)$).

Disponiamo quindi nodi e cariche (si noti anche qui il macronodo $D$):

\begin{center}
	\begin{circuitikz}[scale=1.2]
    \draw (2,-2) 
				-- (0,-2) 
				to[R, l=$R_1$, i = $I_1$] (0,1)
				-- (2, 1);	
		\draw (2,1)
			to [R, l = $R_2$, i = $I_2$] (4, 1)
			to [R, l = $R_3$, i = $I_3$] (6, 1)
			to [R, l = $R_4$, i = $I_4$] (8, 1)
			to [ european voltage source, v=$E_2$] (8, -2)
			-- (0, -2);

		\draw (2, -2)
			to [ european voltage source, v_=$E_1$, i=$I_5$] (2, 1);
		\draw (4, 1)
			to [ european current source, i=$J_1$] (4, -2);
		\draw (6, -2)
			to [ european controlled current source, i=$\alpha I_4$] (6, 1);


		\draw (2,1) node[circ] {};
		\draw (2,1) node[above] {A};

		\draw (4,1) node[circ] {};
		\draw (4,1) node[above] {B};

		\draw (6,1) node[circ] {};
		\draw (6,1) node[above] {C};

		\draw (2,-2) node[circ] {};
		\draw (6,-2) node[circ] {};
		\draw[line width=1.25mm] (2, -2) -- (6,-2);

		\draw (4,-2) node[below] {D};
\end{circuitikz}
\end{center}

Calcoliamo con la prima legge di Kirchoff:
$$
A: I_1 = I_2 + I_5
$$
$$
B: I_2 = I_3 + J_1
$$
$$
D: I_5 + J_1 + I_4 = I_1 + \alpha I_4
$$
e con la seconda, applicata alla maglia sinistra e alle tre rimanenti (ricordando che i generatori di corrente vanno ignorati):
$$
E_1 + R_1 = 0
$$
$$
-E_1 - E_2 + R_4 I_4 + R_3 I_3 + R_2 I_2 = 0
$$
da cui il sistema:
\[
	\begin{cases}			
I_1 = I_2 + I_5 \\ 
I_2 = I_3 + J_1 \\ 
I_5 + J_1 + I_4 = I_1 + \alpha I_4 \\
E_1 + R_1 I_1 = 0 \\ 
-E_1 - E_2 + R_4 I_4 + R_3 I_3 + R_2 I_2 = 0 
	\end{cases}
\]

ovvero:

\[
\begin{pmatrix}
	1 & -2 & 0 & 0 & 1 \\ 
	0 & 1 & -1 & 0 & 0 \\ 
	-1 & 0 & 0 & 1 - \alpha & 1 \\ 
	R_1 & 0 & 0 & 0 & 0 \\
	0 & R_2 & R_3 & R_4 & 0
\end{pmatrix} 
\begin{pmatrix}
		I_1 \\
		I_2 \\
		I_3 \\
		I_4 \\
		I_5
\end{pmatrix} 
= 
\begin{pmatrix}
	0 \\ 
	J_1 \\ 
	-J_1 \\ 
	-E_1 \\ 
	E_1 + E_2
\end{pmatrix}
\]

\end{document}
