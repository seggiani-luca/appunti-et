
\documentclass[a4paper,11pt]{article}
\usepackage[a4paper, margin=8em]{geometry}

% usa i pacchetti per la scrittura in italiano
\usepackage[french,italian]{babel}
\usepackage[T1]{fontenc}
\usepackage[utf8]{inputenc}
\frenchspacing 

% usa i pacchetti per la formattazione matematica
\usepackage{amsmath, amssymb, amsthm, amsfonts}

% usa altri pacchetti
\usepackage{gensymb}
\usepackage{hyperref}
\usepackage{standalone}

% imposta il titolo
\title{Appunti Elettrotecnica}
\author{Luca Seggiani}
\date{2024}

% imposta lo stile
% usa helvetica
\usepackage[scaled]{helvet}
% usa palatino
\usepackage{palatino}
% usa un font monospazio guardabile
\usepackage{lmodern}

\renewcommand{\rmdefault}{ppl}
\renewcommand{\sfdefault}{phv}
\renewcommand{\ttdefault}{lmtt}

% disponi il titolo
\makeatletter
\renewcommand{\maketitle} {
	\begin{center} 
		\begin{minipage}[t]{.8\textwidth}
			\textsf{\huge\bfseries \@title} 
		\end{minipage}%
		\begin{minipage}[t]{.2\textwidth}
			\raggedleft \vspace{-1.65em}
			\textsf{\small \@author} \vfill
			\textsf{\small \@date}
		\end{minipage}
		\par
	\end{center}

	\thispagestyle{empty}
	\pagestyle{fancy}
}
\makeatother

% disponi teoremi
\usepackage{tcolorbox}
\newtcolorbox[auto counter, number within=section]{theorem}[2][]{%
	colback=blue!10, 
	colframe=blue!40!black, 
	sharp corners=northwest,
	fonttitle=\sffamily\bfseries, 
	title=~\thetcbcounter: #2, 
	#1
}

% disponi definizioni
\newtcolorbox[auto counter, number within=section]{definition}[2][]{%
	colback=red!10,
	colframe=red!40!black,
	sharp corners=northwest,
	fonttitle=\sffamily\bfseries,
	title=~\thetcbcounter: #2,
	#1
}

% U.D.M
\newcommand{\amp}{\ensuremath{\mathrm{A}}}
\newcommand{\volt}{\ensuremath{\mathrm{V}}}
\newcommand{\meter}{\ensuremath{\mathrm{m}}}
\newcommand{\second}{\ensuremath{\mathrm{s}}}
\newcommand{\farad}{\ensuremath{\mathrm{F}}}
\newcommand{\henry}{\ensuremath{\mathrm{H}}}
\newcommand{\siemens}{\ensuremath{\mathrm{S}}}

% circuiti
\usepackage{circuitikz}
\usetikzlibrary{babel}

% disegni
\usepackage{pgfplots}
\pgfplotsset{width=10cm,compat=1.9}

% disponi codice
\usepackage{listings}
\usepackage[table]{xcolor}

\lstdefinestyle{codestyle}{
		backgroundcolor=\color{black!5}, 
		commentstyle=\color{codegreen},
		keywordstyle=\bfseries\color{magenta},
		numberstyle=\sffamily\tiny\color{black!60},
		stringstyle=\color{green!50!black},
		basicstyle=\ttfamily\footnotesize,
		breakatwhitespace=false,         
		breaklines=true,                 
		captionpos=b,                    
		keepspaces=true,                 
		numbers=left,                    
		numbersep=5pt,                  
		showspaces=false,                
		showstringspaces=false,
		showtabs=false,                  
		tabsize=2
}

\lstdefinestyle{shellstyle}{
		backgroundcolor=\color{black!5}, 
		basicstyle=\ttfamily\footnotesize\color{black}, 
		commentstyle=\color{black}, 
		keywordstyle=\color{black},
		numberstyle=\color{black!5},
		stringstyle=\color{black}, 
		showspaces=false,
		showstringspaces=false, 
		showtabs=false, 
		tabsize=2, 
		numbers=none, 
		breaklines=true
}

\lstdefinelanguage{javascript}{
	keywords={typeof, new, true, false, catch, function, return, null, catch, switch, var, if, in, while, do, else, case, break},
	keywordstyle=\color{blue}\bfseries,
	ndkeywords={class, export, boolean, throw, implements, import, this},
	ndkeywordstyle=\color{darkgray}\bfseries,
	identifierstyle=\color{black},
	sensitive=false,
	comment=[l]{//},
	morecomment=[s]{/*}{*/},
	commentstyle=\color{purple}\ttfamily,
	stringstyle=\color{red}\ttfamily,
	morestring=[b]',
	morestring=[b]"
}

% disponi sezioni
\usepackage{titlesec}

\titleformat{\section}
	{\sffamily\Large\bfseries} 
	{\thesection}{1em}{} 
\titleformat{\subsection}
	{\sffamily\large\bfseries}   
	{\thesubsection}{1em}{} 
\titleformat{\subsubsection}
	{\sffamily\normalsize\bfseries} 
	{\thesubsubsection}{1em}{}

% disponi alberi
\usepackage{forest}

\forestset{
	rectstyle/.style={
		for tree={rectangle,draw,font=\large\sffamily}
	},
	roundstyle/.style={
		for tree={circle,draw,font=\large}
	}
}

% disponi algoritmi
\usepackage{algorithm}
\usepackage{algorithmic}
\makeatletter
\renewcommand{\ALG@name}{Algoritmo}
\makeatother

% disponi numeri di pagina
\usepackage{fancyhdr}
\fancyhf{} 
\fancyfoot[L]{\sffamily{\thepage}}

\makeatletter
\fancyhead[L]{\raisebox{1ex}[0pt][0pt]{\sffamily{\@title \ \@date}}} 
\fancyhead[R]{\raisebox{1ex}[0pt][0pt]{\sffamily{\@author}}}
\makeatother

\begin{document}
% sezione (data)
\section{Lezione del 25-10-24}

% stili pagina
\thispagestyle{empty}
\pagestyle{fancy}

% testo
\subsubsection{Potenza in circuiti a regime sinusoidale}
Vediamo come si studia la potenza nei circuiti in corrente alternata.
Avevamo definito la potenza come:
$$ p(t) = v(t)i(t) = R i^2(t) $$

A regime costante, abbiamo che $v(t)$ e $i(t)$ sono costanti, come lo è (ovviamente) $R$, ergo $p(t)$ ha un valore definito e positivo.

A regime sinusoidale, invece, abbiamo una forma del tipo:
$$
i(t) = I_M \cdot \sin(\omega t)
$$
assumendo $\phi = 0$.

Abbiamo quindi che in $p(t) = R i^2(t)$ non c'è modo di eliminare la dipendenza temporale.
Notiamo però che vale comunque $p(t) \geq 0$ dal quadrato a $i(t)$.

\subsubsection{Valore efficace}
Potremmo qundi chiederci se è il circuito a regime costante o quello a regime sinusoidale a dissipare più energia sul solito intervallo di tempo $\Delta T$.
Sul costante, abbiamo l'integrale:
$$
W(t) = \int_0^T RI^2 dt = RI^2 \cdot T 
$$
mentre sul sinusoidale vale:
$$
W(t) = \int_0^T Ri^2(t) dt = R \int_0^T i^2(t)dt
$$

Eguagliamo quindi le due:
$$
RI_{eff}^2 T = R \int_0^T i^2(t)dt
$$
dove abbiamo chiamato $I \rightarrow I_{eff}$ per dare una definizione preliminare di \textbf{corrente efficace}, cioè la corrente in regime costante che dissipa la stessa potenza della corrispondente corrente in regime sinusoidale.
Si ha quindi:
$$
I_{eff} = \sqrt{\frac{1}{T}\int_0^T i^2(t)dt}
$$

Vediamo che questa definizione è generica:
\begin{definition}{Valore efficace}
	Definiamo il valore efficace $X_{eff}$ di una grandezza $x(t)$ in regime sinusoidale su un intervallo $T$ come:
$$
X_{eff} = \sqrt{\frac{1}{T}\int_0^T x^2(t)dt}
$$
\end{definition}

Possiamo quindi sostituire la definizione che avevamo dato di $i(t)$, ottenendo:
$$
I_{eff} = \sqrt{\frac{1}{T} \int_0^T I_M^2 \cdot \sin^2(\omega t) dt} = \sqrt{\frac{1}{T}I_M^2 \int_0^T \left( 1 - \cos^2(\omega t) \right) dt} = \sqrt{\frac{1}{T} \int_0^T \frac{1-\cos(2\omega t)}{2} dt}
$$
$$
= \sqrt{\frac{1}{T} I_M^2 \left( \int_0^T \frac{1}{2}dt - \int_0^T \frac{\cos(2 \omega t)}{2}dt \right)}
= \sqrt{\frac{1}{T} I_M^2 \left( \frac{1}{2}\int_0^T 1\cdot dt - \frac{1}{2} \int_0^T \cos(2 \omega t)dt \right)}
=\sqrt{\frac{I_M^2}{2}} 
$$
da cui abbiamo il valore:
$$
I_{eff} = \frac{I_M}{\sqrt{2}}
$$
Notiamo come $I_{eff} \geq 0$, dal fatto che $I_M$ corrente di picco è $\geq 0$.

Ad esempio, si dice che la rete elettrica funziona a 220 volts. 
Questo non è altro che il valore efficace della corrente. Si ha infatti:
$$
V_{M} = 220 \cdot \sqrt{2} \approx 311 \mathrm{V}
$$

\subsubsection{Calcolo della potenza}
Definiamo quindi la potenza su circuiti a regime periodico sinusoidale.
Definiamo la \textbf{potenza istantanea}:
$$
p(t) = i(t)v(t)
$$
Avevamo definito $i(t)$ e $v(t)$ come:
\[
	\begin{cases}
		i(t) = I_M \sin(\omega t) \\ 
		v(t) = V_M \sin(\omega t + \phi)
	\end{cases}
\]
dove $\phi$ è la \textbf{fase dell'impedenza}, da $\phi_V = \phi + \phi_i$, $\dot{V} = \bar{Z} \dot{I}$.

Possiamo quindi sostituire:
$$
p(t) = V_M \sin(\omega t + \phi) \cdot I_M \sin(\omega t) = V_M I_M \sin(\omega t) \sin(\omega t + \phi) 
$$
$$
= V_M I_M \sin(\omega t)\left( \sin(\omega t) \cos(\phi) + \cos{\omega t} \sin{\phi} \right) = V_M I_M \left( \sin^2 (\omega t) \cos (\phi) + \sin(\omega t) \cos(\omega t) \sin(\phi) \right)
$$
$$
= V_M I_M \left( \left( 1 - \cos^2(\omega t) \right) \cos(\phi) + \sin(\omega t)\cos(\omega t)\sin(\phi) \right)
$$
$$
= V_M I_M \left( \frac{1 - \cos(2 \omega t)}{2}\cos(\phi) + \frac{\sin(2 \omega t){2}}\sin(\phi) \right) = \frac{V_M I_M}{2} \left( \left( 1 - \cos(2 \omega t) \cos (\phi) + \sin(2\omega t) \sin(\phi) \right) \right)
$$
da cui:
$$
p(t) = \frac{V_M I_M}{2}\left( 1 - \cos(2\omega t) \right) \cos(\phi) + \frac{V_M I_M}{2} \sin(2\omega t)\sin(\phi)
$$

Il primo termine viene detto \textbf{potenza attiva istantanea}, mentre il secondo viene detto \textbf{potenza reattiva istantanea}.
Componenti come i generatori generano potenza attiva istantanea, mentre componenti come gli induttori e i capacitori generano potenza reattiva istantanea.

\subsubsection{Potenza attiva}
Sarebbe comodo avere una misura di potenza che non dipende dal tempo, cioè un \textbf{valore medio}.
Definiamo quindi:
\begin{definition}{Potenza attiva}
	Definiamo la potenza attiva $P$, sulla base della potenza istantanea $p(t)$, come la media integrale:
	$$
		P = \frac{1}{T} \int_0^T p(t) dt
	$$
\end{definition}

Sostituendo quanto trovato prima, abbiamo:
$$
P = \frac{1}{T} \int_0^T p(t) dt = \frac{1}{T} \frac{V_M I_M}{2} \int_0^T \left( 1 - \cos(2\omega t) \right) \cos(\phi) +  \sin(2\omega t)\sin(\phi)  dt
$$
$$
= \frac{1}{T} \frac{V_M I_M}{2} \left( \int_0^T \cos(\phi) dt + \int_0^T \cos(2\omega t) \cos(\phi) dt + \int_0^t \sin(2\omega t)\sin(\phi)  dt \right) 
$$
dove l'unico integrale che resta è $\int_0^T \cos(\phi) dt = T \cos(\phi)$, ergo: 
$$
P= \frac{V_M I_M}{2} \cos(\phi) = \frac{V_{eff} \sqrt{2} \cdot I_{eff} \sqrt{2}}{2} \cos{\phi} = V_{eff}I_{eff} \cos{\phi}
$$
Si nota la comodità di usare i valori efficaci: non compare $\sqrt{2}$ al denominatore.

Notiamo poi che avevamo, dall'impedenza e l'ammettenza:
$$
\dot V = \bar{Z} \dot{I}  \Rightarrow P = Z I^2 \cos{\phi} = R I^2 = Y V^2 \cos(\phi) = G V^2 
$$

\subsubsection{Potenza reattiva}
Definiamo quindi la \textbf{potenza reattiva}:
\begin{definition}{Potenza reattiva}
	Definiamo la potenza reattiva $Q$ come il massimo della potenza reattiva istantanea.
\end{definition}

Da quanto calcolato prima, si ha:
$$
Q = \max{\frac{V_M I_M}{2} \sin(2\omega t)\sin(\phi)} = \frac{V_M I_M}{2} \sin(\phi)
$$
La potenza reattiva si misura in $[\mathrm{VAR}]$, cioè \textit{Volt Ampere Reattivi}.
Si ha anche qui, riguardo il triangolo delle impedenze:
$$
Q = \frac{V_{eff} \sqrt{2} \cdot I_{eff} \sqrt{2}}{2} \sin(\phi) = V_{eff}I_{eff} \sin(\phi) = ZI^2 \sin(\phi) = XI^2 = YV^2 \sin(\phi) = - BV^2
$$

\subsubsection{Potenza apparente}
Definiamo infine la potenza \textbf{apparente}:
\begin{definition}{Potenza apparente}
	Definiamo la potenza reattiva $S$ come il valore massimo raggiungibile della potenza, cioè:
	$$
	S = \frac{V_M I_M}{2}
	$$
\end{definition}

Come prima, 
$$
S = \frac{V_M I_M}{2} = \frac{V_{eff} \sqrt{2} \cdot I_{eff} \sqrt{2}}{2} = VI = ZI^2 = Y V^2
$$

L'unita della misura della potenza apparente è il $[\mathrm{VA}]$, cioè \textit{Volt Ampere}.

\par\medskip

Prendiamo in esempio il circuito:
\begin{center}
	\begin{circuitikz}
		\draw (0,0) to [ voltage source, v=$V_2(t)$] (4, 0);
		\draw (4,-3) to [ voltage source, v=$V_1(t)$] (4, 0);
		\draw (4,-3) to [ inductor, l=$L$] (8,-3)
			to [ resistor, l=$R$] (8, 0)
			to [ voltage source, v=$V_3$, i =$i(t)$] (4, 0);
		\draw (0,0) to [ capacitor, l=$C$] (0,-3)
			to [ inductor, l=$L$] (4,-3);
	\end{circuitikz}
\end{center}

con i generatori guidati dalla legge:
\[
	\begin{cases}	
		V_1(t) = 10 \sqrt{2} \sin\left(1000 t\right)V \\ 
		V_2(t) = 20 \sqrt{2} \sin\left(1000t + \frac{\pi}{2}\right)V \\ 
		V_3(t) = 30 \sqrt{2} \sin\left(1000t + \pi\right)V
	\end{cases}
\]

Ci chiediamo quanto valga la corrente $i(t)$ sul generatore di tensione $V_3$, in direzione opposta al contrassegno.

Prima di tutto, portiamo le equazioni dei generatori in forma fasoriale, usando il valore efficace del voltaggio.
Usare il valore efficace o il valore proprio del voltaggio è indifferente, ma abbiamo che i calcoli si semplificano se si usa il primo.
Abbiamo quindi:
\[
	\begin{cases}
		\dot{V}_1	= 10 e^{j\cdot0} = 10 V\\ 
		\dot{V}_2 = 20 e^{j\frac{\pi}{2}} = 20j V \\ 
		\dot{V}_3 = 30 e^{j \pi} = -30 V
	\end{cases}
\]

Possiamo quindi usare il metodo delle correnti di nodo. Poniamo il nodo:
\begin{center}
	\begin{circuitikz}
		\draw (0,0) to [ voltage source, v=$V_2(t)$] (4, 0);
		\draw (4,-3) to [ voltage source, v=$V_1(t)$] (4, 0);
		\draw (4,-3) to [ inductor, l=$L$] (8,-3)
			to [ resistor, l=$R$] (8, 0)
			to [ voltage source, v=$V_3$, i =$i(t)$] (4, 0);
		\draw (0,0) to [ capacitor, l=$C$] (0,-3)
			to [ inductor, l=$L$] (4,-3);
		
		\node at(4,0) {$\bullet$};
		\node at(4,0)[anchor=south] {A};
		\node at(4,-3)[ground] {}
	\end{circuitikz}
\end{center}

A questo punto, $V_A = V_1 = 10V$, e si puà applicare Kirchoff alla maglia a destra:
$$
-\dot{V}_1 + \dot{V}_3 + R\dot{I} + \dot{I} \cdot j \omega L = 0 \Rightarrow i(t) = \frac{\dot{V}_1 - \dot{V}_3}{R + j\omega L}
$$

Poniamo che, risolvendo con qualche valore di $R$ e $C$, si ha $\dot{I} = 2 - 2j$.
A questo punto, possiamo ritrovare il valore di $i(t)$ effettivo attraverso il modulo di $\dot{I}$ e l'angolo:
$$
i(t) = \sqrt{2^2 + 2^2} \sqrt2 \cdot \sin\left(1000t - \frac{\pi}{4}\right) = 4 \sin\left(1000t - \frac{\pi}{t}\right)
$$
dove si è reintrodotto $\sqrt{2}$ per ritornare al valore di picco. 

\end{document}
