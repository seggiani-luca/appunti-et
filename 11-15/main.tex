
\documentclass[a4paper,11pt]{article}
\usepackage[a4paper, margin=8em]{geometry}

% usa i pacchetti per la scrittura in italiano
\usepackage[french,italian]{babel}
\usepackage[T1]{fontenc}
\usepackage[utf8]{inputenc}
\frenchspacing 

% usa i pacchetti per la formattazione matematica
\usepackage{amsmath, amssymb, amsthm, amsfonts}

% usa altri pacchetti
\usepackage{gensymb}
\usepackage{hyperref}
\usepackage{standalone}

% imposta il titolo
\title{Appunti Elettrotecnica}
\author{Luca Seggiani}
\date{2024}

% imposta lo stile
% usa helvetica
\usepackage[scaled]{helvet}
% usa palatino
\usepackage{palatino}
% usa un font monospazio guardabile
\usepackage{lmodern}

\renewcommand{\rmdefault}{ppl}
\renewcommand{\sfdefault}{phv}
\renewcommand{\ttdefault}{lmtt}

% disponi il titolo
\makeatletter
\renewcommand{\maketitle} {
	\begin{center} 
		\begin{minipage}[t]{.8\textwidth}
			\textsf{\huge\bfseries \@title} 
		\end{minipage}%
		\begin{minipage}[t]{.2\textwidth}
			\raggedleft \vspace{-1.65em}
			\textsf{\small \@author} \vfill
			\textsf{\small \@date}
		\end{minipage}
		\par
	\end{center}

	\thispagestyle{empty}
	\pagestyle{fancy}
}
\makeatother

% disponi teoremi
\usepackage{tcolorbox}
\newtcolorbox[auto counter, number within=section]{theorem}[2][]{%
	colback=blue!10, 
	colframe=blue!40!black, 
	sharp corners=northwest,
	fonttitle=\sffamily\bfseries, 
	title=~\thetcbcounter: #2, 
	#1
}

% disponi definizioni
\newtcolorbox[auto counter, number within=section]{definition}[2][]{%
	colback=red!10,
	colframe=red!40!black,
	sharp corners=northwest,
	fonttitle=\sffamily\bfseries,
	title=~\thetcbcounter: #2,
	#1
}

% U.D.M
\newcommand{\amp}{\ensuremath{\mathrm{A}}}
\newcommand{\volt}{\ensuremath{\mathrm{V}}}
\newcommand{\meter}{\ensuremath{\mathrm{m}}}
\newcommand{\second}{\ensuremath{\mathrm{s}}}
\newcommand{\farad}{\ensuremath{\mathrm{F}}}
\newcommand{\henry}{\ensuremath{\mathrm{H}}}
\newcommand{\siemens}{\ensuremath{\mathrm{S}}}

% circuiti
\usepackage{circuitikz}
\usetikzlibrary{babel}

% disegni
\usepackage{pgfplots}
\pgfplotsset{width=10cm,compat=1.9}

% disponi codice
\usepackage{listings}
\usepackage[table]{xcolor}

\lstdefinestyle{codestyle}{
		backgroundcolor=\color{black!5}, 
		commentstyle=\color{codegreen},
		keywordstyle=\bfseries\color{magenta},
		numberstyle=\sffamily\tiny\color{black!60},
		stringstyle=\color{green!50!black},
		basicstyle=\ttfamily\footnotesize,
		breakatwhitespace=false,         
		breaklines=true,                 
		captionpos=b,                    
		keepspaces=true,                 
		numbers=left,                    
		numbersep=5pt,                  
		showspaces=false,                
		showstringspaces=false,
		showtabs=false,                  
		tabsize=2
}

\lstdefinestyle{shellstyle}{
		backgroundcolor=\color{black!5}, 
		basicstyle=\ttfamily\footnotesize\color{black}, 
		commentstyle=\color{black}, 
		keywordstyle=\color{black},
		numberstyle=\color{black!5},
		stringstyle=\color{black}, 
		showspaces=false,
		showstringspaces=false, 
		showtabs=false, 
		tabsize=2, 
		numbers=none, 
		breaklines=true
}

\lstdefinelanguage{javascript}{
	keywords={typeof, new, true, false, catch, function, return, null, catch, switch, var, if, in, while, do, else, case, break},
	keywordstyle=\color{blue}\bfseries,
	ndkeywords={class, export, boolean, throw, implements, import, this},
	ndkeywordstyle=\color{darkgray}\bfseries,
	identifierstyle=\color{black},
	sensitive=false,
	comment=[l]{//},
	morecomment=[s]{/*}{*/},
	commentstyle=\color{purple}\ttfamily,
	stringstyle=\color{red}\ttfamily,
	morestring=[b]',
	morestring=[b]"
}

% disponi sezioni
\usepackage{titlesec}

\titleformat{\section}
	{\sffamily\Large\bfseries} 
	{\thesection}{1em}{} 
\titleformat{\subsection}
	{\sffamily\large\bfseries}   
	{\thesubsection}{1em}{} 
\titleformat{\subsubsection}
	{\sffamily\normalsize\bfseries} 
	{\thesubsubsection}{1em}{}

% disponi alberi
\usepackage{forest}

\forestset{
	rectstyle/.style={
		for tree={rectangle,draw,font=\large\sffamily}
	},
	roundstyle/.style={
		for tree={circle,draw,font=\large}
	}
}

% disponi algoritmi
\usepackage{algorithm}
\usepackage{algorithmic}
\makeatletter
\renewcommand{\ALG@name}{Algoritmo}
\makeatother

% disponi numeri di pagina
\usepackage{fancyhdr}
\fancyhf{} 
\fancyfoot[L]{\sffamily{\thepage}}

\makeatletter
\fancyhead[L]{\raisebox{1ex}[0pt][0pt]{\sffamily{\@title \ \@date}}} 
\fancyhead[R]{\raisebox{1ex}[0pt][0pt]{\sffamily{\@author}}}
\makeatother

\begin{document}
% sezione (data)
\section{Lezione del 15-11-24}

% stili pagina
\thispagestyle{empty}
\pagestyle{fancy}

% testo
\subsection{Rappresentazione a parametri T}
Vediamo un ultimo tipo di parametrizzazione, la \textbf{parametrizzazione T}.
Le equazioni di rappresentazione sono:
\[
	\begin{cases}
		\dot{V}_1	= \overline{A} \dot{V}_2 + \overline{B} ( - \dot{I}_2 ) \\ 
		\dot{V}_2	= \overline{C} \dot{V}_2 + \overline{D} ( - \dot{I}_2 ) \\ 
	\end{cases}
\]

Notiamo come le grandezze indipendenti qui sono sempre sia tensioni e correnti, ma non in alternanza come nella parametrizzazione h.
Inoltre, notiamo che il termine $\dot{I}_2$ compare con segno negato.

Potremmo pensare di calcolare i parametri T come segue:
$$
T:
\begin{pmatrix}
	\overline{A} = \frac{\dot{V_1}}{\dot{V_2}} \Big|_{-\dot{I}_2 = 0}	& ... \\
\end{pmatrix}
$$
ma notiamo che è impossibile calcolare, ad esempio $\overline{A}$, in quanto si chiede di mettere sia un generatore che un aperto alla porta 2 # riporta tutti
Scriviamo quindi una matrice del tipo:
$$
T:
\begin{pmatrix}
	\frac{1}{\overline{A}} = \frac{\dot{V}_2}{\dot{V}_1} \Big|_{- \dot{I}_2 = 0} & \frac{1}{\overline{B}} = -\frac{\dot{I}_2}{\dot{V}_1} \Big|_{- \dot{V}_2 = 0} \\
	\frac{1}{\overline{C}} = \frac{\dot{V}_2}{\dot{I}_1} \Big|_{- \dot{I}_2 = 0} & \frac{1}{\overline{D}} = -\frac{\dot{I}_2}{\dot{I}_1} \Big|_{- \dot{V}_2 = 0} \\
\end{pmatrix}
$$

Riscriviamo la seconda equazione di rappresentazione come:
$$
\dot{I}_1 = \overline{C} \dot{V}_2 + \overline{D} (-\dot{I}_2) \Rightarrow -\dot{I}_2 = \frac{1}{\overline{D}} \dot{I}_1 - \frac{\overline{C}}{\overline{D}}\dot{V}_2
$$
Un possibile circuito equivalente di una parametrizzazione T sarà quindi il seguente:
\begin{center}
	\begin{circuitikz}
		\draw (-4, 1) -- (-3, 1) 
		to [ controlled voltage source, v<=$\overline{A}\dot{V}_2$, i>=$I_1$] (-1, 1)
		to [ controlled voltage source, v<=$-\overline{B} \dot{I}_2$ ] (-1, -1) 
			to [ short, i=$I_1$ ] (-3, -1)	
			-- (-4, -1);
			
		\draw (-4.6, 1) node[anchor=west] {$+$};
		\draw (-4.6, 0) node[anchor=west] {$V_1$};
		\draw (-4.6, -1) node[anchor=west] {$-$};

		\draw (6,1) to [ short, i=$I_2$] (5, 1) 
			to [ short] (3, 1)
			to [ controlled current source, cI_<=$\frac{\dot{I}_1}{\overline{D}}$ ] (3, -1) 
			to [ short] (5, -1)
			-- (6, -1);
	
		\draw (6.6, 1) node[anchor=east] {$+$};
		\draw (6.6, 0) node[anchor=east] {$V_2$};
		\draw (6.6, -1) node[anchor=east] {$-$};
		
		\draw (4, 1) to [ european resistor, l=$\frac{\overline{D}}{\overline{C}}$] (4, -1);

		\node[rectangle, draw, minimum width = 8.5cm, minimum height = 5cm] (a) at (1,0) {};
	\end{circuitikz}
\end{center}

# ce sempre un errore sui circuiti in ammettenza

\subsubsection{Condizioni di reciprocità}
Troviamo quindi le condizioni di reciprocità.
Come avevamo fatto per i parametri h, riportiamoci in parameri Z, ad esempio partendo dalla seconda equazione:
$$
\dot{V}_2 = \frac{1}{\overline{C}} \dot{I}_1 + \frac{\overline{D}}{\overline{C}} \dot{I}_2
$$
da cui $\frac{1}{\overline{C}} = \overline{z_{21}}$ e $\frac{\overline{D}}{\overline{C}} = \overline{z_{22}}$, e analogamente per la prima:
$$
\dot{V}_1 = \overline{A} \left( \frac{1}{\overline{C}} \dot{I}_1 + \frac{\overline{D}}{\overline{C}} \dot{I}_2 \right) + \overline{B} \left( - \dot{I}_2 \right) = \dot{I}_1 \frac{1}{\overline{C}} + \dot{I}_2 \left( \frac{\overline{D}}{\overline{C}} - \overline{B} \right)
$$
da cui # finisci

\subsection{Circuiti a due porte in serie}
Poniamo di avere due circuiti a due porte $A$ e $B$, percorsi rispettivamente dalle correnti $I_{1a}$ e $I_{2a}$, e $I_{1b}$ e $I_{2b}$. collegati fra di loro in \textbf{serie}, cioè su cui scorre la \textit{stessa corrente}:

\begin{center}
	\begin{circuitikz}
		\node[rectangle, draw, minimum width = 2cm, minimum height = 2cm] (a) at (0,0) {};
		\draw (-2, 0.6) to [ short, i=$I_{1a}$] (-1, 0.6);
		\draw(-1, -0.6) to [ short, i=$I_{1a}$ ] (-2, -0.6);	
	
		\draw (-2.6, 0.6) node[anchor=west] {$+$};
		\draw (-2.6, 0) node[anchor=west] {$V_{1a}$};
		\draw (-2.6, -0.6) node[anchor=west] {$-$};
		
		\draw (2, 0.6) to [ short, i_=$I_{2a}$] (1, 0.6);
		\draw(1, -0.6) to [ short, i_=$I_{2a}$ ] (2, -0.6);	
	
		\draw (2.6, 0.6) node[anchor=east] {$+$};
		\draw (2.6, 0) node[anchor=east] {$V_{2b}$};
		\draw (2.6, -0.6) node[anchor=east] {$-$};


		\node[rectangle, draw, minimum width = 2cm, minimum height = 2cm] (a) at (0,-3) {};
		\draw (-2, -2.4) to [ short, i=$I_{1b}$] (-1, -2.4);
		\draw(-1, -3.6) to [ short, i=$I_{1b}$ ] (-2, -3.6);	
	
		\draw (-2.6, -2.4) node[anchor=west] {$+$};
		\draw (-2.6, -3) node[anchor=west] {$V_{1b}$};
		\draw (-2.6, -3.6) node[anchor=west] {$-$};
		
		\draw (2, -2.4) to [ short, i_=$I_{2b}$] (1, -2.4);
		\draw(1, -3.6) to [ short, i_=$I_{2b}$ ] (2, -3.6);	
	
		\draw (2.6, -2.4) node[anchor=east] {$+$};
		\draw (2.6, -3) node[anchor=east] {$V_{2b}$};
		\draw (2.6, -3.6) node[anchor=east] {$-$};

		\node at (0, 0) {$Z_A$};
		\node at (0, -3) {$Z_B$};

		\draw (2, -2.4) -- (2, -0.6);
		\draw (-2, -2.4) -- (-2, -0.6);
	\end{circuitikz}
\end{center}

# nota che questa è sempre una porta

Calcoliamo la caduta di potenziale sulle due porte ($1$ e $2$):
$$
\begin{pmatrix}
	\dot{V}_{1s} \\ \dot{V}_{2s}
\end{pmatrix}
=
\begin{pmatrix}
	\dot{V}_{1a} \\ \dot{V}_{2a}
\end{pmatrix}
+
\begin{pmatrix}
	\dot{V}_{1b} \\ \dot{V}_{2b}
\end{pmatrix}
=
\overline{Z_{a}} 
\begin{pmatrix}
	\dot{I}_{1a} \\ \dot{I}_{1b}
\end{pmatrix}
+
\overline{Z_{b}}
\begin{pmatrix}
	\dot{I}_{2a} \\ \dot{I}_{2b}
\end{pmatrix}
=
\left( \overline{Z_a} + \overline{Z_b} \right) 
\begin{pmatrix}
	\dot{I_{1s}} \\ \dot{I_{2s}}
\end{pmatrix}
$$
che è quello che ci aspettavamo: la matrice dei parametri Z di due circuiti a due porte in serie è data dalla \textit{somma} delle matrici dei parametri Z dei singoli circuiti.

\subsubsection{Circuiti a due porte in parallelo}
Poniamo adesso di avere due circuiti a due porte $A$ e $B$, percorsi sempre dalle correnti $I_{1a}$ e $I_{2a}$, e $I_{1b}$ e $I_{2b}$. collegati fra di loro in \textbf{parallelo}, cioè che si trovano allo \textit{stesso potenziale}:

\begin{center}
	\begin{circuitikz}
		\node[rectangle, draw, minimum width = 2cm, minimum height = 2cm] (a) at (0,0) {};
		\draw (-2, 0.6) to [ short, i=$I_{1a}$] (-1, 0.6);
		\draw(-1, -0.6) to [ short, i=$I_{1a}$ ] (-2, -0.6);	
	
		\draw (-2.6, 0.6) node[anchor=west] {$+$};
		\draw (-2.6, 0) node[anchor=west] {$V_{1a}$};
		\draw (-2.6, -0.6) node[anchor=west] {$-$};
		
		\draw (2, 0.6) to [ short, i_=$I_{2a}$] (1, 0.6);
		\draw(1, -0.6) to [ short, i_=$I_{2a}$ ] (2, -0.6);	
	
		\draw (2.6, 0.6) node[anchor=east] {$+$};
		\draw (2.6, 0) node[anchor=east] {$V_{2b}$};
		\draw (2.6, -0.6) node[anchor=east] {$-$};


		\node[rectangle, draw, minimum width = 2cm, minimum height = 2cm] (a) at (0,-3) {};
		\draw (-2, -2.4) to [ short, i=$I_{1b}$] (-1, -2.4);
		\draw(-1, -3.6) to [ short, i=$I_{1b}$ ] (-2, -3.6);	
	
		\draw (-2.6, -2.4) node[anchor=west] {$+$};
		\draw (-2.6, -3) node[anchor=west] {$V_{1b}$};
		\draw (-2.6, -3.6) node[anchor=west] {$-$};
		
		\draw (2, -2.4) to [ short, i_=$I_{2b}$] (1, -2.4);
		\draw(1, -3.6) to [ short, i_=$I_{2b}$ ] (2, -3.6);	
	
		\draw (2.6, -2.4) node[anchor=east] {$+$};
		\draw (2.6, -3) node[anchor=east] {$V_{2b}$};
		\draw (2.6, -3.6) node[anchor=east] {$-$};

		\node at (0, 0) {$Y_A$}; \TODO % non so il perchè delle Y. poi finisci circuito parallelo
		\node at (0, -3) {$Y_B$};

	\end{circuitikz}
\end{center}

# come prima è sempre una porta

Calcoliamo quindi la corrente che attraversa le porte:
$$
\begin{pmatrix}
	\dot{I}_{1s} \\ \dot{I}_{2s} 
\end{pmatrix}
=
\begin{pmatrix}
	\dot{I}_{1a} \\ \dot{I}_{2a}
\end{pmatrix}
+
\begin{pmatrix}
	\dot{I}_{1b} \\ \dot{I}_{2b}
\end{pmatrix}
=
\overline{Y_a} 
\begin{pmatrix}
	\dot{V}_{1a} \\ \dot{V}_{2a}
\end{pmatrix}
+
\overline{Y_b}
\begin{pmatrix}
	\dot{V}_{1b} \\ \dot{V}_{2b}
\end{pmatrix}
=
(\overline{Y_a} + \overline{Y_b})
\begin{pmatrix}
	\dot{I}_{1s} \\ \dot{I}_{2s}
\end{pmatrix}
$$

# riguarda formule, poi ha preso un circuito in serie e l ha risolto coi parametri Y (Y_A^-1 + Y_B^_1)^_1

\subsection{Circuiti a due porte in cascata}
Nel collegamento \textbf{a cascata}, l'uscita di una porta và direttamente in ingresso a una seconda porta, cioè:
\begin{center}
	\begin{circuitikz}
		\node[rectangle, draw, minimum width = 2cm, minimum height = 2cm] (a) at (0,0) {};
		\draw (-2, 0.6) to [ short, i=$I_{1a}$] (-1, 0.6);
		\draw(-1, -0.6) to [ short, i=$I_{1a}$ ] (-2, -0.6);	
	
		\draw (-2.6, 0.6) node[anchor=west] {$+$};
		\draw (-2.6, 0) node[anchor=west] {$V_{1a}$};
		\draw (-2.6, -0.6) node[anchor=west] {$-$};
		
		\draw (2, 0.6) to [ short, i_=$I_{2a}$] (1, 0.6);
		\draw(1, -0.6) to [ short, i_=$I_{2a}$ ] (2, -0.6);	
	
		\draw (2.6, 0.6) node[anchor=east] {$+$};
		\draw (2.6, 0) node[anchor=east] {$V_{2b}$};
		\draw (2.6, -0.6) node[anchor=east] {$-$};


		\node[rectangle, draw, minimum width = 2cm, minimum height = 2cm] (a) at (0,-3) {};
		\draw (-2, -2.4) to [ short, i=$I_{1b}$] (-1, -2.4);
		\draw(-1, -3.6) to [ short, i=$I_{1b}$ ] (-2, -3.6);	
	
		\draw (-2.6, -2.4) node[anchor=west] {$+$};
		\draw (-2.6, -3) node[anchor=west] {$V_{1b}$};
		\draw (-2.6, -3.6) node[anchor=west] {$-$};
		
		\draw (2, -2.4) to [ short, i_=$I_{2b}$] (1, -2.4);
		\draw(1, -3.6) to [ short, i_=$I_{2b}$ ] (2, -3.6);	
	
		\draw (2.6, -2.4) node[anchor=east] {$+$};
		\draw (2.6, -3) node[anchor=east] {$V_{2b}$};
		\draw (2.6, -3.6) node[anchor=east] {$-$};

		\node at (0, 0) {$T_A$}; \TODO % finisci circuito cascata 
		\node at (0, -3) {$T_B$};

	\end{circuitikz}
\end{center}

Non dobbiamo dimostrare che anche questo circuito è una porta, in quanto si prende come ingresso l'ingresso della porta $T_A$ ($I_{1c})$ , $V_{1c}$) e come uscita l'uscita della porta $T_B$ ($I_{2c})$ , $V_{2c}$).

Possiamo quindi esprimere queste relazioni fra le i circuiti interni e le porte esterne come segue:
$$
\begin{pmatrix}
	\dot{V}_{1c} \\ \dot{I}_{1c}
\end{pmatrix}
=
\begin{pmatrix}
	\dot{V}_{1a} \\ \dot{I}_{1a}
\end{pmatrix}
=
\overline{T_a}
\begin{pmatrix}
	\dot{V}_{2a} \\ - \dot{I}_{2a}
\end{pmatrix}
=
\overline{T_a}
\begin{pmatrix}
	\dot{V}_{1b} \\ \dot{I}_{1b}
\end{pmatrix}
=
(\overline{T_a} \overline{T_b}) 
\begin{pmatrix}
	\dot{V}_{2b} \\ -\dot{I}_{2b}
\end{pmatrix}
(\overline{T_a} \overline{T_b})
\begin{pmatrix}
	\dot{V}_{2c} \\ -\dot{I}_{2c}
\end{pmatrix}
$$
da cui:
$$
\begin{pmatrix}
	\dot{V}_{1c} \\ \dot{I}_{1v} 
\end{pmatrix}
=
(\overline{T_a} \overline{T_b})
\begin{pmatrix}
	\dot{V}_{2c} \\ -\dot{I}_{2c}
\end{pmatrix}
$$

# questo sopra boh

\subsection{Collegamento ibrido serie/parallelo}
Attraverso le porte abbiamo a disposizione un ulteriore tipo di collegamento, il cosiddetto collegamento \textbf{ibrido}, cioè dove una coppia di porte viene connessa in serie e l'altra coppia in parallelo:
\begin{center}
	\begin{circuitikz}
		\node[rectangle, draw, minimum width = 2cm, minimum height = 2cm] (a) at (0,0) {};
		\draw (-2, 0.6) to [ short, i=$I_{1a}$] (-1, 0.6);
		\draw(-1, -0.6) to [ short, i=$I_{1a}$ ] (-2, -0.6);	
	
		\draw (-2.6, 0.6) node[anchor=west] {$+$};
		\draw (-2.6, 0) node[anchor=west] {$V_{1a}$};
		\draw (-2.6, -0.6) node[anchor=west] {$-$};
		
		\draw (2, 0.6) to [ short, i_=$I_{2a}$] (1, 0.6);
		\draw(1, -0.6) to [ short, i_=$I_{2a}$ ] (2, -0.6);	
	
		\draw (2.6, 0.6) node[anchor=east] {$+$};
		\draw (2.6, 0) node[anchor=east] {$V_{2b}$};
		\draw (2.6, -0.6) node[anchor=east] {$-$};


		\node[rectangle, draw, minimum width = 2cm, minimum height = 2cm] (a) at (0,-3) {};
		\draw (-2, -2.4) to [ short, i=$I_{1b}$] (-1, -2.4);
		\draw(-1, -3.6) to [ short, i=$I_{1b}$ ] (-2, -3.6);	
	
		\draw (-2.6, -2.4) node[anchor=west] {$+$};
		\draw (-2.6, -3) node[anchor=west] {$V_{1b}$};
		\draw (-2.6, -3.6) node[anchor=west] {$-$};
		
		\draw (2, -2.4) to [ short, i_=$I_{2b}$] (1, -2.4);
		\draw(1, -3.6) to [ short, i_=$I_{2b}$ ] (2, -3.6);	
	
		\draw (2.6, -2.4) node[anchor=east] {$+$};
		\draw (2.6, -3) node[anchor=east] {$V_{2b}$};
		\draw (2.6, -3.6) node[anchor=east] {$-$};

		\node at (0, 0) {$h_A$}; \TODO % finisci circuito ibrido 
		\node at (0, -3) {$h_B$};

	\end{circuitikz}
\end{center}

Per questo tipo di circuiti potremmo dire:
$$
\begin{pmatrix}
	\dot{V}_{1m} \\ \dot{I}_{2m}
\end{pmatrix}
=
\begin{pmatrix}
	\dot{V}_{1a} \\ \dot{I}_{2a}
\end{pmatrix}
+
\begin{pmatrix}
	\dot{V}_{1b} \\ \dot{I}_{2b}
\end{pmatrix}
=
\overline{h_a}
\begin{pmatrix}
	\dot{I}_{1a} \\ \dot{V}_{2a}
\end{pmatrix}
+
\overline{h_b}
\begin{pmatrix}
	\dot{I}_{1b} \\ \dot{V_{2b}}
\end{pmatrix}
=
(\overline{h_a} + \overline{h_b})
\begin{pmatrix}
	\dot{I}_{1m} \\ \dot{V}_{2m}
\end{pmatrix}
$$

\end{document}
# chiudi tutto il discorso che parametrizzazioni diverse aiutano a risolvere circuiti diversi
