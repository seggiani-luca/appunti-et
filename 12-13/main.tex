
\documentclass[a4paper,11pt]{article}
\usepackage[a4paper, margin=8em]{geometry}

% usa i pacchetti per la scrittura in italiano
\usepackage[french,italian]{babel}
\usepackage[T1]{fontenc}
\usepackage[utf8]{inputenc}
\frenchspacing 

% usa i pacchetti per la formattazione matematica
\usepackage{amsmath, amssymb, amsthm, amsfonts}

% usa altri pacchetti
\usepackage{gensymb}
\usepackage{hyperref}
\usepackage{standalone}

% imposta il titolo
\title{Appunti Elettrotecnica}
\author{Luca Seggiani}
\date{2024}

% imposta lo stile
% usa helvetica
\usepackage[scaled]{helvet}
% usa palatino
\usepackage{palatino}
% usa un font monospazio guardabile
\usepackage{lmodern}

\renewcommand{\rmdefault}{ppl}
\renewcommand{\sfdefault}{phv}
\renewcommand{\ttdefault}{lmtt}

% disponi il titolo
\makeatletter
\renewcommand{\maketitle} {
	\begin{center} 
		\begin{minipage}[t]{.8\textwidth}
			\textsf{\huge\bfseries \@title} 
		\end{minipage}%
		\begin{minipage}[t]{.2\textwidth}
			\raggedleft \vspace{-1.65em}
			\textsf{\small \@author} \vfill
			\textsf{\small \@date}
		\end{minipage}
		\par
	\end{center}

	\thispagestyle{empty}
	\pagestyle{fancy}
}
\makeatother

% disponi teoremi
\usepackage{tcolorbox}
\newtcolorbox[auto counter, number within=section]{theorem}[2][]{%
	colback=blue!10, 
	colframe=blue!40!black, 
	sharp corners=northwest,
	fonttitle=\sffamily\bfseries, 
	title=~\thetcbcounter: #2, 
	#1
}

% disponi definizioni
\newtcolorbox[auto counter, number within=section]{definition}[2][]{%
	colback=red!10,
	colframe=red!40!black,
	sharp corners=northwest,
	fonttitle=\sffamily\bfseries,
	title=~\thetcbcounter: #2,
	#1
}

% U.D.M
\newcommand{\amp}{\ensuremath{\mathrm{A}}}
\newcommand{\volt}{\ensuremath{\mathrm{V}}}
\newcommand{\meter}{\ensuremath{\mathrm{m}}}
\newcommand{\second}{\ensuremath{\mathrm{s}}}
\newcommand{\farad}{\ensuremath{\mathrm{F}}}
\newcommand{\henry}{\ensuremath{\mathrm{H}}}
\newcommand{\siemens}{\ensuremath{\mathrm{S}}}

% circuiti
\usepackage{circuitikz}
\usetikzlibrary{babel}

% disegni
\usepackage{pgfplots}
\pgfplotsset{width=10cm,compat=1.9}

% disponi codice
\usepackage{listings}
\usepackage[table]{xcolor}

\lstdefinestyle{codestyle}{
		backgroundcolor=\color{black!5}, 
		commentstyle=\color{codegreen},
		keywordstyle=\bfseries\color{magenta},
		numberstyle=\sffamily\tiny\color{black!60},
		stringstyle=\color{green!50!black},
		basicstyle=\ttfamily\footnotesize,
		breakatwhitespace=false,         
		breaklines=true,                 
		captionpos=b,                    
		keepspaces=true,                 
		numbers=left,                    
		numbersep=5pt,                  
		showspaces=false,                
		showstringspaces=false,
		showtabs=false,                  
		tabsize=2
}

\lstdefinestyle{shellstyle}{
		backgroundcolor=\color{black!5}, 
		basicstyle=\ttfamily\footnotesize\color{black}, 
		commentstyle=\color{black}, 
		keywordstyle=\color{black},
		numberstyle=\color{black!5},
		stringstyle=\color{black}, 
		showspaces=false,
		showstringspaces=false, 
		showtabs=false, 
		tabsize=2, 
		numbers=none, 
		breaklines=true
}

\lstdefinelanguage{javascript}{
	keywords={typeof, new, true, false, catch, function, return, null, catch, switch, var, if, in, while, do, else, case, break},
	keywordstyle=\color{blue}\bfseries,
	ndkeywords={class, export, boolean, throw, implements, import, this},
	ndkeywordstyle=\color{darkgray}\bfseries,
	identifierstyle=\color{black},
	sensitive=false,
	comment=[l]{//},
	morecomment=[s]{/*}{*/},
	commentstyle=\color{purple}\ttfamily,
	stringstyle=\color{red}\ttfamily,
	morestring=[b]',
	morestring=[b]"
}

% disponi sezioni
\usepackage{titlesec}

\titleformat{\section}
	{\sffamily\Large\bfseries} 
	{\thesection}{1em}{} 
\titleformat{\subsection}
	{\sffamily\large\bfseries}   
	{\thesubsection}{1em}{} 
\titleformat{\subsubsection}
	{\sffamily\normalsize\bfseries} 
	{\thesubsubsection}{1em}{}

% disponi alberi
\usepackage{forest}

\forestset{
	rectstyle/.style={
		for tree={rectangle,draw,font=\large\sffamily}
	},
	roundstyle/.style={
		for tree={circle,draw,font=\large}
	}
}

% disponi algoritmi
\usepackage{algorithm}
\usepackage{algorithmic}
\makeatletter
\renewcommand{\ALG@name}{Algoritmo}
\makeatother

% disponi numeri di pagina
\usepackage{fancyhdr}
\fancyhf{} 
\fancyfoot[L]{\sffamily{\thepage}}

\makeatletter
\fancyhead[L]{\raisebox{1ex}[0pt][0pt]{\sffamily{\@title \ \@date}}} 
\fancyhead[R]{\raisebox{1ex}[0pt][0pt]{\sffamily{\@author}}}
\makeatother

\begin{document}
% sezione (data)
\section{Lezione del 13-12-24}

% stili pagina
\thispagestyle{empty}
\pagestyle{fancy}

% testo
\subsection{Circuito equivalente alla macchina asincrona}
Vediamo 3 circuiti equivalenti alla macchina asincrona.

\begin{enumerate}
	\item Di base, possiamo intendere lo statore e il rotore di una macchina asincrona come il primario e il secondario di un trasformatore, e quindi il circuito equivalente alla macchina asincrona è molto simile a quello del trasformatore reale:

\begin{center}
	\begin{circuitikz}
		\draw (-6,0) -- (-3.5, 0)
		to[ short, i=$i_1$] (-3,0)
			to [ resistor, l=$R_{1d}$] (-1,0)
			to [ inductor, l=$L_{1d}$] (1,0)
			to [ inductor , l=$N_1$] (1,-2)
			-- (-6, -2);
		
		\draw (3, 0) to [ resistor, l=$R_{2d}$] (5,0)
			to [ inductor, l=$L_{2d}$] (7,0) 
			to[ short, i<=$i_2$] (8,0);
			
			\draw (3,0) to[ inductor , l_=$N_2$] (3,-2)
			-- (8, -2);

		\draw (-5.5, 0) to [ resistor, l=$R_m$] (-5.5, -2);
		\draw (-4, 0) to [ inductor, l=$L_m$] (-4, -2);

			\draw (0.9,-0.6) node {$\scriptscriptstyle\bullet$};
			\draw (2.9,-0.6) node {$\scriptscriptstyle\bullet$};


			\draw (-6,-0.5) node[anchor=east] {$+$};
			\draw (-6,-1.5) node[anchor=east] {$-$};

			\draw (8,-0.5) node[anchor=west] {$+$};
			\draw (8,-1.5) node[anchor=west] {$-$};

			\draw (-6,-1) node[anchor=east] {$V_1$};
			\draw (8,-1) node[anchor=west] {$V_2$};

			\draw (0.7,-0.5) node[anchor=east] {$+$};
			\draw (0.7,-1.5) node[anchor=east] {$-$};

			\draw (3.3,-0.5) node[anchor=west] {$+$};
			\draw (3.3,-1.5) node[anchor=west] {$-$};

			\draw (0.7,-1) node[anchor=east] {$E_1$};
			\draw (3.3,-1) node[anchor=west] {$E_2$};

	\end{circuitikz}
\end{center}

Le differenze stanno nel fatto che:
\begin{enumerate}
	\item $X_{1d}$ e $X_{2d}$ sono solitamente maggiori rispetto al trasformatore.
Questo è dato dal traferro fra statore e rotore (che possiamo assumere come \textit{primario} e \textit{secondario} di un trasformatore), che ha una permeabilità molto più bassa del blocco cavo del trasformatore.
Per questo motivo è utile realizzare traferri il più sottili possibile (nell'ordine del millimetro);
\item La possibilità di avere \textbf{cortocircuiti} di rotore;
	\item $\omega_2 = s \omega_1$, cioè non ci troviamo a regime puramente sinusoidale ma la pulsazione di $\omega_2$ è $s$ volte $\omega_1$.
\end{enumerate}

Riguardo alla (3), notiamo le grandezze:
$$
X_{2d} = \omega_2 L_{2d}
$$
$$
\dot{E}_2 = j \omega_2 N_2 \dot{\Phi}_2
$$
proporzionali a $\omega_2$.

Rivediamo quindi i casi particolari che avevamo distinto sullo scorrimento $s$:
\begin{itemize}
	\item $s = 0$: l'equivalente è identico al trasformatore per la \textit{prova a vuoto}, in quanto sul lato destro del circuito non scorre corrente, e quindi nemmeno sull'induttanza al lato sinistro. Ricordiamo che questo era il caso di \textbf{rotore libero}.
	\item $s = 1$: l'equivalente è identico al trasformatore per la \textit{prova in cortocircuito}. Ricordiamo che questo era il caso di \textbf{rotore bloccato}. 
\end{itemize}

Abbiamo quindi che, nella macchina asincrona, la \textit{prova a vuoto} corrisponde alla \textbf{prova a rotore libero}, mentre la \textit{prova in cortocircuito} corrisponde alla \textbf{prova a rotore bloccato}.

\item Un'altra modellizzazione può portarci a regime sinusoidale puro.
	Prendiamo il lato destro del circuito, cioè il rotore. La caduta di potenziale su $X_{2d}$ sarà:
$$
j X_{2d} \dot{I}_2 + R_{2d} \dot{I}_2 + \dot{E}_2 = 0 \Rightarrow X_{2d} = \omega_2 L_{2d} = \omega_2 \frac{\omega_1}{\omega_1} L_{2d} = \frac{\omega_2}{\omega_1} (\omega_1 L_{2d}) = s X_{2d0}
$$
Allo stesso modo, riguardo a $\dot{E}_2$, si avrà:
$$
\dot{E}_2 = j \omega_2 N_2 \dot{\Phi}_2 = j \frac{\omega_2}{\omega_1} N_2 \dot{\Phi}_2 \omega_1 = s (j \omega_2 N_2 \dot{\Phi}_2) = s \dot{E}_{20}
$$

Possiamo quindi ridisegnare il circuito equivalente come:

\begin{center}
	\begin{circuitikz}
		\draw (-6,0) -- (-3.5, 0)
		to[ short, i=$i_1$] (-3,0)
			to [ resistor, l=$R_{1d}$] (-1,0)
			to [ inductor, l=$L_{1d}$] (1,0)
			to [ inductor , l=$N_1$] (1,-2)
			-- (-6, -2);
		
		\draw (3, 0) to [ resistor, l=$sR_{2d}$] (5,0)
			to [ inductor, l=$sL_{2d}$] (7,0) 
			to[ short, i<=$i_2$] (8,0);
			
			\draw (3,0) to[ inductor , l_=$N_2$] (3,-2)
			-- (8, -2);

		\draw (-5.5, 0) to [ resistor, l=$R_m$] (-5.5, -2);
		\draw (-4, 0) to [ inductor, l=$L_m$] (-4, -2);

			\draw (0.9,-0.6) node {$\scriptscriptstyle\bullet$};
			\draw (2.9,-0.6) node {$\scriptscriptstyle\bullet$};


			\draw (-6,-0.5) node[anchor=east] {$+$};
			\draw (-6,-1.5) node[anchor=east] {$-$};

			\draw (8,-0.5) node[anchor=west] {$+$};
			\draw (8,-1.5) node[anchor=west] {$-$};

			\draw (-6,-1) node[anchor=east] {$V_1$};
			\draw (8,-1) node[anchor=west] {$V_2$};

			\draw (0.7,-0.5) node[anchor=east] {$+$};
			\draw (0.7,-1.5) node[anchor=east] {$-$};

			\draw (3.3,-0.5) node[anchor=west] {$+$};
			\draw (3.3,-1.5) node[anchor=west] {$-$};

			\draw (0.7,-1) node[anchor=east] {$E_1$};
			\draw (3.3,-1) node[anchor=west] {$sE_{20}$};

	\end{circuitikz}
\end{center}

Con la differenza che in questo caso prendiamo $\omega_s = \omega_r$, cioè "normalizziamo" per avere pulsazione identica sui due lati.

\item Vediamo infine una modellizzazione che tiene conto di un eventuale carico al rotore.
	Si introduce una resistenza, detta \textbf{resistenza di carico}, $R_c$, al secondario:

\begin{center}
	\begin{circuitikz}
		\draw (-6,0) -- (-3.5, 0)
		to[ short, i=$i_1$] (-3,0)
			to [ resistor, l=$R_{1d}$] (-1,0)
			to [ inductor, l=$L_{1d}$] (1,0)
			to [ inductor , l=$N_1$] (1,-2)
			-- (-6, -2);
		
		\draw (3, 0) to [ resistor, l=$sR_{2d}$] (5,0)
			to [ inductor, l=$sL_{2d}$] (7,0) 
			to[ short, i<=$i_2$] (8,0);
			
			\draw (3,0) to[ inductor , l_=$N_2$] (3,-2)
			-- (8, -2);

			\draw (8,0) to [ resistor, l=$R_c$] (8, -2);

		\draw (-5.5, 0) to [ resistor, l=$R_m$] (-5.5, -2);
		\draw (-4, 0) to [ inductor, l=$L_m$] (-4, -2);

			\draw (0.9,-0.6) node {$\scriptscriptstyle\bullet$};
			\draw (2.9,-0.6) node {$\scriptscriptstyle\bullet$};

			\draw (-6,-0.5) node[anchor=east] {$+$};
			\draw (-6,-1.5) node[anchor=east] {$-$};

			\draw (-6,-1) node[anchor=east] {$V_1$};

			\draw (0.7,-0.5) node[anchor=east] {$+$};
			\draw (0.7,-1.5) node[anchor=east] {$-$};

			\draw (3.3,-0.5) node[anchor=west] {$+$};
			\draw (3.3,-1.5) node[anchor=west] {$-$};

			\draw (0.7,-1) node[anchor=east] {$E_1$};
			\draw (3.3,-1) node[anchor=west] {$sE_{20}$};

	\end{circuitikz}
\end{center}

	La $R_C$, nella pratica, va effettivamente a rappresentare un \textit{carico meccanico} collegato al rotore, e non una \textit{resistenza elettrica} vera è proprio. Addirittura, può accadere che $R_C < 0$.
\end{enumerate}

\subsection{Modalità di funzionamento della macchina asincrona}
Vediamo quindi come si comporta la macchina asincrona per diversi valori di scorrimento $s$, ricordando i casi già visti di rotore \textit{libero} e \textit{bloccato}:
\begin{table}[h!]
	\center \rowcolors{2}{white}{black!10}
	\begin{tabular} { c | p{3cm} | c | c | c }
		$s$ & \bfseries Funzionamento & $\Omega_r = (1 - s) \Omega_s$ & $R_c = \frac{1 - s}{s} R_{2d}$ & $P_c = R_c I_2^2$ \\
		\hline 
		$s < 0$ & Generatore & $\Omega_r > \Omega_s$ & $R_c < 0$ & $P_c < 0$ \\
		$s = 0$ & Rotore libero & $\Omega_r = \Omega_s$ & $R_c \rightarrow +\infty$ & $P_c = 0$ \\ 
		$0 < s < 1$ & Motore & $\Omega_r < \Omega_s$ & $R_c > 0$ & $P_c > 0$ \\
		$s = 1$ & Rotore bloccato & $\Omega_r = 0$ & $R_c = 0$ & $P_c = 0$ \\ 
		$s > 1$ & Freno & $\Omega_r \cdot \Omega_s < 0$ & - & - 
	\end{tabular}
\end{table}

\subsection{Coppia nelle macchine asincrone}
Vediamo come calcolare la coppia erogata da una macchina asincrona.
Innanzitutto definiamo la coppia come la potenza $P_m$ su $\Omega_r$ 
$$
C_T = \frac{P_m}{\Omega_r} = \frac{R_c I_2^2}{(1 - s) \Omega_s}
$$
$$
\dot{I}_2 = -\frac{s \dot{E}_{20}}{R_{2d} + js X_{2d0}}
$$

Usiamo il valore efficace della corrente:
$$
I_2 = |\dot{I}_2| = \left| -\frac{s \dot{E_{20}}}{R_{2d} + j X_{2d0}} \right|
$$
Notiamo che, per due complessi $\overline{z}_1$ e $\overline{z}_2$, si ha:
$$
\left| \frac{\overline{z}_1}{\overline{z}_2} \right| = \left| \frac{\rho_1 e^{j \vartheta_1}}{\rho_2 e^{j \vartheta_2}} \right| = \frac{\rho_1}{\rho_2} e^{j(\vartheta_1 - \vartheta_2)}
$$
da cui
$$
I_2 = \frac{s E_{20}}{\sqrt{R_{2d}^2 + s^2 X_{2d0}^2}}
$$

Avremo allora che:
$$
C_T = \frac{\frac{1-s}{s}R_{2d} I_2^2}{(1 - s) \frac{\omega_1}{\rho}} = \frac{\frac{1-s}{s}R_{2d}  \frac{s E_{20}}{\sqrt{R_{2d}^2 + s^2 X_{2d0}^2}} }{(1 - s) \frac{\omega_1}{\rho}} = \frac{\rho R_{2d}}{\omega_1} \cdot \frac{s E_{20}^2}{R_{2d}^2 + s^2 X^2_{2d0}} = \frac{\rho R_{2d} \frac{E_1^2}{n^2}}{\omega_1 \left( \frac{R_{2d}^2}{s}  + s X_{2d0}^2 \right)}
$$
dove $\rho$ è la coppia dipolie $n$ è il rapporto dipoli.

\begin{center}
\begin{tikzpicture}
    \begin{axis}[
        xlabel={$s$},
        ylabel={$C_T$},
        domain=-10:10, % set the x range you want
				samples=100,
        grid=major, % add a grid
				ytick={0, 2},
				yticklabels={ $0$, $C_{max}$},
				xtick={0, 1},
				xticklabels={$0$, $s^*$},
				ymin = -0.9, ymax = 3,
				xmin = -0.9,
				width=15cm,
				height=7cm
    ]
   \addplot[
        red,
        thick,
				domain= 0:10
		] {4/(1/x + x)}; 
    \end{axis}
\end{tikzpicture}
\end{center}


Potremmo voler calcolare la coppia massima nel punto $s = s^*$.
Prendiamo allora la derivata, notando che il numeratore è costante:
$$
\frac{\partial \frac{R_{2d}^2}{s} + s X_{2d0}^2}{\partial s} = 0 \Rightarrow - \frac{R_{2d}^2}{s^2} + X^2_{r2d0} = 0 \Rightarrow  \frac{R_{2d}^2}{s^2} = X^2_{r2d0} \Rightarrow s^2 = \frac{R_{2d}^2}{X_{2d0}^2} \Rightarrow s^* = \frac{R_{2d}}{X_{2d0}}
$$

Un problema comune è quello di volere coppia massima sia all'avvio della macchina che allo stato a regime: all'avvio $s = 1$, e va via via a scendere tendendo a $0$.
Quello che si fa è introdurre una resistenza iniziale:
$$
1 = \frac{R_{2d} + R_{agg}}{x_{2d0}}
$$
che avvicini il più possibile il valore $s^*$ di coppia massima alla $s$ corrente.

\subsection{Rendimento nelle macchine asincrone}
Definiamo il \textbf{rendimento} come:
$$
\eta = \frac{P_{utile}}{P_{assorbita}} \cdot 100 = \frac{R_c I_2^2}{R_c I_2^2 + R_{1d} I_1^2 + R_{2d} I_2^2 + \frac{V_1^2}{R_m} + P_{add}} \cdot 100
$$
dove $P_{add}$ modellizza altre potenze addizionali che vengono dissipate.

\end{document}
