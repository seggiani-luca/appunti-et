
\documentclass[a4paper,11pt]{article}
\usepackage[a4paper, margin=8em]{geometry}

% usa i pacchetti per la scrittura in italiano
\usepackage[french,italian]{babel}
\usepackage[T1]{fontenc}
\usepackage[utf8]{inputenc}
\frenchspacing 

% usa i pacchetti per la formattazione matematica
\usepackage{amsmath, amssymb, amsthm, amsfonts}

% usa altri pacchetti
\usepackage{gensymb}
\usepackage{hyperref}
\usepackage{standalone}

% imposta il titolo
\title{Appunti Elettrotecnica}
\author{Luca Seggiani}
\date{2024}

% imposta lo stile
% usa helvetica
\usepackage[scaled]{helvet}
% usa palatino
\usepackage{palatino}
% usa un font monospazio guardabile
\usepackage{lmodern}

\renewcommand{\rmdefault}{ppl}
\renewcommand{\sfdefault}{phv}
\renewcommand{\ttdefault}{lmtt}

% disponi il titolo
\makeatletter
\renewcommand{\maketitle} {
	\begin{center} 
		\begin{minipage}[t]{.8\textwidth}
			\textsf{\huge\bfseries \@title} 
		\end{minipage}%
		\begin{minipage}[t]{.2\textwidth}
			\raggedleft \vspace{-1.65em}
			\textsf{\small \@author} \vfill
			\textsf{\small \@date}
		\end{minipage}
		\par
	\end{center}

	\thispagestyle{empty}
	\pagestyle{fancy}
}
\makeatother

% disponi teoremi
\usepackage{tcolorbox}
\newtcolorbox[auto counter, number within=section]{theorem}[2][]{%
	colback=blue!10, 
	colframe=blue!40!black, 
	sharp corners=northwest,
	fonttitle=\sffamily\bfseries, 
	title=~\thetcbcounter: #2, 
	#1
}

% disponi definizioni
\newtcolorbox[auto counter, number within=section]{definition}[2][]{%
	colback=red!10,
	colframe=red!40!black,
	sharp corners=northwest,
	fonttitle=\sffamily\bfseries,
	title=~\thetcbcounter: #2,
	#1
}

% U.D.M
\newcommand{\amp}{\ensuremath{\mathrm{A}}}
\newcommand{\volt}{\ensuremath{\mathrm{V}}}
\newcommand{\meter}{\ensuremath{\mathrm{m}}}
\newcommand{\second}{\ensuremath{\mathrm{s}}}
\newcommand{\farad}{\ensuremath{\mathrm{F}}}
\newcommand{\henry}{\ensuremath{\mathrm{H}}}
\newcommand{\siemens}{\ensuremath{\mathrm{S}}}

% circuiti
\usepackage{circuitikz}
\usetikzlibrary{babel}

% disegni
\usepackage{pgfplots}
\pgfplotsset{width=10cm,compat=1.9}

% disponi codice
\usepackage{listings}
\usepackage[table]{xcolor}

\lstdefinestyle{codestyle}{
		backgroundcolor=\color{black!5}, 
		commentstyle=\color{codegreen},
		keywordstyle=\bfseries\color{magenta},
		numberstyle=\sffamily\tiny\color{black!60},
		stringstyle=\color{green!50!black},
		basicstyle=\ttfamily\footnotesize,
		breakatwhitespace=false,         
		breaklines=true,                 
		captionpos=b,                    
		keepspaces=true,                 
		numbers=left,                    
		numbersep=5pt,                  
		showspaces=false,                
		showstringspaces=false,
		showtabs=false,                  
		tabsize=2
}

\lstdefinestyle{shellstyle}{
		backgroundcolor=\color{black!5}, 
		basicstyle=\ttfamily\footnotesize\color{black}, 
		commentstyle=\color{black}, 
		keywordstyle=\color{black},
		numberstyle=\color{black!5},
		stringstyle=\color{black}, 
		showspaces=false,
		showstringspaces=false, 
		showtabs=false, 
		tabsize=2, 
		numbers=none, 
		breaklines=true
}

\lstdefinelanguage{javascript}{
	keywords={typeof, new, true, false, catch, function, return, null, catch, switch, var, if, in, while, do, else, case, break},
	keywordstyle=\color{blue}\bfseries,
	ndkeywords={class, export, boolean, throw, implements, import, this},
	ndkeywordstyle=\color{darkgray}\bfseries,
	identifierstyle=\color{black},
	sensitive=false,
	comment=[l]{//},
	morecomment=[s]{/*}{*/},
	commentstyle=\color{purple}\ttfamily,
	stringstyle=\color{red}\ttfamily,
	morestring=[b]',
	morestring=[b]"
}

% disponi sezioni
\usepackage{titlesec}

\titleformat{\section}
	{\sffamily\Large\bfseries} 
	{\thesection}{1em}{} 
\titleformat{\subsection}
	{\sffamily\large\bfseries}   
	{\thesubsection}{1em}{} 
\titleformat{\subsubsection}
	{\sffamily\normalsize\bfseries} 
	{\thesubsubsection}{1em}{}

% disponi alberi
\usepackage{forest}

\forestset{
	rectstyle/.style={
		for tree={rectangle,draw,font=\large\sffamily}
	},
	roundstyle/.style={
		for tree={circle,draw,font=\large}
	}
}

% disponi algoritmi
\usepackage{algorithm}
\usepackage{algorithmic}
\makeatletter
\renewcommand{\ALG@name}{Algoritmo}
\makeatother

% disponi numeri di pagina
\usepackage{fancyhdr}
\fancyhf{} 
\fancyfoot[L]{\sffamily{\thepage}}

\makeatletter
\fancyhead[L]{\raisebox{1ex}[0pt][0pt]{\sffamily{\@title \ \@date}}} 
\fancyhead[R]{\raisebox{1ex}[0pt][0pt]{\sffamily{\@author}}}
\makeatother

\begin{document}
% sezione (data)
\section{Lezione del 12-12-24}

% stili pagina
\thispagestyle{empty}
\pagestyle{fancy}

% testo
\subsection{Macchina rotante asincrona}
Abbiamo visto una semplice macchina elettrica che sfruttava il moto traslazionale di una bobina mobile.
Adesso vediamo una macchina più comune, che sfrutta il moto rotazionale.

La macchina rotante è composta da una parte statica, detta \textbf{statore}, composta da un materiale ferromagnetico su cui sono ricavati apposite \textbf{cave} dove scorrono fili percorsi da correnti, e una parte rotante, detta \textbf{rotore}, coassiale allo statore.
Fra rotore e statore c'è uno strato di aria, detto \textbf{traferro}.

La macchina rotante che andremo a vedere è detta \textbf{macchina asincrona}, in quanto il rotore ruota con una pulsazione minore della pulsazione del campo magnetico rotante generato dallo statore.
Esistono anche \textbf{macchine sincrone}, comunemente dette \textbf{alternatori}, dove le pulsazioni combaciano.

\subsubsection{Principio di funzionamento}
\begin{itemize}
	\item \textbf{\textsf{Statore}} \\ 
	Assumiamo uno statore con 6 cave disposte esagonalmente. P
	ossiamo immaginare all'interno delle cave singoli grandi fili percorsi da corrente. che si richiudono arrivati all'estremo longitudinale dello statore sulla cava opposta, a formare delle \textit{bobine}. 
	\begin{center}
\begin{tikzpicture}
    \coordinate (A) at (1,0);       
    \coordinate (B) at (0.5,0.866);       
    \coordinate (C) at (-0.5,0.866);      
    \coordinate (D) at (-1,0);      
    \coordinate (E) at (-0.5,-0.866);     
    \coordinate (F) at (0.5,-0.866);      

		\node at (A) {$\times$};
		\node at (B) {$\bullet$};
		\node at (C) {$\times$};
		\node at (D) {$\bullet$};
		\node at (E) {$\times$};
		\node at (F) {$\bullet$};

		\draw[dotted] (A) -- (D);
		\draw[dotted] (B) -- (E);
		\draw[dotted] (C) -- (F);

    \node[right] at (A) {$i_1$};
    \node[above right] at (B) {$i_2$};
    \node[above left] at (C) {$i_3$};
    \node[left] at (D) {$i_1$};
    \node[below left] at (E) {$i_2$};
    \node[below right] at (F) {$i_3$};
\end{tikzpicture}
	\end{center}
Quindi, se su un lato avremo una corrente $I$, sul lato opposto dello statore avremo una corrente $-I$.
Una qualsiasi di queste correnti $i_i$ sarà data dalla funzione sinusoidale:
$$
i_1(t) = I_m \sin{\omega t}
$$ 
Dalla legge di Biot-Savart si andrà a formare un campo di induzione magnetica diretto tangenzialmente rispetto ai fili, secondo la legge della mano destra.
Avremo però che la somma vettoriale dei campi generati dai due fili  è effettivamente un campo radiale, che risulterà uscente sul lato destro dello statore, e entrante sul lato sinistro.
In funzione di un angolo $\theta$ all'origine, questa sarà approssimativamente cosinusoidale, cioè:
$$
B_1(t) = c_1 i_1(t) \cos (\rho \theta)
$$
dove $\rho$ è il numero delle \textbf{paia polari} (nel caso dei sistemi trifase, coppie di triple di avvolgimenti di filo, cioè il numero di coppie \textit{nord-sud} nello statore).
Modificando il numero di paia polari si può variare il periodo di rotazione del campo oltre $\frac{2\pi}{\omega}$, cioè ottenere velocità angolari minori, ma coppie maggiori.

La coppia dei fili percorsi da $i_1(t)$ e $-i_1(t)$ non sarà però l'unica coppia di fili sullo statore.
Assumiamo 6 cave che portano corrente trifase (1 paia polare), quindi 3 correnti in forma:
\[
	\begin{cases}
		i_1(t) = I_m \sin(\omega t) \\
		i_2(t) = I_m \sin(\omega t + \frac{2}{3} \pi)	\\
		i_3(t) = I_m \sin(\omega t + \frac{4}{3} \pi)	
	\end{cases}
\]

Vediamo quindi l'intensità del campo di induzione magnetica generato, rispetto all'angolo $\theta$ sull'asse $x$ da cui viene osservato, da ogni avvolgimento:
\[
	\begin{cases}
B_1(t) = c_1 i_1(t) \sin (\rho \theta) \\ 
B_2(t) = c_2 i_2(t) \sin (\rho \theta + \frac{2}{3} \pi) \\ 
B_3(t) = c_3 i_3(t) \sin (\rho \theta + \frac{4}{3} \pi)
	\end{cases}
\]

Il campo magnetico totale sarà quindi dato dalla somma dei singoli campi generati da coppie di fili, cioè:
$$
B(t) = B_1(t) + B_2(t) + B_3(t) = k I_m \cos(\omega t - \rho \theta) = B(t, \theta)
$$
con $k$ dipendente da $c_1$, $c_2$ e $c_3$.

Il campo $B(t, \theta)$ prende il nome di \textbf{campo magnetico rotante}, \textbf{campo sincronizzato} o \textit{campo magnetico di statore} ed è funzione sia del tempo che dell'angolo $\theta$ su cui lo osserviamo.
Indichiamo la velocità di rotazione del campo, in radianti al secondo, come $\Omega_s$.
Per valutarla prendiamo due valori themporali su due angoli dove il campo è nullo:
\[
	\begin{cases}
		\omega t_1 - \rho \theta_1 = 0 \\
		\omega t_2 - \rho \theta_2 = 0
	\end{cases}
\]
Sarà allora che:
$$
\Omega_s = \frac{\Delta \theta}{\Delta t} = \frac{\theta_2 - \theta_1}{t_2 - t_1} = \frac{\frac{\omega t_2}{\rho} - \frac{\omega t_1}{\rho}}{t_2 - t_1} = \frac{\frac{\omega}{\rho}(t_2 - t_1)}{t_2 - t_1} = \frac{\omega}{\rho}
$$

\item \textbf{\textsf{Statore}} \\ 
Esistono due possibilità realizzative per il rotore:
\begin{itemize}
	\item Costruzione \textit{analoga allo statore}, dove si dice \textbf{rotore avvolto}, con l'unica differenza che gli avvolgimenti di rotore non sono alimentati, ma chiusi in cortocircuito sulle estremità longitudinali del rotore stesso;
	\item Attraverso barre longitudinali al rotore, dove si dice \textbf{rotore a gabbia di scoiattolo}.
\end{itemize}

Qualsiasi sia la modalità di realizzazione del rotore, si ha che le spire disposte longitudinalmente sono chiuse su stesse, cioè quando il rotore inizia a ruotare si ha forza elettromotrice indotta data dalla legge di Faraday:
$$
\varepsilon = -\frac{d \Phi(t)}{dt}
$$

Definiamo allora la velocità di rotazione del campo magnetico di statore come $\Omega_s$. Chiamiamo la $\omega$ associata allo statore \textit{pulsazione delle grandezze elettriche di statore} $\omega_1$, e introduciamo la \textit{pulsazione delle grandezze elettriche di rotore} $\omega_2$, da cui la velocità di rotazione del \textbf{campo magnetico di rotore} o del \textbf{campo asincrono} $\Omega_s$.

Definiremo la differenza di velocità $\Omega_{sr} = \Omega_s - \Omega_r$. 
Questa sarà utile a definire lo \textbf{scorrimento} $s$, con:
$$
s = \frac{\Omega_s - \Omega_r}{\Omega_s} = \frac{\Omega_{sr}}{\Omega_s}
$$
Possiamo distinguere più situazioni sulla base di $s$:
\begin{itemize}
	\item $s = 0$, $\Omega_{sr} = 0$, $\Omega_s = \Omega_r$: si parla di \textbf{rotore libero};
	\item $s = 1$, $\Omega_{sr} = \Omega_s$, $\Omega_r = 0$: si parla di \textbf{rotore bloccato} (chiaramente, da $\Omega_r = 0$).
\end{itemize}

Notiamo che, se:
$$
s = \frac{\Omega_s - \Omega_r}{\Omega_s} \Rightarrow s \Omega_s = \Omega_s - \Omega_r \Rightarrow \Omega_r = (1-s)\Omega_s
$$
e inoltre, se:
$$
\Omega_s = \frac{\omega_1}{\rho} \Rightarrow \Omega_{sr} = \Omega_s - \Omega_r = \frac{\omega_2}{\rho}
$$
$$
s \frac{\omega_1}{\rho} = \frac{\omega_2}{\rho}
$$

\end{itemize}

Possiamo quindi riassumere il principio di funzionamento della macchina rotante in:
\begin{enumerate}
	\item Alimentiamo i circuiti dello statore con 3 correnti trifasem cioè sfasate di 120°. Questo genera un campo magnetico rotante con velocità $\Omega_s = \frac{\omega}{\rho}$;
	\item Sugli avvolgimenti o le barre della gabbia di scoiattolo del rotore si genera una forza elettromotrice indotta, e quindi per momento meccanico una coppia che la porta a ruotare con una velocità di $\Omega_r = (1-s) \Omega_s$ dove $s$ è lo \textit{scorrimento}.
\end{enumerate}

La macchina asincrona lavora in un continuo stato di scorrimento (in quanto quando $s=0$ non si genera f.e.m. per Faraday e quindi non c'è coppia sul rotore).
In seguito vedremo nel dettaglio come questa coppia dipende da $s$, e quindi qual'è la condizione di regime della macchina.

\end{document}
