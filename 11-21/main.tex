
\documentclass[a4paper,11pt]{article}
\usepackage[a4paper, margin=8em]{geometry}

% usa i pacchetti per la scrittura in italiano
\usepackage[french,italian]{babel}
\usepackage[T1]{fontenc}
\usepackage[utf8]{inputenc}
\frenchspacing 

% usa i pacchetti per la formattazione matematica
\usepackage{amsmath, amssymb, amsthm, amsfonts}

% usa altri pacchetti
\usepackage{gensymb}
\usepackage{hyperref}
\usepackage{standalone}

% imposta il titolo
\title{Appunti Elettrotecnica}
\author{Luca Seggiani}
\date{2024}

% imposta lo stile
% usa helvetica
\usepackage[scaled]{helvet}
% usa palatino
\usepackage{palatino}
% usa un font monospazio guardabile
\usepackage{lmodern}

\renewcommand{\rmdefault}{ppl}
\renewcommand{\sfdefault}{phv}
\renewcommand{\ttdefault}{lmtt}

% disponi il titolo
\makeatletter
\renewcommand{\maketitle} {
	\begin{center} 
		\begin{minipage}[t]{.8\textwidth}
			\textsf{\huge\bfseries \@title} 
		\end{minipage}%
		\begin{minipage}[t]{.2\textwidth}
			\raggedleft \vspace{-1.65em}
			\textsf{\small \@author} \vfill
			\textsf{\small \@date}
		\end{minipage}
		\par
	\end{center}

	\thispagestyle{empty}
	\pagestyle{fancy}
}
\makeatother

% disponi teoremi
\usepackage{tcolorbox}
\newtcolorbox[auto counter, number within=section]{theorem}[2][]{%
	colback=blue!10, 
	colframe=blue!40!black, 
	sharp corners=northwest,
	fonttitle=\sffamily\bfseries, 
	title=~\thetcbcounter: #2, 
	#1
}

% disponi definizioni
\newtcolorbox[auto counter, number within=section]{definition}[2][]{%
	colback=red!10,
	colframe=red!40!black,
	sharp corners=northwest,
	fonttitle=\sffamily\bfseries,
	title=~\thetcbcounter: #2,
	#1
}

% U.D.M
\newcommand{\amp}{\ensuremath{\mathrm{A}}}
\newcommand{\volt}{\ensuremath{\mathrm{V}}}
\newcommand{\meter}{\ensuremath{\mathrm{m}}}
\newcommand{\second}{\ensuremath{\mathrm{s}}}
\newcommand{\farad}{\ensuremath{\mathrm{F}}}
\newcommand{\henry}{\ensuremath{\mathrm{H}}}
\newcommand{\siemens}{\ensuremath{\mathrm{S}}}

% circuiti
\usepackage{circuitikz}
\usetikzlibrary{babel}

% disegni
\usepackage{pgfplots}
\pgfplotsset{width=10cm,compat=1.9}

% disponi codice
\usepackage{listings}
\usepackage[table]{xcolor}

\lstdefinestyle{codestyle}{
		backgroundcolor=\color{black!5}, 
		commentstyle=\color{codegreen},
		keywordstyle=\bfseries\color{magenta},
		numberstyle=\sffamily\tiny\color{black!60},
		stringstyle=\color{green!50!black},
		basicstyle=\ttfamily\footnotesize,
		breakatwhitespace=false,         
		breaklines=true,                 
		captionpos=b,                    
		keepspaces=true,                 
		numbers=left,                    
		numbersep=5pt,                  
		showspaces=false,                
		showstringspaces=false,
		showtabs=false,                  
		tabsize=2
}

\lstdefinestyle{shellstyle}{
		backgroundcolor=\color{black!5}, 
		basicstyle=\ttfamily\footnotesize\color{black}, 
		commentstyle=\color{black}, 
		keywordstyle=\color{black},
		numberstyle=\color{black!5},
		stringstyle=\color{black}, 
		showspaces=false,
		showstringspaces=false, 
		showtabs=false, 
		tabsize=2, 
		numbers=none, 
		breaklines=true
}

\lstdefinelanguage{javascript}{
	keywords={typeof, new, true, false, catch, function, return, null, catch, switch, var, if, in, while, do, else, case, break},
	keywordstyle=\color{blue}\bfseries,
	ndkeywords={class, export, boolean, throw, implements, import, this},
	ndkeywordstyle=\color{darkgray}\bfseries,
	identifierstyle=\color{black},
	sensitive=false,
	comment=[l]{//},
	morecomment=[s]{/*}{*/},
	commentstyle=\color{purple}\ttfamily,
	stringstyle=\color{red}\ttfamily,
	morestring=[b]',
	morestring=[b]"
}

% disponi sezioni
\usepackage{titlesec}

\titleformat{\section}
	{\sffamily\Large\bfseries} 
	{\thesection}{1em}{} 
\titleformat{\subsection}
	{\sffamily\large\bfseries}   
	{\thesubsection}{1em}{} 
\titleformat{\subsubsection}
	{\sffamily\normalsize\bfseries} 
	{\thesubsubsection}{1em}{}

% disponi alberi
\usepackage{forest}

\forestset{
	rectstyle/.style={
		for tree={rectangle,draw,font=\large\sffamily}
	},
	roundstyle/.style={
		for tree={circle,draw,font=\large}
	}
}

% disponi algoritmi
\usepackage{algorithm}
\usepackage{algorithmic}
\makeatletter
\renewcommand{\ALG@name}{Algoritmo}
\makeatother

% disponi numeri di pagina
\usepackage{fancyhdr}
\fancyhf{} 
\fancyfoot[L]{\sffamily{\thepage}}

\makeatletter
\fancyhead[L]{\raisebox{1ex}[0pt][0pt]{\sffamily{\@title \ \@date}}} 
\fancyhead[R]{\raisebox{1ex}[0pt][0pt]{\sffamily{\@author}}}
\makeatother

\begin{document}
% sezione (data)
\section{Lezione del 21-11-24}

% stili pagina
\thispagestyle{empty}
\pagestyle{fancy}

% testo
\subsection{Analisi di circuiti aperiodici}
Finora abbiamo studiato circuiti in \textbf{corrente continua} e in \textbf{regime sinusoidale}.
Adesso vedremo come studiare circuiti dove le forme d'onda dei generatori sono arbitrarie.

\subsubsection{Circuito RL}
Poniamo un circuito formato da un resistore in serie a un induttore, come segue:

\begin{center}
	\begin{tikzpicture}
		\draw (0,0) to [ voltage source,  l=$e(t)$] (0,3)
			to [ resistor, l=$R$] (3,3)
			to [ inductor, l=$L$] (3,0)
			-- (0,0);
	\end{tikzpicture}
\end{center}

Diciamo che il generatore di tensione $e(t)$ ha forma d'onda:
\[
	e(t) =
	\begin{cases}
		E_0, \quad t < 0 \\ 
		2E_0, \quad t \geq 0
	\end{cases}
\]
di cui si riporta un grafico:
\begin{center}
\begin{tikzpicture}
    \begin{axis}[
        xlabel={$t$},
        ylabel={$e(t)$},
        domain=-10:10, % set the x range you want,
				samples=100,
        grid=major, % add a grid
				ytick={1, 2},
				yticklabels={$E_0$, $2 E_0$},
				xtick={0},
				xticklabels={$0$},
				width=15cm,
				height=7cm
    ]
    \addplot[
        red,
        thick,
				domain=-10:0
    ] {1};

    \addplot[
        red,
        thick,
				domain=0:10
    ] {2}; 
    \end{axis}
\end{tikzpicture}
\end{center}

Nell'intervallo negativo possiamo assumere che il circuito sia rimasto a $E_0$ costante per un tempo tale da poterlo studiare nello stato di equilibrio.
Cioè, se cercavamo $i(t)$, avremo semplicemente:
$$
i(t) = \frac{E_0}{R}, \quad t < 0
$$

Allo stesso modo, per tempi $t >> 0$, quindi con $t \rightarrow \infty$, potremo immaginare che il circito si trova nuovamente allo stato di equilibrio, cioè:
$$
i(t) = \frac{2 E_0}{R} \quad t >> 0
$$

La domanda è quindi come \textit{varia} la corrente $i(t)$ nell'intervallo immediatamente $t \geq 0$.
Chiamiamo il comportamento della corrente in questa fase \textbf{transitorio}.

Potremo applicare la prima legge di Kirchoff all'unica maglia del circuito, ricordando la caduta di potenziale sull'induttanza in funzione della corrente $i(t)$:
$$
-2 E_0 + R i(t) + L \frac{d \, i(t)}{dt} = 0
$$

Questa è un'equazione differenziale del primo ordine, che sappiamo ha soluzione omogenea e disomogenea (o generale e particolare \textit{specifica}, insomma qualsiasi nomenclatura viene riportata nel testo di Analisi 1 preferito del lettore):
$$
i(t) = i_o(t) + i_p(t)
$$

Risolvendo per $i_o(t)$:
$$
R i_o(t) + L \frac{d \, i_o(t)}{dt} = 0, \quad i_o(t) = Ae^{\lambda t}
$$
$$
R \lambda^0 + L \lambda^1 = 0 \Rightarrow \lambda = -\frac{R}{L}
$$
da cui:
$$
i_o(t) = Ae^{-\frac{R}{L}t}
$$

Troviamo quindi $i_p(t)$, posto $i_p(t) = I$ e dal regime costante in $t < 0$, $\frac{dI}{dt} = 0$:
$$
-2 E_0 + RI = 0 \Rightarrow I = i_p(t) = 2 \frac{E_0}{R}
$$

Otteniamo quindi la soluzione:
$$
i(t) = i_o(t) + i_p(t) = 2 \frac{E_0}{R} + A e^{-\frac{R}{l}t}
$$

Imponiamo la condizione iniziale, cioè $i(t) = \frac{E_0}{R}$ per $t < 0$:
$$
i(0) = \frac{E_0}{R} = \frac{2 E_0}{R} + A \Rightarrow A = \frac{E_0}{R} - \frac{2 E_0}{R} = -\frac{E_0}{R}
$$
da cui:
$$
i(t) = \frac{2 E_0}{R} - \frac{E_0}{R} e^{-\frac{R}{L}t} = \frac{E_0}{R}\left( 2 - e^{-\frac{R}{L}t} \right)
$$

Da cui la soluzione finale.
Possiamo tracciare un grafico dell'andamento della corrente:

\begin{center}
\begin{tikzpicture}
    \begin{axis}[
        xlabel={$t$},
        ylabel={$i(t)$},
        domain=-10:10, % set the x range you want,
				samples=100,
        grid=major, % add a grid
				ytick={1, 2},
				yticklabels={$\frac{E_0}{R}$, $\frac{2 E_0}{R}$},
				xtick={0},
				xticklabels={$0$},
				width=15cm,
				height=7cm
    ]
    \addplot[
        blue,
        thick,
				domain=-10:0
    ] {1};

    \addplot[
        blue,
        thick,
				domain=0:10
    ] {2-exp(-x)};

		\addplot[
        red,
        dotted,
				domain=0:10
    ] {2}; 
    \end{axis}
\end{tikzpicture}
\end{center}

Questa procedura, sebbene sia completamente generale, è difficilmente applicabile su circuiti più complessi.
Conviene quindi spostarsi in un nuovo dominio, seguendo un procedimento simile a quello che avevamo seguito usando i fasori.

\subsection{Trasformata di Laplace}
Rappresentato un segnale come una funzione $f(t)$, la \textbf{trasformata di Laplace} viene indicata come:
$$
F(s) = \mathcal{L} \left\{ f(t) \right\} = \int_{0^-}^{+\infty} f(t)e^{-st} \, dt
$$
dove $s$ è un complesso $\sigma + i \omega$.

Questa trasformata risulta molto utile in quanto ci permette di ricondurre le funzioni con cui lavoriamo nel cosiddetto spazio $s$, all'interno del cui le operazioni di derivazione e integrazione si riducono a divisioni e moltiplicazioni, e quindi all'algebra.

\subsubsection{Proprietà della trasformata di Laplace}

\begin{itemize}
	\item \textbf{Derivata:} la trasformata di Laplace di $f'(t)$ sarà:
		$$
		\mathcal{L}\left\{ f'(t) \right\} = \int_{0^-}^{+\infty} f'(t) e^{-st} dt	
		$$
		dove possiamo applicare la formula di integrazione per parti:
		$$ 
		\int_{0^-}^{+\infty} f'(t) e^{-st} dt = f(t) e^{-st} \Big|_{0^-}^{+\infty} - \int_{0^-}^{+\infty} f(t) \cdot -s e^{-st} dt 
		$$
		$$
		= s \int_{0^-}^{+\infty} f(t) e^{-st} dt + \lim_{t \rightarrow +\infty} f(t)e^{-st} - f(0^-) = s F(s) - f(0^-)
		$$
		dove compare un termine $f(0^-)$ dovuto alle condizioni iniziali.
\item \textbf{Integrale:} la trasformata di Laplace dell'integrale $g(t) = \int_{-\infty}^t f(\tau) d\tau$ sarà:
		$$
		\mathcal{L}\left\{ \frac{d\,g(t)}{dt} \right\} = \mathcal{L}\left\{ f(t) \right\} = s \mathcal{L}\left\{ g(t) \right\} - g(0^-) 
		$$
		$$
		=  s \mathcal{L}\left\{ \int_\infty^t f(\tau) d\tau \right\} - \int_{-\infty}^{0^-} f(\tau) d\tau
		$$
		da cui possiamo ricavare la trasformata dell'integrale:
		$$
		\mathcal{L}\left\{ \int_{-\infty}^t f(\tau) d\tau \right\} = \frac{F(s)}{s} + \frac{1}{s} \int_{-\infty}^{0^-} f(\tau) d \tau
		$$
dove nuovamente compare un termine $\frac{1}{s} \int_{-\infty}^{0^-} f(\tau) d \tau$ dovuto alle condizioni iniziali.
\end{itemize}

\end{document}
