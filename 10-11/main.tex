
\documentclass[a4paper,11pt]{article}
\usepackage[a4paper, margin=8em]{geometry}

% usa i pacchetti per la scrittura in italiano
\usepackage[french,italian]{babel}
\usepackage[T1]{fontenc}
\usepackage[utf8]{inputenc}
\frenchspacing 

% usa i pacchetti per la formattazione matematica
\usepackage{amsmath, amssymb, amsthm, amsfonts}

% usa altri pacchetti
\usepackage{gensymb}
\usepackage{hyperref}
\usepackage{standalone}

% imposta il titolo
\title{Appunti Elettrotecnica}
\author{Luca Seggiani}
\date{2024}

% imposta lo stile
% usa helvetica
\usepackage[scaled]{helvet}
% usa palatino
\usepackage{palatino}
% usa un font monospazio guardabile
\usepackage{lmodern}

\renewcommand{\rmdefault}{ppl}
\renewcommand{\sfdefault}{phv}
\renewcommand{\ttdefault}{lmtt}

% disponi il titolo
\makeatletter
\renewcommand{\maketitle} {
	\begin{center} 
		\begin{minipage}[t]{.8\textwidth}
			\textsf{\huge\bfseries \@title} 
		\end{minipage}%
		\begin{minipage}[t]{.2\textwidth}
			\raggedleft \vspace{-1.65em}
			\textsf{\small \@author} \vfill
			\textsf{\small \@date}
		\end{minipage}
		\par
	\end{center}

	\thispagestyle{empty}
	\pagestyle{fancy}
}
\makeatother

% disponi teoremi
\usepackage{tcolorbox}
\newtcolorbox[auto counter, number within=section]{theorem}[2][]{%
	colback=blue!10, 
	colframe=blue!40!black, 
	sharp corners=northwest,
	fonttitle=\sffamily\bfseries, 
	title=~\thetcbcounter: #2, 
	#1
}

% disponi definizioni
\newtcolorbox[auto counter, number within=section]{definition}[2][]{%
	colback=red!10,
	colframe=red!40!black,
	sharp corners=northwest,
	fonttitle=\sffamily\bfseries,
	title=~\thetcbcounter: #2,
	#1
}

% U.D.M
\newcommand{\amp}{\ensuremath{\mathrm{A}}}
\newcommand{\volt}{\ensuremath{\mathrm{V}}}
\newcommand{\meter}{\ensuremath{\mathrm{m}}}
\newcommand{\second}{\ensuremath{\mathrm{s}}}
\newcommand{\farad}{\ensuremath{\mathrm{F}}}
\newcommand{\henry}{\ensuremath{\mathrm{H}}}
\newcommand{\siemens}{\ensuremath{\mathrm{S}}}

% circuiti
\usepackage{circuitikz}
\usetikzlibrary{babel}

% disegni
\usepackage{pgfplots}
\pgfplotsset{width=10cm,compat=1.9}

% disponi codice
\usepackage{listings}
\usepackage[table]{xcolor}

\lstdefinestyle{codestyle}{
		backgroundcolor=\color{black!5}, 
		commentstyle=\color{codegreen},
		keywordstyle=\bfseries\color{magenta},
		numberstyle=\sffamily\tiny\color{black!60},
		stringstyle=\color{green!50!black},
		basicstyle=\ttfamily\footnotesize,
		breakatwhitespace=false,         
		breaklines=true,                 
		captionpos=b,                    
		keepspaces=true,                 
		numbers=left,                    
		numbersep=5pt,                  
		showspaces=false,                
		showstringspaces=false,
		showtabs=false,                  
		tabsize=2
}

\lstdefinestyle{shellstyle}{
		backgroundcolor=\color{black!5}, 
		basicstyle=\ttfamily\footnotesize\color{black}, 
		commentstyle=\color{black}, 
		keywordstyle=\color{black},
		numberstyle=\color{black!5},
		stringstyle=\color{black}, 
		showspaces=false,
		showstringspaces=false, 
		showtabs=false, 
		tabsize=2, 
		numbers=none, 
		breaklines=true
}

\lstdefinelanguage{javascript}{
	keywords={typeof, new, true, false, catch, function, return, null, catch, switch, var, if, in, while, do, else, case, break},
	keywordstyle=\color{blue}\bfseries,
	ndkeywords={class, export, boolean, throw, implements, import, this},
	ndkeywordstyle=\color{darkgray}\bfseries,
	identifierstyle=\color{black},
	sensitive=false,
	comment=[l]{//},
	morecomment=[s]{/*}{*/},
	commentstyle=\color{purple}\ttfamily,
	stringstyle=\color{red}\ttfamily,
	morestring=[b]',
	morestring=[b]"
}

% disponi sezioni
\usepackage{titlesec}

\titleformat{\section}
	{\sffamily\Large\bfseries} 
	{\thesection}{1em}{} 
\titleformat{\subsection}
	{\sffamily\large\bfseries}   
	{\thesubsection}{1em}{} 
\titleformat{\subsubsection}
	{\sffamily\normalsize\bfseries} 
	{\thesubsubsection}{1em}{}

% disponi alberi
\usepackage{forest}

\forestset{
	rectstyle/.style={
		for tree={rectangle,draw,font=\large\sffamily}
	},
	roundstyle/.style={
		for tree={circle,draw,font=\large}
	}
}

% disponi algoritmi
\usepackage{algorithm}
\usepackage{algorithmic}
\makeatletter
\renewcommand{\ALG@name}{Algoritmo}
\makeatother

% disponi numeri di pagina
\usepackage{fancyhdr}
\fancyhf{} 
\fancyfoot[L]{\sffamily{\thepage}}

\makeatletter
\fancyhead[L]{\raisebox{1ex}[0pt][0pt]{\sffamily{\@title \ \@date}}} 
\fancyhead[R]{\raisebox{1ex}[0pt][0pt]{\sffamily{\@author}}}
\makeatother

\begin{document}
% sezione (data)
\section{Lezione del 11-10-24}

% stili pagina
\thispagestyle{empty}
\pagestyle{fancy}

% testo
\subsection{Metodo delle tensioni di nodo}
Vediamo ora un metodo per la risoluzione dei circuiti dove le incognite non sono le correnti (come nel tableau o nelle correnti di maglia), ma le tensioni.
Calcoleremo quindi le \textbf{tensioni di nodo}, che sappiamo essere relative a qualche zero del potenziale, attraverso la conduttanza anziché la resistenza.

Prendiamo in esempio il circuito:

\begin{center}
\begin{circuitikz}
	\draw (0,0)
		-- (0,1.5)
		to[R, l=$R_1$] (5,1.5)
		-- (5, 0);
	\draw (0,0)
		to[R, l=$R_2$] (2.5, 0)
		to[R, l=$R_3$] (5, 0);
	\draw (0,0)
		to[R, l=$R_4$] (0, -3);
	\draw (2.5,-3)
		to[ european current source, I=$J$ ] (2.5, 0);
	\draw (5,-3)
		to[ R, l_=$R_5$] (5, 0);
	\draw (5, -3)
		to[ short ] (2.5, -3)
		to[ short ] (0, -3);

		\draw (0,0) node[circ] {};
		\draw (0,0) node[left] {A};

		\draw (2.5,0) node[circ] {};
		\draw (2.5,0) node[above] {B};

		\draw (5,0) node[circ] {};
		\draw (5,0) node[right] {C};

		\draw (2.5,-3) node[circ] {};
		\draw (2.5,-3) node[below] {D};
\end{circuitikz}
\end{center}
dove abbiamo già evidenziato i nodi.
Da qui possiamo proseguire in due modi:
\begin{itemize}
	\item \textbf{Col nodo di riferimento:} prendiamo D come nodo di riferimento, e impostiamo il suo potenziale $V_D$ a $0$. 
Le incognite saranno quindi $V_A$, $V_B$ e $V_C$.
Scriviamo un equazione per il potenziale $V_A$:
$$ V_A\left( \frac{1}{R_1} + \frac{1}{R_2} + \frac{1}{R_4} \right) - \frac{V_B}{R_2} - \frac{V_C}{R_1} = 0 $$
Il primo termine è detto \textbf{termine di auto ammettenza}, ed è dato dal prodotto del potenziale sul nodo $V_i$ e la somma delle conduttanze su tutti i rami connessi al nodo.
Il termini sucessivi sono detti \textbf{termini di mutua ammettenza}, e sono dati dalla legge di Ohm applicata ai rami connessi al nodo.

Nel caso uno dei rami contenga un generatore di corrente, si esclude dall'auto e mutua ammettenza e si mette come termine noto sul lato destro dell'equazione. \\
Ad esempio, sul nodo B avremo:
$$ V_B \left( \frac{1}{R_2} + \frac{1}{R_3} \right) - \frac{V_A}{R_2} - \frac{V_C}{R_3} = J $$
dove la $J$ è la corrente erogata dall generatore di corrente sul ramo BD, che abbiamo detto non compare nei termini di auto o mutua ammettenza. 
Si noti che il segno positivo significa corrente \textit{entrante} sul nodo, altrimenti si avrebbe segno negativo. \\
Infine, sul nodo C avremo:
$$ V_C \left( \frac{1}{R_1} + \frac{1}{R_3} + \frac{1}{R-5} \right) - \frac{V_A}{R_1} - \frac{V_B}{R_3} = 0 $$

Abbiamo quindi un sistema sulle 3 tensioni (abbiamo detto che la tensione $D$ vale 0):
\[
	\begin{cases}
 V_A\left( \frac{1}{R_1} + \frac{1}{R_2} + \frac{1}{R_4} \right) - \frac{V_B}{R_2} - \frac{V_C}{R_1} = 0 \\ 
 V_B \left( \frac{1}{R_2} + \frac{1}{R_3} \right) - \frac{V_A}{R_2} - \frac{V_C}{R_3} = J \\
 V_C \left( \frac{1}{R_1} + \frac{1}{R_3} + \frac{1}{R-5} \right) - \frac{V_A}{R_1} - \frac{V_B}{R_3} = 0 
	\end{cases}
\]

che possiamo portare in forma matriciale:
$$
\begin{pmatrix}
	\frac{1}{R_1} + \frac{1}{R_2} + \frac{1}{R_4} & -\frac{1}{R_2} & -\frac{1}{R_1} \\
	-\frac{1}{R_2} & \frac{1}{R_2} + \frac{1}{R_3} & -\frac{1}{R_3} \\
	-\frac{1}{R_1} & -\frac{1}{R3} & \frac{1}{R_1} + \frac{1}{R_3} + \frac{1}{R_5}
\end{pmatrix}
\begin{pmatrix}
V_A \\ V_B \\ V_C
\end{pmatrix}
=
\begin{pmatrix}
0 \\ J \\ 0
\end{pmatrix}
$$
e che possiamo risolvere.

\item \textbf{Senza nodo di riferimento:} possiamo decidere di continuare senza alcun riferimento del potenziale.
	Abbiamo già trovato le equazioni del potenziale sui primi 3 nodi, quindi impostiamo l'equazione per $V_D$:
	$$ V_D \left( \frac{1}{R_4} + \frac{1}{R_5} \right) - \frac{V_A}{R_4} - \frac{V_C}{R_5} = J $$
	e insieriamola nel sistema precedente:
	\[
		\begin{cases}
		 V_A\left( \frac{1}{R_1} + \frac{1}{R_2} + \frac{1}{R_4} \right) - \frac{V_B}{R_2} - \frac{V_C}{R_1} = 0 \\ 
		 V_B \left( \frac{1}{R_2} + \frac{1}{R_3} \right) - \frac{V_A}{R_2} - \frac{V_C}{R_3} = J \\
		 V_C \left( \frac{1}{R_1} + \frac{1}{R_3} + \frac{1}{R-5} \right) - \frac{V_A}{R_1} - \frac{V_B}{R_3} = 0 \\
			V_D \left( \frac{1}{R_4} + \frac{1}{R_5} \right) - \frac{V_A}{R_4} - \frac{V_C}{R_5} = -J
		\end{cases}
	\]
	in forma matriciale, abbiamo un sistema in 4 variabili con 3 equazioni:
$$
\begin{pmatrix}
	\frac{1}{R_1} + \frac{1}{R_2} + \frac{1}{R_4} & -\frac{1}{R_2} & -\frac{1}{R_1} & -\frac{1}{R_4} \\
	-\frac{1}{R_2} & \frac{1}{R_2} + \frac{1}{R_3} & -\frac{1}{R_3} & 0 \\
	-\frac{1}{R_1} & -\frac{1}{R3} & \frac{1}{R_1} + \frac{1}{R_3} + \frac{1}{R_5} & -\frac{1}{R_5} \\
	-\frac{1}{R_4} & 0 & - \frac{1}{R_5} & \frac{1}{R_4} + \frac{1}{R_5}
\end{pmatrix}
\begin{pmatrix}
V_A \\ V_B \\ V_C \\ V_D
\end{pmatrix}
=
\begin{pmatrix}
0 \\ J \\ 0 \\ -J 
\end{pmatrix}
$$

Sappiamo dal teorema di Rouché-Capelli che questo sistema è indeterminato, o almeno ammette soluzioni su una retta, quindi per un certo fattore di spostameno $\lambda$.
Questo rappresenta semplicemente il fatto che non abbiamo scelto uno zero del potenziale: il potenziale dei nodi non esiste se non in riferimento agli altri, quindi qualsiasi valore $\lambda$ aggiunto non cambia i risultati.
Scegliamo il valore $\lambda = 0$ per convenienza, che equivale a impostare $V_D$ a zero.
\end{itemize}

A questo punto, vogliamo risolvere effettivamente il circuito, e quindi trovare le tensioni e le correnti di ramo.
Per le tensioni, prendiamo semplicemente le differenze di potenziale sui nodi:
$$ V_{AB} = V_A - V_B, \quad \text{ecc...} $$
mentre per le correnti applichiamo la legge di Ohm sul ramo scelto. Ad esempio, sul ramo AB avremo:
$$ 
V = IR, \quad I_{AB} = \frac{V_{AB}}{R_2}  
$$

Un caso particolare è rappresentato dalla tensione su BD.
In questo caso si ha un generatore di corrente, e nessuna resistenza, quindi $I_{BD}$ è ben definito, ma non $V_{BD}$.
Quello che vogliamo fare è calcolare la caduta di potenziale $V_J = V_{BD}$ sul generatore di corrente.
Per fare ciò applichiamo la seconda legge di Kirchoff, magari applicata in senso orario sulla maglia in basso a sinistra, con le direzioni di corrente:

\begin{center}
\begin{circuitikz}
	\draw (0,0)
		-- (0,1.5)
		to[R, l=$R_1$, i=$i_1$] (5,1.5)
		-- (5, 0);
	\draw (0,0)
		to[R, l=$R_2$, i=$i_2$] (2.5, 0)
		to[R, l=$R_3$, i=$i_3$] (5, 0);
	\draw (0,0)
		to[R, l=$R_4$, i=$i_4$] (0, -3);
	\draw (2.5,-3)
		to[ european current source, I=$J$ ] (2.5, 0);
	\draw (5,-3)
		to[ R, l_=$R_5$, i=$i_5$] (5, 0);
	\draw (5, -3)
		to[ short ] (2.5, -3)
		to[ short ] (0, -3);

		\draw (0,0) node[circ] {};
		\draw (0,0) node[left] {A};

		\draw (2.5,0) node[circ] {};
		\draw (2.5,0) node[above] {B};

		\draw (5,0) node[circ] {};
		\draw (5,0) node[right] {C};

		\draw (2.5,-3) node[circ] {};
		\draw (2.5,-3) node[below] {D};
\end{circuitikz}
\end{center}

Si ha quindi:
$$
V_j + R_2 i_2 - R_4 i_4 = 0 \quad V_j = R_4 i_4 - R_2 i_2
$$

Abbiamo così determinato tutte le tensioni e tutte le correnti.

\subsubsection{Circuiti con generatori di tensione}
\begin{itemize}
	\item \textbf{\textsf{Generatori ideali di tensione}} \\
Il dipolo particolare quando si applica il metodo delle tensioni di nodo è il generatore ideale di tensione.
In questo caso, possiamo semplificare il sistema: se il generatore collega un nodo al nodo di riferimento, allora il voltaggio del nodo è uguale al voltaggio erogato dal generatore, tenendo conto dei segni in base alle direzioni delle tensioni erogate.

Se si hanno serie di generatori di tensione, si può applicare ripetutamente questa regola sui nodi coinvolti nella serie.

Nel caso non si possano scegliere agevolmente nodi di riferimento che prendono serie di generatori di tensione, si può impostare:
$$ E = V_A - V_B $$
dove $E$ è la tensione erogata dal generatore, e $V_A$ e $V_B$ le tensioni sui nodi che collega, ancora una volta opportunamente segnate in base alle direzioni delle tensioni erogate.
	\item \textbf{\textsf{Generatori reali di tensione}} \\
Un caso particolare è determinato dai generatori reali di tensione, cioè i generatori ideali in serie a resistenze.
In questi casi conviene trasformare i generatori nei loro equivalenti di Norton, e risolvere.
Si ha che per un generatore reale di voltaggio $V$ e resistenza $R$, l'equivalente di Norton è un generatore di corrente $\frac{V}{R}$ in parallelo ad una resistenza $R$.
\end{itemize}

%\subsubsection{Teoria dei grafi e reti elettriche}
%# anche questo magari scrivilo 
\TODO

\end{document}
