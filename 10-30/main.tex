
\documentclass[a4paper,11pt]{article}
\usepackage[a4paper, margin=8em]{geometry}

% usa i pacchetti per la scrittura in italiano
\usepackage[french,italian]{babel}
\usepackage[T1]{fontenc}
\usepackage[utf8]{inputenc}
\frenchspacing 

% usa i pacchetti per la formattazione matematica
\usepackage{amsmath, amssymb, amsthm, amsfonts}

% usa altri pacchetti
\usepackage{gensymb}
\usepackage{hyperref}
\usepackage{standalone}

% imposta il titolo
\title{Appunti Elettrotecnica}
\author{Luca Seggiani}
\date{2024}

% imposta lo stile
% usa helvetica
\usepackage[scaled]{helvet}
% usa palatino
\usepackage{palatino}
% usa un font monospazio guardabile
\usepackage{lmodern}

\renewcommand{\rmdefault}{ppl}
\renewcommand{\sfdefault}{phv}
\renewcommand{\ttdefault}{lmtt}

% disponi il titolo
\makeatletter
\renewcommand{\maketitle} {
	\begin{center} 
		\begin{minipage}[t]{.8\textwidth}
			\textsf{\huge\bfseries \@title} 
		\end{minipage}%
		\begin{minipage}[t]{.2\textwidth}
			\raggedleft \vspace{-1.65em}
			\textsf{\small \@author} \vfill
			\textsf{\small \@date}
		\end{minipage}
		\par
	\end{center}

	\thispagestyle{empty}
	\pagestyle{fancy}
}
\makeatother

% disponi teoremi
\usepackage{tcolorbox}
\newtcolorbox[auto counter, number within=section]{theorem}[2][]{%
	colback=blue!10, 
	colframe=blue!40!black, 
	sharp corners=northwest,
	fonttitle=\sffamily\bfseries, 
	title=~\thetcbcounter: #2, 
	#1
}

% disponi definizioni
\newtcolorbox[auto counter, number within=section]{definition}[2][]{%
	colback=red!10,
	colframe=red!40!black,
	sharp corners=northwest,
	fonttitle=\sffamily\bfseries,
	title=~\thetcbcounter: #2,
	#1
}

% U.D.M
\newcommand{\amp}{\ensuremath{\mathrm{A}}}
\newcommand{\volt}{\ensuremath{\mathrm{V}}}
\newcommand{\meter}{\ensuremath{\mathrm{m}}}
\newcommand{\second}{\ensuremath{\mathrm{s}}}
\newcommand{\farad}{\ensuremath{\mathrm{F}}}
\newcommand{\henry}{\ensuremath{\mathrm{H}}}
\newcommand{\siemens}{\ensuremath{\mathrm{S}}}

% circuiti
\usepackage{circuitikz}
\usetikzlibrary{babel}

% disegni
\usepackage{pgfplots}
\pgfplotsset{width=10cm,compat=1.9}

% disponi codice
\usepackage{listings}
\usepackage[table]{xcolor}

\lstdefinestyle{codestyle}{
		backgroundcolor=\color{black!5}, 
		commentstyle=\color{codegreen},
		keywordstyle=\bfseries\color{magenta},
		numberstyle=\sffamily\tiny\color{black!60},
		stringstyle=\color{green!50!black},
		basicstyle=\ttfamily\footnotesize,
		breakatwhitespace=false,         
		breaklines=true,                 
		captionpos=b,                    
		keepspaces=true,                 
		numbers=left,                    
		numbersep=5pt,                  
		showspaces=false,                
		showstringspaces=false,
		showtabs=false,                  
		tabsize=2
}

\lstdefinestyle{shellstyle}{
		backgroundcolor=\color{black!5}, 
		basicstyle=\ttfamily\footnotesize\color{black}, 
		commentstyle=\color{black}, 
		keywordstyle=\color{black},
		numberstyle=\color{black!5},
		stringstyle=\color{black}, 
		showspaces=false,
		showstringspaces=false, 
		showtabs=false, 
		tabsize=2, 
		numbers=none, 
		breaklines=true
}

\lstdefinelanguage{javascript}{
	keywords={typeof, new, true, false, catch, function, return, null, catch, switch, var, if, in, while, do, else, case, break},
	keywordstyle=\color{blue}\bfseries,
	ndkeywords={class, export, boolean, throw, implements, import, this},
	ndkeywordstyle=\color{darkgray}\bfseries,
	identifierstyle=\color{black},
	sensitive=false,
	comment=[l]{//},
	morecomment=[s]{/*}{*/},
	commentstyle=\color{purple}\ttfamily,
	stringstyle=\color{red}\ttfamily,
	morestring=[b]',
	morestring=[b]"
}

% disponi sezioni
\usepackage{titlesec}

\titleformat{\section}
	{\sffamily\Large\bfseries} 
	{\thesection}{1em}{} 
\titleformat{\subsection}
	{\sffamily\large\bfseries}   
	{\thesubsection}{1em}{} 
\titleformat{\subsubsection}
	{\sffamily\normalsize\bfseries} 
	{\thesubsubsection}{1em}{}

% disponi alberi
\usepackage{forest}

\forestset{
	rectstyle/.style={
		for tree={rectangle,draw,font=\large\sffamily}
	},
	roundstyle/.style={
		for tree={circle,draw,font=\large}
	}
}

% disponi algoritmi
\usepackage{algorithm}
\usepackage{algorithmic}
\makeatletter
\renewcommand{\ALG@name}{Algoritmo}
\makeatother

% disponi numeri di pagina
\usepackage{fancyhdr}
\fancyhf{} 
\fancyfoot[L]{\sffamily{\thepage}}

\makeatletter
\fancyhead[L]{\raisebox{1ex}[0pt][0pt]{\sffamily{\@title \ \@date}}} 
\fancyhead[R]{\raisebox{1ex}[0pt][0pt]{\sffamily{\@author}}}
\makeatother

\begin{document}
% sezione (data)
\section{Lezione del 30-10-24}

% stili pagina
\thispagestyle{empty}
\pagestyle{fancy}

% testo
\subsection{Potenza complessa}
Introduciamo un'ulteriore misura della potenza, comoda perché non necessita (a differenza di quelle studiate finora) di lavorare coi valori efficaci.
\begin{definition}{Potenza complessa}
Definiamo la \textbf{potenza complessa} $\overline{S}$ come il prodotto dei fasori di tensione e corrente, cioè come:
$$
\overline{S} = \dot{V} \dot{I}^*
$$
\end{definition}

Notiamo che si prende il \textbf{coniugato complesso} di $\dot{I}$.
Il suo significato si può mostrare svolgendo il calcolo:
$$
\overline{S} = \dot{V} \dot{I}^* = V \cdot e^{j \phi_v} \cdot I \cdot e^{-j \phi_i} = VI e^{j(\phi_v - \phi_i)} = VI e^{j\phi}
$$
cioè possiamo sfruttare la formula sulle fasi $\phi_v - \phi_i = \phi$.

In forma cartesiana, abbiamo che $\overline{S}$ vale:
$$
\overline{S} = VI e^{j\phi} = VI \cos(\phi) + j VI \sin{\phi} = P + jQ
$$
e quindi la potenza complessa è uguale al complesso dato da potenza attiva e reattiva.
Notiamo che finora nei calcoli si è inteso, con $V$ e $I$, i \textbf{valori efficaci} di queste grandezze: da qui in poi infatti prenderemo qualsiasi versore privo di pedice con valore efficace a modulo.

Possiamo rappresentare questa relazione nel cosiddetto \textbf{triangolo delle potenze}: la base è la potenza attiva, l'altezza la potenza resistiva, e inoltre notiamo che l'ipotenusa è la potenza apparente:

\begin{center}
\begin{tikzpicture}
  \begin{axis}[
    axis lines = center,
    xmin=-1, xmax=5.4,
    ymin=-1, ymax=5.4,
		xtick={0.707*5},
		ytick={0.707*5},
		xticklabel={$P$},
		yticklabel={$Q$},
		grid=minor,
  ]
    % Draw the phasor (example: magnitude=1, angle=45 degrees)
    \addplot[thick,->,black] coordinates {(0,0) (0.707*5,0.707*5)};
    
    % Optional: Add a label for the phasor
		\node at (axis cs: 0.707*5, 0.707*5) [anchor=west] {$\overline{S}$};

		\node at(axis cs: 0.5,0.3) [anchor=west] {$\phi$};
		\node at(axis cs: 1.5,1.7) [anchor=south] {$S$};
  \end{axis}
\end{tikzpicture}
\end{center}

Possiamo poi sfruttare: 
$$
\dot{V} = \overline{Z}\dot{I} \Leftrightarrow
	\begin{cases}
		V = ZI \\ 	
		\phi_v = \phi_z + \phi_i
	\end{cases}, \quad 
\dot{I} = \frac{\dot{V}}{\overline{Z}} = \overline{Y} \dot{V} \Leftrightarrow 
	\begin{cases}
		I = Y V \\ 
		\phi_i = \phi_y + \phi_v
	\end{cases}
$$
e quindi possiamo esprimere $\overline{S}$ come funzione:
\begin{itemize}
	\item In funzione dell'impedenza: 
$$
\overline{S} = \dot{V}\dot{I}^* = \overline{Z} I^2 
$$
	\item In funzione dell'ammettenza:
$$
\overline{S} = \dot{V}\dot{I}^* = \overline{Y} V^2
$$
\end{itemize}

\subsection{Teorema di Tellegen}
Dimostriamo il seguente risultato:
\begin{theorem}{Teorema di Tellegen}
	La somma algebrica delle potenze istantanee impiegate su tutti i rami di una rete elettrica è uguale a 0, ovvero:
$$
\sum_{jk = 1}^n v_{jk} (t) \cdot i_{jk}(t) = 0
$$
\end{theorem}

Notiamo che $j$ e $k$ sono indici che scorrono lungo $n$, dove $n$ è il numero di nodi della rete presa in considerazione, ergo $x_{jk}$ è quella grandezza considerata su ogni arco (ad archi inesistenti sarà evidentemente 0).

Possiamo riformulare la formula precedente come segue:
$$
\sum_{jk = 1}^n  \left( v_{j, 0}(t) - v_{k, 0}(t) \right) \cdot i_{j.k}(t) = 0
$$

Dove si è usato $v_{jk} = v_{j, 0} - v_{k, 0}$ dal secondo principio di Kirchoff.
Abbiamo quindi:
$$
= \sum_{jk = 1}^n  v_{j, 0}(t) \cdot i_{j.k}(t) - \sum_{jk = 1}^n v_{k, 0}(t) \cdot i_{j.k}(t) = \sum_{j=1}^n v_{j,0}(t) \left( \sum_{k=1}^n i_{jk}(t) \right) - v_{k,0}(t) \left( \sum_{j=1}^n i_{jk}(t) \right)
$$

Abbiamo che il secondo termine sommatoria è la somma delle correnti che escono dal nodo $j$, che quindi è 0 dal primo di Kirchoff, e il quarto termine, allo stesso modo, è la somma delle correnti che escono dal nodo $k$, e quindi l'intera espressione è nulla, da cui il teorema.

\subsubsection{Teorema di Boucherot}
Dal teorema di Tellegen sul caso sinusoidale deriva il \textbf{teorema di Boucherot}.
Prendiamo come ipotesi: 
$$
\sum_{jk = 1}^n \dot{V}_{jk} \dot{I}_{jk}^* = 0
$$

Questo deriva dal teorema di Tellegen considerato in regime sinusoidale.
Prendiamo quindi un qualunque ramo $jk$: su questo ramo avremo chiaramente impedenze e generatori, a cui diamo il nome $\overline{Z}_{jk}$ e $\dot{E}_{jk}$, e corrente e voltaggio, a cui diamo il nome di $\dot{I}_{jk}$ e $\dot{V}_{jk}$:
\begin{center}
	\begin{circuitikz}
		\draw (0,0) to [ european resistor, l=$\overline{Z}_{jk}$] (2, 0)
			to [ european voltage source, v=$E_{jk}$] (4, 0);

		\node at (0,0) [anchor=east] {$j$};
		\node at (4,0) [anchor=west] {$k$};
	\end{circuitikz}
\end{center}

Sostituiamo quindi la caduta di potenziale $\dot{V}_{jk}$ nell'ipotesi con:
$$
\sum_{jk = 1}^n \left( \overline{Z}_{jk} \dot{I}_{jk} - \dot{E}_{jk} \right) \cdot \dot{I}_{jk}^* = 0
$$
notando che i due termini coinvolti nella sommatoria sono:
$$
\sum_{jk = 1}^n \overline{Z}_{jk} \cdot I_{jk}^2 = \sum_{jk = 1}^n \dot{E}_{jk} \dot{I}_{jk}^* = \sum_{jk = 1}^n \overline{S}_{jk}^{(G)}
$$
cioè le potenze dissipate sulle impedenze e le potenze erogate dai generatori del ramo.
Da questo si deriva il teorema:
\begin{theorem}{Teorema di Boucherot}
	La somma delle potenze complesse dissipate sulle impedenze di un circuito è uguale alla somma delle potenze complesse erogate dai generatori, cioè:
$$
\sum_{jk = 1}^n \overline{Z}_{jk} \cdot I_{jk}^2 = \sum_{jk = 1}^n \dot{E}_{jk} \dot{I}_{jk}^*
$$
\end{theorem}

Possiamo rendere questo risultato anche come:
$$
\sum_{jk = 1}^n \left( R_{jk} + j X_{jk} \right) I_{jk}^2 = \sum_{jk = 1}^n \left( P_{jk}^{(G)} + j Q_{jk}^{(G)} \right)
$$

Da cui in poi si possono equagliare separatamente parte reale e parte immaginaria:
\[
	\begin{cases}
		\sum\limits_{jk =1}^n R_{jk} I_{jk}^2 = \sum\limits_{jk}^n P_{jk}^{(G)} \\ 
		\sum\limits_{jk =1}^n X_{jk} I_{jk}^2 = \sum\limits_{jk}^n Q_{jk}^{(G)} \\ 
	\end{cases}
\]
ergo, non solo la somma delle potenze potenze complesse dissipate sulle impedenze di un circuito è uguale alla somma delle potenze complesse erogate dai generatori, ma sia la potenza attiva che la potenza reattiva dissipata sulle impedenze sono uguali a le corrispondenti erogate dai generatori.

Sulla potenza attiva, che sappiamo non poter avere segno negativo, si ha che i generatori devono compensare le potenze dissipate sulle impedenze.

\subsubsection{Potenza apparente e Boucherot}
Preso un ramo con due generatori $P_1$ e $P_2$, abbiamo che la potenza attiva e reattive associate valgono quanto le somme delle potenze attive e reattive sui singoli generatori:
$$
P^{(G)} = P_1 + P_2, \quad Q^{(G)} = Q_1 + Q_2 
$$
e quindi, da Boucherot, o semplicemente applicando la definizione di potenza complessa $\overline{S}$:
$$
\overline{S}^{(G)} = P^{(G)} + j Q^{(G)} = (P_1 + P_2) + j (Q_1 + Q_2)
$$

Cioè la potenza complessa si conserva.

Dobbiamo notare che questo non vale per la potenza apparente: si ha che $S^{(G)} = \sqrt{P_{(G)}^2 + Q_{(G)}^2}$, ergo la dipendenza non è lineare, e non possiamo assumere che si conservi, cioè vale:
$$
S^{(G)} = \sqrt{P_{(G)}^2 + Q_{(G)}^2} \neq \sqrt{P_1^2 + Q_1^2} + \sqrt{P_2^2 + Q_2^2} 
$$

\end{document}
