
\documentclass[a4paper,11pt]{article}
\usepackage[a4paper, margin=8em]{geometry}

% usa i pacchetti per la scrittura in italiano
\usepackage[french,italian]{babel}
\usepackage[T1]{fontenc}
\usepackage[utf8]{inputenc}
\frenchspacing 

% usa i pacchetti per la formattazione matematica
\usepackage{amsmath, amssymb, amsthm, amsfonts}

% usa altri pacchetti
\usepackage{gensymb}
\usepackage{hyperref}
\usepackage{standalone}

% imposta il titolo
\title{Appunti Elettrotecnica}
\author{Luca Seggiani}
\date{2024}

% imposta lo stile
% usa helvetica
\usepackage[scaled]{helvet}
% usa palatino
\usepackage{palatino}
% usa un font monospazio guardabile
\usepackage{lmodern}

\renewcommand{\rmdefault}{ppl}
\renewcommand{\sfdefault}{phv}
\renewcommand{\ttdefault}{lmtt}

% disponi il titolo
\makeatletter
\renewcommand{\maketitle} {
	\begin{center} 
		\begin{minipage}[t]{.8\textwidth}
			\textsf{\huge\bfseries \@title} 
		\end{minipage}%
		\begin{minipage}[t]{.2\textwidth}
			\raggedleft \vspace{-1.65em}
			\textsf{\small \@author} \vfill
			\textsf{\small \@date}
		\end{minipage}
		\par
	\end{center}

	\thispagestyle{empty}
	\pagestyle{fancy}
}
\makeatother

% disponi teoremi
\usepackage{tcolorbox}
\newtcolorbox[auto counter, number within=section]{theorem}[2][]{%
	colback=blue!10, 
	colframe=blue!40!black, 
	sharp corners=northwest,
	fonttitle=\sffamily\bfseries, 
	title=~\thetcbcounter: #2, 
	#1
}

% disponi definizioni
\newtcolorbox[auto counter, number within=section]{definition}[2][]{%
	colback=red!10,
	colframe=red!40!black,
	sharp corners=northwest,
	fonttitle=\sffamily\bfseries,
	title=~\thetcbcounter: #2,
	#1
}

% U.D.M
\newcommand{\amp}{\ensuremath{\mathrm{A}}}
\newcommand{\volt}{\ensuremath{\mathrm{V}}}
\newcommand{\meter}{\ensuremath{\mathrm{m}}}
\newcommand{\second}{\ensuremath{\mathrm{s}}}
\newcommand{\farad}{\ensuremath{\mathrm{F}}}
\newcommand{\henry}{\ensuremath{\mathrm{H}}}
\newcommand{\siemens}{\ensuremath{\mathrm{S}}}

% circuiti
\usepackage{circuitikz}
\usetikzlibrary{babel}

% disegni
\usepackage{pgfplots}
\pgfplotsset{width=10cm,compat=1.9}

% disponi codice
\usepackage{listings}
\usepackage[table]{xcolor}

\lstdefinestyle{codestyle}{
		backgroundcolor=\color{black!5}, 
		commentstyle=\color{codegreen},
		keywordstyle=\bfseries\color{magenta},
		numberstyle=\sffamily\tiny\color{black!60},
		stringstyle=\color{green!50!black},
		basicstyle=\ttfamily\footnotesize,
		breakatwhitespace=false,         
		breaklines=true,                 
		captionpos=b,                    
		keepspaces=true,                 
		numbers=left,                    
		numbersep=5pt,                  
		showspaces=false,                
		showstringspaces=false,
		showtabs=false,                  
		tabsize=2
}

\lstdefinestyle{shellstyle}{
		backgroundcolor=\color{black!5}, 
		basicstyle=\ttfamily\footnotesize\color{black}, 
		commentstyle=\color{black}, 
		keywordstyle=\color{black},
		numberstyle=\color{black!5},
		stringstyle=\color{black}, 
		showspaces=false,
		showstringspaces=false, 
		showtabs=false, 
		tabsize=2, 
		numbers=none, 
		breaklines=true
}

\lstdefinelanguage{javascript}{
	keywords={typeof, new, true, false, catch, function, return, null, catch, switch, var, if, in, while, do, else, case, break},
	keywordstyle=\color{blue}\bfseries,
	ndkeywords={class, export, boolean, throw, implements, import, this},
	ndkeywordstyle=\color{darkgray}\bfseries,
	identifierstyle=\color{black},
	sensitive=false,
	comment=[l]{//},
	morecomment=[s]{/*}{*/},
	commentstyle=\color{purple}\ttfamily,
	stringstyle=\color{red}\ttfamily,
	morestring=[b]',
	morestring=[b]"
}

% disponi sezioni
\usepackage{titlesec}

\titleformat{\section}
	{\sffamily\Large\bfseries} 
	{\thesection}{1em}{} 
\titleformat{\subsection}
	{\sffamily\large\bfseries}   
	{\thesubsection}{1em}{} 
\titleformat{\subsubsection}
	{\sffamily\normalsize\bfseries} 
	{\thesubsubsection}{1em}{}

% disponi alberi
\usepackage{forest}

\forestset{
	rectstyle/.style={
		for tree={rectangle,draw,font=\large\sffamily}
	},
	roundstyle/.style={
		for tree={circle,draw,font=\large}
	}
}

% disponi algoritmi
\usepackage{algorithm}
\usepackage{algorithmic}
\makeatletter
\renewcommand{\ALG@name}{Algoritmo}
\makeatother

% disponi numeri di pagina
\usepackage{fancyhdr}
\fancyhf{} 
\fancyfoot[L]{\sffamily{\thepage}}

\makeatletter
\fancyhead[L]{\raisebox{1ex}[0pt][0pt]{\sffamily{\@title \ \@date}}} 
\fancyhead[R]{\raisebox{1ex}[0pt][0pt]{\sffamily{\@author}}}
\makeatother

\begin{document}
% sezione (data)
\section{Lezione del 13-11-24}

% stili pagina
\thispagestyle{empty}
\pagestyle{fancy}

% testo
Avevamo visto il concetto di \textbf{bipolo}, cioè un componente circuitale con due \textit{punti di contatto} col resto del circuito (\textbf{morsetti}), su cui passa una certa \textbf{corrente} $I$ e su cui si trova una certa \textbf{tensione}, cioè una \textit{differenza di potenzale} $V$.
Potremmo avere anche un \textbf{tripolo}, cioè un componente con morsetti, su cui passano (propriamente, da cui \textit{escono} o \textit{entrano}), anzichè una, 3 correnti, e su cui individuiamo 3 tensioni ($A$, $B$ e $C$) e 3 \textbf{cadute} di tensione su ogni percorso che attraversa il bipolo.
Una possibile rappresentazione di un tripolo è la seguente:
\begin{center}
	\begin{circuitikz}
		\node[rectangle, draw, minimum width = 1cm, minimum height = 1cm] (a) at (0,0) {};
		\draw (-2, 0) to [ short, i=$I_A$] (a.west);
		\draw (0, -2) to [ short, i=$I_C$] (a.south);
		\draw (2, 0) to [ short, i_=$I_B$] (a.east);
		

		\node[anchor=east] at(-2, 0) {$V_A$};
		\node[anchor=north] at(0, -2) {$V_C$};
		\node[anchor=west] at(2, 0) {$V_B$};
	\end{circuitikz}
\end{center}
le cui equazioni sono:
\[
	\begin{cases}
		I_A + I_B + I_C = 0 \\ 
		V_{AB} = V_A - V_B \\ 
		V_{AC} = V_A - V_C \\ 
		V_{BC} = V_B - V_C
	\end{cases}
\]

Notiamo che, dalle equazioni ai potenziali, si possono ricavare le relazioni (piuttosto scontate):
\[
	\begin{cases}
		V_{AB} + V_{BC} = V_{AC} \\ 
		V_{BA} + V_{AC} = V_{BC} \\
		V_{AC} + V_{CB} = V_{AB}
	\end{cases}
\]
con $V_{BA} = - V_{AB}$ e $V_{CB} = -V_{BC}$ (e anche se non si è usata, $V_{CA} = -V_{AC}$).

\subsection{Porte}
Definiamo una \textbf{porta} come una coppia di poli di un circuito dove la corrente entrante è uguale a quella uscente.
Rappresentiamo una porta come segue:

\begin{center}
	\begin{circuitikz}
		\node[rectangle, draw, minimum width = 2cm, minimum height = 2cm] (a) at (0,0) {};
		\draw (-2, 0.6) to [ short, i=$I$] (-1, 0.6);
		\draw(-1, -0.6) to [ short, i=$I$ ] (-2, -0.6);	
	
		\draw (-2.6, 0.6) node[anchor=west] {$+$};
		\draw (-2.6, 0) node[anchor=west] {$V$};
		\draw (-2.6, -0.6) node[anchor=west] {$-$};
	\end{circuitikz}
\end{center}

Notiamo che per $n$ poli si hanno al massimo $\frac{n}{2}$ porte (ammesso un numero pari di poli).

Ciò che ci è di interesse sono i circuiti a \textbf{due porte} (o equivalentemente a \textit{quattro poli}):

\begin{center}
	\begin{circuitikz}
		\node[rectangle, draw, minimum width = 2cm, minimum height = 2cm] (a) at (0,0) {};
		\draw (-2, 0.6) to [ short, i=$I_1$] (-1, 0.6);
		\draw(-1, -0.6) to [ short, i=$I_1$ ] (-2, -0.6);	
	
		\draw (-2.6, 0.6) node[anchor=west] {$+$};
		\draw (-2.6, 0) node[anchor=west] {$V_1$};
		\draw (-2.6, -0.6) node[anchor=west] {$-$};
		
		\draw (2, 0.6) to [ short, i_=$I_2$] (1, 0.6);
		\draw(1, -0.6) to [ short, i_=$I_2$ ] (2, -0.6);	
	
		\draw (2.6, 0.6) node[anchor=east] {$+$};
		\draw (2.6, 0) node[anchor=east] {$V_2$};
		\draw (2.6, -0.6) node[anchor=east] {$-$};
	\end{circuitikz}
\end{center}

Possiamo immaginare che un segnale \textit{entra} da una porta, viene \textit{elaborato} all'interno del circuito, e \textit{esce} dalla porta opposta.

Per convenzione, scegliamo le due correnti $I_1(t)$e $I_2(t)$ come rivolte nello stesso senso, e le due tensioni $V_1(t)$ e $V_2(t)$ come con la stessa polarità:

\subsubsection{Rappresentazione in impedenza di circuiti a due porte}
Una coppia di \textbf{induttori mutuamente accoppiati} rappresenta effettivamente un circuito a due porte, in quanto la stessa corrente entra e esce da ogni induttore (cioè si formano due porte).
Avevamo rappresentato questi circuiti come:
\[
	\begin{cases}
		\dot{V}_1 = j \omega L_1 \dot{I}_1 + j \omega M \dot{I}_2 \\	
		\dot{V}_2 = j \omega L_2 \dot{I}_2 + j \omega M \dot{I}_1	
	\end{cases}
\]

Analogamente, decidiamo di rappresentare un circuito a due porte attraverso equazioni che legano la tensione su una porta alla corrente su entrambe le porte:
\[
	\begin{cases}
		\dot{V}_1 = \overline{z_{11}} \dot{I}_1 + \overline{z_{12}} \dot{I}_2 \\ 	
		\dot{V}_2 = \overline{z_{21}} \dot{I}_1 + \overline{z_{22}} \dot{I}_2 	
	\end{cases}
\]

Per esprimere queste relazioni in forma più compatta, possiamo sfruttare il calcolo matriciale:
$$
\dot{V} = \overline{Z} \dot{I}
$$
dove $\dot{V}$ e $\dot{I}$ sono matrici:
$$
\begin{pmatrix}
	\dot{V}_1 \\ \dot{V}_2
\end{pmatrix}
= \overline{Z}
\begin{pmatrix}
	\dot{I}_1 \\ \dot{I}_2
\end{pmatrix}
$$
e $\overline{Z}$ sarà l'\textbf{impedenza} in forma matriciale:
$$
\overline{Z} =
\begin{pmatrix}
	\overline{z_{11}} & \overline{z_{12}} \\ 
	\overline{z_{21}} & \overline{z_{22}} 
\end{pmatrix}
$$

Date le equazioni riportate sopra che legano voltaggio a corrente, possiamo ricavare il valore di ogni componente di $\overline{Z}$ come:
\[
	\begin{cases}
		\overline{z_{11}} = \frac{\dot{V}_1}{\dot{I}_1} \Big|_{\dot{I}_2 = 0} \\
		\overline{z_{12}} = \frac{\dot{V}_1}{\dot{I}_2} \Big|_{\dot{I}_1 = 0} \\
		\overline{z_{21}} = \frac{\dot{V}_2}{\dot{I}_1} \Big|_{\dot{I}_2 = 0} \\
		\overline{z_{22}} = \frac{\dot{V}_2}{\dot{I}_2} \Big|_{\dot{I}_1 = 0}
	\end{cases}
\]
dove la notazione $a \Big|_b$ significa "$a$ quando $b$".

Si ha, attraverso queste relazioni, che basta misurare la tensione sulle porte in due stati ($\dot{I}_1 = 0$ e $\dot{I}_2 = 0$) per ricavare completamente le variabili $\overline{Z}$ del circuito.

\subsubsection{Rappresentazione in ammettenza di circuiti a due porte}
Possiamo usare, anzichè l'impedenza $\overline{Z}$, l'ammettenza $\overline{Y}$: se avevamo espresso il comportamento del circuito come $
\begin{pmatrix}
	\dot{V}_1 \\ \dot{V}_2
\end{pmatrix}
= \overline{Z}
\begin{pmatrix}
	\dot{I}_1 \\ \dot{I}_2
\end{pmatrix}
$, infatti, possiamo trovare l'inverso:
$$
\begin{pmatrix}
	\dot{I}_1 \\ \dot{I}_2
\end{pmatrix}
= \overline{Z}^{-1}
\begin{pmatrix}
	\dot{V}_1 \\ \dot{V}_2
\end{pmatrix}
$$
dove la matrice $\overline{Z}^{-1} = \overline{Y}$ è effettivamente l'\textbf{ammettenza} in forma matriciale del circuito:
$$
\overline{Y} =
\begin{pmatrix}
	\overline{y_{11}} & \overline{y_{12}} \\ 
	\overline{y_{21}} & \overline{y_{22}}
\end{pmatrix}
$$

Questo, in forma sistema, ha l'aspetto:
\[
	\begin{cases}
		\dot{I}_1 = \overline{y_{11}} \dot{V}_1 + \overline{y_{12}} \dot{V}_2 \\ 	
		\dot{I}_2 = \overline{y_{21}} \dot{V}_1 + \overline{y_{22}} \dot{V}_2 	
	\end{cases}
\]

Date le equazioni riportate sopra, possiamo ricavare il valore di ogni componente di $\overline{Y}$ come:
\[
	\begin{cases}
		\overline{y_{11}} = \frac{\dot{I}_1}{\dot{V}_1} \Big|_{\dot{V}_2 = 0} \\
		\overline{y_{12}} = \frac{\dot{I}_1}{\dot{V}_2} \Big|_{\dot{V}_1 = 0} \\
		\overline{y_{21}} = \frac{\dot{I}_2}{\dot{V}_1} \Big|_{\dot{V}_2 = 0} \\
		\overline{y_{22}} = \frac{\dot{I}_2}{\dot{V}_2} \Big|_{\dot{V}_1 = 0}
	\end{cases}
\]

\subsection{Circuiti equivalenti di circuiti a due porte}
Ciò che può interessarci quando studiamo circuiti a due porte è ricavare \textbf{circuiti equivalenti}, cioè che si comportano in maniera equivalente agli effetti esterni.
L'idea è, come sempre, quella di prendere circuiti arbitrariamente complessi e ridurli a circuiti equivalenti relativamente semplici.

\subsubsection{Sintesi a parametri Z}
Sfruttando la rappresentazione in impedenza dei circuiti a due porte, un circuito equivalente potrebbe essere il seguente:

\begin{center}
	\begin{circuitikz}
		\draw (-4, 1) -- (-3, 1) 
			to [ european resistor, l=$\overline{z_{11}}$, i=$I_1$] (-1, 1)
			to [ controlled voltage source, v<=$\overline{z_{12}} \dot{I}_2$ ] (-1, -1) 
			to [ short, i=$I_1$ ] (-3, -1)	
			-- (-4, -1);
			
		\draw (-4.6, 1) node[anchor=west] {$+$};
		\draw (-4.6, 0) node[anchor=west] {$V_1$};
		\draw (-4.6, -1) node[anchor=west] {$-$};

		\draw (6,1) -- (5, 1) 
			to [ european resistor, l_=$\overline{z_{12}}$, i_=$I_2$] (3, 1)
			to [ controlled voltage source, v_<=$\overline{z_{21}} \dot{I}_1$ ] (3, -1) 
			to [ short, i_=$I_2$ ] (5, -1)
			-- (6, -1);
	
		\draw (6.6, 1) node[anchor=east] {$+$};
		\draw (6.6, 0) node[anchor=east] {$V_2$};
		\draw (6.6, -1) node[anchor=east] {$-$};
		
		\node[rectangle, draw, minimum width = 8.5cm, minimum height = 4cm] (a) at (1,0) {};
	\end{circuitikz}
\end{center}
dove si inseriscono i termini di impedenza $\overline{z_{11}}$ e $\overline{z_{22}}$ semplicemente come impedenze in serie alle porte 1 e 2, e i termini "associati" $\overline{z_{12}}$ e $\overline{z_{21}}$ come generatori di tensione pilotati (che generano, appunto, cadute di tensione pilotate, rispettivamente in $\dot{I}_2$ per la porta 1 e in $\dot{I}_1$ per la porta 2).

Il metodo naturale di analisi per questo circuito è correnti di maglia, che possiamo applicare alle due porte per poi eguagliare con la matrice delle impedenze.
Si ha, quindi, rispetto alle due porte:
\[
	\begin{cases}
		\dot{V}_1	= \overline{z_{11}} \dot{I}_1 + \overline{Z_{12}} \dot{I}_2 \\  	
		\dot{V}_2	= \overline{z_{22}} \dot{I}_2 + \overline{Z_{21}} \dot{I}_1 \\  	
	\end{cases}
\]
che combacia con quanto definito sulla rappresentazione in impedenza.

In particolare, nel caso $\overline{z_{12}} = \overline{z_{21}}$ si dice che la rete è \textbf{reciproca} e si può formare il circuito equivalente come:

\begin{center}
	\begin{circuitikz}
		\draw (-4, 1) -- (-3, 1) 
			to [ european resistor, l=$\overline{z_a}$, i=$I_1$] (0, 1)
			to [ european resistor, l=$\overline{z_b}$] (0, -1) 
			to [ short, i=$I_1$ ] (-3, -1)	
			-- (-4, -1);
			
		\draw (-4.6, 1) node[anchor=west] {$+$};
		\draw (-4.6, 0) node[anchor=west] {$V_1$};
		\draw (-4.6, -1) node[anchor=west] {$-$};

		\draw (4, 1) -- (3, 1) 
			to [ european resistor, l_=$\overline{z_c}$, i_=$I_2$] (0, 1);

		\draw (0, -1) to [ short, i_=$I_2$ ] (3, -1)
			-- (4, -1);
	
		\draw (4.6, 1) node[anchor=east] {$+$};
		\draw (4.6, 0) node[anchor=east] {$V_2$};
		\draw (4.6, -1) node[anchor=east] {$-$};
		
		\node[rectangle, draw, minimum width = 6.5cm, minimum height = 4cm] (a) at (0,0) {};
	\end{circuitikz}
\end{center}

Anche qui, applicando correnti di maglia, si ha:
\[
	\begin{cases}
		\dot{V}_1 = \overline{z_a} \dot{I}_1 + \overline{z_b}\left( \dot{I}_1 + \dot{I}_2 \right) = ( \overline{z_a} + \overline{z_b} ) \dot{I}_1 + \overline{z_b} \dot{I}_2 \\	
		\dot{V}_2 = \overline{z_c} \dot{I}_2 + \overline{z_b}\left( \dot{I}_1 + \dot{I}_2 \right) = ( \overline{z_c} + \overline{z_c} ) \dot{I}_2 + \overline{z_b} \dot{I}_1 \\ 	
	\end{cases}
\]
che rappresenta la rete reciproca, ponendo:
\[
	\begin{cases}
		\overline{z_{11}} = \overline{z_a} + \overline{z_b} \\ 
		\overline{z_{12}} = \overline{z_{21}} = \overline{z_b} \\ 
		\overline{z_{22}} = \overline{z_c} + \overline{z_b} 
	\end{cases}
	\Leftrightarrow
	\begin{cases}
		\overline{z_a} = \overline{z_{11}} - \overline{z_{12}} \\ 
		\overline{z_b} = \overline{z_{12}} = \overline{z_{21}} \\
		\overline{z_c} = \overline{z_{22}} - \overline{z_{12}}
	\end{cases}
\]

Notiamo che, per circuiti a due porte generici, non è detto che i potenziali dei morsetti di uscita di entrambe le porte siano allo stesso potenziale: per modellizzare questo comportamento si usa una \textit{mutua induttanza ideale}, cioè un \textbf{trasformatore ideale}.

\subsubsection{Sintesi a parametri Y}
Come si è fatta la sintesi a parametri Z, si può fare la sintesi a parametri Y, cioè secondo le ammettenze:
\begin{center}
	\begin{circuitikz}
		\draw (-4, 1) to [ short, i=$I_1$] (-3, 1) 
			to [ short] (-1, 1);
		\draw (-1, 1) to [ controlled current source, cI<=$\overline{y_{12}} \dot{V}_1$ ] (-1, -1);
		\draw (-1, -1) to [ short] (-3, -1)
		-- (-4, -1);
			
		\draw (-4.6, 1) node[anchor=west] {$+$};
		\draw (-4.6, 0) node[anchor=west] {$V_1$};
		\draw (-4.6, -1) node[anchor=west] {$-$};

		\draw (6,1) to [ short, i=$I_2$] (5, 1) 
			to [ short] (3, 1)
			to [ controlled current source, cI_<=$\overline{y_{21}} \dot{V}_2$ ] (3, -1) 
			to [ short] (5, -1)
			-- (6, -1);
	
		\draw (6.6, 1) node[anchor=east] {$+$};
		\draw (6.6, 0) node[anchor=east] {$V_2$};
		\draw (6.6, -1) node[anchor=east] {$-$};
		
		\draw (-2, 1) to [ european resistor, l_=$\overline{y_{11}}$] (-2, -1);
	\draw (4, 1) to [ european resistor, l=$\overline{y_{22}}$] (4, -1);
		
		\node[rectangle, draw, minimum width = 8.5cm, minimum height = 4cm] (a) at (1,0) {};
	\end{circuitikz}
\end{center}
dove stavolta si inseriscono i termini di ammettenza $\overline{y_{11}}$ e $\overline{y_{22}}$ semplicemente ammettenze in parallelo alle porte 1 e 2, e i termini "associati" $\overline{y_{12}}$ e $\overline{y_{21}}$ come generatori di corrente pilotati (che generano, appunto, cadute di tensione pilotate, rispettivamente in $\dot{I}_2$ per la porta 1 e in $\dot{I}_1$ per la porta 2).

Il metodo naturale di analisi per questo circuito è tensioni di nodo, che possiamo applicare alle due porte per poi eguagliare con la matrice delle ammettenze.
Si ha, quindi, rispetto alle due porte:
\[
	\begin{cases}
		\dot{I}_1	= \overline{y_{11}} \dot{V}_1 + \overline{y_{12}} \dot{V}_2 \\  	
		\dot{I}_2	= \overline{y_{22}} \dot{V}_2 + \overline{y_{21}} \dot{V}_1 \\  	
	\end{cases}
\]
che combacia con quanto definito sulla rappresentazione in ammettenza.

In particolare, vediamo il caso \textbf{reciproco} $\overline{y_{11}} = \overline{y_{21}}$: 

\begin{center}
	\begin{circuitikz}
		\draw (-1, 1) to[ european resistor, l_=$\overline{y_a}$] (-1, -1);
		\draw (1, 1) to[ european resistor, l=$\overline{y_c}$] (1, -1);
		
		\draw (-4, 1) to[ short, i=$I_1$] (-3, 1)
			to[ european resistor, l=$\overline{y_b}$] (3, 1)
			to[ short, i<=$I_2$] (4, 1);		

		\draw (-4, -1) to[ short] (4, -1);

		\draw (-4.6, 1) node[anchor=west] {$+$};
		\draw (-4.6, 0) node[anchor=west] {$V_1$};
		\draw (-4.6, -1) node[anchor=west] {$-$};

		\draw (4.6, 1) node[anchor=east] {$+$};
		\draw (4.6, 0) node[anchor=east] {$V_2$};
		\draw (4.6, -1) node[anchor=east] {$-$};
		
		\node[rectangle, draw, minimum width = 6.5cm, minimum height = 4cm] (a) at (0,0) {};
	\end{circuitikz}
\end{center}

Anche qui, applicando tensioni di nodo, si ha:
\[
	\begin{cases}
		\dot{I}_1 = (\overline{y_a} + \overline{y_b}) \dot{V}_1 - \overline{y_b} \dot{V}_2 \\	
		\dot{I}_2 = (\overline{y_b} + \overline{y_c}) \dot{V}_2 - \overline{y_b} \dot{V}_1
	\end{cases}
\]
che rappresenta la rete reciproca, ponendo:
\[
	\begin{cases}
		\overline{y_{11}} = \overline{y_a} + \overline{y_b} \\ 
		\overline{y_{12}} = \overline{y_{21}} = -\overline{y_b} \\ 
		\overline{y_{22}} = \overline{y_c} + \overline{y_b} 
	\end{cases}
	\Leftrightarrow
	\begin{cases}
		\overline{y_a} = \overline{y_{11}} + \overline{y_{12}} \\ 
		\overline{y_b} = -\overline{y_{12}} = -\overline{y_{21}} \\
		\overline{y_c} = \overline{y_{22}} + \overline{y_{12}}
	\end{cases}
\]

Notiamo che ancora che il circuito più generale si ottiene disaccoppiando i potenziali sul ramo in basso attraverso un trasformatore ideale.


\end{document}
