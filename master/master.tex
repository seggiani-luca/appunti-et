\documentclass[a4paper,14pt]{memoir}
\usepackage[a4paper, margin=4em]{geometry}

% usa i pacchetti per la scrittura in italiano
\usepackage[french,italian]{babel}
\usepackage[T1]{fontenc}
\usepackage[utf8]{inputenc}
\frenchspacing 

% usa i pacchetti per la formattazione matematica
\usepackage{amsmath, amssymb, amsthm, amsfonts}

% usa altri pacchetti
\usepackage{gensymb}
\usepackage{hyperref}
\usepackage{standalone}

% imposta il titolo
\title{Appunti Elettrotecnica}
\author{Luca Seggiani}
\date{2024}

% imposta lo stile
% usa helvetica
\usepackage[scaled]{helvet}
% usa palatino
\usepackage{mathpazo}
\renewcommand{\rmdefault}{ppl}
\renewcommand{\sfdefault}{phv}
\renewcommand{\ttdefault}{lmtt}

% disponi il titolo
\makeatletter
\renewcommand{\maketitle} {
	\begin{center} 
		\begin{minipage}[t]{.8\textwidth}
			\textsf{\huge\bfseries \@title} 
		\end{minipage}%
		\begin{minipage}[t]{.2\textwidth}
			\raggedleft \vspace{-1.65em}
			\textsf{\small \@author} \vfill
			\textsf{\small \@date}
		\end{minipage}
		\par
	\end{center}

	\thispagestyle{empty}
	\pagestyle{fancy}
}
\makeatother

% disponi teoremi
\usepackage{tcolorbox}
\newtcolorbox[auto counter, number within=section]{theorem}[2][]{%
	colback=blue!10, 
	colframe=blue!40!black, 
	sharp corners=northwest,
	fonttitle=\sffamily\bfseries, 
	title=~\thetcbcounter: #2, 
	#1
}

% disponi definizioni
\newtcolorbox[auto counter, number within=section]{definition}[2][]{%
	colback=red!10,
	colframe=red!40!black,
	sharp corners=northwest,
	fonttitle=\sffamily\bfseries,
	title=~\thetcbcounter: #2,
	#1
}

% U.D.M
\newcommand{\amp}{\ensuremath{\mathrm{A}}}
\newcommand{\volt}{\ensuremath{\mathrm{V}}}
\newcommand{\meter}{\ensuremath{\mathrm{m}}}
\newcommand{\second}{\ensuremath{\mathrm{s}}}
\newcommand{\farad}{\ensuremath{\mathrm{F}}}
\newcommand{\henry}{\ensuremath{\mathrm{H}}}
\newcommand{\siemens}{\ensuremath{\mathrm{S}}}

% circuiti
\usepackage{circuitikz}
\usetikzlibrary{babel}

% disegni
\usepackage{pgfplots}
\pgfplotsset{width=10cm,compat=1.9}

% disponi codice
\usepackage{listings}
\usepackage[table]{xcolor}

\lstdefinestyle{codestyle}{
		backgroundcolor=\color{black!5}, 
		commentstyle=\color{codegreen},
		keywordstyle=\bfseries\color{magenta},
		numberstyle=\sffamily\tiny\color{black!60},
		stringstyle=\color{green!50!black},
		basicstyle=\ttfamily\footnotesize,
		breakatwhitespace=false,         
		breaklines=true,                 
		captionpos=b,                    
		keepspaces=true,                 
		numbers=left,                    
		numbersep=5pt,                  
		showspaces=false,                
		showstringspaces=false,
		showtabs=false,                  
		tabsize=2
}

\lstdefinestyle{shellstyle}{
		backgroundcolor=\color{black!5}, 
		basicstyle=\ttfamily\footnotesize\color{black}, 
		commentstyle=\color{black}, 
		keywordstyle=\color{black},
		numberstyle=\color{black!5},
		stringstyle=\color{black}, 
		showspaces=false,
		showstringspaces=false, 
		showtabs=false, 
		tabsize=2, 
		numbers=none, 
		breaklines=true
}

\lstdefinelanguage{javascript}{
	keywords={typeof, new, true, false, catch, function, return, null, catch, switch, var, if, in, while, do, else, case, break},
	keywordstyle=\color{blue}\bfseries,
	ndkeywords={class, export, boolean, throw, implements, import, this},
	ndkeywordstyle=\color{darkgray}\bfseries,
	identifierstyle=\color{black},
	sensitive=false,
	comment=[l]{//},
	morecomment=[s]{/*}{*/},
	commentstyle=\color{purple}\ttfamily,
	stringstyle=\color{red}\ttfamily,
	morestring=[b]',
	morestring=[b]"
}

% disponi sezioni
\usepackage{titlesec}

\titleformat{\section}
	{\sffamily\Large\bfseries} 
	{\thesection}{1em}{} 
\titleformat{\subsection}
	{\sffamily\large\bfseries}   
	{\thesubsection}{1em}{} 
\titleformat{\subsubsection}
	{\sffamily\normalsize\bfseries} 
	{\thesubsubsection}{1em}{}

% disponi alberi
\usepackage{forest}

\forestset{
	rectstyle/.style={
		for tree={rectangle,draw,font=\large\sffamily}
	},
	roundstyle/.style={
		for tree={circle,draw,font=\large}
	}
}

% disponi algoritmi
\usepackage{algorithm}
\usepackage{algorithmic}
\makeatletter
\renewcommand{\ALG@name}{Algoritmo}
\makeatother

% disponi numeri di pagina
\usepackage{fancyhdr}
\fancyhf{} 

\setlength{\headheight}{28pt} 
\setlength{\headsep}{10pt} 
\setlength{\footskip}{28pt} 

\fancyfoot[L]{\raisebox{1ex}[36pt][0pt]{\sffamily{\thepage}}}

\makeatletter
\fancyhead[L]{\raisebox{1ex}[0pt][0pt]{\sffamily{\@title \ \@date}}} 
\fancyhead[R]{\raisebox{1ex}[0pt][0pt]{\sffamily{\@author}}}
\makeatother

\begin{document}

\pagestyle{fancy}
\thispagestyle{empty}
\renewcommand{\thispagestyle}[1]{}

\maketitle
\documentclass[a4paper,11pt]{article}
\usepackage[a4paper, margin=8em]{geometry}

% usa i pacchetti per la scrittura in italiano
\usepackage[french,italian]{babel}
\usepackage[T1]{fontenc}
\usepackage[utf8]{inputenc}
\frenchspacing 

% usa i pacchetti per la formattazione matematica
\usepackage{amsmath, amssymb, amsthm, amsfonts}

% usa altri pacchetti
\usepackage{gensymb}
\usepackage{hyperref}
\usepackage{standalone}

% imposta il titolo
\title{Appunti Elettrotecnica}
\author{Luca Seggiani}
\date{25-09-24}

% imposta lo stile
% usa helvetica
\usepackage[scaled]{helvet}
% usa palatino
\usepackage{palatino}
% usa un font monospazio guardabile
\usepackage{lmodern}

\renewcommand{\rmdefault}{ppl}
\renewcommand{\sfdefault}{phv}
\renewcommand{\ttdefault}{lmtt}

% disponi teoremi
\usepackage{tcolorbox}
\newtcolorbox[auto counter, number within=section]{theorem}[2][]{%
	colback=blue!10, 
	colframe=blue!40!black, 
	sharp corners=northwest,
	fonttitle=\sffamily\bfseries, 
	title=Teorema~\thetcbcounter: #2, 
	#1
}


% disegni
\usepackage{pgfplots}
\pgfplotsset{width=10cm,compat=1.9}

% disponi definizioni
\newtcolorbox[auto counter, number within=section]{definition}[2][]{%
	colback=red!10,
	colframe=red!40!black,
	sharp corners=northwest,
	fonttitle=\sffamily\bfseries,
	title=Definizione~\thetcbcounter: #2,
	#1
}

% disponi codice
\usepackage{listings}
\usepackage[table]{xcolor}

\lstdefinestyle{codestyle}{
		backgroundcolor=\color{black!5}, 
		commentstyle=\color{codegreen},
		keywordstyle=\bfseries\color{magenta},
		numberstyle=\sffamily\tiny\color{black!60},
		stringstyle=\color{green!50!black},
		basicstyle=\ttfamily\footnotesize,
		breakatwhitespace=false,         
		breaklines=true,                 
		captionpos=b,                    
		keepspaces=true,                 
		numbers=left,                    
		numbersep=5pt,                  
		showspaces=false,                
		showstringspaces=false,
		showtabs=false,                  
		tabsize=2
}

\lstdefinestyle{shellstyle}{
		backgroundcolor=\color{black!5}, 
		basicstyle=\ttfamily\footnotesize\color{black}, 
		commentstyle=\color{black}, 
		keywordstyle=\color{black},
		numberstyle=\color{black!5},
		stringstyle=\color{black}, 
		showspaces=false,
		showstringspaces=false, 
		showtabs=false, 
		tabsize=2, 
		numbers=none, 
		breaklines=true
}

\lstdefinelanguage{javascript}{
	keywords={typeof, new, true, false, catch, function, return, null, catch, switch, var, if, in, while, do, else, case, break},
	keywordstyle=\color{blue}\bfseries,
	ndkeywords={class, export, boolean, throw, implements, import, this},
	ndkeywordstyle=\color{darkgray}\bfseries,
	identifierstyle=\color{black},
	sensitive=false,
	comment=[l]{//},
	morecomment=[s]{/*}{*/},
	commentstyle=\color{purple}\ttfamily,
	stringstyle=\color{red}\ttfamily,
	morestring=[b]',
	morestring=[b]"
}

% disponi sezioni
\usepackage{titlesec}

\titleformat{\section}
	{\sffamily\Large\bfseries} 
	{\thesection}{1em}{} 
\titleformat{\subsection}
	{\sffamily\large\bfseries}   
	{\thesubsection}{1em}{} 
\titleformat{\subsubsection}
	{\sffamily\normalsize\bfseries} 
	{\thesubsubsection}{1em}{}

% disponi alberi
\usepackage{forest}

\forestset{
	rectstyle/.style={
		for tree={rectangle,draw,font=\large\sffamily}
	},
	roundstyle/.style={
		for tree={circle,draw,font=\large}
	}
}

% disponi algoritmi
\usepackage{algorithm}
\usepackage{algorithmic}
\makeatletter
\renewcommand{\ALG@name}{Algoritmo}
\makeatother

% disponi numeri di pagina
\usepackage{fancyhdr}
\fancyhf{} 
\fancyfoot[L]{\sffamily{\thepage}}

\makeatletter
\fancyhead[L]{\raisebox{1ex}[0pt][0pt]{\sffamily{\@title \ \@date}}} 
\fancyhead[R]{\raisebox{1ex}[0pt][0pt]{\sffamily{\@author}}}
\makeatother

\begin{document}
% sezione (data)
\section{Lezione del 25-09-24}

% stili pagina
\thispagestyle{empty}
\pagestyle{fancy}

% testo
\subsection{Introduzione}
Il corso di elettrotecnica riguarda lo studio dei \textbf{circuiti elettrici} e dei \textbf{macchinari elettrici}.

\subsubsection{Analisi dei circuiti elettrici}

Le leggi di Maxwell vanno a descrivere come si evolvono, nel tempo e nello spazio, i campi elettrici e magnetici.
Purtroppo,  le equazioni di Maxwell sono equazioni differenziali e legate fra di loro, ergo si possono spesso avere solo soluzioni numeriche.
Esistono però casì particolari in cui si possono fare semplificazioni considerevoli.

Un \textbf{circuito elettrico} è formato da fili conduttori e \textbf{componenti circuitali}.
All'interno di un circuito si va a descrivere un'onda elettrica:

$$
	\psi(s,t)
$$

rappresentata come una funzione di spazio e tempo.
Poniamo ad esempio la funzione, sulla sola posizione x:

$$
	\psi(x, t) = y \sin{\left( \frac{2\pi}{\lambda}x - \frac{2\pi}{T} t \right)}
$$

Questa funzione ha comunque due variabili: la posizione $x$ e il tempo $t$.
Immaginiamo di prendere un punto $x_0$ sul circuito elettrico:

$$
	\psi(t) = y \sin{\left( \frac{2\pi}{\lambda}x_0 - \frac{2\pi}{T} t \right)}
$$

Con $x_0 = 0$, annulliamo il primo termine.
A questo punto abbiamo ottenuto una funzione in una sola variabile:

$$
\psi(x_0, t) = y \sin{\left( - \frac{2\pi}{T} t \right)}
$$

ovvero una sinusoide invertita che oscilla fra un massimo di $y$ e un minimo di $-y$.

Questo significa che, mettendoci sul punto $x_0 = 0$ del circuito elettrico, notiamo che il valore dell'onda elettrica varia nel tempo seguendo questa funzione sinousidale.

Possiamo fare il processo invrso: fissiamo il tempo $t$, e vediamo come varia l'onda elettrica su diverse posizioni $x$ nel circuito.
Abbiamo, simbolicamente:

$$
	\psi(x) = y \sin{\left( \frac{2\pi}{\lambda}x - \frac{2\pi}{T} t_0 \right)}
$$

da cui ricaviamo l'equazione in una sola variabile $t$:

$$
	\psi(x) = y \sin{\left( \frac{2\pi}{\lambda}x \right)}
$$

ovvero una sinusoide che, come prima, oscilla fra un massimo di $y$ e un minimo di $-y$.
Si riporta un grafico:

\begin{center}
\begin{tikzpicture}
    \begin{axis}[
        xlabel={$x$},
        ylabel={$\psi(x)$},
        domain=-10:10, % set the x range you want
				samples=100,
        grid=major, % add a grid
				ytick={-2, 2},
				yticklabels={$-y$, $y$},
				ymin = -3, ymax = 3,
				width=15cm,
				height=7cm
    ]
    \addplot[
        blue,
        thick
    ] {2 * sin(50*x)}; 
    \end{axis}
\end{tikzpicture}
\end{center}

Questo significa che, all'istante $t_0 = 0$ notiamo che il valore dell'onda elettrica varia sulla lunghezza del circuito seguendo ancora questa funzione sinousidale.

Possiamo provare a calcolare lunghezza d'onda e periodo di questa oscillazione: visto che il periodo del seno è $2\pi$, abbiamo che nello spazio la lunghezza d'onda è $\lambda$ e nel tempo il periodo è $T$.

Proviamo a calcolare $\lambda$: sappiamo che la lunghezza d'onda equivale alla velocità di propagazione sulla frequenza dell'oscillazione, ovvero:

$$
\lambda = \frac{v}{f}
$$

Posti i valori $300 \cdot 10^6 \ \mathrm{m/s}$ per $v$ e $50 \ \text{Hz}$ per $f$ (la frequenza della rete elettrica), abbiamo:

$$
\lambda = \frac{3.00 \cdot 10^6 \ \mathrm{m/s}}{50 \ \text{Hz}} = 6000 \ \text{km} 
$$

Questa lunghezza d'onda diventa rilevante in trasmissioni elettriche su larga scala.

Possiamo fare considerazioni diverse se prendiamo in esempio le comunicazioni radio: lì si parla di frequenze $f >> 50 \ \text{Hz}$, nell'ordine dei megahertz o gigahertz.

L'elevata velocità della corrente ci permette di fare un'importante approssimazione e considerare \textbf{circuiti a parametri concetrati}.
Quest'ipotesi, in inglese \textit{lumped element model}, ci permette di ignorare l'estensione fisica del circuito, e quindi le variazioni delle funzioni d'onda sulla variabille spazio $s$, concentradosi sulla variabile tempo $t$.

\subsection{Grandezze}
Si usano le seguenti grandezze:
\subsubsection{Intensità di corrente}

\begin{definition}{Corrente elettrica}	
Si indica con $I$ la corrente elettrica, misurata in Ampere [$\mathrm{A}$], e definita come la variazione di carica:
$$
I = \frac{dq}{dt}
$$
\end{definition}

Si prende come positivo il verso in cui si muovono i portatori di carica positive, anche se sappiamo nella stragrande maggioranza dei casi i portatori di carica essere negativi, e quindi il movimento vero e proprio degli elettroni in direzione opposta.

Notiamo che se un segmento di circuito da $A$ a $B$ si ha una corrente $I_{AB}$, vale:

$$
I_{AB} = -I_{BA}
$$

\subsubsection{Differenza di potenziale}

\begin{definition}{Differenza di potenziale}
Si indica con $V$ la differenza di potenziale o \textit{tensione}, misurata in Volt [$\mathrm{V}$], e definita come il lavoro necessario a spostare una carica elementare positiva da un punto $A$ ad un punto $B$ sulla carica:

$$
	V_{AB}(t) = \frac{L_{AB}(t)}{q(t)}
$$
\end{definition}

Il segno del potenziale è definito come \textit{positivo} quando si deve vincere il campo magnetico per spingere la carica, ergo il campo elettrico svolge lavoro \textit{negativo} sulla carica.
Come prima, su segmenti di circuito da $A$ a $B$ vale:

$$
V_{AB} = -V_{BA}
$$

\subsubsection{Riferimenti associati e non associati}
I componenti circuitali, presi a sé, vengono detti \textbf{bipoli elettrici}, dal fatto che hanno 2 poli.
Di un bipolo elettrico si può misurare la differenza di potenziale ai capi e la corrente che vi scorre attraverso.

Quando si parla di tensione, o si parla di differenze di potenziale, o si assume un riferimento (lo zero del potenziale).
Non possiamo sapere a priori se il potenziale al capo di un bipolo è maggiore del potenziale all'altro capo: bisogna prima scegliere una direzione e poi vedere se il segno ricavato è concorde o meno con la nostra scelta.

Lo stesso vale per la corrente.
I riferimenti concordi al verso della corrente si dicono \textbf{associati}, quelli discordi si dicono \textbf{non associati}.

\subsubsection{Potenza elettrica}

\begin{definition}{Potenza elettrica}
Si indica con $P$ la potenza elettrica, misurata in Watt [$\mathrm{W}$] e definita come il prodotto:
$$
	P = IV
$$
fra corrente e tensione.
\end{definition}

Anche la potenza ha un segno, che in questo caso si riferisce a potenza \textit{erogata} o \textit{dissipata}.
La potenza calcolata sui riferimenti associati positiva è dissipata, quella negativa è erogata.
Viceversa, la potenza calcolata sui riferimenti non associati positiva è erogata, quella negativa è dissipata.

\subsubsection{Energia}

\begin{definition}{Energia}
Si indica con $W$ (non Watt!) l'energia, misurata in Joule [$\mathrm{J}$], o in Kilowatt/ora ($\mathrm{KW/h}$), e definita come l'integrale sul tempo della potenza:
$$
W = \int_{-\infty}^t P \ dt 
$$
\end{definition}

\subsection{Leggi di Kirchoff}
Iniziamo col dare dei nomi a particolari punti del circuito elettrico: i punti di incontro di più fili prendono il nome di \textbf{nodi}, e i fili che collegano i bipoli ai nodi prendono il nome di \textbf{rami}.
Da questo abbiamo che nei nodi si incontrano 3 o più rami.

Da qui possiamo definire la legge:
\begin{theorem}{Prima legge di Kirchoff}
		La somma algebrica delle correnti dei rami tagliati da una linea chiusa è uguale a 0.
		In particolare, la somma algebrica delle correnti entranti e uscenti da un nodo è uguale a 0.
\end{theorem}

Definiamo quindi il concetto di \textbf{maglia}: una maglia è un percorso chiuso di nodi e rami, ovvero un sottoinsieme di rami tali per cui spostandosi da un nodo all'altro si percorre ogni ramo una sola volta.
Sulle maglie si ha:

\begin{theorem}{Seconda legge di Kirchoff}
	La somma algebrica delle cadute di potenziale lungo una maglia è uguale a 0.
\end{theorem}

\end{document}


\documentclass[a4paper,11pt]{article}
\usepackage[a4paper, margin=8em]{geometry}

% usa i pacchetti per la scrittura in italiano
\usepackage[french,italian]{babel}
\usepackage[T1]{fontenc}
\usepackage[utf8]{inputenc}
\frenchspacing 

% usa i pacchetti per la formattazione matematica
\usepackage{amsmath, amssymb, amsthm, amsfonts}

% usa altri pacchetti
\usepackage{gensymb}
\usepackage{hyperref}
\usepackage{standalone}

% imposta il titolo
\title{Appunti Elettrotecnica}
\author{Luca Seggiani}
\date{2024}

% imposta lo stile
% usa helvetica
\usepackage[scaled]{helvet}
% usa palatino
\usepackage{palatino}
% usa un font monospazio guardabile
\usepackage{lmodern}

\renewcommand{\rmdefault}{ppl}
\renewcommand{\sfdefault}{phv}
\renewcommand{\ttdefault}{lmtt}

% disponi il titolo
\makeatletter
\renewcommand{\maketitle} {
	\begin{center} 
		\begin{minipage}[t]{.8\textwidth}
			\textsf{\huge\bfseries \@title} 
		\end{minipage}%
		\begin{minipage}[t]{.2\textwidth}
			\raggedleft \vspace{-1.65em}
			\textsf{\small \@author} \vfill
			\textsf{\small \@date}
		\end{minipage}
		\par
	\end{center}

	\thispagestyle{empty}
	\pagestyle{fancy}
}
\makeatother

% U.D.M
\newcommand{\amp}{\ensuremath{\mathrm{A}}}
\newcommand{\volt}{\ensuremath{\mathrm{V}}}
\newcommand{\meter}{\ensuremath{\mathrm{m}}}
\newcommand{\second}{\ensuremath{\mathrm{s}}}
\newcommand{\farad}{\ensuremath{\mathrm{F}}}
\newcommand{\henry}{\ensuremath{\mathrm{H}}}
\newcommand{\siemens}{\ensuremath{\mathrm{S}}}

% disponi teoremi
\usepackage{tcolorbox}
\newtcolorbox[auto counter, number within=section]{theorem}[2][]{%
	colback=blue!10, 
	colframe=blue!40!black, 
	sharp corners=northwest,
	fonttitle=\sffamily\bfseries, 
	title=Teorema~\thetcbcounter: #2, 
	#1
}

% disponi definizioni
\newtcolorbox[auto counter, number within=section]{definition}[2][]{%
	colback=red!10,
	colframe=red!40!black,
	sharp corners=northwest,
	fonttitle=\sffamily\bfseries,
	title=Definizione~\thetcbcounter: #2,
	#1
}

% disegni
\usepackage{pgfplots}
\pgfplotsset{width=10cm,compat=1.9}

% disponi codice
\usepackage{listings}
\usepackage[table]{xcolor}

\lstdefinestyle{codestyle}{
		backgroundcolor=\color{black!5}, 
		commentstyle=\color{codegreen},
		keywordstyle=\bfseries\color{magenta},
		numberstyle=\sffamily\tiny\color{black!60},
		stringstyle=\color{green!50!black},
		basicstyle=\ttfamily\footnotesize,
		breakatwhitespace=false,         
		breaklines=true,                 
		captionpos=b,                    
		keepspaces=true,                 
		numbers=left,                    
		numbersep=5pt,                  
		showspaces=false,                
		showstringspaces=false,
		showtabs=false,                  
		tabsize=2
}

\lstdefinestyle{shellstyle}{
		backgroundcolor=\color{black!5}, 
		basicstyle=\ttfamily\footnotesize\color{black}, 
		commentstyle=\color{black}, 
		keywordstyle=\color{black},
		numberstyle=\color{black!5},
		stringstyle=\color{black}, 
		showspaces=false,
		showstringspaces=false, 
		showtabs=false, 
		tabsize=2, 
		numbers=none, 
		breaklines=true
}

\lstdefinelanguage{javascript}{
	keywords={typeof, new, true, false, catch, function, return, null, catch, switch, var, if, in, while, do, else, case, break},
	keywordstyle=\color{blue}\bfseries,
	ndkeywords={class, export, boolean, throw, implements, import, this},
	ndkeywordstyle=\color{darkgray}\bfseries,
	identifierstyle=\color{black},
	sensitive=false,
	comment=[l]{//},
	morecomment=[s]{/*}{*/},
	commentstyle=\color{purple}\ttfamily,
	stringstyle=\color{red}\ttfamily,
	morestring=[b]',
	morestring=[b]"
}

% disponi sezioni
\usepackage{titlesec}

\titleformat{\section}
	{\sffamily\Large\bfseries} 
	{\thesection}{1em}{} 
\titleformat{\subsection}
	{\sffamily\large\bfseries}   
	{\thesubsection}{1em}{} 
\titleformat{\subsubsection}
	{\sffamily\normalsize\bfseries} 
	{\thesubsubsection}{1em}{}

% disponi alberi
\usepackage{forest}

\forestset{
	rectstyle/.style={
		for tree={rectangle,draw,font=\large\sffamily}
	},
	roundstyle/.style={
		for tree={circle,draw,font=\large}
	}
}

% disponi algoritmi
\usepackage{algorithm}
\usepackage{algorithmic}
\makeatletter
\renewcommand{\ALG@name}{Algoritmo}
\makeatother

% disponi numeri di pagina
\usepackage{fancyhdr}
\fancyhf{} 
\fancyfoot[L]{\sffamily{\thepage}}

\makeatletter
\fancyhead[L]{\raisebox{1ex}[0pt][0pt]{\sffamily{\@title \ \@date}}} 
\fancyhead[R]{\raisebox{1ex}[0pt][0pt]{\sffamily{\@author}}}
\makeatother

% circuiti
\usepackage{circuitikz}
\usetikzlibrary{babel}

\begin{document}
% sezione (data)
\section{Lezione del 26-09-24}

% stili pagina
\thispagestyle{empty}
\pagestyle{fancy}

% testo
\subsection{Bipolo elettrico}
Abbiamo introdotto i componenti circuitali come \textbf{bipoli elettrici}.
In particolare, diciamo che un bipolo elettrico è un componente, con una certa differenza di potenziale $V_{AB}$ ai suoi capi e una corrente $i_{AB}(t)$ che vi scorre all'interno, tale per cui si può definire una funzione del tipo:
$$
V_{AB} = f(i_{AB}(t))
$$

Possiamo individuare alcune caratteristiche importanti dei bipoli:
\begin{itemize}
	\item \textbf{Linearità:} un bipolo si dice lineare se la funzione che lega voltaggio e corrente è lineare.
		Tutti i bipoli che studieremo sono lineari (resistenze, capacitori, ecc...).
		Esistono però svariati bipoli che hanno risposte non lineari ai voltaggi/correnti a cui vengono sottoposti (diodi (risposte diverse a direzioni diverse della corrente), amplificatori operazionali, ecc...).
	\item \textbf{Tempo invarianza:} un bipolo si dice tempo invariante quando le sue caratteristiche non variano nel tempo.
	\item \textbf{Memoria:} un bipolo si dice dotato di memoria quando i suoi valori di corrente e tensione attuali dipendono da valori di corrente e tensioni ad un'istante $t$ precedente.
		I bipoli dotati di memoria presentano solitamente \textit{cicli di isteresi}.
	\item \textbf{Passività/attività:} si dice \textbf{passivo} un bipolo che dissipa potenza, e \textbf{attivo} un bipolo che la eroga.
		Più propriamente, si ha che un bipolo e passivo quando l'energia su di esso, presa un riferimento associato, è $\geq 0$.
\end{itemize}

\subsection{Resistori}
Un resistore è un componente circuitale caratterizzato dalla legge di Ohm ($ J = \sigma E$), e quindi formato da un materiale \textit{ohmico} che ha risposta lineare in densità di corrente alle variazioni del campo.
Si indica come:

\begin{center}
\begin{circuitikz}
\draw (0,0) to[ resistor ] (2,0); 
\end{circuitikz}
\end{center}

\begin{theorem}{Prima legge di Ohm}	
Il voltaggio è legato alla corrente, in un resistore, secondo la relazione:
$$V_R(t) = R \ i_R(t)$$
\end{theorem}

dove R prende il nome di \textbf{resistenza}, misurata in Ohm [$\ohm$], definita come:
$$
R = \frac{V}{i}
$$

\subsubsection{Resistenza e resistività}
Conosciamo la legge di Ohm sui materiali ohmici riportata prima.
Da questa legge si ricava:
\begin{theorem}{Seconda legge di Ohm}
	La resistenza di un filo di lunghezza $l$ e sezione $s$ è data da:
	$$
		R = \rho \frac{l}{s}
	$$
\end{theorem}
dove $\rho$ prende il nome di \textbf{resistività}, misurata in Ohm per metro [$\ohm \cdot \meter$].

Questo significa che la resistenza cresce con il crescere della lunghezza, e diminuisce con il crescere della sezione.

In verità questa non sono le uniche caratteristiche che influenzano la resistenza: un apporto significativo è dato anche dalla \textbf{temperatura}, alla quale la resistenza ha proporzionalità quasi lineare, ma che noi ignoreremo.

\subsubsection{Conduttanza e conducibilità}
Conviene definire altre due unità di misura: l'inverso della resistenza, detta \textbf{conduttanza}, che si misura in Siemens [$\ohm^{-1} = \siemens$], o in \textbf{mho} [$\mho = \ohm^{-1}$]:
$$
G = \frac{1}{R}
$$
e l'inverso della resistività, detta \textbf{conducibilità}, che si misura in [$\ohm^{-1} \cdot \meter^{-1}$]:
$$
\sigma = \frac{1}{\rho}
$$

\par\smallskip

I resistori sono inoltre:
\begin{itemize}
	\item Tempo invarianti (a patto di trascurare la temperatura);
	\item Senza memoria;
	\item Passivi (dissipano potenza per \textbf{effetto Joule}).
		Ciò si può dimostrare calcolando la potenza dalla prima legge di Ohm:
		$$
		p(t) = v_{AB}(t) \cdot i_{AB}(t) = R \ i_{AB}^2(t) \geq 0
		$$
		e calcolando l'energia come integrale:
		$$
		w(t) = \int_{-\infty}^{t} p(t)dt \Rightarrow w(t) > 0
		$$
\end{itemize}

\subsubsection{Circuiti aperti/chiusi}
Le resistenza, sopratutto nei loro casi limite, aiutano a modellizzare varie parti di un circuito:
\begin{itemize}
	\item \textbf{Cortocircuito:} indicato da una resistenza nulla, ergo:
		$$
			V_{AB}(t) = 0 \Leftrightarrow R = 0
		$$
		
		Modellizza il filo ideale, ergo ciò che per noi è un ramo.
	\item \textbf{Circuito aperto:} indicato da una corrente nulla, ergo:
		$$
			i_{AB} = 0 \Leftrightarrow R = +\infty
		$$

		Modellizza interruzioni nel circuito: si può dimostrare che la corrente attraverso un'interruzione in un circuito è nulla sfruttando la prima legge di Kirchoff: una linea chiusa che comprende il nodo finale di un'interruzione avrà un ramo entrante e 0 uscenti, ovvero corrente entrante nulla.
\end{itemize}

\subsubsection{Resistenze in serie}
Poniamo di avere una configurazione di resistenze del tipo:

\begin{center}
\begin{circuitikz}
    \draw (0,0) node[left] {$A$} 
        to[R, l=$R_1$] (2,0) 
        to[R, l=$R_2$] (4,0) 
        to[short] (5,0)
        node[anchor=west,xshift=0.15cm] {\dots} (6,0) 
        to[short] (7,0)
        to[R, l=$R_n$] (9,0) node[right] {$B$};

    \draw[decorate,decoration={brace,amplitude=10pt,mirror}] (0,-1) -- (9,-1)
        node[midway,below=10pt]{$n$ resistori};
\end{circuitikz}
\end{center}

Vogliamo calcolare una resistenza $R_{eq}$ che valga quando la resistenza cumulativa di tutte e $n$ le resistenze.
Abbiamo allora che la corrente lungo ogni resistenza $i(t)$ è costante, mentre ogni resistenza contribuisce al potenziale $V_{AB}$ con una certa caduta di potenziale $V_1(t), V_2(t), ..., V_n(T)$.
Si applica quindi la prima legge di Ohm:
$$
V_{AB} = V_1{t} + V_2{t} + ... + V_n{t} = R_1 \cdot i(t) + R_2 \cdot i(t) + ... + R_n \cdot i(t) = i(t) \cdot \left( R_1 + R_2 + ... + R_n \right)
$$
quindi, da $V_{AB} = i(t) \ R_{eq}$ si ha:

\begin{theorem}{Resistenze in serie}	
$$
R_{eq} = R_1 + R_2 + ... + R_n
$$
\end{theorem}

\subsubsection{Resistenze in parallelo}
Poniamo di avere le resistenze in parallelo invece che in serie:
\begin{center}
\begin{circuitikz}
    \draw (2,0) 
				to[short] (0,0) node[left] {$A$};
    
    \draw (2,-2) 
				to[short] (0,-2) node[left] {$B$};
		
		\draw (2,0 )to[R, l=$R_1$] (2,-2);
		\draw (4,0 )to[R, l=$R_2$] (4,-2);
		
		
    \draw (4,0) node[anchor=west,xshift=0.7cm] {\dots} (6,0);
    \draw (4,-2) node[anchor=west,xshift=0.7cm] {\dots} (6,-2); 
		
		\draw (8,0 )to[R, l=$R_n$] (8,-2);
        
    % Draw horizontal lines at the top and bottom
    \draw (0,0) -- (4,0);
    \draw (0,-2) -- (4,-2);

    \draw (6,0) -- (8,0);
    \draw (6,-2) -- (8,-2);

    % Bracket below the resistors
    \draw[decorate,decoration={brace,amplitude=10pt,mirror}] (0,-3) -- (8,-3)
        node[midway,below=10pt]{$n$ resistori};
\end{circuitikz}
\end{center}

Vogliamo ancora calcolare una resistenza $R_{eq}$ che valga quando la resistenza cumulativa di tutte e $n$ le resistenze.
Qui abbiamo che la differenza di potenziale lungo ogni resistenza $V(t)$ costante.
Si applica ancora la prima legge di Ohm:
$$
i = i_1(t) + i_2(t) + ... + i_n(t) = \frac{V_{AB}(t)}{R_1} + \frac{V_{AB}(t)}{R_2} + ... + \frac{V_{AB}(t)}{R_n}
$$
conviene raccogliere e passare alle conduttanze:
$$
G_{eq} = V_{AB}(t)(G_1 + G_2 + ... + G_n) = G_{eq} \cdot V_{AB}(t)
$$
Ora, se $G = \frac{1}{R}$:
$$
R_{eq} = G_{eq}^{-1} = \left( \frac{1}{R_1} + \frac{1}{R_2} + ... + \frac{1}{R_n} \right)^{-1} 
$$
quindi, si ha:

\begin{theorem}{Resistenze in parallelo}
$$
R_{eq} = \left( \frac{1}{R_1} + \frac{1}{R_2} + ... + \frac{1}{R_n} \right)^{-1}
$$
\end{theorem}

\end{document}


\documentclass[a4paper,11pt]{article}
\usepackage[a4paper, margin=8em]{geometry}

% usa i pacchetti per la scrittura in italiano
\usepackage[french,italian]{babel}
\usepackage[T1]{fontenc}
\usepackage[utf8]{inputenc}
\frenchspacing 

% usa i pacchetti per la formattazione matematica
\usepackage{amsmath, amssymb, amsthm, amsfonts}

% usa altri pacchetti
\usepackage{gensymb}
\usepackage{hyperref}
\usepackage{standalone}

% imposta il titolo
\title{Appunti Elettrotecnica}
\author{Luca Seggiani}
\date{2024}

% imposta lo stile
% usa helvetica
\usepackage[scaled]{helvet}
% usa palatino
\usepackage{palatino}
% usa un font monospazio guardabile
\usepackage{lmodern}

\renewcommand{\rmdefault}{ppl}
\renewcommand{\sfdefault}{phv}
\renewcommand{\ttdefault}{lmtt}

% disponi il titolo
\makeatletter
\renewcommand{\maketitle} {
	\begin{center} 
		\begin{minipage}[t]{.8\textwidth}
			\textsf{\huge\bfseries \@title} 
		\end{minipage}%
		\begin{minipage}[t]{.2\textwidth}
			\raggedleft \vspace{-1.65em}
			\textsf{\small \@author} \vfill
			\textsf{\small \@date}
		\end{minipage}
		\par
	\end{center}

	\thispagestyle{empty}
	\pagestyle{fancy}
}
\makeatother

% disponi teoremi
\usepackage{tcolorbox}
\newtcolorbox[auto counter, number within=section]{theorem}[2][]{%
	colback=blue!10, 
	colframe=blue!40!black, 
	sharp corners=northwest,
	fonttitle=\sffamily\bfseries, 
	title=~\thetcbcounter: #2, 
	#1
}

% disponi definizioni
\newtcolorbox[auto counter, number within=section]{definition}[2][]{%
	colback=red!10,
	colframe=red!40!black,
	sharp corners=northwest,
	fonttitle=\sffamily\bfseries,
	title=~\thetcbcounter: #2,
	#1
}

% U.D.M
\newcommand{\amp}{\ensuremath{\mathrm{A}}}
\newcommand{\volt}{\ensuremath{\mathrm{V}}}
\newcommand{\meter}{\ensuremath{\mathrm{m}}}
\newcommand{\second}{\ensuremath{\mathrm{s}}}
\newcommand{\farad}{\ensuremath{\mathrm{F}}}
\newcommand{\henry}{\ensuremath{\mathrm{H}}}
\newcommand{\siemens}{\ensuremath{\mathrm{S}}}

% circuiti
\usepackage{circuitikz}
\usetikzlibrary{babel}

% disegni
\usepackage{pgfplots}
\pgfplotsset{width=10cm,compat=1.9}

% disponi codice
\usepackage{listings}
\usepackage[table]{xcolor}

\lstdefinestyle{codestyle}{
		backgroundcolor=\color{black!5}, 
		commentstyle=\color{codegreen},
		keywordstyle=\bfseries\color{magenta},
		numberstyle=\sffamily\tiny\color{black!60},
		stringstyle=\color{green!50!black},
		basicstyle=\ttfamily\footnotesize,
		breakatwhitespace=false,         
		breaklines=true,                 
		captionpos=b,                    
		keepspaces=true,                 
		numbers=left,                    
		numbersep=5pt,                  
		showspaces=false,                
		showstringspaces=false,
		showtabs=false,                  
		tabsize=2
}

\lstdefinestyle{shellstyle}{
		backgroundcolor=\color{black!5}, 
		basicstyle=\ttfamily\footnotesize\color{black}, 
		commentstyle=\color{black}, 
		keywordstyle=\color{black},
		numberstyle=\color{black!5},
		stringstyle=\color{black}, 
		showspaces=false,
		showstringspaces=false, 
		showtabs=false, 
		tabsize=2, 
		numbers=none, 
		breaklines=true
}

\lstdefinelanguage{javascript}{
	keywords={typeof, new, true, false, catch, function, return, null, catch, switch, var, if, in, while, do, else, case, break},
	keywordstyle=\color{blue}\bfseries,
	ndkeywords={class, export, boolean, throw, implements, import, this},
	ndkeywordstyle=\color{darkgray}\bfseries,
	identifierstyle=\color{black},
	sensitive=false,
	comment=[l]{//},
	morecomment=[s]{/*}{*/},
	commentstyle=\color{purple}\ttfamily,
	stringstyle=\color{red}\ttfamily,
	morestring=[b]',
	morestring=[b]"
}

% disponi sezioni
\usepackage{titlesec}

\titleformat{\section}
	{\sffamily\Large\bfseries} 
	{\thesection}{1em}{} 
\titleformat{\subsection}
	{\sffamily\large\bfseries}   
	{\thesubsection}{1em}{} 
\titleformat{\subsubsection}
	{\sffamily\normalsize\bfseries} 
	{\thesubsubsection}{1em}{}

% disponi alberi
\usepackage{forest}

\forestset{
	rectstyle/.style={
		for tree={rectangle,draw,font=\large\sffamily}
	},
	roundstyle/.style={
		for tree={circle,draw,font=\large}
	}
}

% disponi algoritmi
\usepackage{algorithm}
\usepackage{algorithmic}
\makeatletter
\renewcommand{\ALG@name}{Algoritmo}
\makeatother

% disponi numeri di pagina
\usepackage{fancyhdr}
\fancyhf{} 
\fancyfoot[L]{\sffamily{\thepage}}

\makeatletter
\fancyhead[L]{\raisebox{1ex}[0pt][0pt]{\sffamily{\@title \ \@date}}} 
\fancyhead[R]{\raisebox{1ex}[0pt][0pt]{\sffamily{\@author}}}
\makeatother

\begin{document}
% sezione (data)
\section{Lezione del 27-09-24}

% stili pagina
\thispagestyle{empty}
\pagestyle{fancy}

% testo
\subsubsection{Resistenza e cortocircuito in parallelo}
Poniamo di avere la configurazione:

\begin{center}
\begin{circuitikz}
    \draw (2,0) 
				to[short] (0,0) node[left] {$A$};
    
    \draw (2,-2) 
				to[short] (0,-2) node[left] {$B$};
		
		\draw (2,0 )to[R, l=$R$] (2,-2);
		\draw (4,0 )to[short] (4,-2);
		
		    
    % Draw horizontal lines at the top and bottom
    \draw (0,0) -- (4,0);
    \draw (0,-2) -- (4,-2);
\end{circuitikz}
\end{center}

Dove un resistore è in parallelo ad un corto circuito.
Intuitivamente, tutta la corrente passerà dal cortocircuito, e non dalla resistenza.
Possiamo modellizzare questo fatto in due modi:
\begin{itemize}
	\item Attraverso la formula per le resistenze in parallelo, avremo che:
		$$
		R_{eq} = \left( \frac{1}{R_1} + \frac{1}{R_2} \right)^{-1} = \frac{R_1R_2}{R_1 + R_2}, \quad R_1 = 0 \Rightarrow R_{eq} = \frac{0}{R_2} = 0
		$$	
		ergo resistenza nulla.
		
		La prima trasformazione è necessaria in quanto rimuove i vincoli sul dominio di $R_1$ e $R_2$ (che altrimenti non potrebbero essere 0).
	\item Notiamo che A e B sono effettivamente allo stesso potenziale, ergo abbiamo differenza di potenziale $V_{AB} = 0$ ai capi della resistenza.
		Applicando quindi la prima legge di Ohm $V_{AB} = i(t)R$ si ha $i(t) = 0$, cioè corrente costante nulla sulla resistenza.
\end{itemize}

\subsection{Altre configurazioni di resistenze}
Esistono altri modi di configurare le resistenze, che permettono di studiare circuiti su cui i metodi studiati finora non funzionano.

\subsubsection{Resistenze a triangolo}
Nelle resistenze a triangolo, una singola maglia di 3 nodi forma un triangolo con i lati 3 resistenze:

\begin{center}
\begin{circuitikz}
    \draw (1.73,1) node[right] {$1$}
				to[R, l=$R_{12}$] (-1.73,1) ;

    \draw (0,-2) 
				to[R, l=$R_{23}$] (-1.73,1) node[left] {$2$};

    \draw (1.73,1) 
				to[R, l=$R_{13}$] (0,-2) node[below] {$3$};

\end{circuitikz}
\end{center}

\subsubsection{Resistenze a stella}
Nelle resistenze a stella, più resistenze vengono collegate, da un'estremo, ad un singolo nodo centrale:

\begin{center}
\begin{circuitikz}
    \draw (0,0) 
				to[R, l=$R_1$] (1.73,1) node[right] {$1$};

    \draw (-1.73,1) node[left] {$2$}
				to[R, l=$R_2$] (0,0);

    \draw (0,-2) node[below] {$3$}
				to[R, l=$R_3$] (0,0);

\end{circuitikz}
\end{center}

\par\smallskip
Si possono trasformare resistenze a triangolo in resistenze a stella aggiungendo un nodo centrale $O$ e collegandovi i 3 nodi già esistenti attraverso le resistenze interne:

\begin{theorem}{Resistenze da triangolo a stella}	
$$
R_1 = \frac{R_{12}R_{13}}{R_{12} + R_{13} + R_{23}}
$$
$$
R_2 = \frac{R_{12}R_{23}}{R_{12} + R_{13} + R_{23}}
$$
$$
R_3 = \frac{R_{23}R_{13}}{R_{12} + R_{13} + R_{23}}
$$
\end{theorem}

Allo stesso modo, si possono trasformare resistenze a stella in resistenze a triangolo unendo i nodi fra di loro attraverso le resistenze esterne:

\begin{theorem}{Resistenze da stella a triangolo}	
$$
R_{12} = \frac{R_1R_2 + R_1R_3 + R_2R_3}{R_3}
$$
$$
R_{13} = \frac{R_1R_2 + R_1R_3 + R_2R_3}{R_2}
$$
$$
R_{23} = \frac{R_1R_2 + R_1R_3 + R_2R_3}{R_1}
$$
\end{theorem}

\subsection{Algoritmo per la resistenza equivalente}
A questo punto, si possono semplificare circuiti di resistori arbitrari applicando l'algoritmo:
\begin{algorithm}
\caption{Calcolo della resistenza equivalente}
\begin{algorithmic}
	\WHILE{ci sono $>1$ resistenze}

	\STATE Semplificare le resistenze in serie
	\STATE Semplificare le resistenze in parallelo
	
	\STATE Se non hai semplificato niente, trasforma un triangolo in stella o viceversa.

	\ENDWHILE
\end{algorithmic}
\end{algorithm}

La resistenza equivalente è a volte detta anche \textit{resistenza vista}. 
Questo perchè l'intero circuito si comporterà, per una qualsiasi rete esterna, come un singolo resistore di resistenza $R_{eq}$, ovvero avrà la stessa \textbf{risposta} di un singolo resistore di resistenza $R_{eq}$.
Analiticamente, questo significa che la funzione $f$ in $v(t) = f(i(t))$ (o la sua inversa) sono uguali per i due circuiti.
\end{document}


\documentclass[a4paper,11pt]{article}
\usepackage[a4paper, margin=8em]{geometry}

% usa i pacchetti per la scrittura in italiano
\usepackage[french,italian]{babel}
\usepackage[T1]{fontenc}
\usepackage[utf8]{inputenc}
\frenchspacing 

% usa i pacchetti per la formattazione matematica
\usepackage{amsmath, amssymb, amsthm, amsfonts}

% usa altri pacchetti
\usepackage{gensymb}
\usepackage{hyperref}
\usepackage{standalone}

% imposta il titolo
\title{Appunti Elettrotecnica}
\author{Luca Seggiani}
\date{2024}

% imposta lo stile
% usa helvetica
\usepackage[scaled]{helvet}
% usa palatino
\usepackage{palatino}
% usa un font monospazio guardabile
\usepackage{lmodern}

\renewcommand{\rmdefault}{ppl}
\renewcommand{\sfdefault}{phv}
\renewcommand{\ttdefault}{lmtt}

% disponi il titolo
\makeatletter
\renewcommand{\maketitle} {
	\begin{center} 
		\begin{minipage}[t]{.8\textwidth}
			\textsf{\huge\bfseries \@title} 
		\end{minipage}%
		\begin{minipage}[t]{.2\textwidth}
			\raggedleft \vspace{-1.65em}
			\textsf{\small \@author} \vfill
			\textsf{\small \@date}
		\end{minipage}
		\par
	\end{center}

	\thispagestyle{empty}
	\pagestyle{fancy}
}
\makeatother

% disponi teoremi
\usepackage{tcolorbox}
\newtcolorbox[auto counter, number within=section]{theorem}[2][]{%
	colback=blue!10, 
	colframe=blue!40!black, 
	sharp corners=northwest,
	fonttitle=\sffamily\bfseries, 
	title=~\thetcbcounter: #2, 
	#1
}

% disponi definizioni
\newtcolorbox[auto counter, number within=section]{definition}[2][]{%
	colback=red!10,
	colframe=red!40!black,
	sharp corners=northwest,
	fonttitle=\sffamily\bfseries,
	title=~\thetcbcounter: #2,
	#1
}

% U.D.M
\newcommand{\amp}{\ensuremath{\mathrm{A}}}
\newcommand{\volt}{\ensuremath{\mathrm{V}}}
\newcommand{\meter}{\ensuremath{\mathrm{m}}}
\newcommand{\second}{\ensuremath{\mathrm{s}}}
\newcommand{\farad}{\ensuremath{\mathrm{F}}}
\newcommand{\henry}{\ensuremath{\mathrm{H}}}
\newcommand{\siemens}{\ensuremath{\mathrm{S}}}

% circuiti
\usepackage{circuitikz}
\usetikzlibrary{babel}

% disegni
\usepackage{pgfplots}
\pgfplotsset{width=10cm,compat=1.9}

% disponi codice
\usepackage{listings}
\usepackage[table]{xcolor}

\lstdefinestyle{codestyle}{
		backgroundcolor=\color{black!5}, 
		commentstyle=\color{codegreen},
		keywordstyle=\bfseries\color{magenta},
		numberstyle=\sffamily\tiny\color{black!60},
		stringstyle=\color{green!50!black},
		basicstyle=\ttfamily\footnotesize,
		breakatwhitespace=false,         
		breaklines=true,                 
		captionpos=b,                    
		keepspaces=true,                 
		numbers=left,                    
		numbersep=5pt,                  
		showspaces=false,                
		showstringspaces=false,
		showtabs=false,                  
		tabsize=2
}

\lstdefinestyle{shellstyle}{
		backgroundcolor=\color{black!5}, 
		basicstyle=\ttfamily\footnotesize\color{black}, 
		commentstyle=\color{black}, 
		keywordstyle=\color{black},
		numberstyle=\color{black!5},
		stringstyle=\color{black}, 
		showspaces=false,
		showstringspaces=false, 
		showtabs=false, 
		tabsize=2, 
		numbers=none, 
		breaklines=true
}

\lstdefinelanguage{javascript}{
	keywords={typeof, new, true, false, catch, function, return, null, catch, switch, var, if, in, while, do, else, case, break},
	keywordstyle=\color{blue}\bfseries,
	ndkeywords={class, export, boolean, throw, implements, import, this},
	ndkeywordstyle=\color{darkgray}\bfseries,
	identifierstyle=\color{black},
	sensitive=false,
	comment=[l]{//},
	morecomment=[s]{/*}{*/},
	commentstyle=\color{purple}\ttfamily,
	stringstyle=\color{red}\ttfamily,
	morestring=[b]',
	morestring=[b]"
}

% disponi sezioni
\usepackage{titlesec}

\titleformat{\section}
	{\sffamily\Large\bfseries} 
	{\thesection}{1em}{} 
\titleformat{\subsection}
	{\sffamily\large\bfseries}   
	{\thesubsection}{1em}{} 
\titleformat{\subsubsection}
	{\sffamily\normalsize\bfseries} 
	{\thesubsubsection}{1em}{}

% disponi alberi
\usepackage{forest}

\forestset{
	rectstyle/.style={
		for tree={rectangle,draw,font=\large\sffamily}
	},
	roundstyle/.style={
		for tree={circle,draw,font=\large}
	}
}

% disponi algoritmi
\usepackage{algorithm}
\usepackage{algorithmic}
\makeatletter
\renewcommand{\ALG@name}{Algoritmo}
\makeatother

% disponi numeri di pagina
\usepackage{fancyhdr}
\fancyhf{} 
\fancyfoot[L]{\sffamily{\thepage}}

\makeatletter
\fancyhead[L]{\raisebox{1ex}[0pt][0pt]{\sffamily{\@title \ \@date}}} 
\fancyhead[R]{\raisebox{1ex}[0pt][0pt]{\sffamily{\@author}}}
\makeatother

\begin{document}
% sezione (data)
\section{Lezione del 02-10-24}

% stili pagina
\thispagestyle{empty}
\pagestyle{fancy}

% testo
\subsection{Generatori}
I generatori sono i componenti che spostano le cariche attraverso le reti elettriche.
Dividiamo i generatori in due macrocategorie, in base alle loro caratteristiche:
\begin{itemize}
	\item \textbf{Indipendenti:} hanno sempre le stesse caratteristiche, e portano energia all'interno del circuito;
	\item \textbf{Dipendenti:} hanno caratteristiche \textit{pilotate} da altri fattori del circuito, non portano energia in esso e quindi non sono diversi dagli altri bipoli passivi già visti.
\end{itemize}

Inoltre dividiamo entrambe in altre due categorie, in base al tipo di operazione che svolgono:
\begin{itemize}
	\item \textbf{Generatori di tensione:} mantengono i loro capi a differenza di potenziale costante;
	\item \textbf{Generatori di corrente:} mantengono una corrente costante al loro interno.
\end{itemize}

Infine, dividiamo in due ulteriori modalità di operazione:
\begin{itemize}
	\item \textbf{Corrente continua:} mantengono la corrente costante. Si dicono C.C. (Corrente Continua), o D.C. (Direct Current). Il grafico della corrente sarà:
		\begin{center}
\begin{tikzpicture}
    \begin{axis}[
        xlabel={$t$},
        ylabel={$i(t)$},
        domain=-10:10, % set the x range you want
				samples=100,
        grid=major, % add a grid
				ytick={0, 2},
				yticklabels={0, I},
				ymin = -1, ymax = 3,
				width=14cm,
				height=7cm
    ]
    \addplot[
        blue,
        thick
    ] {2}; 
    \end{axis}
\end{tikzpicture}
\end{center}

	\item \textbf{Corrente alternata:} mantengono la corrente in regime sinousidale. Si dicono C.A. (Corrente Alternata), o A.C. (Alternating Current). Il grafico della corrente alternata è stato già visto all'inizio del corso, ha equazione: 
$$
	i(t) = A \sin{\left(\frac{2\pi}{T} t \right)}
$$
con $A$ ampiezza e $T$ periodo, e grafico:

\begin{center}
\begin{tikzpicture}
    \begin{axis}[
        xlabel={$t$},
        ylabel={$i(t)$},
        domain=-10:10, % set the x range you want
				samples=100,
        grid=major, % add a grid
				ytick={-2, 2},
				yticklabels={$-A$, $A$},
				ymin = -3, ymax = 3,
				width=14cm,
				height=7cm
    ]
    \addplot[
        blue,
        thick
    ] {2 * sin(50*x)}; 
    \end{axis}
\end{tikzpicture}
\end{center}
\end{itemize}

Esistono poi altri regimi di applicazione della corrente, che vedremo per casi specifici (impulsi, gradini, ecc...).

Riportiamo intanto ogni combinazione delle prime quattro tipologie nel dettaglio.

\subsubsection{Generatori di tensione}
Un generatore di tensione (o voltaggio) ideale è un componente circuitale che mantiene i suoi capi $A$ e $B$ ad una differenza di potenziale $V_{AB}$ costante, ovvero:
$$ v(t) = E(t) = V $$
dove con $E$ si indica la forza elettromotrice. 
Si indica come:

\begin{center}
\begin{circuitikz}
\draw (0,0) to[ european voltage source ] (2,0); 
\end{circuitikz}
\end{center}

Si nota che a voltaggio nullo, un generatore di tensione equivale a un corto circuito (un filo ideale).

\par\medskip
\noindent
\textbf{\textsf{Correlazione con la corrente}} \\
La tensione erogata da un generatore di tensione è costante, qualsiasi sia la corrente che lo attraversa:
$$ v(i) = \mathrm{const.} $$
Il grafico di correlazione corrente-voltaggio sarà quindi:

\begin{center}
\begin{tikzpicture}
    \begin{axis}[
        xlabel={$i$},
        ylabel={$v$},
        domain=0:4, % set the x range you want
				samples=100,
        grid=major, % add a grid
				ytick={0, 2},
				yticklabels={0, $V$},
				xtick={0},
				ymin = -0.5, ymax = 3,
				xmin = -0.5,
				width=15cm,
				height=7cm
    ]
    \addplot[
        blue,
        thick
    ] {2}; 

    \end{axis}
\end{tikzpicture}
\end{center}

\par\medskip
\noindent
\textbf{\textsf{Correlazione con la potenza}} \\
Tradizionalmente si descrivono i generatori di tensione attraverso riferimenti non associati di corrente e tensione.
Resta il fatto che la potenza:
$$ p(t) = v(t)i(t) = E(t)i(t) $$
quando è erogata dal generatore, è $> 0$.

\par\medskip
\noindent
\textbf{\textsf{Collegamenti in serie}} \\
Per sommare i contributi al voltaggio di più generatori di voltaggio, li disponiamo in serie:

\begin{center}
\begin{circuitikz}
    \draw (0,0) node[left] {$A$} 
        to[european voltage source, l=$V_1$] (2,0) 
        to[european voltage source, l=$V_2$] (4,0) 
        to[short] (5,0)
        node[anchor=west,xshift=0.15cm] {\dots} (6,0) 
        to[short] (7,0)
        to[european voltage source, l=$V_n$] (9,0) node[right] {$B$};

    \draw[decorate,decoration={brace,amplitude=10pt,mirror}] (0,-1) -- (9,-1)
        node[midway,below=10pt]{$n$ generatori};
\end{circuitikz}
\end{center}

Abbiamo che il contributo totale dei generatori equivale a quello di un singolo generatore $E_T$ di voltaggio:
$$ V_T = V_1 + V_2 + ... + V_n $$

\par\medskip
\noindent
\textbf{\textsf{Collegamenti in parallelo}} \\
Non si possono collegare generatori di voltaggio in parallelo, a meno che questi non abbiano lo stesso voltaggio (e quindi risultino in movimento nullo di carica):

\begin{center}
\begin{circuitikz}
    \draw (2,0) 
				to[short] (0,0) node[left] {$A$};
    
    \draw (2,-2) 
				to[short] (0,-2) node[left] {$B$};
		
		\draw (2,0 )to[european voltage source, l=$V_1$] (2,-2);
		\draw (4,0 )to[european voltage source, l=$V_2$] (4,-2);
		
        
    % Draw horizontal lines at the top and bottom
    \draw (0,0) -- (4,0);
    \draw (0,-2) -- (4,-2);
\end{circuitikz}
\end{center}

Dove si ha, dall'applicazione della seconda legge di Kirchoff:
$$
V_1 - V_2 = 0 \Rightarrow V_1 = V_2 
$$
che sarebbe altrimenti violata.

Nella realtà, se si provasse a collegare due generatori di tensione di voltaggio diverso in parallelo, questi proverebbero a imporre la loro differenza di potenziale sui due rami del circuito, creando forti correnti, e probabilmente causando danni termici ad esso o a loro stessi.


\subsubsection{Generatori di corrente}
Un generatore di corrente ideale è un componente circuitale che mantiene attraverso di sé una corrente costante, ovvero:
$$ i(t) = I $$
Si indica come:

\begin{center}
\begin{circuitikz}
\draw (0,0) to[ european current source ] (2,0); 
\end{circuitikz}
\end{center}

Si nota che a corrente nulla, un generatore di corrente equivale a un circuito aperto.

\par\medskip
\noindent
\textbf{\textsf{Correlazione con il voltaggio}} \\
Un generatore di corrente mantiene la stessa corrente qualsiasi sia il voltaggio.
$$ i(v) = \mathrm{const.} $$
Il grafico di correlazione corrente-voltaggio sarà quindi:

\begin{center}
\begin{tikzpicture}
    \begin{axis}[
        xlabel={$i$},
        ylabel={$v$},
        domain=0:4, % set the x range you want
				samples=100,
        grid=major, % add a grid
				ytick={0},
				yticklabels={0},
				xtick={0, 1.5},
				xticklabels={0, $I$},
				ymin = -0.5, ymax = 4,
				xmin = -0.25, xmax = 1.75,
				width=15cm,
				height=7cm
    ]
    \addplot[
        red,
        thick,
        domain=-0.5:3 % set the y range you want
    ] coordinates {(1.5,0) (1.5,3)};


    \end{axis}
\end{tikzpicture}
\end{center}

\par\medskip
\noindent
\textbf{\textsf{Correlazione con la potenza}} \\
Come per i generatori di tensione, si descrivono i generatori di corrente attraverso riferimenti non associati di corrente e tensione.
Resta comunque il fatto che la potenza:
$$ p(t) = v(t)i(t) = v(t)I(t) $$
quando è erogata dal generatore, è $> 0$.

\par\medskip
\noindent
\textbf{\textsf{Collegamenti in serie}} \\
Non si possono collegare generatori di corrente in serie, a meno che questi non abbiano la stessa carica (e quindi risultino in movimento uniforme di carica):

\begin{center}
\begin{circuitikz}
    \draw (0,0) node[left] {$A$} 
        to[european current source, l=$I_1$] (2,0) 
        to[european current source, l=$I_2$] (4,0) node[right] {$B$};
\end{circuitikz}
\end{center}

Dove si ha, dall'applicazione della prima legge di Kirchoff:
$$
I_1 - I_2 = 0 \Rightarrow I_1 = I_2 
$$
che sarebbe altrimenti violata.

Come prima, questa situazione non è effettivamente modellizzabile nella realtà usando il modello studiato.
In verità il generatore di corrente in sé per sé è più uno strumento teorico che serve a modelizzare fenomeni diversi (transistor, amplificatori, ecc...).

\par\medskip
\noindent
\textbf{\textsf{Collegamenti in parallelo}} \\
Per sommare i contributi alla corrente di più generatori di corrente, li disponiamo in parallelo:

\begin{center}
\begin{circuitikz}
    \draw (2,0) 
				to[short] (0,0) node[left] {$A$};
    
    \draw (2,-2) 
				to[short] (0,-2) node[left] {$B$};
		
		\draw (2,0 )to[european current source, l=$I_1$] (2,-2);
		\draw (4,0 )to[european current source, l=$I_2$] (4,-2);
		
		
    \draw (4,0) node[anchor=west,xshift=0.7cm] {\dots} (6,0);
    \draw (4,-2) node[anchor=west,xshift=0.7cm] {\dots} (6,-2); 
		
		\draw (8,0 )to[european current source, l=$I_n$] (8,-2);
        
    % Draw horizontal lines at the top and bottom
    \draw (0,0) -- (4,0);
    \draw (0,-2) -- (4,-2);

    \draw (6,0) -- (8,0);
    \draw (6,-2) -- (8,-2);

    % Bracket below the resistors
    \draw[decorate,decoration={brace,amplitude=10pt,mirror}] (0,-3) -- (8,-3)
        node[midway,below=10pt]{$n$ generatori};
\end{circuitikz}
\end{center}

Abbiamo che il contributo totale dei generatori equivale a quello di un singolo generatore $E_T$ di corrente:
$$ I_T = I_1 + I_2 + ... + I_n $$

\subsubsection{Resistenza interna}
Possiamo combinare i componenti visti finora per creare modelli più realistici.
Innanzitutto, è improbabile che un generatore reali applichi resistenza nulla alle cariche che vi scorrono dentro.
Aggiungiamo quindi una resistenza (solitamente piccola per i generatori di tensione ed elevata per i generatori di corrente) al generatore, che chiameremo \textbf{resistenza interna}.
Questa resistenza rappresenterà la potenza che viene dissipata per effetto Joule.

La resistenza si disporrà come segue per i diversi tipi di generatore:
\begin{itemize}
	\item \textbf{Generatore di tensione:} resistenza in serie;
\begin{center}
\begin{circuitikz}
    \draw (0,0) node[left] {$A$} 
        to[european voltage source, l=$V$] (2,0) 
				to[R, l=$R$] (4,0) node[right] {$B$};
\end{circuitikz}
\end{center}
	\item \textbf{Generatore di corrente:} resistenza in parallelo.
\begin{center}
\begin{circuitikz}
    \draw (2,0) 
				to[short] (0,0) node[left] {$A$};
    
    \draw (2,-2) 
				to[short] (0,-2) node[left] {$B$};
		
		\draw (2,0 )to[european current source, l=$I$] (2,-2);
		\draw (4,0 )to[R, l=$R$] (4,-2);
		
        
    % Draw horizontal lines at the top and bottom
    \draw (0,0) -- (4,0);
    \draw (0,-2) -- (4,-2);
\end{circuitikz}
\end{center}
\end{itemize}

Notiamo che i casi visti prima come impossibili, di generatori di tensione in parallelo e di generatori di corrente in serie, sono rappresentabili quando si rilascia l'ipotesi che i generatori siano ideali e si introducono resistenze interne.

\subsubsection{Generatori dipendenti}
I generatori dipendenti, detti anche controllati o pilotati, sono particolari tipi di generatore il cui voltaggio (o corrente) dipende dal valore del voltaggio (o corrente) di un'altro punto del circuito, scalato di un qualche coefficiente. 
Si indicano come i generatori indipendenti ma all'interno di un rombo invece che di un cerchio.

Abbiamo quindi 4 tipi fondamentali di generatori dipendenti:
\begin{itemize}
	\item \textbf{Generatori di tensione,} si indicano come:
\begin{center}
\begin{circuitikz}
    \draw (0,0) 
        to[european controlled voltage source] (2,0);
\end{circuitikz}
\end{center}
		\begin{itemize}
			\item \textbf{Generatore di tensione pilotato in tensione:} comandato dalla funzione:
				$$
				v(t) = \alpha \cdot v(t)
				$$
				su un punto arbitrario dove si calcola $i(t)$.
			\item \textbf{Generatore di tensione pilotato in corrente:} comandato dalla funzione:
				$$
				v(t) = \alpha \cdot i(t)
				$$
				su un punto arbitrario dove si calcola $v(t)$.
		\end{itemize}
	\item \textbf{Generatori di corrente,} si indicano come:
\begin{center}
\begin{circuitikz}
    \draw (0,0) 
        to[european controlled current source] (2,0);
\end{circuitikz}
\end{center}
		\begin{itemize}
			\item \textbf{Generatore di corrente pilotato in tensione:} comandato dalla funzione:
				$$
				i(t) = \alpha \cdot v(t)
				$$
				su un punto arbitrario dove si calcola $v(t)$.
			\item \textbf{Generatore di corrente pilotato in corrente:} comandato dalla funzione:
				$$
				i(t) = \alpha \cdot i(t)
				$$
				su un punto arbitrario dove si calcola $i(t)$.
		\end{itemize}
\end{itemize}

\par\smallskip 

Bisogna notare che, come già riportato, un generatore dipendente non è diverso da un bipolo passivo in termini di potenza: non porta nessuna energia esterna all'interno del circuito.
Si può anzi dire che è necessario avere almeno un generatore indipendente per avere spostamento di carica all'interno del circuito.

\subsection{Partitore di tensione}
Analizziamo il seguente circuito:

\begin{center}
\begin{circuitikz} 
	\draw
  (0,2) to[V, v=$e(t)$] (0,12) % Voltage source
  -- (2,12) 
  to[R, l=$R_1$] (2,10) 
  to[R, l=$R_2$] (2,8);

	\draw
	(2,7) to[R, l=$R_J$] (2,5);

	\draw
	(2,4) to[R, l=$R_n$] (2,2)
	-- (0, 2);

\draw (2,7.5) node[xshift=0.1] {\dots} (2,5);
	\draw (2,4.5) node[xshift=0.1] {\dots} (2,4); 
\end{circuitikz}
\end{center}

Reti di questo tipo prendono il nome di \textbf{partitori di tensione}, e hanno lo scopo di partizionare una certa differenza di potenziale in diverse frazioni proprie.

Poniamo di voler calcolare la caduta di potenziale su una particolare resistenza, diciamo la $R_J$. Avremo allora, dalla seconda legge di Kirchoff:
$$
-e(t) + R_1(t) i(t)+ R_2(t) i(t) + ... + R_J(t) i(t) + ... + R_n(t) i(t) = 0
$$
che raccogliendo la corrente comune diventa:
$$
e(t) = (R_1 + R_2 + ... + R_J + R_n) i(t) = i(t) \sum_{i=1}^n R_i
$$
somma delle resistenze per corrente.
A questo punto possiamo applicare la legge di Ohm per ottenere la caduta di potenziale:
$$
V_J(t) = R_J i(t) = e(t)\frac{R_j}{\sum_{i=1}^n R_i}
$$
cioè il rapporto fra la resistenza interessata e la resistenza complessiva del circuito, moltiplicata per la tensione.

\par\smallskip

Vediamo l'esempio di un partitore di tensione con due resistenze, in due casi particolari:

\begin{center}
\begin{circuitikz} 
	\draw
  (0,8) to[V, v=$e(t)$] (0,12) % Voltage source
  -- (2,12) 
  to[R, l=$R_1$] (2,10) 
  to[R, l=$R_2$] (2,8)
	-- (0, 8);
\end{circuitikz}
\end{center}

\begin{itemize}
	\item \textbf{Caso 1:} $R_1 = R_2$ \\
		Applicando la formula per $V_1$, si avrà:
		$$
		V_1 = \frac{R_1}{R_1 + R_2} e(t) = \frac{1}{2} e(t) = ... = V_2
		$$
		uguale per $V_2$, cioè a resistenze uguali partizioniamo il voltaggio in parti uguali.
	\item \textbf{Caso 2:} $R_2 = 0$ \\ 
		Prendiamo il caso dove una resistenza è nulla, ergo corrisponde ad un cortocircuito.
		Avremo, dalla formula per $V_1$:
		$$ 
		V_1 = \frac{R_1}{R_1 + 0} e(t) = e(t) 
		$$
		cioè l'intero potenziale si distribuisce su $R_1$.
		Su $R_2$ avremo invece:
		$$
		V_2 = \frac{0}{R_1 + 0} e(t) = 0
		$$
		cioè, non c'è caduta di potenziale su $V_2$.
\end{itemize}

Questi ultimi due calcoli, che potevamo chiaramente svolgere applicando semplicemente la legge di Ohm, dimostrano comunque che le leggi funzionano anche in questi casi semplici.

\subsection{Partitore di corrente}
Analizziamo quindi il seguente circuito:

\begin{center}
\begin{circuitikz}
    \draw (2,-2) 
				-- (0,-2) 
				to[ european current source, i=$I(t)$] (0, 0);	

		\draw (2,0 )to[R, l=$R_1$] (2,-2);
		\draw (4,0 )to[R, l=$R_2$] (4,-2);
		
		
    \draw (4,0) node[anchor=west,xshift=0.7cm] {\dots} (6,0);
    \draw (4,-2) node[anchor=west,xshift=0.7cm] {\dots} (6,-2); 
		
		\draw (8,0 )to[R, l=$R_J$] (8,-2);
       
    \draw (8,0) node[anchor=west,xshift=0.7cm] {\dots} (10,0);
    \draw (8,-2) node[anchor=west,xshift=0.7cm] {\dots} (10,-2); 
		
		\draw (12,0 )to[R, l=$R_n$] (12,-2);

    % Draw horizontal lines at the top and bottom
    \draw (0,0) -- (4,0);
    \draw (0,-2) -- (4,-2);

    \draw (6,0) -- (8,0);
    \draw (6,-2) -- (8,-2);

    \draw (10,0) -- (12,0);
    \draw (10,-2) -- (12,-2);

\end{circuitikz}
\end{center}

Reti di questo tipo hanno uno scopo simile a quello della rete vista prima, solo riguardo alla corrente: prendono infatti il nome di \textbf{partitori di corrente}.

Poniamo di voler calcolare la corrente su una singola resistenza. 
Potremo dire che la corrente complessiva è, dalla prima legge di Kirchoff:

$$
I_T(t) = I_1(t) + I_2(t) + ... + I_J(t) + ... + I_n(t)
$$

Un'altro modo di ottenere queste correnti è dalla legge di Ohm, usando le conduttanze invece delle resistenze:
$$ 
I = \frac{V}{R} \Rightarrow I = GR, \quad I(t) = v(t) \sum_{i=1}^n G_i
$$
A questo punto, possiamo dire che la corrente nella J-esima resistenza vale:
$$
I_J(t) = v(t) G_n = I(t) \frac{G_J}{\sum_{i=1}^n G_i} 
$$
cioè il rapporto fra la conduttanza (della resistenza) interessata e la conduttanza complessiva del circuito, moltiplicata per la corrente.

Vediamo, come prima, l'esempio di un partitore di corrente con due resistenze, in due casi particolari:

\begin{center}
\begin{circuitikz}
    \draw (2,-2) 
				-- (0,-2) 
				to[ european current source, i=$I(t)$] (0, 0);	
		
		\draw (2,0 )to[R, l=$R_1$] (2,-2);
		\draw (4,0 )to[R, l=$R_2$] (4,-2);
		
    % Draw horizontal lines at the top and bottom
    \draw (0,0) -- (4,0);
    \draw (0,-2) -- (4,-2);

\end{circuitikz}
\end{center}

\begin{itemize}
	\item \textbf{Caso 1:} $R_1 = R_2$ \\
		Applicando la formula per $I_1$, si avrà:
		$$
		I_1 = \frac{\frac{1}{R_1}}{\frac{1}{R_1} + \frac{1}{R_2}} I(t) = \frac{\frac{1}{R_1}}{\frac{R_2 + R_1}{R_1 R_2}} I(t) = \frac{R_1 + R_2}{R_1(R_1 + R_2)} I(t) = \frac{R_2}{R_1 + R_2} I(t) = \frac{1}{2} I(t) = ... = I_2
		$$
		uguale per $I_2$, cioè a resistenze uguali partizioniamo la corrente in parti uguali.
	\item \textbf{Caso 2:} $R_2 = 0$ \\ 
		Prendiamo il caso dove una resistenza è nulla, ergo corrisponde ad un cortocircuito.
		Avremo, dalla formula per $I_1$:
		$$
		I_1 = \frac{\frac{1}{R_1}}{\frac{1}{R_1} + \frac{1}{R_2}} I(t) = ... = \frac{R_2}{R_1 + 0} I(t) = 0
		$$
		cioè, non c'è corrente su $R_1$.
		Su $R_2$ avremo invece:
		$$
	I_2 = ... = \frac{R_1}{R_1 + 0} I(t) = I(t)
		$$
		cioè, l'intera corrente passa dal cortocircuito.
\end{itemize}

Come prima, questi calcoli dimostrano che le leggi funzionano anche su casi semplici.

\end{document}


\documentclass[a4paper,11pt]{article}
\usepackage[a4paper, margin=8em]{geometry}

% usa i pacchetti per la scrittura in italiano
\usepackage[french,italian]{babel}
\usepackage[T1]{fontenc}
\usepackage[utf8]{inputenc}
\frenchspacing 

% usa i pacchetti per la formattazione matematica
\usepackage{amsmath, amssymb, amsthm, amsfonts}

% usa altri pacchetti
\usepackage{gensymb}
\usepackage{hyperref}
\usepackage{standalone}

% imposta il titolo
\title{Appunti Elettrotecnica}
\author{Luca Seggiani}
\date{2024}

% imposta lo stile
% usa helvetica
\usepackage[scaled]{helvet}
% usa palatino
\usepackage{palatino}
% usa un font monospazio guardabile
\usepackage{lmodern}

\renewcommand{\rmdefault}{ppl}
\renewcommand{\sfdefault}{phv}
\renewcommand{\ttdefault}{lmtt}

% disponi il titolo
\makeatletter
\renewcommand{\maketitle} {
	\begin{center} 
		\begin{minipage}[t]{.8\textwidth}
			\textsf{\huge\bfseries \@title} 
		\end{minipage}%
		\begin{minipage}[t]{.2\textwidth}
			\raggedleft \vspace{-1.65em}
			\textsf{\small \@author} \vfill
			\textsf{\small \@date}
		\end{minipage}
		\par
	\end{center}

	\thispagestyle{empty}
	\pagestyle{fancy}
}
\makeatother

% disponi teoremi
\usepackage{tcolorbox}
\newtcolorbox[auto counter, number within=section]{theorem}[2][]{%
	colback=blue!10, 
	colframe=blue!40!black, 
	sharp corners=northwest,
	fonttitle=\sffamily\bfseries, 
	title=~\thetcbcounter: #2, 
	#1
}

% disponi definizioni
\newtcolorbox[auto counter, number within=section]{definition}[2][]{%
	colback=red!10,
	colframe=red!40!black,
	sharp corners=northwest,
	fonttitle=\sffamily\bfseries,
	title=~\thetcbcounter: #2,
	#1
}

% U.D.M
\newcommand{\amp}{\ensuremath{\mathrm{A}}}
\newcommand{\volt}{\ensuremath{\mathrm{V}}}
\newcommand{\meter}{\ensuremath{\mathrm{m}}}
\newcommand{\second}{\ensuremath{\mathrm{s}}}
\newcommand{\farad}{\ensuremath{\mathrm{F}}}
\newcommand{\henry}{\ensuremath{\mathrm{H}}}
\newcommand{\siemens}{\ensuremath{\mathrm{S}}}

% circuiti
\usepackage{circuitikz}
\usetikzlibrary{babel}

% disegni
\usepackage{pgfplots}
\pgfplotsset{width=10cm,compat=1.9}

% disponi codice
\usepackage{listings}
\usepackage[table]{xcolor}

\lstdefinestyle{codestyle}{
		backgroundcolor=\color{black!5}, 
		commentstyle=\color{codegreen},
		keywordstyle=\bfseries\color{magenta},
		numberstyle=\sffamily\tiny\color{black!60},
		stringstyle=\color{green!50!black},
		basicstyle=\ttfamily\footnotesize,
		breakatwhitespace=false,         
		breaklines=true,                 
		captionpos=b,                    
		keepspaces=true,                 
		numbers=left,                    
		numbersep=5pt,                  
		showspaces=false,                
		showstringspaces=false,
		showtabs=false,                  
		tabsize=2
}

\lstdefinestyle{shellstyle}{
		backgroundcolor=\color{black!5}, 
		basicstyle=\ttfamily\footnotesize\color{black}, 
		commentstyle=\color{black}, 
		keywordstyle=\color{black},
		numberstyle=\color{black!5},
		stringstyle=\color{black}, 
		showspaces=false,
		showstringspaces=false, 
		showtabs=false, 
		tabsize=2, 
		numbers=none, 
		breaklines=true
}

\lstdefinelanguage{javascript}{
	keywords={typeof, new, true, false, catch, function, return, null, catch, switch, var, if, in, while, do, else, case, break},
	keywordstyle=\color{blue}\bfseries,
	ndkeywords={class, export, boolean, throw, implements, import, this},
	ndkeywordstyle=\color{darkgray}\bfseries,
	identifierstyle=\color{black},
	sensitive=false,
	comment=[l]{//},
	morecomment=[s]{/*}{*/},
	commentstyle=\color{purple}\ttfamily,
	stringstyle=\color{red}\ttfamily,
	morestring=[b]',
	morestring=[b]"
}

% disponi sezioni
\usepackage{titlesec}

\titleformat{\section}
	{\sffamily\Large\bfseries} 
	{\thesection}{1em}{} 
\titleformat{\subsection}
	{\sffamily\large\bfseries}   
	{\thesubsection}{1em}{} 
\titleformat{\subsubsection}
	{\sffamily\normalsize\bfseries} 
	{\thesubsubsection}{1em}{}

% disponi alberi
\usepackage{forest}

\forestset{
	rectstyle/.style={
		for tree={rectangle,draw,font=\large\sffamily}
	},
	roundstyle/.style={
		for tree={circle,draw,font=\large}
	}
}

% disponi algoritmi
\usepackage{algorithm}
\usepackage{algorithmic}
\makeatletter
\renewcommand{\ALG@name}{Algoritmo}
\makeatother

% disponi numeri di pagina
\usepackage{fancyhdr}
\fancyhf{} 
\fancyfoot[L]{\sffamily{\thepage}}

\makeatletter
\fancyhead[L]{\raisebox{1ex}[0pt][0pt]{\sffamily{\@title \ \@date}}} 
\fancyhead[R]{\raisebox{1ex}[0pt][0pt]{\sffamily{\@author}}}
\makeatother

\begin{document}
% sezione (data)
\section{Lezione del 13-11-24}

% stili pagina
\thispagestyle{empty}
\pagestyle{fancy}

% testo
Avevamo visto il concetto di \textbf{bipolo}, cioè un componente circuitale con due \textit{punti di contatto} col resto del circuito (\textbf{morsetti}), su cui passa una certa \textbf{corrente} $I$ e su cui si trova una certa \textbf{tensione}, cioè una \textit{differenza di potenzale} $V$.
Potremmo avere anche un \textbf{tripolo}, cioè un componente con morsetti, su cui passano (propriamente, da cui \textit{escono} o \textit{entrano}), anzichè una, 3 correnti, e su cui individuiamo 3 tensioni ($A$, $B$ e $C$) e 3 \textbf{cadute} di tensione su ogni percorso che attraversa il bipolo.
Una possibile rappresentazione di un tripolo è la seguente:
\begin{center}
	\begin{circuitikz}
		\node[rectangle, draw, minimum width = 1cm, minimum height = 1cm] (a) at (0,0) {};
		\draw (-2, 0) to [ short, i=$I_A$] (a.west);
		\draw (0, -2) to [ short, i=$I_C$] (a.south);
		\draw (2, 0) to [ short, i_=$I_B$] (a.east);
		

		\node[anchor=east] at(-2, 0) {$V_A$};
		\node[anchor=north] at(0, -2) {$V_C$};
		\node[anchor=west] at(2, 0) {$V_B$};
	\end{circuitikz}
\end{center}
le cui equazioni sono:
\[
	\begin{cases}
		I_A + I_B + I_C = 0 \\ 
		V_{AB} = V_A - V_B \\ 
		V_{AC} = V_A - V_C \\ 
		V_{BC} = V_B - V_C
	\end{cases}
\]

Notiamo che, dalle equazioni ai potenziali, si possono ricavare le relazioni (piuttosto scontate):
\[
	\begin{cases}
		V_{AB} + V_{BC} = V_{AC} \\ 
		V_{BA} + V_{AC} = V_{BC} \\
		V_{AC} + V_{CB} = V_{AB}
	\end{cases}
\]
con $V_{BA} = - V_{AB}$ e $V_{CB} = -V_{BC}$ (e anche se non si è usata, $V_{CA} = -V_{AC}$).

\subsection{Porte}
Definiamo una \textbf{porta} come una coppia di poli di un circuito dove la corrente entrante è uguale a quella uscente.
Rappresentiamo una porta come segue:

\begin{center}
	\begin{circuitikz}
		\node[rectangle, draw, minimum width = 2cm, minimum height = 2cm] (a) at (0,0) {};
		\draw (-2, 0.6) to [ short, i=$I$] (-1, 0.6);
		\draw(-1, -0.6) to [ short, i=$I$ ] (-2, -0.6);	
	
		\draw (-2.6, 0.6) node[anchor=west] {$+$};
		\draw (-2.6, 0) node[anchor=west] {$V$};
		\draw (-2.6, -0.6) node[anchor=west] {$-$};
	\end{circuitikz}
\end{center}

Notiamo che per $n$ poli si hanno al massimo $\frac{n}{2}$ porte (ammesso un numero pari di poli).

Ciò che ci è di interesse sono i circuiti a \textbf{due porte} (o equivalentemente a \textit{quattro poli}):

\begin{center}
	\begin{circuitikz}
		\node[rectangle, draw, minimum width = 2cm, minimum height = 2cm] (a) at (0,0) {};
		\draw (-2, 0.6) to [ short, i=$I_1$] (-1, 0.6);
		\draw(-1, -0.6) to [ short, i=$I_1$ ] (-2, -0.6);	
	
		\draw (-2.6, 0.6) node[anchor=west] {$+$};
		\draw (-2.6, 0) node[anchor=west] {$V_1$};
		\draw (-2.6, -0.6) node[anchor=west] {$-$};
		
		\draw (2, 0.6) to [ short, i_=$I_2$] (1, 0.6);
		\draw(1, -0.6) to [ short, i_=$I_2$ ] (2, -0.6);	
	
		\draw (2.6, 0.6) node[anchor=east] {$+$};
		\draw (2.6, 0) node[anchor=east] {$V_2$};
		\draw (2.6, -0.6) node[anchor=east] {$-$};
	\end{circuitikz}
\end{center}

Possiamo immaginare che un segnale \textit{entra} da una porta, viene \textit{elaborato} all'interno del circuito, e \textit{esce} dalla porta opposta.

Per convenzione, scegliamo le due correnti $I_1(t)$e $I_2(t)$ come rivolte nello stesso senso, e le due tensioni $V_1(t)$ e $V_2(t)$ come con la stessa polarità:

\subsection{Circuiti equivalenti di circuiti a due porte}
Ciò che può interessarci quando studiamo circuiti a due porte è ricavare \textbf{circuiti equivalenti}, cioè che si comportano in maniera equivalente agli effetti esterni.
L'idea è, come sempre, quella di prendere circuiti arbitrariamente complessi e ridurli a circuiti equivalenti relativamente semplici.

\subsubsection{Rappresentazione a parametri Z}
Una coppia di \textbf{induttori mutuamente accoppiati} rappresenta effettivamente un circuito a due porte, in quanto la stessa corrente entra e esce da ogni induttore (cioè si formano due porte).
\begin{center}
	\begin{circuitikz}
		\draw (0,0) to[ short, i=$i_1$] (1,0)
			to[ inductor , l=$L_1$] (1,-2)
			-- (0, -2);

		\draw (4,0) to[ short, i_=$i_2$] (3,0)
			to[ inductor , l_=$L_2$] (3,-2)
			-- (4, -2);

			\draw (0.9,-0.6) node {$\scriptscriptstyle\bullet$};
			\draw (2.9,-0.6) node {$\scriptscriptstyle\bullet$};

			\draw (0,0) node[anchor=east] {$A$};
			\draw (0,-2) node[anchor=east] {$B$};

			\draw (0,-0.5) node[anchor=east] {$+$};
			\draw (0,-1.5) node[anchor=east] {$-$};

			\draw (4,0) node[anchor=west] {$C$};
			\draw (4,-2) node[anchor=west] {$D$};

			\draw (4,-0.5) node[anchor=west] {$+$};
			\draw (4,-1.5) node[anchor=west] {$-$};

			\draw (0,-1) node[anchor=east] {$v_1$};
			\draw (4,-1) node[anchor=west] {$v_2$};
	\end{circuitikz}
\end{center}

Avevamo rappresentato questi circuiti come:
\[
	\begin{cases}
		\dot{V}_1 = j \omega L_1 \dot{I}_1 + j \omega M \dot{I}_2 \\	
		\dot{V}_2 = j \omega L_2 \dot{I}_2 + j \omega M \dot{I}_1	
	\end{cases}
\]

Analogamente, decidiamo di rappresentare un circuito a due porte attraverso equazioni che legano la tensione su una porta alla corrente su entrambe le porte:
\[
	\begin{cases}
		\dot{V}_1 = \overline{z_{11}} \dot{I}_1 + \overline{z_{12}} \dot{I}_2 \\ 	
		\dot{V}_2 = \overline{z_{21}} \dot{I}_1 + \overline{z_{22}} \dot{I}_2 	
	\end{cases}
\]

Per esprimere queste relazioni in forma più compatta, possiamo sfruttare il calcolo matriciale:
$$
\dot{V} = \overline{Z} \dot{I}
$$
dove $\dot{V}$ e $\dot{I}$ sono matrici:
$$
\begin{pmatrix}
	\dot{V}_1 \\ \dot{V}_2
\end{pmatrix}
= \overline{Z}
\begin{pmatrix}
	\dot{I}_1 \\ \dot{I}_2
\end{pmatrix}
$$
e $\overline{Z}$ sarà l'\textbf{impedenza} in forma matriciale:
$$
\overline{Z} =
\begin{pmatrix}
	\overline{z_{11}} & \overline{z_{12}} \\ 
	\overline{z_{21}} & \overline{z_{22}} 
\end{pmatrix}
$$

Date le equazioni riportate sopra che legano voltaggio a corrente, possiamo ricavare il valore di ogni componente di $\overline{Z}$ come:
$$
\overline{Z} =
\begin{pmatrix}
	\overline{z_{11}} = \frac{\dot{V}_1}{\dot{I}_1} \Big|_{\dot{I}_2 = 0} & 
	\overline{z_{12}} = \frac{\dot{V}_1}{\dot{I}_2} \Big|_{\dot{I}_1 = 0} \\
	\overline{z_{21}} = \frac{\dot{V}_2}{\dot{I}_1} \Big|_{\dot{I}_2 = 0} &
	\overline{z_{22}} = \frac{\dot{V}_2}{\dot{I}_2} \Big|_{\dot{I}_1 = 0}
\end{pmatrix}
$$
dove la notazione $a \Big|_b$ significa "$a$ quando $b$".

Si ha, attraverso queste relazioni, che basta misurare la tensione sulle porte in due stati ($\dot{I}_1 = 0$ e $\dot{I}_2 = 0$) per ricavare completamente i parametri $\overline{Z}$ del circuito, e ricavare quindi un circuito equivalente del tipo:

\begin{center}
	\begin{circuitikz}
		\draw (-4, 1) -- (-3, 1) 
			to [ european resistor, l=$\overline{z_{11}}$, i=$I_1$] (-1, 1)
			to [ controlled voltage source, v<=$\overline{z_{12}} \dot{I}_2$ ] (-1, -1) 
			to [ short, i=$I_1$ ] (-3, -1)	
			-- (-4, -1);
			
		\draw (-4.6, 1) node[anchor=west] {$+$};
		\draw (-4.6, 0) node[anchor=west] {$V_1$};
		\draw (-4.6, -1) node[anchor=west] {$-$};

		\draw (6,1) -- (5, 1) 
			to [ european resistor, l_=$\overline{z_{12}}$, i_=$I_2$] (3, 1)
			to [ controlled voltage source, v_<=$\overline{z_{21}} \dot{I}_1$ ] (3, -1) 
			to [ short, i_=$I_2$ ] (5, -1)
			-- (6, -1);
	
		\draw (6.6, 1) node[anchor=east] {$+$};
		\draw (6.6, 0) node[anchor=east] {$V_2$};
		\draw (6.6, -1) node[anchor=east] {$-$};
		
		\node[rectangle, draw, minimum width = 8.5cm, minimum height = 4cm] (a) at (1,0) {};
	\end{circuitikz}
\end{center}
dove si inseriscono i termini di impedenza $\overline{z_{11}}$ e $\overline{z_{22}}$ semplicemente come impedenze in serie alle porte 1 e 2, e i termini "associati" $\overline{z_{12}}$ e $\overline{z_{21}}$ come generatori di tensione pilotati (che generano, appunto, cadute di tensione pilotate, rispettivamente in $\dot{I}_2$ per la porta 1 e in $\dot{I}_1$ per la porta 2).

Il metodo naturale di analisi per questo circuito è correnti di maglia, che possiamo applicare alle due porte per poi eguagliare con la matrice delle impedenze:
\[
	\begin{cases}
		\dot{V}_1	= \overline{z_{11}} \dot{I}_1 + \overline{Z_{12}} \dot{I}_2 \\  	
		\dot{V}_2	= \overline{z_{22}} \dot{I}_2 + \overline{Z_{21}} \dot{I}_1 \\  	
	\end{cases}
\]
che combacia con quanto definito sulla rappresentazione in impedenza.

In particolare, nel caso $\overline{z_{12}} = \overline{z_{21}}$ si dice che la rete è \textbf{reciproca} e si può formare il circuito equivalente come:

\begin{center}
	\begin{circuitikz}
		\draw (-4, 1) -- (-3, 1) 
			to [ european resistor, l=$\overline{z_a}$, i=$I_1$] (0, 1)
			to [ european resistor, l=$\overline{z_b}$] (0, -1) 
			to [ short, i=$I_1$ ] (-3, -1)	
			-- (-4, -1);
			
		\draw (-4.6, 1) node[anchor=west] {$+$};
		\draw (-4.6, 0) node[anchor=west] {$V_1$};
		\draw (-4.6, -1) node[anchor=west] {$-$};

		\draw (4, 1) -- (3, 1) 
			to [ european resistor, l_=$\overline{z_c}$, i_=$I_2$] (0, 1);

		\draw (0, -1) to [ short, i_=$I_2$ ] (3, -1)
			-- (4, -1);
	
		\draw (4.6, 1) node[anchor=east] {$+$};
		\draw (4.6, 0) node[anchor=east] {$V_2$};
		\draw (4.6, -1) node[anchor=east] {$-$};
		
		\node[rectangle, draw, minimum width = 6.5cm, minimum height = 4cm] (a) at (0,0) {};
	\end{circuitikz}
\end{center}

Anche qui, applicando correnti di maglia, si ha:
\[
	\begin{cases}
		\dot{V}_1 = \overline{z_a} \dot{I}_1 + \overline{z_b}\left( \dot{I}_1 + \dot{I}_2 \right) = ( \overline{z_a} + \overline{z_b} ) \dot{I}_1 + \overline{z_b} \dot{I}_2 \\	
		\dot{V}_2 = \overline{z_c} \dot{I}_2 + \overline{z_b}\left( \dot{I}_1 + \dot{I}_2 \right) = ( \overline{z_c} + \overline{z_c} ) \dot{I}_2 + \overline{z_b} \dot{I}_1 \\ 	
	\end{cases}
\]
che rappresenta la rete reciproca, ponendo:
\[
	\overline{Z} =
	\begin{pmatrix}
		\overline{z_a} + \overline{z_b} & \overline{z_b} \\ 
		\overline{z_b} & \overline{z_b + z_c}
	\end{pmatrix}
	\Leftrightarrow
	\begin{cases}
		\overline{z_a} = \overline{z_{11}} - \overline{z_{12}} = \overline{z_{11}} - \overline{z_{21}} \\ 
		\overline{z_b} = \overline{z_{12}} = \overline{z_{21}} \\
		\overline{z_c} = \overline{z_{22}} - \overline{z_{12}} = \overline{z_{22}} - \overline{z_{21}}
	\end{cases}
\]

Notiamo che, per circuiti a due porte generici, non è detto che i potenziali dei morsetti di uscita di entrambe le porte siano allo stesso potenziale: per modellizzare questo comportamento si usa una \textit{mutua induttanza ideale}, cioè un \textbf{trasformatore ideale}.


\subsubsection{Rappresentazione a parametri Y}
Nella rappresentazione di un circuito a due porte possiamo parametrizzare, anzichè l'impedenza $\overline{Z}$, l'ammettenza $\overline{Y}$: se avevamo espresso il comportamento del circuito come $
\begin{pmatrix}
	\dot{V}_1 \\ \dot{V}_2
\end{pmatrix}
= \overline{Z}
\begin{pmatrix}
	\dot{I}_1 \\ \dot{I}_2
\end{pmatrix}
$, infatti, possiamo trovare l'inverso:
$$
\begin{pmatrix}
	\dot{I}_1 \\ \dot{I}_2
\end{pmatrix}
= \overline{Z}^{-1}
\begin{pmatrix}
	\dot{V}_1 \\ \dot{V}_2
\end{pmatrix}
$$
dove la matrice $\overline{Z}^{-1} = \overline{Y}$ è effettivamente l'\textbf{ammettenza} in forma matriciale del circuito:
$$
\overline{Y} =
\begin{pmatrix}
	\overline{y_{11}} & \overline{y_{12}} \\ 
	\overline{y_{21}} & \overline{y_{22}}
\end{pmatrix}
$$

Da cui il sistema lineare:
\[
	\begin{cases}
		\dot{I}_1 = \overline{y_{11}} \dot{V}_1 + \overline{y_{12}} \dot{V}_2 \\ 	
		\dot{I}_2 = \overline{y_{21}} \dot{V}_1 + \overline{y_{22}} \dot{V}_2 	
	\end{cases}
\]

Date le equazioni riportate sopra, possiamo ricavare il valore di ogni componente di $\overline{Y}$ come:
$$
\overline{Y} =
\begin{pmatrix}
	\overline{y_{11}} = \frac{\dot{I}_1}{\dot{V}_1} \Big|_{\dot{V}_2 = 0} &
	\overline{y_{12}} = \frac{\dot{I}_1}{\dot{V}_2} \Big|_{\dot{V}_1 = 0} \\
	\overline{y_{21}} = \frac{\dot{I}_2}{\dot{V}_1} \Big|_{\dot{V}_2 = 0} &
	\overline{y_{22}} = \frac{\dot{I}_2}{\dot{V}_2} \Big|_{\dot{V}_1 = 0}
\end{pmatrix}
$$

Possiamo quindi disporre un circuito equivalente come segue:
\begin{center}
	\begin{circuitikz}
		\draw (-4, 1) to [ short, i=$I_1$] (-3, 1) 
			to [ short] (-1, 1);
		\draw (-1, 1) to [ controlled current source, cI>=$\overline{y_{12}} \dot{V}_1$ ] (-1, -1);
		\draw (-1, -1) to [ short] (-3, -1)
		-- (-4, -1);
			
		\draw (-4.6, 1) node[anchor=west] {$+$};
		\draw (-4.6, 0) node[anchor=west] {$V_1$};
		\draw (-4.6, -1) node[anchor=west] {$-$};

		\draw (6,1) to [ short, i=$I_2$] (5, 1) 
			to [ short] (3, 1)
			to [ controlled current source, cI_>=$\overline{y_{21}} \dot{V}_2$ ] (3, -1) 
			to [ short] (5, -1)
			-- (6, -1);
	
		\draw (6.6, 1) node[anchor=east] {$+$};
		\draw (6.6, 0) node[anchor=east] {$V_2$};
		\draw (6.6, -1) node[anchor=east] {$-$};
		
		\draw (-2, 1) to [ european resistor, l_=$\overline{y_{11}}$] (-2, -1);
	\draw (4, 1) to [ european resistor, l=$\overline{y_{22}}$] (4, -1);
		
		\node[rectangle, draw, minimum width = 8.5cm, minimum height = 4cm] (a) at (1,0) {};
	\end{circuitikz}
\end{center}
dove stavolta si inseriscono i termini di ammettenza $\overline{y_{11}}$ e $\overline{y_{22}}$ semplicemente ammettenze in parallelo alle porte 1 e 2, e i termini "associati" $\overline{y_{12}}$ e $\overline{y_{21}}$ come generatori di corrente pilotati.
Possiamo analizzare questo circuito considerando le corrente sui rami impedenza e generatore di entrambe le porte, da cui si ottiene:
\[
	\begin{cases}
		\dot{I}_1	= \overline{y_{11}} \dot{V}_1 + \overline{y_{12}} \dot{V}_2 \\  	
		\dot{I}_2	= \overline{y_{22}} \dot{V}_2 + \overline{y_{21}} \dot{V}_1 \\  	
	\end{cases}
\]
che combacia con quanto definito sulla rappresentazione in ammettenza.

In particolare, vediamo il caso \textbf{reciproco} $\overline{y_{11}} = \overline{y_{21}}$: 
\begin{center}
	\begin{circuitikz}
		\draw (-1, 1) to[ european resistor, l_=$\overline{y_a}$] (-1, -1);
		\draw (1, 1) to[ european resistor, l=$\overline{y_c}$] (1, -1);
		
		\draw (-4, 1) to[ short, i=$I_1$] (-3, 1)
			to[ european resistor, l=$\overline{y_b}$] (3, 1)
			to[ short, i<=$I_2$] (4, 1);		

		\draw (-4, -1) to[ short] (4, -1);

		\draw (-4.6, 1) node[anchor=west] {$+$};
		\draw (-4.6, 0) node[anchor=west] {$V_1$};
		\draw (-4.6, -1) node[anchor=west] {$-$};

		\draw (4.6, 1) node[anchor=east] {$+$};
		\draw (4.6, 0) node[anchor=east] {$V_2$};
		\draw (4.6, -1) node[anchor=east] {$-$};
	
		\node [ground] at (0, -1) {};
		
		\draw (-1,1) node[circ] {};
		\draw (-1,1) node[above] {$V_1$};

		\draw (1,1) node[circ] {};
		\draw (1,1) node[above] {$V_2$};
		
		\node[rectangle, draw, minimum width = 6.5cm, minimum height = 4cm] (a) at (0,0) {};
	\end{circuitikz}
\end{center}

Il metodo naturale di analisi per questo circuito è tensioni di nodo, che possiamo applicare alle due porte per poi eguagliare con la matrice delle ammettenze.
Prendiamo i due nodi in alto come $\dot{V}_1$ e $\dot{V}_2$, il nodo in basso come terra, e scriviamo le equazioni:
\[
	\begin{cases}
		\dot{I}_1 = (\overline{y_a} + \overline{y_b}) \dot{V}_1 - \overline{y_b} \dot{V}_2 \\	
		\dot{I}_2 = (\overline{y_b} + \overline{y_c}) \dot{V}_2 - \overline{y_b} \dot{V}_1
	\end{cases}
\]
che rappresenta la rete reciproca, ponendo:
\[
	\overline{Y} = 
	\begin{pmatrix}
		\overline{y_a} + \overline{y_b} & -\overline{y_b} \\ 
		-\overline{y_b} & \overline{y_b} + \overline{y_c}
	\end{pmatrix}
	\Leftrightarrow
	\begin{cases}
		\overline{y_a} = \overline{y_{11}} + \overline{y_{12}} \\ 
		\overline{y_b} = -\overline{y_{12}} = -\overline{y_{21}} \\
		\overline{y_c} = \overline{y_{22}} + \overline{y_{12}}
	\end{cases}
\]

Notiamo che ancora che il circuito più generale si ottiene disaccoppiando i potenziali sul ramo in basso attraverso un trasformatore ideale.
\end{document}


\documentclass[a4paper,11pt]{article}
\usepackage[a4paper, margin=8em]{geometry}

% usa i pacchetti per la scrittura in italiano
\usepackage[french,italian]{babel}
\usepackage[T1]{fontenc}
\usepackage[utf8]{inputenc}
\frenchspacing 

% usa i pacchetti per la formattazione matematica
\usepackage{amsmath, amssymb, amsthm, amsfonts}

% usa altri pacchetti
\usepackage{gensymb}
\usepackage{hyperref}
\usepackage{standalone}

% imposta il titolo
\title{Appunti Elettrotecnica}
\author{Luca Seggiani}
\date{2024}

% imposta lo stile
% usa helvetica
\usepackage[scaled]{helvet}
% usa palatino
\usepackage{palatino}
% usa un font monospazio guardabile
\usepackage{lmodern}

\renewcommand{\rmdefault}{ppl}
\renewcommand{\sfdefault}{phv}
\renewcommand{\ttdefault}{lmtt}

% disponi il titolo
\makeatletter
\renewcommand{\maketitle} {
	\begin{center} 
		\begin{minipage}[t]{.8\textwidth}
			\textsf{\huge\bfseries \@title} 
		\end{minipage}%
		\begin{minipage}[t]{.2\textwidth}
			\raggedleft \vspace{-1.65em}
			\textsf{\small \@author} \vfill
			\textsf{\small \@date}
		\end{minipage}
		\par
	\end{center}

	\thispagestyle{empty}
	\pagestyle{fancy}
}
\makeatother

% disponi teoremi
\usepackage{tcolorbox}
\newtcolorbox[auto counter, number within=section]{theorem}[2][]{%
	colback=blue!10, 
	colframe=blue!40!black, 
	sharp corners=northwest,
	fonttitle=\sffamily\bfseries, 
	title=~\thetcbcounter: #2, 
	#1
}

% disponi definizioni
\newtcolorbox[auto counter, number within=section]{definition}[2][]{%
	colback=red!10,
	colframe=red!40!black,
	sharp corners=northwest,
	fonttitle=\sffamily\bfseries,
	title=~\thetcbcounter: #2,
	#1
}

% U.D.M
\newcommand{\amp}{\ensuremath{\mathrm{A}}}
\newcommand{\volt}{\ensuremath{\mathrm{V}}}
\newcommand{\meter}{\ensuremath{\mathrm{m}}}
\newcommand{\second}{\ensuremath{\mathrm{s}}}
\newcommand{\farad}{\ensuremath{\mathrm{F}}}
\newcommand{\henry}{\ensuremath{\mathrm{H}}}
\newcommand{\siemens}{\ensuremath{\mathrm{S}}}

% circuiti
\usepackage{circuitikz}
\usetikzlibrary{babel}

% disegni
\usepackage{pgfplots}
\pgfplotsset{width=10cm,compat=1.9}

% disponi codice
\usepackage{listings}
\usepackage[table]{xcolor}

\lstdefinestyle{codestyle}{
		backgroundcolor=\color{black!5}, 
		commentstyle=\color{codegreen},
		keywordstyle=\bfseries\color{magenta},
		numberstyle=\sffamily\tiny\color{black!60},
		stringstyle=\color{green!50!black},
		basicstyle=\ttfamily\footnotesize,
		breakatwhitespace=false,         
		breaklines=true,                 
		captionpos=b,                    
		keepspaces=true,                 
		numbers=left,                    
		numbersep=5pt,                  
		showspaces=false,                
		showstringspaces=false,
		showtabs=false,                  
		tabsize=2
}

\lstdefinestyle{shellstyle}{
		backgroundcolor=\color{black!5}, 
		basicstyle=\ttfamily\footnotesize\color{black}, 
		commentstyle=\color{black}, 
		keywordstyle=\color{black},
		numberstyle=\color{black!5},
		stringstyle=\color{black}, 
		showspaces=false,
		showstringspaces=false, 
		showtabs=false, 
		tabsize=2, 
		numbers=none, 
		breaklines=true
}

\lstdefinelanguage{javascript}{
	keywords={typeof, new, true, false, catch, function, return, null, catch, switch, var, if, in, while, do, else, case, break},
	keywordstyle=\color{blue}\bfseries,
	ndkeywords={class, export, boolean, throw, implements, import, this},
	ndkeywordstyle=\color{darkgray}\bfseries,
	identifierstyle=\color{black},
	sensitive=false,
	comment=[l]{//},
	morecomment=[s]{/*}{*/},
	commentstyle=\color{purple}\ttfamily,
	stringstyle=\color{red}\ttfamily,
	morestring=[b]',
	morestring=[b]"
}

% disponi sezioni
\usepackage{titlesec}

\titleformat{\section}
	{\sffamily\Large\bfseries} 
	{\thesection}{1em}{} 
\titleformat{\subsection}
	{\sffamily\large\bfseries}   
	{\thesubsection}{1em}{} 
\titleformat{\subsubsection}
	{\sffamily\normalsize\bfseries} 
	{\thesubsubsection}{1em}{}

% disponi alberi
\usepackage{forest}

\forestset{
	rectstyle/.style={
		for tree={rectangle,draw,font=\large\sffamily}
	},
	roundstyle/.style={
		for tree={circle,draw,font=\large}
	}
}

% disponi algoritmi
\usepackage{algorithm}
\usepackage{algorithmic}
\makeatletter
\renewcommand{\ALG@name}{Algoritmo}
\makeatother

% disponi numeri di pagina
\usepackage{fancyhdr}
\fancyhf{} 
\fancyfoot[L]{\sffamily{\thepage}}

\makeatletter
\fancyhead[L]{\raisebox{1ex}[0pt][0pt]{\sffamily{\@title \ \@date}}} 
\fancyhead[R]{\raisebox{1ex}[0pt][0pt]{\sffamily{\@author}}}
\makeatother

\begin{document}
% sezione (data)
\section{Lezione del 13-11-24}

% stili pagina
\thispagestyle{empty}
\pagestyle{fancy}

% testo
Avevamo visto il concetto di \textbf{bipolo}, cioè un componente circuitale con due \textit{punti di contatto} col resto del circuito (\textbf{morsetti}), su cui passa una certa \textbf{corrente} $I$ e su cui si trova una certa \textbf{tensione}, cioè una \textit{differenza di potenzale} $V$.
Potremmo avere anche un \textbf{tripolo}, cioè un componente con morsetti, su cui passano (propriamente, da cui \textit{escono} o \textit{entrano}), anzichè una, 3 correnti, e su cui individuiamo 3 tensioni ($A$, $B$ e $C$) e 3 \textbf{cadute} di tensione su ogni percorso che attraversa il bipolo.
Una possibile rappresentazione di un tripolo è la seguente:
\begin{center}
	\begin{circuitikz}
		\node[rectangle, draw, minimum width = 1cm, minimum height = 1cm] (a) at (0,0) {};
		\draw (-2, 0) to [ short, i=$I_A$] (a.west);
		\draw (0, -2) to [ short, i=$I_C$] (a.south);
		\draw (2, 0) to [ short, i_=$I_B$] (a.east);
		

		\node[anchor=east] at(-2, 0) {$V_A$};
		\node[anchor=north] at(0, -2) {$V_C$};
		\node[anchor=west] at(2, 0) {$V_B$};
	\end{circuitikz}
\end{center}
le cui equazioni sono:
\[
	\begin{cases}
		I_A + I_B + I_C = 0 \\ 
		V_{AB} = V_A - V_B \\ 
		V_{AC} = V_A - V_C \\ 
		V_{BC} = V_B - V_C
	\end{cases}
\]

Notiamo che, dalle equazioni ai potenziali, si possono ricavare le relazioni (piuttosto scontate):
\[
	\begin{cases}
		V_{AB} + V_{BC} = V_{AC} \\ 
		V_{BA} + V_{AC} = V_{BC} \\
		V_{AC} + V_{CB} = V_{AB}
	\end{cases}
\]
con $V_{BA} = - V_{AB}$ e $V_{CB} = -V_{BC}$ (e anche se non si è usata, $V_{CA} = -V_{AC}$).

\subsection{Porte}
Definiamo una \textbf{porta} come una coppia di poli di un circuito dove la corrente entrante è uguale a quella uscente.
Rappresentiamo una porta come segue:

\begin{center}
	\begin{circuitikz}
		\node[rectangle, draw, minimum width = 2cm, minimum height = 2cm] (a) at (0,0) {};
		\draw (-2, 0.6) to [ short, i=$I$] (-1, 0.6);
		\draw(-1, -0.6) to [ short, i=$I$ ] (-2, -0.6);	
	
		\draw (-2.6, 0.6) node[anchor=west] {$+$};
		\draw (-2.6, 0) node[anchor=west] {$V$};
		\draw (-2.6, -0.6) node[anchor=west] {$-$};
	\end{circuitikz}
\end{center}

Notiamo che per $n$ poli si hanno al massimo $\frac{n}{2}$ porte (ammesso un numero pari di poli).

Ciò che ci è di interesse sono i circuiti a \textbf{due porte} (o equivalentemente a \textit{quattro poli}):

\begin{center}
	\begin{circuitikz}
		\node[rectangle, draw, minimum width = 2cm, minimum height = 2cm] (a) at (0,0) {};
		\draw (-2, 0.6) to [ short, i=$I_1$] (-1, 0.6);
		\draw(-1, -0.6) to [ short, i=$I_1$ ] (-2, -0.6);	
	
		\draw (-2.6, 0.6) node[anchor=west] {$+$};
		\draw (-2.6, 0) node[anchor=west] {$V_1$};
		\draw (-2.6, -0.6) node[anchor=west] {$-$};
		
		\draw (2, 0.6) to [ short, i_=$I_2$] (1, 0.6);
		\draw(1, -0.6) to [ short, i_=$I_2$ ] (2, -0.6);	
	
		\draw (2.6, 0.6) node[anchor=east] {$+$};
		\draw (2.6, 0) node[anchor=east] {$V_2$};
		\draw (2.6, -0.6) node[anchor=east] {$-$};
	\end{circuitikz}
\end{center}

Possiamo immaginare che un segnale \textit{entra} da una porta, viene \textit{elaborato} all'interno del circuito, e \textit{esce} dalla porta opposta.

Per convenzione, scegliamo le due correnti $I_1(t)$e $I_2(t)$ come rivolte nello stesso senso, e le due tensioni $V_1(t)$ e $V_2(t)$ come con la stessa polarità:

\subsection{Circuiti equivalenti di circuiti a due porte}
Ciò che può interessarci quando studiamo circuiti a due porte è ricavare \textbf{circuiti equivalenti}, cioè che si comportano in maniera equivalente agli effetti esterni.
L'idea è, come sempre, quella di prendere circuiti arbitrariamente complessi e ridurli a circuiti equivalenti relativamente semplici.

\subsubsection{Rappresentazione a parametri Z}
Una coppia di \textbf{induttori mutuamente accoppiati} rappresenta effettivamente un circuito a due porte, in quanto la stessa corrente entra e esce da ogni induttore (cioè si formano due porte).
\begin{center}
	\begin{circuitikz}
		\draw (0,0) to[ short, i=$i_1$] (1,0)
			to[ inductor , l=$L_1$] (1,-2)
			-- (0, -2);

		\draw (4,0) to[ short, i_=$i_2$] (3,0)
			to[ inductor , l_=$L_2$] (3,-2)
			-- (4, -2);

			\draw (0.9,-0.6) node {$\scriptscriptstyle\bullet$};
			\draw (2.9,-0.6) node {$\scriptscriptstyle\bullet$};

			\draw (0,0) node[anchor=east] {$A$};
			\draw (0,-2) node[anchor=east] {$B$};

			\draw (0,-0.5) node[anchor=east] {$+$};
			\draw (0,-1.5) node[anchor=east] {$-$};

			\draw (4,0) node[anchor=west] {$C$};
			\draw (4,-2) node[anchor=west] {$D$};

			\draw (4,-0.5) node[anchor=west] {$+$};
			\draw (4,-1.5) node[anchor=west] {$-$};

			\draw (0,-1) node[anchor=east] {$v_1$};
			\draw (4,-1) node[anchor=west] {$v_2$};
	\end{circuitikz}
\end{center}

Avevamo rappresentato questi circuiti come:
\[
	\begin{cases}
		\dot{V}_1 = j \omega L_1 \dot{I}_1 + j \omega M \dot{I}_2 \\	
		\dot{V}_2 = j \omega L_2 \dot{I}_2 + j \omega M \dot{I}_1	
	\end{cases}
\]

Analogamente, decidiamo di rappresentare un circuito a due porte attraverso equazioni che legano la tensione su una porta alla corrente su entrambe le porte:
\[
	\begin{cases}
		\dot{V}_1 = \overline{z_{11}} \dot{I}_1 + \overline{z_{12}} \dot{I}_2 \\ 	
		\dot{V}_2 = \overline{z_{21}} \dot{I}_1 + \overline{z_{22}} \dot{I}_2 	
	\end{cases}
\]

Per esprimere queste relazioni in forma più compatta, possiamo sfruttare il calcolo matriciale:
$$
\dot{V} = \overline{Z} \dot{I}
$$
dove $\dot{V}$ e $\dot{I}$ sono matrici:
$$
\begin{pmatrix}
	\dot{V}_1 \\ \dot{V}_2
\end{pmatrix}
= \overline{Z}
\begin{pmatrix}
	\dot{I}_1 \\ \dot{I}_2
\end{pmatrix}
$$
e $\overline{Z}$ sarà l'\textbf{impedenza} in forma matriciale:
$$
\overline{Z} =
\begin{pmatrix}
	\overline{z_{11}} & \overline{z_{12}} \\ 
	\overline{z_{21}} & \overline{z_{22}} 
\end{pmatrix}
$$

Date le equazioni riportate sopra che legano voltaggio a corrente, possiamo ricavare il valore di ogni componente di $\overline{Z}$ come:
$$
\overline{Z} =
\begin{pmatrix}
	\overline{z_{11}} = \frac{\dot{V}_1}{\dot{I}_1} \Big|_{\dot{I}_2 = 0} & 
	\overline{z_{12}} = \frac{\dot{V}_1}{\dot{I}_2} \Big|_{\dot{I}_1 = 0} \\
	\overline{z_{21}} = \frac{\dot{V}_2}{\dot{I}_1} \Big|_{\dot{I}_2 = 0} &
	\overline{z_{22}} = \frac{\dot{V}_2}{\dot{I}_2} \Big|_{\dot{I}_1 = 0}
\end{pmatrix}
$$
dove la notazione $a \Big|_b$ significa "$a$ quando $b$".

Si ha, attraverso queste relazioni, che basta misurare la tensione sulle porte in due stati ($\dot{I}_1 = 0$ e $\dot{I}_2 = 0$) per ricavare completamente i parametri $\overline{Z}$ del circuito, e ricavare quindi un circuito equivalente del tipo:

\begin{center}
	\begin{circuitikz}
		\draw (-4, 1) -- (-3, 1) 
			to [ european resistor, l=$\overline{z_{11}}$, i=$I_1$] (-1, 1)
			to [ controlled voltage source, v<=$\overline{z_{12}} \dot{I}_2$ ] (-1, -1) 
			to [ short, i=$I_1$ ] (-3, -1)	
			-- (-4, -1);
			
		\draw (-4.6, 1) node[anchor=west] {$+$};
		\draw (-4.6, 0) node[anchor=west] {$V_1$};
		\draw (-4.6, -1) node[anchor=west] {$-$};

		\draw (6,1) -- (5, 1) 
			to [ european resistor, l_=$\overline{z_{12}}$, i_=$I_2$] (3, 1)
			to [ controlled voltage source, v_<=$\overline{z_{21}} \dot{I}_1$ ] (3, -1) 
			to [ short, i_=$I_2$ ] (5, -1)
			-- (6, -1);
	
		\draw (6.6, 1) node[anchor=east] {$+$};
		\draw (6.6, 0) node[anchor=east] {$V_2$};
		\draw (6.6, -1) node[anchor=east] {$-$};
		
		\node[rectangle, draw, minimum width = 8.5cm, minimum height = 4cm] (a) at (1,0) {};
	\end{circuitikz}
\end{center}
dove si inseriscono i termini di impedenza $\overline{z_{11}}$ e $\overline{z_{22}}$ semplicemente come impedenze in serie alle porte 1 e 2, e i termini "associati" $\overline{z_{12}}$ e $\overline{z_{21}}$ come generatori di tensione pilotati (che generano, appunto, cadute di tensione pilotate, rispettivamente in $\dot{I}_2$ per la porta 1 e in $\dot{I}_1$ per la porta 2).

Il metodo naturale di analisi per questo circuito è correnti di maglia, che possiamo applicare alle due porte per poi eguagliare con la matrice delle impedenze:
\[
	\begin{cases}
		\dot{V}_1	= \overline{z_{11}} \dot{I}_1 + \overline{Z_{12}} \dot{I}_2 \\  	
		\dot{V}_2	= \overline{z_{22}} \dot{I}_2 + \overline{Z_{21}} \dot{I}_1 \\  	
	\end{cases}
\]
che combacia con quanto definito sulla rappresentazione in impedenza.

In particolare, nel caso $\overline{z_{12}} = \overline{z_{21}}$ si dice che la rete è \textbf{reciproca} e si può formare il circuito equivalente come:

\begin{center}
	\begin{circuitikz}
		\draw (-4, 1) -- (-3, 1) 
			to [ european resistor, l=$\overline{z_a}$, i=$I_1$] (0, 1)
			to [ european resistor, l=$\overline{z_b}$] (0, -1) 
			to [ short, i=$I_1$ ] (-3, -1)	
			-- (-4, -1);
			
		\draw (-4.6, 1) node[anchor=west] {$+$};
		\draw (-4.6, 0) node[anchor=west] {$V_1$};
		\draw (-4.6, -1) node[anchor=west] {$-$};

		\draw (4, 1) -- (3, 1) 
			to [ european resistor, l_=$\overline{z_c}$, i_=$I_2$] (0, 1);

		\draw (0, -1) to [ short, i_=$I_2$ ] (3, -1)
			-- (4, -1);
	
		\draw (4.6, 1) node[anchor=east] {$+$};
		\draw (4.6, 0) node[anchor=east] {$V_2$};
		\draw (4.6, -1) node[anchor=east] {$-$};
		
		\node[rectangle, draw, minimum width = 6.5cm, minimum height = 4cm] (a) at (0,0) {};
	\end{circuitikz}
\end{center}

Anche qui, applicando correnti di maglia, si ha:
\[
	\begin{cases}
		\dot{V}_1 = \overline{z_a} \dot{I}_1 + \overline{z_b}\left( \dot{I}_1 + \dot{I}_2 \right) = ( \overline{z_a} + \overline{z_b} ) \dot{I}_1 + \overline{z_b} \dot{I}_2 \\	
		\dot{V}_2 = \overline{z_c} \dot{I}_2 + \overline{z_b}\left( \dot{I}_1 + \dot{I}_2 \right) = ( \overline{z_c} + \overline{z_c} ) \dot{I}_2 + \overline{z_b} \dot{I}_1 \\ 	
	\end{cases}
\]
che rappresenta la rete reciproca, ponendo:
\[
	\overline{Z} =
	\begin{pmatrix}
		\overline{z_a} + \overline{z_b} & \overline{z_b} \\ 
		\overline{z_b} & \overline{z_b + z_c}
	\end{pmatrix}
	\Leftrightarrow
	\begin{cases}
		\overline{z_a} = \overline{z_{11}} - \overline{z_{12}} = \overline{z_{11}} - \overline{z_{21}} \\ 
		\overline{z_b} = \overline{z_{12}} = \overline{z_{21}} \\
		\overline{z_c} = \overline{z_{22}} - \overline{z_{12}} = \overline{z_{22}} - \overline{z_{21}}
	\end{cases}
\]

Notiamo che, per circuiti a due porte generici, non è detto che i potenziali dei morsetti di uscita di entrambe le porte siano allo stesso potenziale: per modellizzare questo comportamento si usa una \textit{mutua induttanza ideale}, cioè un \textbf{trasformatore ideale}.


\subsubsection{Rappresentazione a parametri Y}
Nella rappresentazione di un circuito a due porte possiamo parametrizzare, anzichè l'impedenza $\overline{Z}$, l'ammettenza $\overline{Y}$: se avevamo espresso il comportamento del circuito come $
\begin{pmatrix}
	\dot{V}_1 \\ \dot{V}_2
\end{pmatrix}
= \overline{Z}
\begin{pmatrix}
	\dot{I}_1 \\ \dot{I}_2
\end{pmatrix}
$, infatti, possiamo trovare l'inverso:
$$
\begin{pmatrix}
	\dot{I}_1 \\ \dot{I}_2
\end{pmatrix}
= \overline{Z}^{-1}
\begin{pmatrix}
	\dot{V}_1 \\ \dot{V}_2
\end{pmatrix}
$$
dove la matrice $\overline{Z}^{-1} = \overline{Y}$ è effettivamente l'\textbf{ammettenza} in forma matriciale del circuito:
$$
\overline{Y} =
\begin{pmatrix}
	\overline{y_{11}} & \overline{y_{12}} \\ 
	\overline{y_{21}} & \overline{y_{22}}
\end{pmatrix}
$$

Da cui il sistema lineare:
\[
	\begin{cases}
		\dot{I}_1 = \overline{y_{11}} \dot{V}_1 + \overline{y_{12}} \dot{V}_2 \\ 	
		\dot{I}_2 = \overline{y_{21}} \dot{V}_1 + \overline{y_{22}} \dot{V}_2 	
	\end{cases}
\]

Date le equazioni riportate sopra, possiamo ricavare il valore di ogni componente di $\overline{Y}$ come:
$$
\overline{Y} =
\begin{pmatrix}
	\overline{y_{11}} = \frac{\dot{I}_1}{\dot{V}_1} \Big|_{\dot{V}_2 = 0} &
	\overline{y_{12}} = \frac{\dot{I}_1}{\dot{V}_2} \Big|_{\dot{V}_1 = 0} \\
	\overline{y_{21}} = \frac{\dot{I}_2}{\dot{V}_1} \Big|_{\dot{V}_2 = 0} &
	\overline{y_{22}} = \frac{\dot{I}_2}{\dot{V}_2} \Big|_{\dot{V}_1 = 0}
\end{pmatrix}
$$

Possiamo quindi disporre un circuito equivalente come segue:
\begin{center}
	\begin{circuitikz}
		\draw (-4, 1) to [ short, i=$I_1$] (-3, 1) 
			to [ short] (-1, 1);
		\draw (-1, 1) to [ controlled current source, cI>=$\overline{y_{12}} \dot{V}_1$ ] (-1, -1);
		\draw (-1, -1) to [ short] (-3, -1)
		-- (-4, -1);
			
		\draw (-4.6, 1) node[anchor=west] {$+$};
		\draw (-4.6, 0) node[anchor=west] {$V_1$};
		\draw (-4.6, -1) node[anchor=west] {$-$};

		\draw (6,1) to [ short, i=$I_2$] (5, 1) 
			to [ short] (3, 1)
			to [ controlled current source, cI_>=$\overline{y_{21}} \dot{V}_2$ ] (3, -1) 
			to [ short] (5, -1)
			-- (6, -1);
	
		\draw (6.6, 1) node[anchor=east] {$+$};
		\draw (6.6, 0) node[anchor=east] {$V_2$};
		\draw (6.6, -1) node[anchor=east] {$-$};
		
		\draw (-2, 1) to [ european resistor, l_=$\overline{y_{11}}$] (-2, -1);
	\draw (4, 1) to [ european resistor, l=$\overline{y_{22}}$] (4, -1);
		
		\node[rectangle, draw, minimum width = 8.5cm, minimum height = 4cm] (a) at (1,0) {};
	\end{circuitikz}
\end{center}
dove stavolta si inseriscono i termini di ammettenza $\overline{y_{11}}$ e $\overline{y_{22}}$ semplicemente ammettenze in parallelo alle porte 1 e 2, e i termini "associati" $\overline{y_{12}}$ e $\overline{y_{21}}$ come generatori di corrente pilotati.
Possiamo analizzare questo circuito considerando le corrente sui rami impedenza e generatore di entrambe le porte, da cui si ottiene:
\[
	\begin{cases}
		\dot{I}_1	= \overline{y_{11}} \dot{V}_1 + \overline{y_{12}} \dot{V}_2 \\  	
		\dot{I}_2	= \overline{y_{22}} \dot{V}_2 + \overline{y_{21}} \dot{V}_1 \\  	
	\end{cases}
\]
che combacia con quanto definito sulla rappresentazione in ammettenza.

In particolare, vediamo il caso \textbf{reciproco} $\overline{y_{11}} = \overline{y_{21}}$: 
\begin{center}
	\begin{circuitikz}
		\draw (-1, 1) to[ european resistor, l_=$\overline{y_a}$] (-1, -1);
		\draw (1, 1) to[ european resistor, l=$\overline{y_c}$] (1, -1);
		
		\draw (-4, 1) to[ short, i=$I_1$] (-3, 1)
			to[ european resistor, l=$\overline{y_b}$] (3, 1)
			to[ short, i<=$I_2$] (4, 1);		

		\draw (-4, -1) to[ short] (4, -1);

		\draw (-4.6, 1) node[anchor=west] {$+$};
		\draw (-4.6, 0) node[anchor=west] {$V_1$};
		\draw (-4.6, -1) node[anchor=west] {$-$};

		\draw (4.6, 1) node[anchor=east] {$+$};
		\draw (4.6, 0) node[anchor=east] {$V_2$};
		\draw (4.6, -1) node[anchor=east] {$-$};
	
		\node [ground] at (0, -1) {};
		
		\draw (-1,1) node[circ] {};
		\draw (-1,1) node[above] {$V_1$};

		\draw (1,1) node[circ] {};
		\draw (1,1) node[above] {$V_2$};
		
		\node[rectangle, draw, minimum width = 6.5cm, minimum height = 4cm] (a) at (0,0) {};
	\end{circuitikz}
\end{center}

Il metodo naturale di analisi per questo circuito è tensioni di nodo, che possiamo applicare alle due porte per poi eguagliare con la matrice delle ammettenze.
Prendiamo i due nodi in alto come $\dot{V}_1$ e $\dot{V}_2$, il nodo in basso come terra, e scriviamo le equazioni:
\[
	\begin{cases}
		\dot{I}_1 = (\overline{y_a} + \overline{y_b}) \dot{V}_1 - \overline{y_b} \dot{V}_2 \\	
		\dot{I}_2 = (\overline{y_b} + \overline{y_c}) \dot{V}_2 - \overline{y_b} \dot{V}_1
	\end{cases}
\]
che rappresenta la rete reciproca, ponendo:
\[
	\overline{Y} = 
	\begin{pmatrix}
		\overline{y_a} + \overline{y_b} & -\overline{y_b} \\ 
		-\overline{y_b} & \overline{y_b} + \overline{y_c}
	\end{pmatrix}
	\Leftrightarrow
	\begin{cases}
		\overline{y_a} = \overline{y_{11}} + \overline{y_{12}} \\ 
		\overline{y_b} = -\overline{y_{12}} = -\overline{y_{21}} \\
		\overline{y_c} = \overline{y_{22}} + \overline{y_{12}}
	\end{cases}
\]

Notiamo che ancora che il circuito più generale si ottiene disaccoppiando i potenziali sul ramo in basso attraverso un trasformatore ideale.
\end{document}


\documentclass[a4paper,11pt]{article}
\usepackage[a4paper, margin=8em]{geometry}

% usa i pacchetti per la scrittura in italiano
\usepackage[french,italian]{babel}
\usepackage[T1]{fontenc}
\usepackage[utf8]{inputenc}
\frenchspacing 

% usa i pacchetti per la formattazione matematica
\usepackage{amsmath, amssymb, amsthm, amsfonts}

% usa altri pacchetti
\usepackage{gensymb}
\usepackage{hyperref}
\usepackage{standalone}

% imposta il titolo
\title{Appunti Elettrotecnica}
\author{Luca Seggiani}
\date{2024}

% imposta lo stile
% usa helvetica
\usepackage[scaled]{helvet}
% usa palatino
\usepackage{palatino}
% usa un font monospazio guardabile
\usepackage{lmodern}

\renewcommand{\rmdefault}{ppl}
\renewcommand{\sfdefault}{phv}
\renewcommand{\ttdefault}{lmtt}

% disponi il titolo
\makeatletter
\renewcommand{\maketitle} {
	\begin{center} 
		\begin{minipage}[t]{.8\textwidth}
			\textsf{\huge\bfseries \@title} 
		\end{minipage}%
		\begin{minipage}[t]{.2\textwidth}
			\raggedleft \vspace{-1.65em}
			\textsf{\small \@author} \vfill
			\textsf{\small \@date}
		\end{minipage}
		\par
	\end{center}

	\thispagestyle{empty}
	\pagestyle{fancy}
}
\makeatother

% disponi teoremi
\usepackage{tcolorbox}
\newtcolorbox[auto counter, number within=section]{theorem}[2][]{%
	colback=blue!10, 
	colframe=blue!40!black, 
	sharp corners=northwest,
	fonttitle=\sffamily\bfseries, 
	title=~\thetcbcounter: #2, 
	#1
}

% disponi definizioni
\newtcolorbox[auto counter, number within=section]{definition}[2][]{%
	colback=red!10,
	colframe=red!40!black,
	sharp corners=northwest,
	fonttitle=\sffamily\bfseries,
	title=~\thetcbcounter: #2,
	#1
}

% U.D.M
\newcommand{\amp}{\ensuremath{\mathrm{A}}}
\newcommand{\volt}{\ensuremath{\mathrm{V}}}
\newcommand{\meter}{\ensuremath{\mathrm{m}}}
\newcommand{\second}{\ensuremath{\mathrm{s}}}
\newcommand{\farad}{\ensuremath{\mathrm{F}}}
\newcommand{\henry}{\ensuremath{\mathrm{H}}}
\newcommand{\siemens}{\ensuremath{\mathrm{S}}}

% circuiti
\usepackage{circuitikz}
\usetikzlibrary{babel}

% disegni
\usepackage{pgfplots}
\pgfplotsset{width=10cm,compat=1.9}

% disponi codice
\usepackage{listings}
\usepackage[table]{xcolor}

\lstdefinestyle{codestyle}{
		backgroundcolor=\color{black!5}, 
		commentstyle=\color{codegreen},
		keywordstyle=\bfseries\color{magenta},
		numberstyle=\sffamily\tiny\color{black!60},
		stringstyle=\color{green!50!black},
		basicstyle=\ttfamily\footnotesize,
		breakatwhitespace=false,         
		breaklines=true,                 
		captionpos=b,                    
		keepspaces=true,                 
		numbers=left,                    
		numbersep=5pt,                  
		showspaces=false,                
		showstringspaces=false,
		showtabs=false,                  
		tabsize=2
}

\lstdefinestyle{shellstyle}{
		backgroundcolor=\color{black!5}, 
		basicstyle=\ttfamily\footnotesize\color{black}, 
		commentstyle=\color{black}, 
		keywordstyle=\color{black},
		numberstyle=\color{black!5},
		stringstyle=\color{black}, 
		showspaces=false,
		showstringspaces=false, 
		showtabs=false, 
		tabsize=2, 
		numbers=none, 
		breaklines=true
}

\lstdefinelanguage{javascript}{
	keywords={typeof, new, true, false, catch, function, return, null, catch, switch, var, if, in, while, do, else, case, break},
	keywordstyle=\color{blue}\bfseries,
	ndkeywords={class, export, boolean, throw, implements, import, this},
	ndkeywordstyle=\color{darkgray}\bfseries,
	identifierstyle=\color{black},
	sensitive=false,
	comment=[l]{//},
	morecomment=[s]{/*}{*/},
	commentstyle=\color{purple}\ttfamily,
	stringstyle=\color{red}\ttfamily,
	morestring=[b]',
	morestring=[b]"
}

% disponi sezioni
\usepackage{titlesec}

\titleformat{\section}
	{\sffamily\Large\bfseries} 
	{\thesection}{1em}{} 
\titleformat{\subsection}
	{\sffamily\large\bfseries}   
	{\thesubsection}{1em}{} 
\titleformat{\subsubsection}
	{\sffamily\normalsize\bfseries} 
	{\thesubsubsection}{1em}{}

% disponi alberi
\usepackage{forest}

\forestset{
	rectstyle/.style={
		for tree={rectangle,draw,font=\large\sffamily}
	},
	roundstyle/.style={
		for tree={circle,draw,font=\large}
	}
}

% disponi algoritmi
\usepackage{algorithm}
\usepackage{algorithmic}
\makeatletter
\renewcommand{\ALG@name}{Algoritmo}
\makeatother

% disponi numeri di pagina
\usepackage{fancyhdr}
\fancyhf{} 
\fancyfoot[L]{\sffamily{\thepage}}

\makeatletter
\fancyhead[L]{\raisebox{1ex}[0pt][0pt]{\sffamily{\@title \ \@date}}} 
\fancyhead[R]{\raisebox{1ex}[0pt][0pt]{\sffamily{\@author}}}
\makeatother

\begin{document}
% sezione (data)
\section{Lezione del 09-10-24}

% stili pagina
\thispagestyle{empty}
\pagestyle{fancy}

% testo
\subsection{Teorema di Thevenin}
Poniamo di avere un certo bipolo, inteso come una sottorete con due morsetti.
All'interno del bipolo avremo resistori, generatori di corrente e generatori di tensioni.
Quello che vogliamo fare è rappresentare la sottorete con un circuito semplificato, come avevamo fatto con circuiti di solo resistori.

Pensiamo di mettere un generatore di corrente fra i morsetti del bipolo, in modo da imporre una certa corrente, e calcolare la tensione risultante.
Facendo misure diverse a correnti diverse, dovremmo essere in grado di trovare un legame fra tensione $v(t)$ e corrente $i(t)$ attraverso, ad esempio, la legge di Ohm.
A questo punto sapremmo sostituire l'intera rete con un singolo dispositivo che rappresenti la sua risposta interna a qualsiasi stimolo esterno.

Questo è quello che ci permette di fare il teorema di Thevenin:
\begin{theorem}{Teorema di Thevenin}
	Qualsiasi circuito lineare, visto da due morsetti, è equivalente a un generatore di tensione in serie a un resistore.

\begin{itemize}
	\item La \textbf{resistenza di Thevenin} $R_{th}$, equivale alla resistenza vista da dai morsetti calcolata dopo aver disattivato tutti i generatori indipendenti della sottorete;
	\item La \textbf{tensione di Thevenin} $V_{th}$, è pari al valore della tensione a vuoto misurata fra i morsetti, dove per tensione a vuoto si intende la tensione della sottorete isolata da altri circuiti. Questo equivale a \textbf{tagliare} fuori i rami del circuito che vanno ai morsetti, compresi eventuali bipoli presenti su quei rami (non vi scorrerà alcuna corrente).
\end{itemize}
\end{theorem}

Poste $R_{th}$ e $V_{th}$, si può dire che la correlazione fra voltaggio e corrente del circuito è:
$$
v(t) = V_{th} - R_{th} i(t)
$$
dove il segno meno alla corrente è lì perchè, all'aumentare della corrente fatta passare attraverso i morsetti (considerando, per il principio di sovrapposizione, i generatori interni disattivati), aumenta la differenza di potenziale fra di loro ($v_{load} = R_{th} i(t)$), e quindi diminuisce quella sul generatore equivalente di Thevenin $v(t)$.
Infatti, $V_{th}$ era stata definita come la differenza di potenziale a vuoto fra i morsetti, e quindi dovrà essere:
$$
v_{load} + v(t) = V_{th}
$$

\subsection{Teorema di Norton}
Il teorema di Thevenin ammette un duale: come possiamo usare un generatore di tensione, possiamo infatti usare un generatore di corrente per rappresentare un'intera sottorete.
Formuliamo quindi:
\begin{theorem}{Teorema di Norton}
	Qualsiasi circuito lineare, visto da due morsetti, è equivalente a un generatore di corrente in parallelo a un resistore.

\begin{itemize}
	\item La \textbf{resistenza di Norton} $R_{no}$, equivale alla resistenza di Thevenin, cioè alla resistenza vista da dai morsetti calcolata dopo aver disattivato tutti i generatori indipendenti della sottorete;
	\item La \textbf{corrente di Norton} $I_{no}$, è pari al valore della corrente misurata su un ramo (posto da noi per effettuare la misura) che unisce i morsetti.\end{itemize}
\end{theorem}

Con Norton, anzichè la differenza di potenziale a vuoto $V_{th}$ ricavata per Thevenin, si considera la corrente che scorrerebbe fra i morsetti se fossero collegati fra di loro.
Avremo quindi che la corrente alla variazione della differenza di potenziale fra le maglie è:
$$
i(t) = I_{no} - \frac{v(t)}{R_{no}}
$$
Questo perché, come prima, dovrà essere che la corrente fra i morsetti, disattivati i generatori interni, $i_load$ e la corrente sul circuito equivalente di Norton $i(t)$ sono legate da:
$$
i_{load} + i(t) = I_{no}
$$

\par\medskip 

Possiamo stabilire una relazione fra la resistenza e la tensione di Thevenin, $R_{th}$ e $V_{th}$, e la resistenza e la corrente di Norton, $R_{no}$ e $I_{no}$:
\[
	\begin{cases}
		R_{th} = R_{no} \\ 
		V_{th} = I_{no} \cdot R_{no}
	\end{cases}
\]

In ogni caso, vale $R_{th} = R_{no} = R_{eq}$ del circuito.

Si può quindi ricavare il corollario:
\begin{theorem}{Corollario da Thevenin e Norton}
	Un generatore di corrente in parallelo a una resistenza equivale ad un generatore di tensione in serie alla stessa resistenza (salvo casi limite).
\end{theorem}

\par\medskip 

\subsubsection{Generatori dipendenti}
Nel caso si incontrino generatori di corrente o di tensione dipendenti, bisogna considerare che potrebbe essere impossibile applicare Thevenin o Norton su una parte di circuito, poiché la grandezza pilota del generatore potrebbe trovarsi nell'altra parte del circuito.

Nel caso questo sia possibile, invece, si adotta un \textbf{generatore di prova} che può essere arbitrariamente di tensione o di corrente.
Potremo infatti ricavare l'una dall'altra come:
$$
R_{eq} = \frac{V_p}{I_p}
$$
dalla legge di Ohm.


\end{document}


\documentclass[a4paper,11pt]{article}
\usepackage[a4paper, margin=8em]{geometry}

% usa i pacchetti per la scrittura in italiano
\usepackage[french,italian]{babel}
\usepackage[T1]{fontenc}
\usepackage[utf8]{inputenc}
\frenchspacing 

% usa i pacchetti per la formattazione matematica
\usepackage{amsmath, amssymb, amsthm, amsfonts}

% usa altri pacchetti
\usepackage{gensymb}
\usepackage{hyperref}
\usepackage{standalone}

% imposta il titolo
\title{Appunti Elettrotecnica}
\author{Luca Seggiani}
\date{2024}

% imposta lo stile
% usa helvetica
\usepackage[scaled]{helvet}
% usa palatino
\usepackage{palatino}
% usa un font monospazio guardabile
\usepackage{lmodern}

\renewcommand{\rmdefault}{ppl}
\renewcommand{\sfdefault}{phv}
\renewcommand{\ttdefault}{lmtt}

% disponi il titolo
\makeatletter
\renewcommand{\maketitle} {
	\begin{center} 
		\begin{minipage}[t]{.8\textwidth}
			\textsf{\huge\bfseries \@title} 
		\end{minipage}%
		\begin{minipage}[t]{.2\textwidth}
			\raggedleft \vspace{-1.65em}
			\textsf{\small \@author} \vfill
			\textsf{\small \@date}
		\end{minipage}
		\par
	\end{center}

	\thispagestyle{empty}
	\pagestyle{fancy}
}
\makeatother

% disponi teoremi
\usepackage{tcolorbox}
\newtcolorbox[auto counter, number within=section]{theorem}[2][]{%
	colback=blue!10, 
	colframe=blue!40!black, 
	sharp corners=northwest,
	fonttitle=\sffamily\bfseries, 
	title=~\thetcbcounter: #2, 
	#1
}

% disponi definizioni
\newtcolorbox[auto counter, number within=section]{definition}[2][]{%
	colback=red!10,
	colframe=red!40!black,
	sharp corners=northwest,
	fonttitle=\sffamily\bfseries,
	title=~\thetcbcounter: #2,
	#1
}

% U.D.M
\newcommand{\amp}{\ensuremath{\mathrm{A}}}
\newcommand{\volt}{\ensuremath{\mathrm{V}}}
\newcommand{\meter}{\ensuremath{\mathrm{m}}}
\newcommand{\second}{\ensuremath{\mathrm{s}}}
\newcommand{\farad}{\ensuremath{\mathrm{F}}}
\newcommand{\henry}{\ensuremath{\mathrm{H}}}
\newcommand{\siemens}{\ensuremath{\mathrm{S}}}

% circuiti
\usepackage{circuitikz}
\usetikzlibrary{babel}

% disegni
\usepackage{pgfplots}
\pgfplotsset{width=10cm,compat=1.9}

% disponi codice
\usepackage{listings}
\usepackage[table]{xcolor}

\lstdefinestyle{codestyle}{
		backgroundcolor=\color{black!5}, 
		commentstyle=\color{codegreen},
		keywordstyle=\bfseries\color{magenta},
		numberstyle=\sffamily\tiny\color{black!60},
		stringstyle=\color{green!50!black},
		basicstyle=\ttfamily\footnotesize,
		breakatwhitespace=false,         
		breaklines=true,                 
		captionpos=b,                    
		keepspaces=true,                 
		numbers=left,                    
		numbersep=5pt,                  
		showspaces=false,                
		showstringspaces=false,
		showtabs=false,                  
		tabsize=2
}

\lstdefinestyle{shellstyle}{
		backgroundcolor=\color{black!5}, 
		basicstyle=\ttfamily\footnotesize\color{black}, 
		commentstyle=\color{black}, 
		keywordstyle=\color{black},
		numberstyle=\color{black!5},
		stringstyle=\color{black}, 
		showspaces=false,
		showstringspaces=false, 
		showtabs=false, 
		tabsize=2, 
		numbers=none, 
		breaklines=true
}

\lstdefinelanguage{javascript}{
	keywords={typeof, new, true, false, catch, function, return, null, catch, switch, var, if, in, while, do, else, case, break},
	keywordstyle=\color{blue}\bfseries,
	ndkeywords={class, export, boolean, throw, implements, import, this},
	ndkeywordstyle=\color{darkgray}\bfseries,
	identifierstyle=\color{black},
	sensitive=false,
	comment=[l]{//},
	morecomment=[s]{/*}{*/},
	commentstyle=\color{purple}\ttfamily,
	stringstyle=\color{red}\ttfamily,
	morestring=[b]',
	morestring=[b]"
}

% disponi sezioni
\usepackage{titlesec}

\titleformat{\section}
	{\sffamily\Large\bfseries} 
	{\thesection}{1em}{} 
\titleformat{\subsection}
	{\sffamily\large\bfseries}   
	{\thesubsection}{1em}{} 
\titleformat{\subsubsection}
	{\sffamily\normalsize\bfseries} 
	{\thesubsubsection}{1em}{}

% disponi alberi
\usepackage{forest}

\forestset{
	rectstyle/.style={
		for tree={rectangle,draw,font=\large\sffamily}
	},
	roundstyle/.style={
		for tree={circle,draw,font=\large}
	}
}

% disponi algoritmi
\usepackage{algorithm}
\usepackage{algorithmic}
\makeatletter
\renewcommand{\ALG@name}{Algoritmo}
\makeatother

% disponi numeri di pagina
\usepackage{fancyhdr}
\fancyhf{} 
\fancyfoot[L]{\sffamily{\thepage}}

\makeatletter
\fancyhead[L]{\raisebox{1ex}[0pt][0pt]{\sffamily{\@title \ \@date}}} 
\fancyhead[R]{\raisebox{1ex}[0pt][0pt]{\sffamily{\@author}}}
\makeatother

\begin{document}
% sezione (data)
\section{Lezione del 10-10-24}

% stili pagina
\thispagestyle{empty}
\pagestyle{fancy}

% testo
\subsection{Dimostrazione di Thevenin}
Avevamo detto che, attraverso Thevenin, si può trasformare qualsiasi rete come vista da due morsetti in una rete equivalente formata da un resistore e un generatore di tensione in serie.
Dimostriamo questo risultato.

Supponiamo di avere un circuito elettrico che andiamo a dividere in due sottoreti.
Poniamo di voler semplificare una di queste sottoreti (magari quella di sinistra), presi due morsetti, $1$ e $2$, che la collegano all'altra.
Avremo che da questi morsetti passa una corrente $i(t)$ e che essi si trovano ad una differenza di potenziale $v(t)$.

\begin{center}
	\begin{circuitikz}
    % First rectangle (component)
    \draw (0,0) rectangle (3,3) node[midway] {};
    
    % Second rectangle (component)
    \draw (7,0) rectangle (10,3) node[midway] {};

		\node at(5,2) {$+$};
		\node at(5,1.5) {$v(t)$};
		\node at(5,1) {$-$};

		\node at(3.2,2.8) {$1$};
		\node at(3.2,0.8) {$2$};
		
		\draw (3,0.5) to[ short ] (7,0.5);
		\draw (3,2.5) to[ short, i = $i(t)$ ] (7,2.5);
	\end{circuitikz}
\end{center}

Possiamo quindi pensare di separare le due sottoreti, e di sostituire ai morsetti dei generatori di corrente che replicano la corrente $i(t)$ vista prima.
Se la tensione ai capi dei generatori resta $v(t)$, allora i circuiti non sono cambiati.

\begin{center}
	\begin{circuitikz}
    % First rectangle (component)
    \draw (0,0) rectangle (3,3) node[midway] {};
    
    % Second rectangle (component)
    \draw (7,0) rectangle (10,3) node[midway] {};

		\node at(5,2) {$+$};
		\node at(5,1.5) {$v(t)$};
		\node at(5,1) {$-$};

		\node at(3.2,2.8) {$1$};
		\node at(3.2,0.8) {$2$};
		
		\draw (7,0.5) to[ short ] (6,0.5)
			to [ short, I = $i(t)$ ] (6, 2.5)
			to [ short ] (7, 2.5);
		\draw (3,2.5) to[ short ] (4,2.5)
			to[ short, I = $i(t)$] (4,0.5)
			to [short] (3, 0.5);
	\end{circuitikz}
\end{center}

Questo prende il nome di \textbf{principio di sostituzione}.

A questo punto possiamo dire che all'interno delle sottoreti troveremo bipoli attivi (genratori di tensione e di corrente), e passivi (resistori e generatori pilotati di tensione e di corrente):

\begin{center}
	\begin{circuitikz}
    % First rectangle (component)
    \draw (0,0) rectangle (3,3) node[midway] {};
    
		\node at(5,2) {$+$};
		\node at(5,1.5) {$v(t)$};
		\node at(5,1) {$-$};

		\node at(3.2,2.8) {$1$};
		\node at(3.2,0.8) {$2$};
		
		\draw (3,2.5) to[ short ] (4,2.5)
			to[ short, I = $i(t)$] (4,0.5)
			to [short] (3, 0.5);

		\draw (0,0.5) to[ short ] (-1,0.5)
			to[ short, I = $I$] (-1,2.5)
			to [short] (0, 2.5);

		\draw (0.5,0) to[ short ] (0.5,-1)
			to[ short, V_ = $E$] (2.5,-1)
			to [short] (2.5, 0);

		\draw (0.4,0.2) to [ R ] (0.4, 2.8);
		\draw (1.3,0.2) to [ controlled voltage source ] (1.3, 2.8);
		\draw (2.4,0.2) to [ controlled current source ] (2.4, 2.8);
	\end{circuitikz}
\end{center}

Si decide di applicare il principio di sovrapposizione, ponendo il voltaggio dato dai generatori interni come $v'(t)$ e quello dato dal generatore esterno come $v''(t)$, da cui:
$$
v(t) = v'(t) + v''(t)
$$

Staccate dal resto del circuito si ha che le sottoreti portano i morsetti a tensione $v'(t) = v_0(t)$.
Incluso il generatore di corrente di prima, possiamo pensare di disattivare i generatori dipendenti propri della sottorete, per ricavare un'altra tensione $v''(t)$.
Visto che il circuito è formato effettivamente da componenti passivi, possiamo calcolare una resistenza vista, che chiamiamo $R_{12}$.
Abbiamo allora che, incluso il generatore di corrente di prova $i(t)$:
$$
v''(t) = -R_{12} \cdot i(t)
$$
dove il segno meno è dato dal fatto che abbiamo preso il generatore esterno come un riferimento non associato (spinge carica dal potenziale positivo a quello negativo), e quindi la tensione totale è:
$$
v(t) = v_0(t) - R_{12} \cdot i(t)
$$

Questo equivale a rappresentare il circuito come un generatore di tensione $v_{TH} = v_0(t)$ in serie ad una resistenza $v_{TH} = R_{12}$.

\subsection{Metodo delle correnti di maglia}
Vediamo un metodo alternativo per la risoluzione dei circuiti, simile a quello delle correnti di ramo ma che genera meno variabili.
Questo metodo presuppone di prendere, anziché le correnti su ogni ramo, un numero minore di correnti, prese su intere maglie.
Le maglie vengono scelte secondo lo stesso metodo del tableau: dopo aver usato una maglia, se ne taglia un ramo.

Prendiamo in esempio il circuito:

\begin{center}
\begin{circuitikz}
	\draw (0,0)
		-- (0,1.5)
		to[R, l=$R_1$] (5,1.5)
		-- (5, 0);
	\draw (0,0)
		to[R, l=$R_2$] (2.5, 0)
		to[R, l=$R_3$] (5, 0);
	\draw (0,0)
		to[R, l=$R_4$] (0, -3);
	\draw (2.5,0)
		to[R, l=$R_5$] (2.5, -3);
	\draw (5,-3)
		to[ european voltage source, v=$E_2$] (5, 0);
	\draw (5, -3)
		to[R, l=$R_6$] (2.5, -3)
		to[ european voltage source, v=$E_1$] (0, -3);
\end{circuitikz}
\end{center}

Avevamo già risolto questa rete usando il metodo del tableau, che potremmo dire \textit{"metodo delle correnti di ramo"}, trovando un sistema lineare dalla prima e seconda legge di Kirchoff.
Il problema di tale metodo era la grande quantità di variabili: una per ogni ramo.
Applichiamo adesso il metodo delle correnti di maglia.
Scegliamo ogni maglia a sé stante: in basso a sinistra, in basso a destra e in alto.
Diamo un nome alla corrente su ognuna di queste maglie, rispettivamente $I_a$, $I_b$ e $I_c$.
Le direzioni delle correnti sono come in figura (scelte ad arbitrio):

\begin{center}
\begin{circuitikz}
	\draw (0,0)
		-- (0,1.5)
		to[R, l=$R_1$] (5,1.5)
		-- (5, 0);
	\draw (0,0)
		to[R, l=$R_2$] (2.5, 0)
		to[R, l=$R_3$] (5, 0);
	\draw (0,0)
		to[R, l=$R_4$] (0, -3);
	\draw (2.5,0)
		to[R, l=$R_5$] (2.5, -3);
	\draw (5,-3)
		to[ european voltage source, v_=$E_2$] (5, 0);
	\draw (5, -3)
		to[R, l=$R_6$] (2.5, -3) 
		to[ european voltage source, v=$E_1$] (0, -3);

	\draw[->, thick] (3,0.6) arc[start angle=0, end angle=-270, radius=0.4cm]; % Loop direction
	\node at (3, 0.8) {$I_c$}; % Adjust the position (0.5, 0.5) to place the label where desired

	\draw[->, thick] (1.8,-1.6) arc[start angle=0, end angle=-270, radius=0.4cm]; % Loop direction
	\node at (1.8, -1.2) {$I_a$}; % Adjust the position (0.5, 0.5) to place the label where desired

	\draw[->, thick] (4.2,-1.6) arc[start angle=0, end angle=270, radius=0.4cm]; % Loop direction
	\node at (4.2, -1.9) {$I_b$}; % Adjust the position (0.5, 0.5) to place the label where desired
\end{circuitikz}
\end{center}

Iniziamo con la maglia in basso a sinistra: percorsa in senso orario, avremo che la seconda legge di Kirchoff dà:
$$
E_1 - R_4 I_a + R_2 ( I_c - I_a ) + R_5 ( -I_a - I_b) = 0
$$
dove si nota che per bipoli percorsi da più correnti di maglia, si prende come corrente la somma algebrica delle correnti sulla base del loro segno.
Applichiamo quindi lo stesso metodo alle altre due maglie, percorrendo quella in basso a destra in senso antiorario e quella in alto in senso orario:
$$
E_2 + R_3(-I_b - I_c) + R_5 (-I_b - I_a) - R_6 I_b = 0
$$
$$
-R_1 I_c + R_3 (-I_c - I_b) + R_2 (I_a - I_c) = 0
$$

Da cui si ricava il sistema lineare:
\[
	\begin{cases}
			E_1 - R_4 I_a + R_2 ( I_c - I_a ) + R_5 ( -I_a - I_b) = 0 \\
			E_2 + R_3(-I_b - I_c) + R_5 (-I_b - I_a) - R_6 I_b = 0 \\
			-R_1 I_c + R_3 (-I_c - I_b) + R_2 (I_a - I_c) = 0
	\end{cases}
\]
in forma matriciale:
$$
\begin{pmatrix}
	-R_4-R_2-R_5 & -R_5 & R_2 \\ 
	-R_5 & -R_3-R_5-R_6 -R_3 \\ 
	R_2 & -R_3 & -R_1-R_3-R_2
\end{pmatrix}
\begin{pmatrix}
I_a \\ I_b \\ I_v
\end{pmatrix}
=
\begin{pmatrix}
-E_1 \\ -E_2 \\ 0
\end{pmatrix}
$$

Come sempre, questo sistema si risolve con qualsiasi metodo di risoluzione dei sistemi lineari.

\subsubsection{Circuiti con generatori di corrente}
Notiamo un caso particolare per l'applicazione di questo problema: nel caso il circuito contenga generatori di corrente, bisogna evitare di introdurre tensioni incognite, come nel metodo del tabeau.
Si sceglie quindi una maglia contenente il generatore di corrente per prima, e poi si taglia quel ramo.
Prendiamo ad esempio:

\begin{center}
\begin{circuitikz}
	\draw (0,0)
		to[R, l=$R_2$] (2.5, 0)
		to[R, l=$R_3$] (5, 0);
	\draw (0,-3)
		to[ european voltage source, v=$E$] (0, 0);
	\draw (2.5,-3)
		to[ european current source, I=$J$] (2.5, 0);
	\draw (5,-3)
		to[ short ] (5, 0);
	\draw (5, -3)
		to[ short ] (2.5, -3)
		to[ short ] (0, -3);
\end{circuitikz}
\end{center}

Qui il generatore di corrente $J$ genera una differenza di potenziale $V_J$, che non vogliamo come incognita.
Prenderemo quindi per prima una maglia che lo contiene (in questo caso qualsiasi tranne la più esterna), e poi taglieremo il suo ramo.
Ad esempio, prendendo prima la maglia a destra, si ha al primo passo:
$$
-E + R I_x + 2R(I_x + J) = 0
$$
dove si è potuto chiamare l'unica corrente incognita $I_x$.

Non ci è effettivamente negato scegliere di scegliere le maglie in quest'ordine, ma se lo facciamo, allora non dobbiamo usare la corrente $J$ nel calcolo, ma introdurre una corrente incognita per ogni maglia.
Ad esempio, se volessimo considerare le due maglie a sinistra ($I_s$) e a destra ($I_d$) separatamente:
$$ 
R I_s + E - V_j = 0 
$$
$$
J = I_x + I_b
$$

\end{document}


\documentclass[a4paper,11pt]{article}
\usepackage[a4paper, margin=8em]{geometry}

% usa i pacchetti per la scrittura in italiano
\usepackage[french,italian]{babel}
\usepackage[T1]{fontenc}
\usepackage[utf8]{inputenc}
\frenchspacing 

% usa i pacchetti per la formattazione matematica
\usepackage{amsmath, amssymb, amsthm, amsfonts}

% usa altri pacchetti
\usepackage{gensymb}
\usepackage{hyperref}
\usepackage{standalone}

% imposta il titolo
\title{Appunti Elettrotecnica}
\author{Luca Seggiani}
\date{2024}

% imposta lo stile
% usa helvetica
\usepackage[scaled]{helvet}
% usa palatino
\usepackage{palatino}
% usa un font monospazio guardabile
\usepackage{lmodern}

\renewcommand{\rmdefault}{ppl}
\renewcommand{\sfdefault}{phv}
\renewcommand{\ttdefault}{lmtt}

% disponi il titolo
\makeatletter
\renewcommand{\maketitle} {
	\begin{center} 
		\begin{minipage}[t]{.8\textwidth}
			\textsf{\huge\bfseries \@title} 
		\end{minipage}%
		\begin{minipage}[t]{.2\textwidth}
			\raggedleft \vspace{-1.65em}
			\textsf{\small \@author} \vfill
			\textsf{\small \@date}
		\end{minipage}
		\par
	\end{center}

	\thispagestyle{empty}
	\pagestyle{fancy}
}
\makeatother

% disponi teoremi
\usepackage{tcolorbox}
\newtcolorbox[auto counter, number within=section]{theorem}[2][]{%
	colback=blue!10, 
	colframe=blue!40!black, 
	sharp corners=northwest,
	fonttitle=\sffamily\bfseries, 
	title=~\thetcbcounter: #2, 
	#1
}

% disponi definizioni
\newtcolorbox[auto counter, number within=section]{definition}[2][]{%
	colback=red!10,
	colframe=red!40!black,
	sharp corners=northwest,
	fonttitle=\sffamily\bfseries,
	title=~\thetcbcounter: #2,
	#1
}

% U.D.M
\newcommand{\amp}{\ensuremath{\mathrm{A}}}
\newcommand{\volt}{\ensuremath{\mathrm{V}}}
\newcommand{\meter}{\ensuremath{\mathrm{m}}}
\newcommand{\second}{\ensuremath{\mathrm{s}}}
\newcommand{\farad}{\ensuremath{\mathrm{F}}}
\newcommand{\henry}{\ensuremath{\mathrm{H}}}
\newcommand{\siemens}{\ensuremath{\mathrm{S}}}

% circuiti
\usepackage{circuitikz}
\usetikzlibrary{babel}

% disegni
\usepackage{pgfplots}
\pgfplotsset{width=10cm,compat=1.9}

% disponi codice
\usepackage{listings}
\usepackage[table]{xcolor}

\lstdefinestyle{codestyle}{
		backgroundcolor=\color{black!5}, 
		commentstyle=\color{codegreen},
		keywordstyle=\bfseries\color{magenta},
		numberstyle=\sffamily\tiny\color{black!60},
		stringstyle=\color{green!50!black},
		basicstyle=\ttfamily\footnotesize,
		breakatwhitespace=false,         
		breaklines=true,                 
		captionpos=b,                    
		keepspaces=true,                 
		numbers=left,                    
		numbersep=5pt,                  
		showspaces=false,                
		showstringspaces=false,
		showtabs=false,                  
		tabsize=2
}

\lstdefinestyle{shellstyle}{
		backgroundcolor=\color{black!5}, 
		basicstyle=\ttfamily\footnotesize\color{black}, 
		commentstyle=\color{black}, 
		keywordstyle=\color{black},
		numberstyle=\color{black!5},
		stringstyle=\color{black}, 
		showspaces=false,
		showstringspaces=false, 
		showtabs=false, 
		tabsize=2, 
		numbers=none, 
		breaklines=true
}

\lstdefinelanguage{javascript}{
	keywords={typeof, new, true, false, catch, function, return, null, catch, switch, var, if, in, while, do, else, case, break},
	keywordstyle=\color{blue}\bfseries,
	ndkeywords={class, export, boolean, throw, implements, import, this},
	ndkeywordstyle=\color{darkgray}\bfseries,
	identifierstyle=\color{black},
	sensitive=false,
	comment=[l]{//},
	morecomment=[s]{/*}{*/},
	commentstyle=\color{purple}\ttfamily,
	stringstyle=\color{red}\ttfamily,
	morestring=[b]',
	morestring=[b]"
}

% disponi sezioni
\usepackage{titlesec}

\titleformat{\section}
	{\sffamily\Large\bfseries} 
	{\thesection}{1em}{} 
\titleformat{\subsection}
	{\sffamily\large\bfseries}   
	{\thesubsection}{1em}{} 
\titleformat{\subsubsection}
	{\sffamily\normalsize\bfseries} 
	{\thesubsubsection}{1em}{}

% disponi alberi
\usepackage{forest}

\forestset{
	rectstyle/.style={
		for tree={rectangle,draw,font=\large\sffamily}
	},
	roundstyle/.style={
		for tree={circle,draw,font=\large}
	}
}

% disponi algoritmi
\usepackage{algorithm}
\usepackage{algorithmic}
\makeatletter
\renewcommand{\ALG@name}{Algoritmo}
\makeatother

% disponi numeri di pagina
\usepackage{fancyhdr}
\fancyhf{} 
\fancyfoot[L]{\sffamily{\thepage}}

\makeatletter
\fancyhead[L]{\raisebox{1ex}[0pt][0pt]{\sffamily{\@title \ \@date}}} 
\fancyhead[R]{\raisebox{1ex}[0pt][0pt]{\sffamily{\@author}}}
\makeatother

\begin{document}
% sezione (data)
\section{Lezione del 11-10-24}

% stili pagina
\thispagestyle{empty}
\pagestyle{fancy}

% testo
\subsection{Metodo delle tensioni di nodo}
Vediamo ora un metodo per la risoluzione dei circuiti dove le incognite non sono le correnti (come nel tableau o nelle correnti di maglia), ma le tensioni.
Calcoleremo quindi le \textbf{tensioni di nodo}, che sappiamo essere relative a qualche zero del potenziale, attraverso la conduttanza anziché la resistenza.

Prendiamo in esempio il circuito:

\begin{center}
\begin{circuitikz}
	\draw (0,0)
		-- (0,1.5)
		to[R, l=$R_1$] (5,1.5)
		-- (5, 0);
	\draw (0,0)
		to[R, l=$R_2$] (2.5, 0)
		to[R, l=$R_3$] (5, 0);
	\draw (0,0)
		to[R, l=$R_4$] (0, -3);
	\draw (2.5,-3)
		to[ european current source, I=$J$ ] (2.5, 0);
	\draw (5,-3)
		to[ R, l_=$R_5$] (5, 0);
	\draw (5, -3)
		to[ short ] (2.5, -3)
		to[ short ] (0, -3);

		\draw (0,0) node[circ] {};
		\draw (0,0) node[left] {A};

		\draw (2.5,0) node[circ] {};
		\draw (2.5,0) node[above] {B};

		\draw (5,0) node[circ] {};
		\draw (5,0) node[right] {C};

		\draw (2.5,-3) node[circ] {};
		\draw (2.5,-3) node[below] {D};
\end{circuitikz}
\end{center}
dove abbiamo già evidenziato i nodi.
Da qui possiamo proseguire in due modi:
\begin{itemize}
	\item \textbf{Col nodo di riferimento:} prendiamo D come nodo di riferimento, e impostiamo il suo potenziale $V_D$ a $0$. 
Le incognite saranno quindi $V_A$, $V_B$ e $V_C$.
Scriviamo un equazione per il potenziale $V_A$:
$$ V_A\left( \frac{1}{R_1} + \frac{1}{R_2} + \frac{1}{R_4} \right) - \frac{V_B}{R_2} - \frac{V_C}{R_1} = 0 $$
Il primo termine è detto \textbf{termine di auto ammettenza}, ed è dato dal prodotto del potenziale sul nodo $V_i$ e la somma delle conduttanze su tutti i rami connessi al nodo.
Il termini sucessivi sono detti \textbf{termini di mutua ammettenza}, e sono dati dalla legge di Ohm applicata ai rami connessi al nodo.

Nel caso uno dei rami contenga un generatore di corrente, si esclude dall'auto e mutua ammettenza e si mette come termine noto sul lato destro dell'equazione. \\
Ad esempio, sul nodo B avremo:
$$ V_B \left( \frac{1}{R_2} + \frac{1}{R_3} \right) - \frac{V_A}{R_2} - \frac{V_C}{R_3} = J $$
dove la $J$ è la corrente erogata dall generatore di corrente sul ramo BD, che abbiamo detto non compare nei termini di auto o mutua ammettenza. 
Si noti che il segno positivo significa corrente \textit{entrante} sul nodo, altrimenti si avrebbe segno negativo. \\
Infine, sul nodo C avremo:
$$ V_C \left( \frac{1}{R_1} + \frac{1}{R_3} + \frac{1}{R-5} \right) - \frac{V_A}{R_1} - \frac{V_B}{R_3} = 0 $$

Abbiamo quindi un sistema sulle 3 tensioni (abbiamo detto che la tensione $D$ vale 0):
\[
	\begin{cases}
 V_A\left( \frac{1}{R_1} + \frac{1}{R_2} + \frac{1}{R_4} \right) - \frac{V_B}{R_2} - \frac{V_C}{R_1} = 0 \\ 
 V_B \left( \frac{1}{R_2} + \frac{1}{R_3} \right) - \frac{V_A}{R_2} - \frac{V_C}{R_3} = J \\
 V_C \left( \frac{1}{R_1} + \frac{1}{R_3} + \frac{1}{R-5} \right) - \frac{V_A}{R_1} - \frac{V_B}{R_3} = 0 
	\end{cases}
\]

che possiamo portare in forma matriciale:
$$
\begin{pmatrix}
	\frac{1}{R_1} + \frac{1}{R_2} + \frac{1}{R_4} & -\frac{1}{R_2} & -\frac{1}{R_1} \\
	-\frac{1}{R_2} & \frac{1}{R_2} + \frac{1}{R_3} & -\frac{1}{R_3} \\
	-\frac{1}{R_1} & -\frac{1}{R3} & \frac{1}{R_1} + \frac{1}{R_3} + \frac{1}{R_5}
\end{pmatrix}
\begin{pmatrix}
V_A \\ V_B \\ V_C
\end{pmatrix}
=
\begin{pmatrix}
0 \\ J \\ 0
\end{pmatrix}
$$
e che possiamo risolvere.

\item \textbf{Senza nodo di riferimento:} possiamo decidere di continuare senza alcun riferimento del potenziale.
	Abbiamo già trovato le equazioni del potenziale sui primi 3 nodi, quindi impostiamo l'equazione per $V_D$:
	$$ V_D \left( \frac{1}{R_4} + \frac{1}{R_5} \right) - \frac{V_A}{R_4} - \frac{V_C}{R_5} = J $$
	e insieriamola nel sistema precedente:
	\[
		\begin{cases}
		 V_A\left( \frac{1}{R_1} + \frac{1}{R_2} + \frac{1}{R_4} \right) - \frac{V_B}{R_2} - \frac{V_C}{R_1} = 0 \\ 
		 V_B \left( \frac{1}{R_2} + \frac{1}{R_3} \right) - \frac{V_A}{R_2} - \frac{V_C}{R_3} = J \\
		 V_C \left( \frac{1}{R_1} + \frac{1}{R_3} + \frac{1}{R-5} \right) - \frac{V_A}{R_1} - \frac{V_B}{R_3} = 0 \\
			V_D \left( \frac{1}{R_4} + \frac{1}{R_5} \right) - \frac{V_A}{R_4} - \frac{V_C}{R_5} = -J
		\end{cases}
	\]
	in forma matriciale, abbiamo un sistema in 4 variabili con 3 equazioni:
$$
\begin{pmatrix}
	\frac{1}{R_1} + \frac{1}{R_2} + \frac{1}{R_4} & -\frac{1}{R_2} & -\frac{1}{R_1} & -\frac{1}{R_4} \\
	-\frac{1}{R_2} & \frac{1}{R_2} + \frac{1}{R_3} & -\frac{1}{R_3} & 0 \\
	-\frac{1}{R_1} & -\frac{1}{R3} & \frac{1}{R_1} + \frac{1}{R_3} + \frac{1}{R_5} & -\frac{1}{R_5} \\
	-\frac{1}{R_4} & 0 & - \frac{1}{R_5} & \frac{1}{R_4} + \frac{1}{R_5}
\end{pmatrix}
\begin{pmatrix}
V_A \\ V_B \\ V_C \\ V_D
\end{pmatrix}
=
\begin{pmatrix}
0 \\ J \\ 0 \\ -J 
\end{pmatrix}
$$

Sappiamo dal teorema di Rouché-Capelli che questo sistema è indeterminato, o almeno ammette soluzioni su una retta, quindi per un certo fattore di spostameno $\lambda$.
Questo rappresenta semplicemente il fatto che non abbiamo scelto uno zero del potenziale: il potenziale dei nodi non esiste se non in riferimento agli altri, quindi qualsiasi valore $\lambda$ aggiunto non cambia i risultati.
Scegliamo il valore $\lambda = 0$ per convenienza, che equivale a impostare $V_D$ a zero.
\end{itemize}

A questo punto, vogliamo risolvere effettivamente il circuito, e quindi trovare le tensioni e le correnti di ramo.
Per le tensioni, prendiamo semplicemente le differenze di potenziale sui nodi:
$$ V_{AB} = V_A - V_B, \quad \text{ecc...} $$
mentre per le correnti applichiamo la legge di Ohm sul ramo scelto. Ad esempio, sul ramo AB avremo:
$$ 
V = IR, \quad I_{AB} = \frac{V_{AB}}{R_2}  
$$

Un caso particolare è rappresentato dalla tensione su BD.
In questo caso si ha un generatore di corrente, e nessuna resistenza, quindi $I_{BD}$ è ben definito, ma non $V_{BD}$.
Quello che vogliamo fare è calcolare la caduta di potenziale $V_J = V_{BD}$ sul generatore di corrente.
Per fare ciò applichiamo la seconda legge di Kirchoff, magari applicata in senso orario sulla maglia in basso a sinistra, con le direzioni di corrente:

\begin{center}
\begin{circuitikz}
	\draw (0,0)
		-- (0,1.5)
		to[R, l=$R_1$, i=$i_1$] (5,1.5)
		-- (5, 0);
	\draw (0,0)
		to[R, l=$R_2$, i=$i_2$] (2.5, 0)
		to[R, l=$R_3$, i=$i_3$] (5, 0);
	\draw (0,0)
		to[R, l=$R_4$, i=$i_4$] (0, -3);
	\draw (2.5,-3)
		to[ european current source, I=$J$ ] (2.5, 0);
	\draw (5,-3)
		to[ R, l_=$R_5$, i=$i_5$] (5, 0);
	\draw (5, -3)
		to[ short ] (2.5, -3)
		to[ short ] (0, -3);

		\draw (0,0) node[circ] {};
		\draw (0,0) node[left] {A};

		\draw (2.5,0) node[circ] {};
		\draw (2.5,0) node[above] {B};

		\draw (5,0) node[circ] {};
		\draw (5,0) node[right] {C};

		\draw (2.5,-3) node[circ] {};
		\draw (2.5,-3) node[below] {D};
\end{circuitikz}
\end{center}

Si ha quindi:
$$
V_j + R_2 i_2 - R_4 i_4 = 0 \quad V_j = R_4 i_4 - R_2 i_2
$$

Abbiamo così determinato tutte le tensioni e tutte le correnti.

\subsubsection{Circuiti con generatori di tensione}
\begin{itemize}
	\item \textbf{\textsf{Generatori ideali di tensione}} \\
Il dipolo particolare quando si applica il metodo delle tensioni di nodo è il generatore ideale di tensione.
In questo caso, possiamo semplificare il sistema: se il generatore collega un nodo al nodo di riferimento, allora il voltaggio del nodo è uguale al voltaggio erogato dal generatore, tenendo conto dei segni in base alle direzioni delle tensioni erogate.

Se si hanno serie di generatori di tensione, si può applicare ripetutamente questa regola sui nodi coinvolti nella serie.

Nel caso non si possano scegliere agevolmente nodi di riferimento che prendono serie di generatori di tensione, si può impostare:
$$ E = V_A - V_B $$
dove $E$ è la tensione erogata dal generatore, e $V_A$ e $V_B$ le tensioni sui nodi che collega, ancora una volta opportunamente segnate in base alle direzioni delle tensioni erogate.
	\item \textbf{\textsf{Generatori reali di tensione}} \\
Un caso particolare è determinato dai generatori reali di tensione, cioè i generatori ideali in serie a resistenze.
In questi casi conviene trasformare i generatori nei loro equivalenti di Norton, e risolvere.
Si ha che per un generatore reale di voltaggio $V$ e resistenza $R$, l'equivalente di Norton è un generatore di corrente $\frac{V}{R}$ in parallelo ad una resistenza $R$.
\end{itemize}

%\subsubsection{Teoria dei grafi e reti elettriche}
%# anche questo magari scrivilo 
\TODO

\end{document}


\documentclass[a4paper,11pt]{article}
\usepackage[a4paper, margin=8em]{geometry}

% usa i pacchetti per la scrittura in italiano
\usepackage[french,italian]{babel}
\usepackage[T1]{fontenc}
\usepackage[utf8]{inputenc}
\frenchspacing 

% usa i pacchetti per la formattazione matematica
\usepackage{amsmath, amssymb, amsthm, amsfonts}

% usa altri pacchetti
\usepackage{gensymb}
\usepackage{hyperref}
\usepackage{standalone}

% imposta il titolo
\title{Appunti Elettrotecnica}
\author{Luca Seggiani}
\date{2024}

% imposta lo stile
% usa helvetica
\usepackage[scaled]{helvet}
% usa palatino
\usepackage{palatino}
% usa un font monospazio guardabile
\usepackage{lmodern}

\renewcommand{\rmdefault}{ppl}
\renewcommand{\sfdefault}{phv}
\renewcommand{\ttdefault}{lmtt}

% disponi il titolo
\makeatletter
\renewcommand{\maketitle} {
	\begin{center} 
		\begin{minipage}[t]{.8\textwidth}
			\textsf{\huge\bfseries \@title} 
		\end{minipage}%
		\begin{minipage}[t]{.2\textwidth}
			\raggedleft \vspace{-1.65em}
			\textsf{\small \@author} \vfill
			\textsf{\small \@date}
		\end{minipage}
		\par
	\end{center}

	\thispagestyle{empty}
	\pagestyle{fancy}
}
\makeatother

% disponi teoremi
\usepackage{tcolorbox}
\newtcolorbox[auto counter, number within=section]{theorem}[2][]{%
	colback=blue!10, 
	colframe=blue!40!black, 
	sharp corners=northwest,
	fonttitle=\sffamily\bfseries, 
	title=~\thetcbcounter: #2, 
	#1
}

% disponi definizioni
\newtcolorbox[auto counter, number within=section]{definition}[2][]{%
	colback=red!10,
	colframe=red!40!black,
	sharp corners=northwest,
	fonttitle=\sffamily\bfseries,
	title=~\thetcbcounter: #2,
	#1
}

% U.D.M
\newcommand{\amp}{\ensuremath{\mathrm{A}}}
\newcommand{\volt}{\ensuremath{\mathrm{V}}}
\newcommand{\meter}{\ensuremath{\mathrm{m}}}
\newcommand{\second}{\ensuremath{\mathrm{s}}}
\newcommand{\farad}{\ensuremath{\mathrm{F}}}
\newcommand{\henry}{\ensuremath{\mathrm{H}}}
\newcommand{\siemens}{\ensuremath{\mathrm{S}}}

% circuiti
\usepackage{circuitikz}
\usetikzlibrary{babel}

% disegni
\usepackage{pgfplots}
\pgfplotsset{width=10cm,compat=1.9}

% disponi codice
\usepackage{listings}
\usepackage[table]{xcolor}

\lstdefinestyle{codestyle}{
		backgroundcolor=\color{black!5}, 
		commentstyle=\color{codegreen},
		keywordstyle=\bfseries\color{magenta},
		numberstyle=\sffamily\tiny\color{black!60},
		stringstyle=\color{green!50!black},
		basicstyle=\ttfamily\footnotesize,
		breakatwhitespace=false,         
		breaklines=true,                 
		captionpos=b,                    
		keepspaces=true,                 
		numbers=left,                    
		numbersep=5pt,                  
		showspaces=false,                
		showstringspaces=false,
		showtabs=false,                  
		tabsize=2
}

\lstdefinestyle{shellstyle}{
		backgroundcolor=\color{black!5}, 
		basicstyle=\ttfamily\footnotesize\color{black}, 
		commentstyle=\color{black}, 
		keywordstyle=\color{black},
		numberstyle=\color{black!5},
		stringstyle=\color{black}, 
		showspaces=false,
		showstringspaces=false, 
		showtabs=false, 
		tabsize=2, 
		numbers=none, 
		breaklines=true
}

\lstdefinelanguage{javascript}{
	keywords={typeof, new, true, false, catch, function, return, null, catch, switch, var, if, in, while, do, else, case, break},
	keywordstyle=\color{blue}\bfseries,
	ndkeywords={class, export, boolean, throw, implements, import, this},
	ndkeywordstyle=\color{darkgray}\bfseries,
	identifierstyle=\color{black},
	sensitive=false,
	comment=[l]{//},
	morecomment=[s]{/*}{*/},
	commentstyle=\color{purple}\ttfamily,
	stringstyle=\color{red}\ttfamily,
	morestring=[b]',
	morestring=[b]"
}

% disponi sezioni
\usepackage{titlesec}

\titleformat{\section}
	{\sffamily\Large\bfseries} 
	{\thesection}{1em}{} 
\titleformat{\subsection}
	{\sffamily\large\bfseries}   
	{\thesubsection}{1em}{} 
\titleformat{\subsubsection}
	{\sffamily\normalsize\bfseries} 
	{\thesubsubsection}{1em}{}

% disponi alberi
\usepackage{forest}

\forestset{
	rectstyle/.style={
		for tree={rectangle,draw,font=\large\sffamily}
	},
	roundstyle/.style={
		for tree={circle,draw,font=\large}
	}
}

% disponi algoritmi
\usepackage{algorithm}
\usepackage{algorithmic}
\makeatletter
\renewcommand{\ALG@name}{Algoritmo}
\makeatother

% disponi numeri di pagina
\usepackage{fancyhdr}
\fancyhf{} 
\fancyfoot[L]{\sffamily{\thepage}}

\makeatletter
\fancyhead[L]{\raisebox{1ex}[0pt][0pt]{\sffamily{\@title \ \@date}}} 
\fancyhead[R]{\raisebox{1ex}[0pt][0pt]{\sffamily{\@author}}}
\makeatother

\begin{document}
% sezione (data)
\section{Lezione del 17-10-24}

% stili pagina
\thispagestyle{empty}
\pagestyle{fancy}

% testo
\subsection{Condensatori}
Introduciamo un nuovo bipolo: il \textbf{condensatore} o \textit{capacitore}. 
Si indica come:

\begin{center}
	\begin{circuitikz}
		\draw (0,0) to [ C ] (2, 0); 
	\end{circuitikz}
\end{center}
ed è costituito da due armature di materiale conduttore, inframezzate da un \textbf{dielettrico}.

Il verso di percorrenza nei condensatori, come nei resistori, è irrilevante.
La loro funzione è quella di accumulare energia, secondo la legge:
$$
q(t) = C \cdot v(t)
$$
dove $C$ è la \textbf{capacità}, misurata in Farad ($\mathrm{F}$).

Nota la superficie delle armature e la distanza fra di esse, si può calcolare la capacità come:
$$
C = \varepsilon \cdot \frac{s}{d}
$$
Nel caso di dielettrici, si indica con $\varepsilon_0$ la costante dielettrica introdotta e si scrive:
$$
C = \varepsilon \cdot \varepsilon_0 \cdot \frac{s}{d}
$$

Si ricorda che il Farad è un'unita di misura molto grande, e solitamente si usano i sottomultipli:
\begin{table}[H]
	\center \rowcolors{2}{white}{black!10}
	\begin{tabular} { c | c }
		\bfseries Simbolo & \bfseries Ordine \\
		\hline 
		$ \mathrm{mF} $ & $10^{-3}$ \\
		$ \mathrm{\mu F} $ & $10^{-6}$ \\
		$ \mathrm{nF} $ & $10^{-9}$ \\
		$ \mathrm{pF} $ & $10^{-12}$ \\
	\end{tabular}
\end{table}

Diciamo che il condensatore ideale è:
\begin{itemize}
	\item \textbf{Lineare:} dalla legge $ q(t) = C \cdot v(t)$;
	\item \textbf{Tempo-invariante:} trascurando cambi di temperatura, si comportano come i resistori;
	\item \textbf{Con memoria:} visto che legano tensione a carica, dobbiamo prendere la corrente come derivata della carica:
		$$
		i(t) = \frac{dq(t)}{dt} = \frac{d}{dt}(C \cdot v(t)) = C \frac{dv(t)}{dt}
		$$
		Possiamo quindi integrare:
		$$
		v(t) = \int_{-\infty}^{t} \frac{1}{C} \cdot i(\tau) \, d\tau = \frac{1}{C} \int_{-\infty}^{t} i(\tau) \, d\tau = \frac{1}{C} \left[ \int_{-\infty}^{t_0} i(\tau) + \int_{t_0}^{t} i(\tau) \right] 
		$$
		$$
		= v(t_0) + \frac{1}{C} \int_{t_0}^{t} i(\tau) \, d\tau
		$$

		Abbiamo quindi che la tensione sul condensatore dipende dalla tensione iniziale $v(t_0)$ e dalle correnti precedeneti $i(t')$ a $t' < t$, ergo è un componente con memoria.
	\item \textbf{Passivo:} anche qui possiamo definire $p(t)$ e derivare:
		$$
		p(t) = v_C(t)i_C(t) = v_C(t) \cdot C\frac{dv_C(t)}{dt}
		$$
		da cui si ottiene che $p(t)$ ha qualsiasi segno.
		Vediamo quindi l'energia:
		$$
		w(t) = \int_{-\infty}^t p(\tau) \, d\tau = \int_{-\infty}^t C v_C(\tau) \frac{dv_C(\tau)}{d\tau} \, d\tau 
		$$
		$$
		= C \int_{-\infty}^t v_C(\tau) dv_C(\tau) = C \left[ \frac{1}{2} v_C^2 (\tau) \right]_{-\infty}^{t} 
		$$
		da cui si ha, risolvendo:
		$$
		w(t) = \frac{1}{2} C v_C^2(t) - \frac{1}{2} C v_C^2 (-\infty)
		$$
		
		Assumendo $v_C^2 (-\infty) = 0$, cioè condensatore inizialmente scarico, si ha $w(t) \geq 0$, ergo è un componente passivo.
		Solo nel caso in cui parte da carico il condensatore può (temporaneamente) erogare energia.
\end{itemize}

\subsection{Induttori}
Introduciamo un nuovo bipolo: l'\textbf{induttore} o \textit{induttanza}. 
Si indica come:

\begin{center}
	\begin{circuitikz}
		\draw (0,0) to [ inductor ] (2, 0); 
	\end{circuitikz}
\end{center}
ed è costituito da spire di materiale ferromagnetico avvolte attorno a un dielettrico.

La loro funzione è ancora quella di accumulare energia, secondo la legge:
$$
\phi(t) = L \cdot i_L (t)
$$
dove $L$ è l'\textbf{induttanza}, misurata in Henry ($\mathrm{H}$), e $\phi$ è il \textbf{flusso magnetico}, misurata in Weber ($\mathrm{Wb}$).
L'induttanza dipende dalla geometria dell'induttore, e ad esempio in un solenoide di $N$ spire di superficie $s$ su una lunghezza $l$ è:
$$ 
L = \mu \cdot \frac{S}{l} \cdot N^2 = \mu_0 \cdot \mu_r \frac{S}{l} N^2
$$

Vediamo quindi le proprietà:
\begin{itemize}
	\item \textbf{Lineare:} sempre per la legge $\phi(t) = L \cdot i_L (t)$;
	\item \textbf{Tempo-invariante:} il flusso interno dipende solo dalla corrente;
	\item \textbf{Con memoria:} possamo derivare la legge:
		$$ 
			v_L (t) = \frac{d \phi(t)}{dt} = \frac{d (L i_L(t))}{dt} = L\frac{d i_L(t)}{dt}
		$$
		e ricavare e derivare la corrente $i_L(t)$:
		$$
		i_L(t) = \frac{1}{L} \int_{-\infty}^t v_L (\tau) \, d\tau = i_L(t_0) + \frac{1}{L} \int_{t_0}^{t} v_L(\tau) \, d\tau
		$$
		da dove si ricava che, come il condensatore, l'induttore ha memoria di uno stato iniziale a $t_0$.
	\item \textbf{Passivo:} ritroviamo la potenza:
		$$
			p(t) = v_C(t) i_C(t) = L \frac{d i_C(t)}{dt} \cdot i_L(t)
		$$
		da cui si ottiene che $p(t)$ ha qualsiasi segno. Vediamo quindi l'energia:
		$$
		w(t) = \int_{-\infty}^{t} p(\tau) \, d \tau = \int_{-\infty}^{t} L \frac{d i_L(\tau)}{d\tau} \cdot i_L(\tau) d\tau = L \int_{-\infty}^t i_L(\tau) d i_L(\tau)
		$$
		da cui si ha:
		$$
		w(t) = \frac{1}{2} L i_L^2(t) - \frac{1}{2} L i_L^2 (-\infty)
		$$
		Come prima, assumendo $i_L^2(\infty) = 0$, cioè induttore inizialmente scarico, si ha $w(t) \geq 0$, e che l'induttore è un componente passivo (salvo parta da carico).
\end{itemize}

\subsubsection{Induttori mutuamente accoppiati}
Si possono avere più induttanze (prendiamo 2 per semplicità) accoppiate fra di loro attraverso l'effetto del magnetico generato da entrambe sulle reciproche spire, cioè:

\begin{center}
	\begin{circuitikz}
		\draw (0,0) to[ short, i=$i_1$] (1,0)
			to[ inductor , l=$L_1$] (1,-2)
			-- (0, -2);

		\draw (4,0) to[ short, i_=$i_2$] (3,0)
			to[ inductor , l_=$L_2$] (3,-2)
			-- (4, -2);

			\draw (0.9,-0.6) node {$\scriptscriptstyle\bullet$};
			\draw (2.9,-0.6) node {$\scriptscriptstyle\bullet$};

			\draw (0,0) node[anchor=east] {$A$};
			\draw (0,-2) node[anchor=east] {$B$};

			\draw (0,-0.5) node[anchor=east] {$+$};
			\draw (0,-1.5) node[anchor=east] {$-$};

			\draw (4,0) node[anchor=west] {$C$};
			\draw (4,-2) node[anchor=west] {$D$};

			\draw (4,-0.5) node[anchor=west] {$+$};
			\draw (4,-1.5) node[anchor=west] {$-$};

			\draw (0,-1) node[anchor=east] {$v_1$};
			\draw (4,-1) node[anchor=west] {$v_2$};
	\end{circuitikz}
\end{center}

Dove il flusso su una e l'altra induttanza è espresso come:
\[
	\begin{cases}
			
  \phi_1(t) = \phi_{1.1} \pm \phi_{1.2} = L_1 \cdot i_1 (t) \pm M i_2 (t) \\
  \phi_2(t) = \phi_{1.2} \pm \phi_{1.1} = L_2 \cdot i_2 (t) \pm M i_1 (t) 
	\end{cases}
\]

Qui $M$ prende il nome di \textbf{coefficiente di mutua induzione}.
Anche questo coefficiente dipende dalla geometria della configurazione degli induttori.

Chiamiamo quindi la componente $L_i \cdot i_i$ \textbf{caduta di auto}, per \textit{caduta di auto induttanza}, e la componente $M \cdot i_i$ \textbf{caduta di mutua}, per \textit{caduta di mutua induttanza}.
Conviene fare una riflessione sui segni delle cadute di auto e di mutua: 
\begin{itemize}
	\item Prendendo riferimenti associati, cioè percorrendo le induttanze nella direzione della corrente propria, si ha che le cadute di auto sono positive. In caso contrario, si prendono come negative;
	\item Per i segni delle cadute di mutua si usa la regola dei \textbf{contrassegni}:
		\begin{itemize}
			\item Se la corrente entra al terminale contrassegnato di un induttore e induce una forza elettromotrice $\mathcal{E}$ positiva al terminale contrassegnato dell'altro induttore, si ha che $M>0$;
			\item Altrimenti, se la corrente entra al terminale contrassegnato di un induttore e induce una forza elettromotrice $\mathcal{E}$ negativa al terminale contrassegnato dell'altro induttore, si ha che $M<0$;
		\end{itemize}
	ergo, scelti riferimenti associati per le cadute di auto, si ha che se le correnti raggiungono il contrassegno della propria induttanza entrambe entrando o uscendo dall'induttore, le cadute di mutua sono positive.
	Altrimenti, se le correnti raggiungono il contrassegno della propria induttanza una entrando e una uscendo dall'induttore, o viceversa, le cadute di mutua sono negative.
\end{itemize}

Si ha quindi, derivando:
\[
	\begin{cases}
		v_1(t) = L_1 \frac{d i_1(t)}{dt} \pm M \frac{d i_2(t)}{dt} \\ 
		v_2(t) = L_2 \frac{d i_2(t)}{dt} \pm M \frac{d i_1(t)}{dt}
	\end{cases}
\]

Gli induttori mutuamente accoppiati vengono detti \textbf{quadripoli} o \textbf{doppi bipoli}, in quanto hanno effettivamente 4 poli.

Si calcola la potenza semplicemente sommando le potenze sulle singole induttanze:
$$
p(t) = v_1(t)i_1(t) + v_2(t)i_2(t)  
$$
$$
= \left( L_1 \frac{d i_1(t)}{dt} \pm M \frac{d i_2 (t)}{dt} \right) \cdot i_2(t) +  \left( L_2 \frac{d i_2(t)}{dt} \pm M \frac{d i_1 (t)}{dt} \right) \cdot i_1(t)
$$
$$
= L_1 i_1(t) \frac{d i_1(t)}{dt} + L_2 i_2 \frac{d i_2(t)}{dt} \pm M \left( i_1(t) \frac{d i_2(t)}{dt} + \frac{i_2 d i_1(t)}{dt} \right)
$$
E l'energia integrando la potenza:
$$
w(t) = \int_{-\infty}^t p(\tau) d\tau 
= L_1 \int_{-\infty}^t i_1(\tau) \frac{d i_1 (\tau)}{d\tau}d\tau + L_2 \int_{-\infty}^t i_2(\tau) \frac{d i_2 (\tau)}{d\tau}d\tau 
$$
$$
\pm M \int_{-\infty}^t \left[ i_1(\tau) \frac{d i_2(\tau)}{d \tau} + i_2(\tau) \frac{d i_1(\tau)}{d \tau} \right] d\tau
$$
$$
= L_1 \cdot \frac{1}{2} i_1^2(t) + L_2 \cdot \frac{1}{2} i_2^2(t) \pm M \int_{-\infty}^t \frac{d(i_1(\tau) \cdot i_2(\tau))}{d\tau}d\tau
$$
$$
= L_1 \cdot \frac{1}{2} i_1^2(t) + L_2 \cdot \frac{1}{2} i_2^2(t) \pm M i_1(t)i_2(t) 
$$

Si può dimostrare che $ M \leq \sqrt{L_1 L_2}$, e nel caso $M = \sqrt{L_1 L_2}$ si parla di \textbf{accoppiamento ideale}. Nel caso peggiore si ha quindi:
$$
w = \frac{1}{2} L_1 i_1^2(t) + \frac{1}{2} L_2 i_2^2(t) - \sqrt{L_1 L_2} i_1(t) i_2(t) = \frac{1}{2}\left( \sqrt{L_1} i_1(t) - \sqrt{L_2} i_2(t) \right)^2 \geq 0
$$
ergo, salvo caricamenti iniziali, le induttanze mutuamente accoppiate sono componenti \textbf{passivi}.

\subsection{Circuiti con impedenze a regime costante}
Vediamo il comportamento dei condensatori e degli induttori in circuiti a regime costante, cioè come quelli che abbiamo studiato finora.
\begin{itemize}
	\item \textbf{Condensatori:} il condensatore ha legge $q(t) = C \cdot v_C(t) \Rightarrow i_C(t) = C \frac{dv_C(t)}{dt}$, ergo se siamo in continua, $i_C(t) = C \cdot 0 = 0$ e il condensatore si comporta come un \textbf{aperto};
	\item \textbf{Induttori:} l'induttore ha legge $\phi(t) = L \cdot i_L(t) \Rightarrow V_L(t) = L \frac{di_L(t)}{dt}$. ergo se siamo in continua, $v_L (t) = L \cdot 0 = 0$, e l'induttore si comporta come un \textbf{cortocircuito}.
\end{itemize}
Abbiamo quindi che non si possono apprezzare gli effetti di questi nuovi bipoli finché si studiano circuiti a regime stazionario.
Infatti vedremo più nel dettaglio quali sono le loro caratteristiche quando parleremo di circuiti in regime sinusoidale.
\end{document}


\documentclass[a4paper,11pt]{article}
\usepackage[a4paper, margin=8em]{geometry}

% usa i pacchetti per la scrittura in italiano
\usepackage[french,italian]{babel}
\usepackage[T1]{fontenc}
\usepackage[utf8]{inputenc}
\frenchspacing 

% usa i pacchetti per la formattazione matematica
\usepackage{amsmath, amssymb, amsthm, amsfonts}

% usa altri pacchetti
\usepackage{gensymb}
\usepackage{hyperref}
\usepackage{standalone}

% imposta il titolo
\title{Appunti Elettrotecnica}
\author{Luca Seggiani}
\date{2024}

% imposta lo stile
% usa helvetica
\usepackage[scaled]{helvet}
% usa palatino
\usepackage{palatino}
% usa un font monospazio guardabile
\usepackage{lmodern}

\renewcommand{\rmdefault}{ppl}
\renewcommand{\sfdefault}{phv}
\renewcommand{\ttdefault}{lmtt}

% disponi il titolo
\makeatletter
\renewcommand{\maketitle} {
	\begin{center} 
		\begin{minipage}[t]{.8\textwidth}
			\textsf{\huge\bfseries \@title} 
		\end{minipage}%
		\begin{minipage}[t]{.2\textwidth}
			\raggedleft \vspace{-1.65em}
			\textsf{\small \@author} \vfill
			\textsf{\small \@date}
		\end{minipage}
		\par
	\end{center}

	\thispagestyle{empty}
	\pagestyle{fancy}
}
\makeatother

% disponi teoremi
\usepackage{tcolorbox}
\newtcolorbox[auto counter, number within=section]{theorem}[2][]{%
	colback=blue!10, 
	colframe=blue!40!black, 
	sharp corners=northwest,
	fonttitle=\sffamily\bfseries, 
	title=~\thetcbcounter: #2, 
	#1
}

% disponi definizioni
\newtcolorbox[auto counter, number within=section]{definition}[2][]{%
	colback=red!10,
	colframe=red!40!black,
	sharp corners=northwest,
	fonttitle=\sffamily\bfseries,
	title=~\thetcbcounter: #2,
	#1
}

% U.D.M
\newcommand{\amp}{\ensuremath{\mathrm{A}}}
\newcommand{\volt}{\ensuremath{\mathrm{V}}}
\newcommand{\meter}{\ensuremath{\mathrm{m}}}
\newcommand{\second}{\ensuremath{\mathrm{s}}}
\newcommand{\farad}{\ensuremath{\mathrm{F}}}
\newcommand{\henry}{\ensuremath{\mathrm{H}}}
\newcommand{\siemens}{\ensuremath{\mathrm{S}}}

% circuiti
\usepackage{circuitikz}
\usetikzlibrary{babel}

% disegni
\usepackage{pgfplots}
\pgfplotsset{width=10cm,compat=1.9}

% disponi codice
\usepackage{listings}
\usepackage[table]{xcolor}

\lstdefinestyle{codestyle}{
		backgroundcolor=\color{black!5}, 
		commentstyle=\color{codegreen},
		keywordstyle=\bfseries\color{magenta},
		numberstyle=\sffamily\tiny\color{black!60},
		stringstyle=\color{green!50!black},
		basicstyle=\ttfamily\footnotesize,
		breakatwhitespace=false,         
		breaklines=true,                 
		captionpos=b,                    
		keepspaces=true,                 
		numbers=left,                    
		numbersep=5pt,                  
		showspaces=false,                
		showstringspaces=false,
		showtabs=false,                  
		tabsize=2
}

\lstdefinestyle{shellstyle}{
		backgroundcolor=\color{black!5}, 
		basicstyle=\ttfamily\footnotesize\color{black}, 
		commentstyle=\color{black}, 
		keywordstyle=\color{black},
		numberstyle=\color{black!5},
		stringstyle=\color{black}, 
		showspaces=false,
		showstringspaces=false, 
		showtabs=false, 
		tabsize=2, 
		numbers=none, 
		breaklines=true
}

\lstdefinelanguage{javascript}{
	keywords={typeof, new, true, false, catch, function, return, null, catch, switch, var, if, in, while, do, else, case, break},
	keywordstyle=\color{blue}\bfseries,
	ndkeywords={class, export, boolean, throw, implements, import, this},
	ndkeywordstyle=\color{darkgray}\bfseries,
	identifierstyle=\color{black},
	sensitive=false,
	comment=[l]{//},
	morecomment=[s]{/*}{*/},
	commentstyle=\color{purple}\ttfamily,
	stringstyle=\color{red}\ttfamily,
	morestring=[b]',
	morestring=[b]"
}

% disponi sezioni
\usepackage{titlesec}

\titleformat{\section}
	{\sffamily\Large\bfseries} 
	{\thesection}{1em}{} 
\titleformat{\subsection}
	{\sffamily\large\bfseries}   
	{\thesubsection}{1em}{} 
\titleformat{\subsubsection}
	{\sffamily\normalsize\bfseries} 
	{\thesubsubsection}{1em}{}

% disponi alberi
\usepackage{forest}

\forestset{
	rectstyle/.style={
		for tree={rectangle,draw,font=\large\sffamily}
	},
	roundstyle/.style={
		for tree={circle,draw,font=\large}
	}
}

% disponi algoritmi
\usepackage{algorithm}
\usepackage{algorithmic}
\makeatletter
\renewcommand{\ALG@name}{Algoritmo}
\makeatother

% disponi numeri di pagina
\usepackage{fancyhdr}
\fancyhf{} 
\fancyfoot[L]{\sffamily{\thepage}}

\makeatletter
\fancyhead[L]{\raisebox{1ex}[0pt][0pt]{\sffamily{\@title \ \@date}}} 
\fancyhead[R]{\raisebox{1ex}[0pt][0pt]{\sffamily{\@author}}}
\makeatother

\begin{document}
% sezione (data)
\section{Lezione del 23-10-24}

% stili pagina
\thispagestyle{empty}
\pagestyle{fancy}

% testo
\subsection{Circuiti RC a regime sinusoidale}
Veniamo quindi a studiare circuiti dove i generatori non sono in regime costante, ma \textbf{sinusoidali}.
Un circuito si dice in \textbf{regime periodico sinusoidale} se tutte le correnti $i(t)$ e tutte le tensioni $v(t)$ possono essere espresse come funzioni sinusoidali con la stessa pulsazione.
Le grandezze devono quindi essere descritte da funzioni \textbf{periodiche}, ergo:
		$$ 
		f(t + T) = f(t) \quad \forall t
		$$
		dove $T$ si dice \textbf{periodo}.
		Notiamo che periodico non significa sinusoidale, in quanto altri tipi di onde (quadre, a dente di sega, ecc...) sono ugualmente periodiche.
	In particolare, noi tratteremo funzioni in forma:
	$$ 
		f(t) = F \cdot \sin(\omega t + \phi)	
	$$
	dove $F$ è l'\textbf{ampiezza} dell'oscillazione, $\omega$ la \textbf{pulsazione} e $\phi$ \textbf{fase}.

\begin{center}
\begin{tikzpicture}
    \begin{axis}[
        xlabel={$t$},
        ylabel={$f(t)$},
        domain=-10:10, % set the x range you want
				samples=100,
        grid=major, % add a grid
				ytick={-2, 2},
				yticklabels={$-F$, $F$},
				xtick={0, 7},
				xticklabels={$\phi$, $\phi + T$},
				width=15cm,
				height=7cm
    ]
    \addplot[
        blue,
        thick
    ] {2 * sin(50*x)}; 
    \end{axis}
\end{tikzpicture}
\end{center}

	Abbiamo che esiste una relazione fra pulsazione $\omega$ e periodo $T$, e che si può introdurre la \textbf{frequenza} $f$:
	$$
	\omega = \frac{2\pi}{T} = 2\pi f \quad f = \frac{1}{T}
	$$
	Si ha che il periodo si misura in secondi, la frequenza in Herz ($\mathrm{Hz}$), e la pulsazione in $\mathrm{rad}/\mathrm{s}$.

\subsection{Fasori}
Si ha che per lo studio delle equazioni differenziali date da sistemi a regime sinusoidale, torna utile passare ad un dominio \textbf{fasoriale} delle grandezze in questione.

Prendiamo una funzione sinusoidale in forma:
$$
x(t) = x_M \cdot \sin (\omega t + \phi)
$$
e trasformiamola in un numero complesso, nella sua forma esponenziale:
$$
x(t) = x_m \cdot e^{j(\omega t + \phi)}
$$
dove $j$ rappresenta la \textbf{costante immaginaria}.

Lo stesso complesso (più propriamente, una \textbf{funzione complessa}) potrà essere espresso in forma cartesiana, come:
$$
x(t) = a(t) \cdot j b(t)
$$

Potremo allora ricavare, dalla \textbf{formula di Eulero}:
$$
e^{j\phi} = \cos(\phi) + j \sin(\phi)
$$
le seguenti trasformazioni:

\par\medskip

\noindent
\begin{minipage}{0.45\textwidth}
	\begin{center}
		$$
			\begin{cases}
				x_M = \sqrt{a^2 + b^2} \\ 
				\phi = 
					\mathrm{atan2}\left(\frac{b}{a}\right)	
			\end{cases}
	$$
	\end{center}
\end{minipage}
\hfill
\begin{minipage}{0.45\textwidth}
	\begin{center}
		$$	
		\begin{cases}
				a = x_M \cos(\phi) \\ 
				b = x_M \sin(\phi) \\ 
			\end{cases}
			$$
	\end{center}
\end{minipage}

dove si è definito $\mathrm{atan2}$ come:
$$
\mathrm{atan2}(\phi) =
	\begin{cases}
		\arctan{\frac{b}{a}}, \quad \quad a > 0, b \in \mathbb{R} \\ 
					\pi + \arctan{\frac{b}{a}}, \quad \quad a < 0, b > 0 \\ 
					- \pi + \arctan{\frac{b}{a}}, \quad \quad a < 0, b < 0 \\ 
					\frac{\pi}{2}, \quad \quad \quad a = 0, b > 0 \\
					-\frac{\pi}{2}, \quad \quad \quad a = 0, b < 0 \\
					0 \quad \quad \quad a > 0, b > 0 \\ 
					\pi / -\pi \quad \quad \quad a < 0, b > 0 \\
					\emptyset \quad \quad \quad a = 0, b = 0	
	\end{cases}
$$

\par\medskip

Chiamiamo $x(t)$ \textbf{vettore rotante}.
Si ha che il vettore rotante ha distanza dall'origine pari all'ampiezza dell'oscillazione, e si muove di moto rotante sul piano di Gauss con velocità angolare pari alla pulsazione dell'oscillazione, con relativa fase.

Infine, definiamo quando ci interessa:
\begin{definition}{Fasore}
Chiamiamo il valore assunto da un vettore rotante $x(t)$ in un dato momento $t_0 = 0$ \textbf{fasore} (da \textit{phase vector}), e lo indichiamo come $\dot{X}$.
\end{definition}
Chiaramente, il vettore rotante assume valori complessi e il fasore stesso è un numero complesso:

\begin{center}
\begin{tikzpicture}
  \begin{axis}[
    axis lines = center,
    xlabel={$Re$},
    ylabel={$Im$},
    xmin=-5.4, xmax=5.4,
    ymin=-5.4, ymax=5.4,
    grid=minor,
  ]
    % Draw the phasor (example: magnitude=1, angle=45 degrees)
    \addplot[thick,->,red] coordinates {(0,0) (0.707*5,0.707*5)};
    
    % Optional: Add a label for the phasor
		\node at (axis cs: 0.707*5, 0.707*5) [anchor=west] {$\dot{I}$};

		\node at(axis cs: 0.5,0.3) [anchor=west] {$\phi$};
  \end{axis}
\end{tikzpicture}
\end{center}

Nell'esempio, posto $i(t) = 5 \cdot \sin (1000 t + \frac{\pi}{4})A$, avremo il fasore a $t_0 = 0$:
$$
\dot{I} = 5 \cdot e^{j \frac{\pi}{4}}
$$

Dal grafico vediamo poi che il modulo del vettore coincide col valore della corrente o della tensione che rappresenta, e che l'angolo che forma con l'asse delle ascisse rappresenta la fase $\phi$.

Visto che le pulsazioni sono comuni fra fasori diversi (circuiti a regime sinusoidale), abbiamo che le uniche informazioni che conserviamo nel diagramma dei fasori sono l'ampiezza e la fase.

\subsubsection{Proprietà dei fasori}
\begin{enumerate}
	\item \textbf{Addizione e sottrazione:} dati due fasori $\dot{I_1}$ e $\dot{I_2}$ si ha che:
		$$
		\dot{I_1} + \dot{I_2} = I_1 \cdot e^{j\phi_1} + I_2 \cdot e^{j\phi_2}
		$$
		$$
		= I_1 \cdot \cos(\phi_1) + j \cdot I_1 \cdot \sin(\phi_1) + I_2 \cdot \cos(\phi_2) + j \cdot I_2 \cdot \sin(\phi_2)
		$$
		$$
		= \left( I_1 \cdot \cos(\phi_1) \pm I_2 \cdot \cos(\phi_2) \right) + j (I_1 \cdot \sin(\phi_1) \pm I_2 \cdot \sin(\phi_2))
		$$
		dove $I_1$ e $I_2$ sono i moduli rispettivamente di $\dot{I_1}$ e $\dot{I_2}$.
		Abbiamo che si sommano separatamente parti reali e immaginarie, ergo si si va calcolare banalmente la \textbf{somma vettoriale} nel piano di Gauss dei due fasori.
	\item \textbf{Derivata:} definiamo la derivata di $x(t)$ rispetto al tempo come:
		$$
		y(t) = \frac{d\left(x(t)\right)}{dt} = \mathrm{Im}\left\{ \dot{X} e^{j \omega t} \right\} = \frac{d\left( \mathrm{Im}\left\{\dot{X}e^{j\omega t}\right\} \right)}{dt} = \mathrm{Im}\left\{ \frac{d\left( \dot{X} e^{j\omega t} \right)}{dt} \right\} 
		$$
		$$
		= \mathrm{Im}\left\{ \dot{X} \cdot \frac{d \left( e^{j \omega t} \right)}{dt} \right\} = \mathrm{Im} \{ \dot{X} e^{j \omega t} \cdot j \omega \} = \mathrm{Im} \{ j\omega \cdot \dot{X} \cdot e^{j\omega t} \}
		$$
		da cui:
		$$
		\dot{Y} = j \omega \dot{X}
		$$
	\item \textbf{Integrale:} definiamo l'integrale di un fasore come:
		$$
		y(t) = \int x(t) dt = \mathrm{Im}\left\{ \hat{Y} e^{j\omega t} \right\} = \int \mathrm{Im} \left\{ \dot{X} e^{j\omega t} \right\} dt = \mathrm{Im} \left\{ \int \dot{X} e^{j \omega t} dt \right\} 
		$$
		$$
		= \mathrm{Im} \left\{ \dot{X} \cdot \int e^{j \omega t} dt \right\} = \mathrm{Im} \left\{ \dot{X} \cdot \frac{e^{j \omega t}}{j\omega} \right\} = \mathrm{Im}\left\{ \frac{\dot{X}}{j\omega} e^{j \omega t} \right\}
		$$
		da cui:
		$$
		\dot{Y} = \frac{\dot{X}}{j\omega}
		$$
\end{enumerate}

\subsection{Bipoli in regime sinusoidale}
Vediamo come si comportano i bipoli visti finora in regime sinusoidale.

\subsubsection{Resistori}
Poniamo di avere un resistore in regime sinsuoidale.
Avevamo che:
$$
v_R(t) = R i_R(t)
$$
In quanto a fasori, avremo la stesssa legge:
$$
\dot{V}_R = R \dot{I}_R
$$

Ergo non cambia niente rispetto a quanto avevamo in regime costante.
Sul diagramma dei fasori, avremmo che i fasori tensione e corrente saranno fra di loro paralleli:
\begin{center}
\begin{tikzpicture}
  \begin{axis}[
    axis lines = center,
    xlabel={$Re$},
    ylabel={$Im$},
    xmin=-5.4, xmax=5.4,
    ymin=-5.4, ymax=5.4,
		xtick={0},
		ytick={0},
		grid=minor,
  ]
    % Draw the phasor (example: magnitude=1, angle=45 degrees)
    \addplot[thick,->,blue] coordinates {(0,0) (0.707*5,0.707*5)};
    \addplot[thick,->,red] coordinates {(0,0) (0.707*3,0.707*3)};
    
    % Optional: Add a label for the phasor
		\node at (axis cs: 0.707*5, 0.707*5) [anchor=south] {$\dot{V}$};
		\node at (axis cs: 0.707*3, 0.707*3) [anchor=south] {$\dot{I}$};
  \end{axis}
\end{tikzpicture}
\end{center}

\subsubsection{Induttori}
Un induttore è governato dalla legge:
$$
v_L(t) = L \frac{d \, i_L(t)}{dt}
$$
Sui fasori si avrà, applicando la legge di derivazione dei fasori:
$$
\dot{V}_L = L j \omega \dot{I}_L = j \omega L \cdot \dot{I}_L
$$

Notiamo come il legame differenziale della prima legge diventa algebrico nella seconda, e anzi si riconduce ad una forma che ricorda la legge di Ohm.
Sul diagramma dei fasori, avremo che i fasori tensione e corrente formano fra di loro un angolo di $\frac{\pi}{2}$:
\begin{center}
\begin{tikzpicture}
  \begin{axis}[
    axis lines = center,
    xlabel={$Re$},
    ylabel={$Im$},
    xmin=-5.4, xmax=5.4,
    ymin=-5.4, ymax=5.4,
		xtick={0},
		ytick={0},
		grid=minor,
  ]
    % Draw the phasor (example: magnitude=1, angle=45 degrees)
    \addplot[thick,->,blue] coordinates {(0,0) (0.707*5,0.707*5)};
    \addplot[thick,->,red] coordinates {(0,0) (-0.707*3,0.707*3)};
    
    % Optional: Add a label for the phasor
		\node at (axis cs: 0.707*5, 0.707*5) [anchor=west] {$\dot{V}$};
		\node at (axis cs: -0.707*3, 0.707*3) [anchor=east] {$\dot{I}$};
  \end{axis}
\end{tikzpicture}
\end{center}

\subsubsection{Condensatori}
Un condensatore è governato dalla legge:
$$
v_C(t) = \frac{1}{C} \int i_C(t) dt
$$
Sui fasori si avrà, applicando la legge di integrazione dei fasori:
$$
\dot{V}_C = \frac{1}{C} \frac{\dot{I}_C}{j\omega} = \frac{\dot{I}_C}{j \omega C}
$$

Come prima, ci riportiamo in una forma lineare.
Sul diagramma dei fasori, avremo che i fasori tensione e corrente formano fra di loro un angolo di $-\frac{\pi}{2}$:
\begin{center}
\begin{tikzpicture}
  \begin{axis}[
    axis lines = center,
    xlabel={$Re$},
    ylabel={$Im$},
    xmin=-5.4, xmax=5.4,
    ymin=-5.4, ymax=5.4,
		xtick={0},
		ytick={0},
		grid=minor,
  ]
    % Draw the phasor (example: magnitude=1, angle=45 degrees)
    \addplot[thick,->,blue] coordinates {(0,0) (0.707*5,0.707*5)};
    \addplot[thick,->,red] coordinates {(0,0) (0.707*3,-0.707*3)};
    
    % Optional: Add a label for the phasor
		\node at (axis cs: 0.707*5, 0.707*5) [anchor=west] {$\dot{V}$};
		\node at (axis cs: 0.707*3, -0.707*3) [anchor=west] {$\dot{I}$};
  \end{axis}
\end{tikzpicture}
\end{center}

\subsection{Analisi circuitale coi fasori}
La definizione dei fasori ci permette di studiare circuiti in regime sinusoidali attraverso gli stessi strumenti che abbiamo studiato finora, a patto di dover risolvere un sistema di equazioni complesse (che possiamo sempre dividere in parte reale e immaginaria).
Prendiamo ad esempio il circuito:
\begin{center}
	\begin{circuitikz}
		\draw (0,0) to[ voltage source, v=$e(t)$ ] (0,2)
			to[ resistor, l=$R_1$ ] (2,2) to[ inductor, l=$L$ ] (4,2)
			to[ resistor, l_=$R_2$, i=$I_R$ ] (4,0)
			to[ short, i=$I_L$] (0,0);
		\draw (4,2) to [ short, i=$I_C$ ] (6,2)
			to[ capacitor, l=$C$] (6,0)
			-- (4,0);
	\end{circuitikz}
\end{center}

La legge che governa il generatore di tensione è:
$$
e(t) = A\sin\left(\omega t + \phi \right), \quad \dot{E} = A e^{j \phi}
$$

Vediamo come risolvere il circuito usando il \textbf{tableau} e le \textbf{correnti di maglia}.

\begin{itemize}
	\item \textbf{Metodo del tableau:} prendendo il nodo in basso, si ha dalla prima legge di Kirchoff:
$$
I_R + I_C - I_L = 0
$$

Ricordando che, nel campo complesso, un'equazione di questo tipo significa:
\[
	\begin{cases}			
		\mathrm{Re}\{I_R\} + \mathrm{Re}\{I_C\} - \mathrm{Re}\{I_L\} = 0 \\ 
		\mathrm{Im}\{I_R\} + \mathrm{Im}\{I_C\} - \mathrm{Im}\{I_L\} = 0  
	\end{cases}
\]

In ogni caso, ossiamo poi applicare la seconda legge di Kirchoff sulle due maglie, in modo da ottenere il sistema completo:
\[
	\begin{cases}
		I_R + I_C - I_L = 0 \\
		\dot{E} + R_1 \dot{I}_L + j \omega L \dot{I}_L + R_2 \dot{I}_R = 0 \\ 
		-R_2 \dot{I}_R + \frac{1}{j\omega C} \dot{I}_C = 0
	\end{cases}
\]

	Questo si riconduce ad un sistema lineare:
$$
\begin{pmatrix}
	1 & -1 & 1 \\ R_2 & R_1 + j\omega L & 0 \\ -R_2 & 0 & \frac{1}{j\omega L}
\end{pmatrix}
\begin{pmatrix}
	\dot{I}_R \\ \dot{I}_L \\ \dot{I}_C
\end{pmatrix} = 
\begin{pmatrix}
	0 \\ \dot{E} \\ 0
\end{pmatrix}
$$
che possiamo risolvere con qualsiasi risolutore di sistemi lineari.

Assumiamo $\phi = \frac{\pi}{2}$, così che $\dot{E} = Aj$.
In modalità simbolica, allora, MATLAB restituisce:
$$
\begin{cases}
     I_{L} = -\frac{A\,\left(C\,R_{2}\,\omega -\mathrm{i}\right)}{R_{1}+R_{2}+L\,w\,1{}\mathrm{i}-C\,L\,R_{2}\,w^2+C\,R_{1}\,R_{2}\,w\,1{}\mathrm{i}} \\ 
     I_{R} = \frac{A\,1{}\mathrm{i}}{R_{1}+R_{2}+L\,\omega \,1{}\mathrm{i}-C\,L\,R_{2}\,w^2+C\,R_{1}\,R_{2}\,w\,1{}\mathrm{i}} \\
     I_{C} = -\frac{A\,C\,R_{2}\,\omega }{R_{1}+R_{2}+L\,w\,1{}\mathrm{i}-C\,L\,R_{2}\,w^2+C\,R_{1}\,R_{2}\,w\,1{}\mathrm{i}}
\end{cases}
$$

\item \textbf{Metodo delle correnti di maglia:} prendiamo la maglia a sinistra in senso orario e quella destra in senso antiorario, con le correnti rispettivamente $I_1$ e $I_2$:
\[
	\begin{cases}
		\dot{E} = R_1 \dot{I}_1 + j \omega L \dot{I}_1 + R_2\left(\dot{I}_1 - \dot{I}_2 \right) \\ 
		0 = \frac{\dot{I}_2}{j \omega C} + R_2\left(\dot{I_2} - \dot{I_1}\right)
	\end{cases}
\]
da cui il sistema:
$$
\begin{pmatrix}
	R_1 + R_2 + j \omega L & -R_2 \\ 
	-R_2 & \frac{1}{j \omega C} + R_2
\end{pmatrix}
\begin{pmatrix}
	\dot{I_1} \\ \dot{I_2}
\end{pmatrix}
=
\begin{pmatrix}
	\dot{E} \\ 0
\end{pmatrix}
$$
che dà lo stesso risultato.
\end{itemize}

\end{document}


\documentclass[a4paper,11pt]{article}
\usepackage[a4paper, margin=8em]{geometry}

% usa i pacchetti per la scrittura in italiano
\usepackage[french,italian]{babel}
\usepackage[T1]{fontenc}
\usepackage[utf8]{inputenc}
\frenchspacing 

% usa i pacchetti per la formattazione matematica
\usepackage{amsmath, amssymb, amsthm, amsfonts}

% usa altri pacchetti
\usepackage{gensymb}
\usepackage{hyperref}
\usepackage{standalone}

% imposta il titolo
\title{Appunti Elettrotecnica}
\author{Luca Seggiani}
\date{2024}

% imposta lo stile
% usa helvetica
\usepackage[scaled]{helvet}
% usa palatino
\usepackage{palatino}
% usa un font monospazio guardabile
\usepackage{lmodern}

\renewcommand{\rmdefault}{ppl}
\renewcommand{\sfdefault}{phv}
\renewcommand{\ttdefault}{lmtt}

% disponi il titolo
\makeatletter
\renewcommand{\maketitle} {
	\begin{center} 
		\begin{minipage}[t]{.8\textwidth}
			\textsf{\huge\bfseries \@title} 
		\end{minipage}%
		\begin{minipage}[t]{.2\textwidth}
			\raggedleft \vspace{-1.65em}
			\textsf{\small \@author} \vfill
			\textsf{\small \@date}
		\end{minipage}
		\par
	\end{center}

	\thispagestyle{empty}
	\pagestyle{fancy}
}
\makeatother

% disponi teoremi
\usepackage{tcolorbox}
\newtcolorbox[auto counter, number within=section]{theorem}[2][]{%
	colback=blue!10, 
	colframe=blue!40!black, 
	sharp corners=northwest,
	fonttitle=\sffamily\bfseries, 
	title=~\thetcbcounter: #2, 
	#1
}

% disponi definizioni
\newtcolorbox[auto counter, number within=section]{definition}[2][]{%
	colback=red!10,
	colframe=red!40!black,
	sharp corners=northwest,
	fonttitle=\sffamily\bfseries,
	title=~\thetcbcounter: #2,
	#1
}

% U.D.M
\newcommand{\amp}{\ensuremath{\mathrm{A}}}
\newcommand{\volt}{\ensuremath{\mathrm{V}}}
\newcommand{\meter}{\ensuremath{\mathrm{m}}}
\newcommand{\second}{\ensuremath{\mathrm{s}}}
\newcommand{\farad}{\ensuremath{\mathrm{F}}}
\newcommand{\henry}{\ensuremath{\mathrm{H}}}
\newcommand{\siemens}{\ensuremath{\mathrm{S}}}

% circuiti
\usepackage{circuitikz}
\usetikzlibrary{babel}

% disegni
\usepackage{pgfplots}
\pgfplotsset{width=10cm,compat=1.9}

% disponi codice
\usepackage{listings}
\usepackage[table]{xcolor}

\lstdefinestyle{codestyle}{
		backgroundcolor=\color{black!5}, 
		commentstyle=\color{codegreen},
		keywordstyle=\bfseries\color{magenta},
		numberstyle=\sffamily\tiny\color{black!60},
		stringstyle=\color{green!50!black},
		basicstyle=\ttfamily\footnotesize,
		breakatwhitespace=false,         
		breaklines=true,                 
		captionpos=b,                    
		keepspaces=true,                 
		numbers=left,                    
		numbersep=5pt,                  
		showspaces=false,                
		showstringspaces=false,
		showtabs=false,                  
		tabsize=2
}

\lstdefinestyle{shellstyle}{
		backgroundcolor=\color{black!5}, 
		basicstyle=\ttfamily\footnotesize\color{black}, 
		commentstyle=\color{black}, 
		keywordstyle=\color{black},
		numberstyle=\color{black!5},
		stringstyle=\color{black}, 
		showspaces=false,
		showstringspaces=false, 
		showtabs=false, 
		tabsize=2, 
		numbers=none, 
		breaklines=true
}

\lstdefinelanguage{javascript}{
	keywords={typeof, new, true, false, catch, function, return, null, catch, switch, var, if, in, while, do, else, case, break},
	keywordstyle=\color{blue}\bfseries,
	ndkeywords={class, export, boolean, throw, implements, import, this},
	ndkeywordstyle=\color{darkgray}\bfseries,
	identifierstyle=\color{black},
	sensitive=false,
	comment=[l]{//},
	morecomment=[s]{/*}{*/},
	commentstyle=\color{purple}\ttfamily,
	stringstyle=\color{red}\ttfamily,
	morestring=[b]',
	morestring=[b]"
}

% disponi sezioni
\usepackage{titlesec}

\titleformat{\section}
	{\sffamily\Large\bfseries} 
	{\thesection}{1em}{} 
\titleformat{\subsection}
	{\sffamily\large\bfseries}   
	{\thesubsection}{1em}{} 
\titleformat{\subsubsection}
	{\sffamily\normalsize\bfseries} 
	{\thesubsubsection}{1em}{}

% disponi alberi
\usepackage{forest}

\forestset{
	rectstyle/.style={
		for tree={rectangle,draw,font=\large\sffamily}
	},
	roundstyle/.style={
		for tree={circle,draw,font=\large}
	}
}

% disponi algoritmi
\usepackage{algorithm}
\usepackage{algorithmic}
\makeatletter
\renewcommand{\ALG@name}{Algoritmo}
\makeatother

% disponi numeri di pagina
\usepackage{fancyhdr}
\fancyhf{} 
\fancyfoot[L]{\sffamily{\thepage}}

\makeatletter
\fancyhead[L]{\raisebox{1ex}[0pt][0pt]{\sffamily{\@title \ \@date}}} 
\fancyhead[R]{\raisebox{1ex}[0pt][0pt]{\sffamily{\@author}}}
\makeatother

\begin{document}
% sezione (data)
\section{Lezione del 24-10-24}

% stili pagina
\thispagestyle{empty}
\pagestyle{fancy}

% testo
\subsubsection{Mutua induttanza nel dominio fasoriale}
Avevamo le formule per la mutua induttanza:
\[
	\begin{cases}
		v_1(t) = L_1 \frac{d i_1(t)}{dt} \pm M \frac{d i_2(t)}{dt} \\
		v_2(t) = L_2 \frac{d i_2(t)}{dt} \pm M \frac{d i_1(t)}{dt}	
	\end{cases}
\]
nel dominio tempo.
Portandoci nel dominio fasoriale, abbiamo:
\[
	\begin{cases}
		\dot{V}_1 = j\omega L_1 \dot{I}_1	\pm j\omega M \dot{I}_2 \\
		\dot{V}_2 = j\omega L_2 \dot{I}_2	\pm j\omega M \dot{I}_1
	\end{cases}
\]

\subsection{Circuito RLC}
Poniamo di avere un circuito con un resistore, un induttore e un capacitore.
\begin{center}
	\begin{circuitikz}
		\draw (0,0) to [ resistor, l=$R$] (2,0) to [ inductor, l=$L$] (4,0) to [ capacitor, l=$C$ ] (6,0);
	\end{circuitikz}
\end{center}

Avremo le cadute di potenziale $\dot{V}_R$, $\dot{V}_L$ e $\dot{V}_C$ sui singoli componenti, da cui:
$$
\dot{V} = \dot{V}_R + \dot{V}_L + \dot{V}_C = R \dot{I} + j\omega L \dot{I} + \frac{1}{j \omega c} \dot{I} = \left( R + j\omega l+ \frac{1}{j\omega C} \right) \dot{I}
$$
Portando in forma cartesiana, si ha:
$$
= (R + j\omega L - \frac{j}{\omega C}) \dot{I} = \left( R + j \left( \omega L - \frac{1}{\omega C} \right) \right) \dot{I}
$$
che possiamo riscrivere come:
$$
\dot{V} = \bar{z} \dot{I}, \quad z = R + j \left( \omega L - \frac{1}{\omega C} \right)
$$

\subsection{Impedenza}
Il numero $z$, un complesso, è chiamato \textbf{impedenza} del circuito.
Abbiamo che l'equazione $\dot{V} = \bar{z} \dot{I}$ è un'equazione complessa, ergo che vorremo riscrivere come:
$$
\Rightarrow v_M \cdot e^{j \phi_v} = z \cdot e^{j \phi_z} I_M e^{j \phi_i}
$$
e quindi ridurre al sistema:
\[
	\begin{cases}
		V_M = z I_M \\ 
		\phi_v = \phi_z + \phi_i
	\end{cases}
\]

Spesso si indica $\phi_z$ come semplicemente $\phi$, cioè la \textbf{fase dell'impedenza}, definita quindi come:
$$
\phi = \phi_v - \phi_i
$$

Possiamo scrivere l'impedenza come:
$$
\bar{Z} = Z \cdot e^{j \phi} = R + j X, \quad z = |\bar{z}|
$$
dove $R$ corrisponde alla \textbf{resistenza} che già conosciamo, mentre $X$ viene detta \textbf{reattanza}.

Vediamo le reattanze dei dipoli studiati finora:
\begin{itemize}
	\item \textbf{Resistenza:} da $\dot{V} = R \dot{I}$ abbiamo $X_R = 0$, cioè reattanza nulla (e chiaramente $R_R = R$);
	\item \textbf{Induttore:} da $\dot{V} = j \omega L \dot{I}$, ricaviamo:
		\[
			\begin{cases}
				R_L = 0	\\ 
				X_L = \omega L
			\end{cases}
		\]
	\item \textbf{Condensatore:} da $\dot{V} = \frac{1}{j \omega C}$, ricaviamo:
		\[
			\begin{cases}
				R_C = 0	\\ 
				X_C = -\frac{1}{\omega C}
			\end{cases}
		\]
		dove l'ultima $X_C$ si è ricavata da $-\frac{j}{\omega C}$, già usato sopra nella forma cartesiana dell'RLC in serie.
\end{itemize}

Avremo che, riguardo all impededenza, avremo solitamente i valori di fase:
$$
\bar{Z} = R \cdot j X = z e^{j \phi}:
	\begin{cases}
		\phi = \frac{\pi}{2} \\ 
		\phi = 0 \\ 
		\phi = -\frac{pi}{2}
	\end{cases}
$$

Classifichiamo questi valori.
\begin{itemize}
	\item $\phi = \frac{\pi}{2}$, si dice che l'impedenza è \textbf{induttiva};
	\item $0 < \phi < \frac{\pi}{2}$, si dice che l'impedenza è \textbf{ohmico-induttiva};
	\item $\phi = 0$, si dice che l'impedenza è \textbf{resistiva};
	\item $-\frac{\pi}{2} < \phi < 0$, si dice che l'impedenza è \textbf{ohmico-capacitiva}; 
	\item $\phi = -\frac{\pi}{2}$, si dice che l'impedenza è \textbf{capacitiva};
\end{itemize}

\subsubsection{Ammettenza}
Possiamo definire l'opposto dell'impedenza:
$$
\bar{Y} = \frac{1}{\hat{Z}}
$$

Chiamiamo $Y$ \textbf{ammettenza}.
Possiamo dividere anche l'ammettenza in componenti cartesiane, cioè:
$$
\bar{Y} = G + j B
$$
dove $G$ è la \textbf{conduttanza}, e $B$ viene detta \textbf{suscettanza}.

Possiamo eliminare il complesso al denominatore come:
$$
\bar{Y} = \frac{1}{\hat{Z}} = \frac{1}{R + jX} = \frac{R - jX}{R^2 + X^2} = \frac{R}{R^2 + X^2} + j \left(-\frac{X}{R^2 + X^2}\right)
$$
da cui:
\[
	\begin{cases}
		G = \frac{R}{R^2 + X^2} \\ 	
		B = -\frac{X}{R^2 + X^2} \\ 	
	\end{cases}
\]

\subsubsection{Unità di misura}
Abbiamo che, essendo quantità omegenee (le abbiamo sommate fra di loro senza problemi), l'impedenza $Z$, la reattanza $X$ e la resistenza $R$ si misurano in Ohm, mentre l'ammettanza $Y$, la suscettanza $B$ e la conduttanza $G$ si misurano in Siemens.

\subsubsection{Rappresentazione grafica dell'impedenza}
Abbiamo che il vettore di $Z$ rappresentato sul piano di Argand-Gauss rappresenta in componenti $R$ e $X$ (com'è ovvio), e che l'angolo che forma con l'asse delle x rappresenta $\phi$.

Inoltre, si  ha che l'ammettenza corrispondente è un vettore con modulo $\frac{1}{Z}$ e angolo $-\phi$, da:
$$
\bar{Y} = \frac{1}{\bar{Z}} = \frac{1}{Ze^{j\phi}} = \frac{1}{Z} e^{-j\phi}
$$

\begin{center}
\begin{tikzpicture}
  \begin{axis}[
    axis lines = center,
    xlabel={$R$},
    ylabel={$X$},
    xmin=-5.4, xmax=5.4,
    ymin=-5.4, ymax=5.4,
    grid=minor,
  ]
    % Draw the phasor (example: magnitude=1, angle=45 degrees)
    \addplot[thick,->,black] coordinates {(0,0) (0.707*5,0.707*5)};
    
    % Optional: Add a label for the phasor
		\node at (axis cs: 0.707*5, 0.707*5) [anchor=west] {$Z$};

		\node at(axis cs: 0.5,0.3) [anchor=west] {$\phi$};
  \end{axis}
\end{tikzpicture}
\end{center}

Notiamo il significato dei quadranti: 
\begin{itemize}
	\item \textbf{Primo quadrante:} $R > 0$, $X > 0$ abbiamo che sia la resistenza che la reattanza sono positive, quindi si parla di impedenza ohmico-induttiva.
	\item \textbf{Secondo quadrante:} $R < 0$, $X > 0$ abbiamo che la resistenza è negativa e la reattanza è positiva, quindi si parla di impedenza ohmico-induttiva negativa.
	\item \textbf{Terzo quadrante:} $R < 0$, $X < 0$ abbiamo che sia la resistenza che la reattanza sono negative, quindi si parla di impedenza ohmico-capacitiva negativa.
	\item \textbf{Quarto quadrante:} $R < 0$, $X > 0$ abbiamo che la resistenza è positivae la reattanza è negativa, quindi si parla di impedenza ohmico-capacitiva.
\end{itemize}

In parallelo a quanto detto prima sull'angolo $\phi$.
Ai vettori paralleli agli assi corrispondono, come ci si aspetterebbe, abbiamo l'impedenza puramente induttiva, puramente resistiva e puramente capacitiva.

\end{document}


\documentclass[a4paper,11pt]{article}
\usepackage[a4paper, margin=8em]{geometry}

% usa i pacchetti per la scrittura in italiano
\usepackage[french,italian]{babel}
\usepackage[T1]{fontenc}
\usepackage[utf8]{inputenc}
\frenchspacing 

% usa i pacchetti per la formattazione matematica
\usepackage{amsmath, amssymb, amsthm, amsfonts}

% usa altri pacchetti
\usepackage{gensymb}
\usepackage{hyperref}
\usepackage{standalone}

% imposta il titolo
\title{Appunti Elettrotecnica}
\author{Luca Seggiani}
\date{2024}

% imposta lo stile
% usa helvetica
\usepackage[scaled]{helvet}
% usa palatino
\usepackage{palatino}
% usa un font monospazio guardabile
\usepackage{lmodern}

\renewcommand{\rmdefault}{ppl}
\renewcommand{\sfdefault}{phv}
\renewcommand{\ttdefault}{lmtt}

% disponi il titolo
\makeatletter
\renewcommand{\maketitle} {
	\begin{center} 
		\begin{minipage}[t]{.8\textwidth}
			\textsf{\huge\bfseries \@title} 
		\end{minipage}%
		\begin{minipage}[t]{.2\textwidth}
			\raggedleft \vspace{-1.65em}
			\textsf{\small \@author} \vfill
			\textsf{\small \@date}
		\end{minipage}
		\par
	\end{center}

	\thispagestyle{empty}
	\pagestyle{fancy}
}
\makeatother

% disponi teoremi
\usepackage{tcolorbox}
\newtcolorbox[auto counter, number within=section]{theorem}[2][]{%
	colback=blue!10, 
	colframe=blue!40!black, 
	sharp corners=northwest,
	fonttitle=\sffamily\bfseries, 
	title=~\thetcbcounter: #2, 
	#1
}

% disponi definizioni
\newtcolorbox[auto counter, number within=section]{definition}[2][]{%
	colback=red!10,
	colframe=red!40!black,
	sharp corners=northwest,
	fonttitle=\sffamily\bfseries,
	title=~\thetcbcounter: #2,
	#1
}

% U.D.M
\newcommand{\amp}{\ensuremath{\mathrm{A}}}
\newcommand{\volt}{\ensuremath{\mathrm{V}}}
\newcommand{\meter}{\ensuremath{\mathrm{m}}}
\newcommand{\second}{\ensuremath{\mathrm{s}}}
\newcommand{\farad}{\ensuremath{\mathrm{F}}}
\newcommand{\henry}{\ensuremath{\mathrm{H}}}
\newcommand{\siemens}{\ensuremath{\mathrm{S}}}

% circuiti
\usepackage{circuitikz}
\usetikzlibrary{babel}

% disegni
\usepackage{pgfplots}
\pgfplotsset{width=10cm,compat=1.9}

% disponi codice
\usepackage{listings}
\usepackage[table]{xcolor}

\lstdefinestyle{codestyle}{
		backgroundcolor=\color{black!5}, 
		commentstyle=\color{codegreen},
		keywordstyle=\bfseries\color{magenta},
		numberstyle=\sffamily\tiny\color{black!60},
		stringstyle=\color{green!50!black},
		basicstyle=\ttfamily\footnotesize,
		breakatwhitespace=false,         
		breaklines=true,                 
		captionpos=b,                    
		keepspaces=true,                 
		numbers=left,                    
		numbersep=5pt,                  
		showspaces=false,                
		showstringspaces=false,
		showtabs=false,                  
		tabsize=2
}

\lstdefinestyle{shellstyle}{
		backgroundcolor=\color{black!5}, 
		basicstyle=\ttfamily\footnotesize\color{black}, 
		commentstyle=\color{black}, 
		keywordstyle=\color{black},
		numberstyle=\color{black!5},
		stringstyle=\color{black}, 
		showspaces=false,
		showstringspaces=false, 
		showtabs=false, 
		tabsize=2, 
		numbers=none, 
		breaklines=true
}

\lstdefinelanguage{javascript}{
	keywords={typeof, new, true, false, catch, function, return, null, catch, switch, var, if, in, while, do, else, case, break},
	keywordstyle=\color{blue}\bfseries,
	ndkeywords={class, export, boolean, throw, implements, import, this},
	ndkeywordstyle=\color{darkgray}\bfseries,
	identifierstyle=\color{black},
	sensitive=false,
	comment=[l]{//},
	morecomment=[s]{/*}{*/},
	commentstyle=\color{purple}\ttfamily,
	stringstyle=\color{red}\ttfamily,
	morestring=[b]',
	morestring=[b]"
}

% disponi sezioni
\usepackage{titlesec}

\titleformat{\section}
	{\sffamily\Large\bfseries} 
	{\thesection}{1em}{} 
\titleformat{\subsection}
	{\sffamily\large\bfseries}   
	{\thesubsection}{1em}{} 
\titleformat{\subsubsection}
	{\sffamily\normalsize\bfseries} 
	{\thesubsubsection}{1em}{}

% disponi alberi
\usepackage{forest}

\forestset{
	rectstyle/.style={
		for tree={rectangle,draw,font=\large\sffamily}
	},
	roundstyle/.style={
		for tree={circle,draw,font=\large}
	}
}

% disponi algoritmi
\usepackage{algorithm}
\usepackage{algorithmic}
\makeatletter
\renewcommand{\ALG@name}{Algoritmo}
\makeatother

% disponi numeri di pagina
\usepackage{fancyhdr}
\fancyhf{} 
\fancyfoot[L]{\sffamily{\thepage}}

\makeatletter
\fancyhead[L]{\raisebox{1ex}[0pt][0pt]{\sffamily{\@title \ \@date}}} 
\fancyhead[R]{\raisebox{1ex}[0pt][0pt]{\sffamily{\@author}}}
\makeatother

\begin{document}
% sezione (data)
\section{Lezione del 25-10-24}

% stili pagina
\thispagestyle{empty}
\pagestyle{fancy}

% testo
\subsubsection{Potenza in circuiti a regime sinusoidale}
Vediamo come si studia la potenza nei circuiti in corrente alternata.
Avevamo definito la potenza come:
$$ p(t) = v(t)i(t) = R i^2(t) $$

A regime costante, abbiamo che $v(t)$ e $i(t)$ sono costanti, come lo è (ovviamente) $R$, ergo $p(t)$ ha un valore definito e positivo.

A regime sinusoidale, invece, abbiamo una forma del tipo:
$$
i(t) = I_M \cdot \sin(\omega t)
$$
assumendo $\phi = 0$.

Abbiamo quindi che in $p(t) = R i^2(t)$ non c'è modo di eliminare la dipendenza temporale.
Notiamo però che vale comunque $p(t) \geq 0$ dal quadrato a $i(t)$.

\subsubsection{Valore efficace}
Potremmo qundi chiederci se è il circuito a regime costante o quello a regime sinusoidale a dissipare più energia sul solito intervallo di tempo $\Delta T$.
Sul costante, abbiamo l'integrale:
$$
W(t) = \int_0^T RI^2 dt = RI^2 \cdot T 
$$
mentre sul sinusoidale vale:
$$
W(t) = \int_0^T Ri^2(t) dt = R \int_0^T i^2(t)dt
$$

Eguagliamo quindi le due:
$$
RI_{eff}^2 T = R \int_0^T i^2(t)dt
$$
dove abbiamo chiamato $I \rightarrow I_{eff}$ per dare una definizione preliminare di \textbf{corrente efficace}, cioè la corrente in regime costante che dissipa la stessa potenza della corrispondente corrente in regime sinusoidale.
Si ha quindi:
$$
I_{eff} = \sqrt{\frac{1}{T}\int_0^T i^2(t)dt}
$$

Vediamo che questa definizione è generica:
\begin{definition}{Valore efficace}
	Definiamo il valore efficace $X_{eff}$ di una grandezza $x(t)$ in regime sinusoidale su un intervallo $T$ come:
$$
X_{eff} = \sqrt{\frac{1}{T}\int_0^T x^2(t)dt}
$$
\end{definition}

Possiamo quindi sostituire la definizione che avevamo dato di $i(t)$, ottenendo:
$$
I_{eff} = \sqrt{\frac{1}{T} \int_0^T I_M^2 \cdot \sin^2(\omega t) dt} = \sqrt{\frac{1}{T}I_M^2 \int_0^T \left( 1 - \cos^2(\omega t) \right) dt} = \sqrt{\frac{1}{T} \int_0^T \frac{1-\cos(2\omega t)}{2} dt}
$$
$$
= \sqrt{\frac{1}{T} I_M^2 \left( \int_0^T \frac{1}{2}dt - \int_0^T \frac{\cos(2 \omega t)}{2}dt \right)}
= \sqrt{\frac{1}{T} I_M^2 \left( \frac{1}{2}\int_0^T 1\cdot dt - \frac{1}{2} \int_0^T \cos(2 \omega t)dt \right)}
=\sqrt{\frac{I_M^2}{2}} 
$$
da cui abbiamo il valore:
$$
I_{eff} = \frac{I_M}{\sqrt{2}}
$$
Notiamo come $I_{eff} \geq 0$, dal fatto che $I_M$ corrente di picco è $\geq 0$.

Ad esempio, si dice che la rete elettrica funziona a 220 volts. 
Questo non è altro che il valore efficace della corrente. Si ha infatti:
$$
V_{M} = 220 \cdot \sqrt{2} \approx 311 \mathrm{V}
$$

\subsubsection{Calcolo della potenza}
Definiamo quindi la potenza su circuiti a regime periodico sinusoidale.
Definiamo la \textbf{potenza istantanea}:
$$
p(t) = i(t)v(t)
$$
Avevamo definito $i(t)$ e $v(t)$ come:
\[
	\begin{cases}
		i(t) = I_M \sin(\omega t) \\ 
		v(t) = V_M \sin(\omega t + \phi)
	\end{cases}
\]
dove $\phi$ è la \textbf{fase dell'impedenza}, da $\phi_V = \phi + \phi_i$, $\dot{V} = \bar{Z} \dot{I}$.

Possiamo quindi sostituire:
$$
p(t) = V_M \sin(\omega t + \phi) \cdot I_M \sin(\omega t) = V_M I_M \sin(\omega t) \sin(\omega t + \phi) 
$$
$$
= V_M I_M \sin(\omega t)\left( \sin(\omega t) \cos(\phi) + \cos{\omega t} \sin{\phi} \right) = V_M I_M \left( \sin^2 (\omega t) \cos (\phi) + \sin(\omega t) \cos(\omega t) \sin(\phi) \right)
$$
$$
= V_M I_M \left( \left( 1 - \cos^2(\omega t) \right) \cos(\phi) + \sin(\omega t)\cos(\omega t)\sin(\phi) \right)
$$
$$
= V_M I_M \left( \frac{1 - \cos(2 \omega t)}{2}\cos(\phi) + \frac{\sin(2 \omega t){2}}\sin(\phi) \right) = \frac{V_M I_M}{2} \left( \left( 1 - \cos(2 \omega t) \cos (\phi) + \sin(2\omega t) \sin(\phi) \right) \right)
$$
da cui:
$$
p(t) = \frac{V_M I_M}{2}\left( 1 - \cos(2\omega t) \right) \cos(\phi) + \frac{V_M I_M}{2} \sin(2\omega t)\sin(\phi)
$$

Il primo termine viene detto \textbf{potenza attiva istantanea}, mentre il secondo viene detto \textbf{potenza reattiva istantanea}.
Componenti come i generatori generano potenza attiva istantanea, mentre componenti come gli induttori e i capacitori generano potenza reattiva istantanea.

\subsubsection{Potenza attiva}
Sarebbe comodo avere una misura di potenza che non dipende dal tempo, cioè un \textbf{valore medio}.
Definiamo quindi:
\begin{definition}{Potenza attiva}
	Definiamo la potenza attiva $P$, sulla base della potenza istantanea $p(t)$, come la media integrale:
	$$
		P = \frac{1}{T} \int_0^T p(t) dt
	$$
\end{definition}

Sostituendo quanto trovato prima, abbiamo:
$$
P = \frac{1}{T} \int_0^T p(t) dt = \frac{1}{T} \frac{V_M I_M}{2} \int_0^T \left( 1 - \cos(2\omega t) \right) \cos(\phi) +  \sin(2\omega t)\sin(\phi)  dt
$$
$$
= \frac{1}{T} \frac{V_M I_M}{2} \left( \int_0^T \cos(\phi) dt + \int_0^T \cos(2\omega t) \cos(\phi) dt + \int_0^t \sin(2\omega t)\sin(\phi)  dt \right) 
$$
dove l'unico integrale che resta è $\int_0^T \cos(\phi) dt = T \cos(\phi)$, ergo: 
$$
P= \frac{V_M I_M}{2} \cos(\phi) = \frac{V_{eff} \sqrt{2} \cdot I_{eff} \sqrt{2}}{2} \cos{\phi} = V_{eff}I_{eff} \cos{\phi}
$$
Si nota la comodità di usare i valori efficaci: non compare $\sqrt{2}$ al denominatore.

Notiamo poi che avevamo, dall'impedenza e l'ammettenza:
$$
\dot V = \bar{Z} \dot{I}  \Rightarrow P = Z I^2 \cos{\phi} = R I^2 = Y V^2 \cos(\phi) = G V^2 
$$

\subsubsection{Potenza reattiva}
Definiamo quindi la \textbf{potenza reattiva}:
\begin{definition}{Potenza reattiva}
	Definiamo la potenza reattiva $Q$ come il massimo della potenza reattiva istantanea.
\end{definition}

Da quanto calcolato prima, si ha:
$$
Q = \max{\frac{V_M I_M}{2} \sin(2\omega t)\sin(\phi)} = \frac{V_M I_M}{2} \sin(\phi)
$$
La potenza reattiva si misura in $[\mathrm{VAR}]$, cioè \textit{Volt Ampere Reattivi}.
Si ha anche qui, riguardo il triangolo delle impedenze:
$$
Q = \frac{V_{eff} \sqrt{2} \cdot I_{eff} \sqrt{2}}{2} \sin(\phi) = V_{eff}I_{eff} \sin(\phi) = ZI^2 \sin(\phi) = XI^2 = YV^2 \sin(\phi) = - BV^2
$$

\subsubsection{Potenza apparente}
Definiamo infine la potenza \textbf{apparente}:
\begin{definition}{Potenza apparente}
	Definiamo la potenza reattiva $S$ come il valore massimo raggiungibile della potenza, cioè:
	$$
	S = \frac{V_M I_M}{2}
	$$
\end{definition}

Come prima, 
$$
S = \frac{V_M I_M}{2} = \frac{V_{eff} \sqrt{2} \cdot I_{eff} \sqrt{2}}{2} = VI = ZI^2 = Y V^2
$$

L'unita della misura della potenza apparente è il $[\mathrm{VA}]$, cioè \textit{Volt Ampere}.

\par\medskip

Prendiamo in esempio il circuito:
\begin{center}
	\begin{circuitikz}
		\draw (0,0) to [ voltage source, v=$V_2(t)$] (4, 0);
		\draw (4,-3) to [ voltage source, v=$V_1(t)$] (4, 0);
		\draw (4,-3) to [ inductor, l=$L$] (8,-3)
			to [ resistor, l=$R$] (8, 0)
			to [ voltage source, v=$V_3$, i =$i(t)$] (4, 0);
		\draw (0,0) to [ capacitor, l=$C$] (0,-3)
			to [ inductor, l=$L$] (4,-3);
	\end{circuitikz}
\end{center}

con i generatori guidati dalla legge:
\[
	\begin{cases}	
		V_1(t) = 10 \sqrt{2} \sin\left(1000 t\right)V \\ 
		V_2(t) = 20 \sqrt{2} \sin\left(1000t + \frac{\pi}{2}\right)V \\ 
		V_3(t) = 30 \sqrt{2} \sin\left(1000t + \pi\right)V
	\end{cases}
\]

Ci chiediamo quanto valga la corrente $i(t)$ sul generatore di tensione $V_3$, in direzione opposta al contrassegno.

Prima di tutto, portiamo le equazioni dei generatori in forma fasoriale, usando il valore efficace del voltaggio.
Usare il valore efficace o il valore proprio del voltaggio è indifferente, ma abbiamo che i calcoli si semplificano se si usa il primo.
Abbiamo quindi:
\[
	\begin{cases}
		\dot{V}_1	= 10 e^{j\cdot0} = 10 V\\ 
		\dot{V}_2 = 20 e^{j\frac{\pi}{2}} = 20j V \\ 
		\dot{V}_3 = 30 e^{j \pi} = -30 V
	\end{cases}
\]

Possiamo quindi usare il metodo delle correnti di nodo. Poniamo il nodo:
\begin{center}
	\begin{circuitikz}
		\draw (0,0) to [ voltage source, v=$V_2(t)$] (4, 0);
		\draw (4,-3) to [ voltage source, v=$V_1(t)$] (4, 0);
		\draw (4,-3) to [ inductor, l=$L$] (8,-3)
			to [ resistor, l=$R$] (8, 0)
			to [ voltage source, v=$V_3$, i =$i(t)$] (4, 0);
		\draw (0,0) to [ capacitor, l=$C$] (0,-3)
			to [ inductor, l=$L$] (4,-3);
		
		\node at(4,0) {$\bullet$};
		\node at(4,0)[anchor=south] {A};
		\node at(4,-3)[ground] {}
	\end{circuitikz}
\end{center}

A questo punto, $V_A = V_1 = 10V$, e si puà applicare Kirchoff alla maglia a destra:
$$
-\dot{V}_1 + \dot{V}_3 + R\dot{I} + \dot{I} \cdot j \omega L = 0 \Rightarrow i(t) = \frac{\dot{V}_1 - \dot{V}_3}{R + j\omega L}
$$

Poniamo che, risolvendo con qualche valore di $R$ e $C$, si ha $\dot{I} = 2 - 2j$.
A questo punto, possiamo ritrovare il valore di $i(t)$ effettivo attraverso il modulo di $\dot{I}$ e l'angolo:
$$
i(t) = \sqrt{2^2 + 2^2} \sqrt2 \cdot \sin\left(1000t - \frac{\pi}{4}\right) = 4 \sin\left(1000t - \frac{\pi}{t}\right)
$$
dove si è reintrodotto $\sqrt{2}$ per ritornare al valore di picco. 

\end{document}


\documentclass[a4paper,11pt]{article}
\usepackage[a4paper, margin=8em]{geometry}

% usa i pacchetti per la scrittura in italiano
\usepackage[french,italian]{babel}
\usepackage[T1]{fontenc}
\usepackage[utf8]{inputenc}
\frenchspacing 

% usa i pacchetti per la formattazione matematica
\usepackage{amsmath, amssymb, amsthm, amsfonts}

% usa altri pacchetti
\usepackage{gensymb}
\usepackage{hyperref}
\usepackage{standalone}

% imposta il titolo
\title{Appunti Elettrotecnica}
\author{Luca Seggiani}
\date{2024}

% imposta lo stile
% usa helvetica
\usepackage[scaled]{helvet}
% usa palatino
\usepackage{palatino}
% usa un font monospazio guardabile
\usepackage{lmodern}

\renewcommand{\rmdefault}{ppl}
\renewcommand{\sfdefault}{phv}
\renewcommand{\ttdefault}{lmtt}

% disponi il titolo
\makeatletter
\renewcommand{\maketitle} {
	\begin{center} 
		\begin{minipage}[t]{.8\textwidth}
			\textsf{\huge\bfseries \@title} 
		\end{minipage}%
		\begin{minipage}[t]{.2\textwidth}
			\raggedleft \vspace{-1.65em}
			\textsf{\small \@author} \vfill
			\textsf{\small \@date}
		\end{minipage}
		\par
	\end{center}

	\thispagestyle{empty}
	\pagestyle{fancy}
}
\makeatother

% disponi teoremi
\usepackage{tcolorbox}
\newtcolorbox[auto counter, number within=section]{theorem}[2][]{%
	colback=blue!10, 
	colframe=blue!40!black, 
	sharp corners=northwest,
	fonttitle=\sffamily\bfseries, 
	title=~\thetcbcounter: #2, 
	#1
}

% disponi definizioni
\newtcolorbox[auto counter, number within=section]{definition}[2][]{%
	colback=red!10,
	colframe=red!40!black,
	sharp corners=northwest,
	fonttitle=\sffamily\bfseries,
	title=~\thetcbcounter: #2,
	#1
}

% U.D.M
\newcommand{\amp}{\ensuremath{\mathrm{A}}}
\newcommand{\volt}{\ensuremath{\mathrm{V}}}
\newcommand{\meter}{\ensuremath{\mathrm{m}}}
\newcommand{\second}{\ensuremath{\mathrm{s}}}
\newcommand{\farad}{\ensuremath{\mathrm{F}}}
\newcommand{\henry}{\ensuremath{\mathrm{H}}}
\newcommand{\siemens}{\ensuremath{\mathrm{S}}}

% circuiti
\usepackage{circuitikz}
\usetikzlibrary{babel}

% disegni
\usepackage{pgfplots}
\pgfplotsset{width=10cm,compat=1.9}

% disponi codice
\usepackage{listings}
\usepackage[table]{xcolor}

\lstdefinestyle{codestyle}{
		backgroundcolor=\color{black!5}, 
		commentstyle=\color{codegreen},
		keywordstyle=\bfseries\color{magenta},
		numberstyle=\sffamily\tiny\color{black!60},
		stringstyle=\color{green!50!black},
		basicstyle=\ttfamily\footnotesize,
		breakatwhitespace=false,         
		breaklines=true,                 
		captionpos=b,                    
		keepspaces=true,                 
		numbers=left,                    
		numbersep=5pt,                  
		showspaces=false,                
		showstringspaces=false,
		showtabs=false,                  
		tabsize=2
}

\lstdefinestyle{shellstyle}{
		backgroundcolor=\color{black!5}, 
		basicstyle=\ttfamily\footnotesize\color{black}, 
		commentstyle=\color{black}, 
		keywordstyle=\color{black},
		numberstyle=\color{black!5},
		stringstyle=\color{black}, 
		showspaces=false,
		showstringspaces=false, 
		showtabs=false, 
		tabsize=2, 
		numbers=none, 
		breaklines=true
}

\lstdefinelanguage{javascript}{
	keywords={typeof, new, true, false, catch, function, return, null, catch, switch, var, if, in, while, do, else, case, break},
	keywordstyle=\color{blue}\bfseries,
	ndkeywords={class, export, boolean, throw, implements, import, this},
	ndkeywordstyle=\color{darkgray}\bfseries,
	identifierstyle=\color{black},
	sensitive=false,
	comment=[l]{//},
	morecomment=[s]{/*}{*/},
	commentstyle=\color{purple}\ttfamily,
	stringstyle=\color{red}\ttfamily,
	morestring=[b]',
	morestring=[b]"
}

% disponi sezioni
\usepackage{titlesec}

\titleformat{\section}
	{\sffamily\Large\bfseries} 
	{\thesection}{1em}{} 
\titleformat{\subsection}
	{\sffamily\large\bfseries}   
	{\thesubsection}{1em}{} 
\titleformat{\subsubsection}
	{\sffamily\normalsize\bfseries} 
	{\thesubsubsection}{1em}{}

% disponi alberi
\usepackage{forest}

\forestset{
	rectstyle/.style={
		for tree={rectangle,draw,font=\large\sffamily}
	},
	roundstyle/.style={
		for tree={circle,draw,font=\large}
	}
}

% disponi algoritmi
\usepackage{algorithm}
\usepackage{algorithmic}
\makeatletter
\renewcommand{\ALG@name}{Algoritmo}
\makeatother

% disponi numeri di pagina
\usepackage{fancyhdr}
\fancyhf{} 
\fancyfoot[L]{\sffamily{\thepage}}

\makeatletter
\fancyhead[L]{\raisebox{1ex}[0pt][0pt]{\sffamily{\@title \ \@date}}} 
\fancyhead[R]{\raisebox{1ex}[0pt][0pt]{\sffamily{\@author}}}
\makeatother

\begin{document}
% sezione (data)
\section{Lezione del 30-10-24}

% stili pagina
\thispagestyle{empty}
\pagestyle{fancy}

% testo
\subsection{Potenza complessa}
Introduciamo un'ulteriore misura della potenza, comoda perché non necessita (a differenza di quelle studiate finora) di lavorare coi valori efficaci.
\begin{definition}{Potenza complessa}
Definiamo la \textbf{potenza complessa} $\overline{S}$ come il prodotto dei fasori di tensione e corrente, cioè come:
$$
\overline{S} = \dot{V} \dot{I}^*
$$
\end{definition}

Notiamo che si prende il \textbf{coniugato complesso} di $\dot{I}$.
Il suo significato si può mostrare svolgendo il calcolo:
$$
\overline{S} = \dot{V} \dot{I}^* = V \cdot e^{j \phi_v} \cdot I \cdot e^{-j \phi_i} = VI e^{j(\phi_v - \phi_i)} = VI e^{j\phi}
$$
cioè possiamo sfruttare la formula sulle fasi $\phi_v - \phi_i = \phi$.

In forma cartesiana, abbiamo che $\overline{S}$ vale:
$$
\overline{S} = VI e^{j\phi} = VI \cos(\phi) + j VI \sin{\phi} = P + jQ
$$
e quindi la potenza complessa è uguale al complesso dato da potenza attiva e reattiva.
Notiamo che finora nei calcoli si è inteso, con $V$ e $I$, i \textbf{valori efficaci} di queste grandezze: da qui in poi infatti prenderemo qualsiasi versore privo di pedice con valore efficace a modulo.

Possiamo rappresentare questa relazione nel cosiddetto \textbf{triangolo delle potenze}: la base è la potenza attiva, l'altezza la potenza resistiva, e inoltre notiamo che l'ipotenusa è la potenza apparente:

\begin{center}
\begin{tikzpicture}
  \begin{axis}[
    axis lines = center,
    xmin=-1, xmax=5.4,
    ymin=-1, ymax=5.4,
		xtick={0.707*5},
		ytick={0.707*5},
		xticklabel={$P$},
		yticklabel={$Q$},
		grid=minor,
  ]
    % Draw the phasor (example: magnitude=1, angle=45 degrees)
    \addplot[thick,->,black] coordinates {(0,0) (0.707*5,0.707*5)};
    
    % Optional: Add a label for the phasor
		\node at (axis cs: 0.707*5, 0.707*5) [anchor=west] {$\overline{S}$};

		\node at(axis cs: 0.5,0.3) [anchor=west] {$\phi$};
		\node at(axis cs: 1.5,1.7) [anchor=south] {$S$};
  \end{axis}
\end{tikzpicture}
\end{center}

Possiamo poi sfruttare: 
$$
\dot{V} = \overline{Z}\dot{I} \Leftrightarrow
	\begin{cases}
		V = ZI \\ 	
		\phi_v = \phi_z + \phi_i
	\end{cases}, \quad 
\dot{I} = \frac{\dot{V}}{\overline{Z}} = \overline{Y} \dot{V} \Leftrightarrow 
	\begin{cases}
		I = Y V \\ 
		\phi_i = \phi_y + \phi_v
	\end{cases}
$$
e quindi possiamo esprimere $\overline{S}$ come funzione:
\begin{itemize}
	\item In funzione dell'impedenza: 
$$
\overline{S} = \dot{V}\dot{I}^* = \overline{Z} I^2 
$$
	\item In funzione dell'ammettenza:
$$
\overline{S} = \dot{V}\dot{I}^* = \overline{Y} V^2
$$
\end{itemize}

\subsection{Teorema di Tellegen}
Dimostriamo il seguente risultato:
\begin{theorem}{Teorema di Tellegen}
	La somma algebrica delle potenze istantanee impiegate su tutti i rami di una rete elettrica è uguale a 0, ovvero:
$$
\sum_{jk = 1}^n v_{jk} (t) \cdot i_{jk}(t) = 0
$$
\end{theorem}

Notiamo che $j$ e $k$ sono indici che scorrono lungo $n$, dove $n$ è il numero di nodi della rete presa in considerazione, ergo $x_{jk}$ è quella grandezza considerata su ogni arco (ad archi inesistenti sarà evidentemente 0).

Possiamo riformulare la formula precedente come segue:
$$
\sum_{jk = 1}^n  \left( v_{j, 0}(t) - v_{k, 0}(t) \right) \cdot i_{j.k}(t) = 0
$$

Dove si è usato $v_{jk} = v_{j, 0} - v_{k, 0}$ dal secondo principio di Kirchoff.
Abbiamo quindi:
$$
= \sum_{jk = 1}^n  v_{j, 0}(t) \cdot i_{j.k}(t) - \sum_{jk = 1}^n v_{k, 0}(t) \cdot i_{j.k}(t) = \sum_{j=1}^n v_{j,0}(t) \left( \sum_{k=1}^n i_{jk}(t) \right) - v_{k,0}(t) \left( \sum_{j=1}^n i_{jk}(t) \right)
$$

Abbiamo che il secondo termine sommatoria è la somma delle correnti che escono dal nodo $j$, che quindi è 0 dal primo di Kirchoff, e il quarto termine, allo stesso modo, è la somma delle correnti che escono dal nodo $k$, e quindi l'intera espressione è nulla, da cui il teorema.

\subsubsection{Teorema di Boucherot}
Dal teorema di Tellegen sul caso sinusoidale deriva il \textbf{teorema di Boucherot}.
Prendiamo come ipotesi: 
$$
\sum_{jk = 1}^n \dot{V}_{jk} \dot{I}_{jk}^* = 0
$$

Questo deriva dal teorema di Tellegen considerato in regime sinusoidale.
Prendiamo quindi un qualunque ramo $jk$: su questo ramo avremo chiaramente impedenze e generatori, a cui diamo il nome $\overline{Z}_{jk}$ e $\dot{E}_{jk}$, e corrente e voltaggio, a cui diamo il nome di $\dot{I}_{jk}$ e $\dot{V}_{jk}$:
\begin{center}
	\begin{circuitikz}
		\draw (0,0) to [ european resistor, l=$\overline{Z}_{jk}$] (2, 0)
			to [ european voltage source, v=$E_{jk}$] (4, 0);

		\node at (0,0) [anchor=east] {$j$};
		\node at (4,0) [anchor=west] {$k$};
	\end{circuitikz}
\end{center}

Sostituiamo quindi la caduta di potenziale $\dot{V}_{jk}$ nell'ipotesi con:
$$
\sum_{jk = 1}^n \left( \overline{Z}_{jk} \dot{I}_{jk} - \dot{E}_{jk} \right) \cdot \dot{I}_{jk}^* = 0
$$
notando che i due termini coinvolti nella sommatoria sono:
$$
\sum_{jk = 1}^n \overline{Z}_{jk} \cdot I_{jk}^2 = \sum_{jk = 1}^n \dot{E}_{jk} \dot{I}_{jk}^* = \sum_{jk = 1}^n \overline{S}_{jk}^{(G)}
$$
cioè le potenze dissipate sulle impedenze e le potenze erogate dai generatori del ramo.
Da questo si deriva il teorema:
\begin{theorem}{Teorema di Boucherot}
	La somma delle potenze complesse dissipate sulle impedenze di un circuito è uguale alla somma delle potenze complesse erogate dai generatori, cioè:
$$
\sum_{jk = 1}^n \overline{Z}_{jk} \cdot I_{jk}^2 = \sum_{jk = 1}^n \dot{E}_{jk} \dot{I}_{jk}^*
$$
\end{theorem}

Possiamo rendere questo risultato anche come:
$$
\sum_{jk = 1}^n \left( R_{jk} + j X_{jk} \right) I_{jk}^2 = \sum_{jk = 1}^n \left( P_{jk}^{(G)} + j Q_{jk}^{(G)} \right)
$$

Da cui in poi si possono equagliare separatamente parte reale e parte immaginaria:
\[
	\begin{cases}
		\sum\limits_{jk =1}^n R_{jk} I_{jk}^2 = \sum\limits_{jk}^n P_{jk}^{(G)} \\ 
		\sum\limits_{jk =1}^n X_{jk} I_{jk}^2 = \sum\limits_{jk}^n Q_{jk}^{(G)} \\ 
	\end{cases}
\]
ergo, non solo la somma delle potenze potenze complesse dissipate sulle impedenze di un circuito è uguale alla somma delle potenze complesse erogate dai generatori, ma sia la potenza attiva che la potenza reattiva dissipata sulle impedenze sono uguali a le corrispondenti erogate dai generatori.

Sulla potenza attiva, che sappiamo non poter avere segno negativo, si ha che i generatori devono compensare le potenze dissipate sulle impedenze.

\subsubsection{Potenza apparente e Boucherot}
Preso un ramo con due generatori $P_1$ e $P_2$, abbiamo che la potenza attiva e reattive associate valgono quanto le somme delle potenze attive e reattive sui singoli generatori:
$$
P^{(G)} = P_1 + P_2, \quad Q^{(G)} = Q_1 + Q_2 
$$
e quindi, da Boucherot, o semplicemente applicando la definizione di potenza complessa $\overline{S}$:
$$
\overline{S}^{(G)} = P^{(G)} + j Q^{(G)} = (P_1 + P_2) + j (Q_1 + Q_2)
$$

Cioè la potenza complessa si conserva.

Dobbiamo notare che questo non vale per la potenza apparente: si ha che $S^{(G)} = \sqrt{P_{(G)}^2 + Q_{(G)}^2}$, ergo la dipendenza non è lineare, e non possiamo assumere che si conservi, cioè vale:
$$
S^{(G)} = \sqrt{P_{(G)}^2 + Q_{(G)}^2} \neq \sqrt{P_1^2 + Q_1^2} + \sqrt{P_2^2 + Q_2^2} 
$$

\end{document}


\documentclass[a4paper,11pt]{article}
\usepackage[a4paper, margin=8em]{geometry}

% usa i pacchetti per la scrittura in italiano
\usepackage[french,italian]{babel}
\usepackage[T1]{fontenc}
\usepackage[utf8]{inputenc}
\frenchspacing 

% usa i pacchetti per la formattazione matematica
\usepackage{amsmath, amssymb, amsthm, amsfonts}

% usa altri pacchetti
\usepackage{gensymb}
\usepackage{hyperref}
\usepackage{standalone}

% imposta il titolo
\title{Appunti Elettrotecnica}
\author{Luca Seggiani}
\date{2024}

% imposta lo stile
% usa helvetica
\usepackage[scaled]{helvet}
% usa palatino
\usepackage{palatino}
% usa un font monospazio guardabile
\usepackage{lmodern}

\renewcommand{\rmdefault}{ppl}
\renewcommand{\sfdefault}{phv}
\renewcommand{\ttdefault}{lmtt}

% disponi il titolo
\makeatletter
\renewcommand{\maketitle} {
	\begin{center} 
		\begin{minipage}[t]{.8\textwidth}
			\textsf{\huge\bfseries \@title} 
		\end{minipage}%
		\begin{minipage}[t]{.2\textwidth}
			\raggedleft \vspace{-1.65em}
			\textsf{\small \@author} \vfill
			\textsf{\small \@date}
		\end{minipage}
		\par
	\end{center}

	\thispagestyle{empty}
	\pagestyle{fancy}
}
\makeatother

% disponi teoremi
\usepackage{tcolorbox}
\newtcolorbox[auto counter, number within=section]{theorem}[2][]{%
	colback=blue!10, 
	colframe=blue!40!black, 
	sharp corners=northwest,
	fonttitle=\sffamily\bfseries, 
	title=~\thetcbcounter: #2, 
	#1
}

% disponi definizioni
\newtcolorbox[auto counter, number within=section]{definition}[2][]{%
	colback=red!10,
	colframe=red!40!black,
	sharp corners=northwest,
	fonttitle=\sffamily\bfseries,
	title=~\thetcbcounter: #2,
	#1
}

% U.D.M
\newcommand{\amp}{\ensuremath{\mathrm{A}}}
\newcommand{\volt}{\ensuremath{\mathrm{V}}}
\newcommand{\meter}{\ensuremath{\mathrm{m}}}
\newcommand{\second}{\ensuremath{\mathrm{s}}}
\newcommand{\farad}{\ensuremath{\mathrm{F}}}
\newcommand{\henry}{\ensuremath{\mathrm{H}}}
\newcommand{\siemens}{\ensuremath{\mathrm{S}}}

% circuiti
\usepackage{circuitikz}
\usetikzlibrary{babel}

% disegni
\usepackage{pgfplots}
\pgfplotsset{width=10cm,compat=1.9}

% disponi codice
\usepackage{listings}
\usepackage[table]{xcolor}

\lstdefinestyle{codestyle}{
		backgroundcolor=\color{black!5}, 
		commentstyle=\color{codegreen},
		keywordstyle=\bfseries\color{magenta},
		numberstyle=\sffamily\tiny\color{black!60},
		stringstyle=\color{green!50!black},
		basicstyle=\ttfamily\footnotesize,
		breakatwhitespace=false,         
		breaklines=true,                 
		captionpos=b,                    
		keepspaces=true,                 
		numbers=left,                    
		numbersep=5pt,                  
		showspaces=false,                
		showstringspaces=false,
		showtabs=false,                  
		tabsize=2
}

\lstdefinestyle{shellstyle}{
		backgroundcolor=\color{black!5}, 
		basicstyle=\ttfamily\footnotesize\color{black}, 
		commentstyle=\color{black}, 
		keywordstyle=\color{black},
		numberstyle=\color{black!5},
		stringstyle=\color{black}, 
		showspaces=false,
		showstringspaces=false, 
		showtabs=false, 
		tabsize=2, 
		numbers=none, 
		breaklines=true
}

\lstdefinelanguage{javascript}{
	keywords={typeof, new, true, false, catch, function, return, null, catch, switch, var, if, in, while, do, else, case, break},
	keywordstyle=\color{blue}\bfseries,
	ndkeywords={class, export, boolean, throw, implements, import, this},
	ndkeywordstyle=\color{darkgray}\bfseries,
	identifierstyle=\color{black},
	sensitive=false,
	comment=[l]{//},
	morecomment=[s]{/*}{*/},
	commentstyle=\color{purple}\ttfamily,
	stringstyle=\color{red}\ttfamily,
	morestring=[b]',
	morestring=[b]"
}

% disponi sezioni
\usepackage{titlesec}

\titleformat{\section}
	{\sffamily\Large\bfseries} 
	{\thesection}{1em}{} 
\titleformat{\subsection}
	{\sffamily\large\bfseries}   
	{\thesubsection}{1em}{} 
\titleformat{\subsubsection}
	{\sffamily\normalsize\bfseries} 
	{\thesubsubsection}{1em}{}

% disponi alberi
\usepackage{forest}

\forestset{
	rectstyle/.style={
		for tree={rectangle,draw,font=\large\sffamily}
	},
	roundstyle/.style={
		for tree={circle,draw,font=\large}
	}
}

% disponi algoritmi
\usepackage{algorithm}
\usepackage{algorithmic}
\makeatletter
\renewcommand{\ALG@name}{Algoritmo}
\makeatother

% disponi numeri di pagina
\usepackage{fancyhdr}
\fancyhf{} 
\fancyfoot[L]{\sffamily{\thepage}}

\makeatletter
\fancyhead[L]{\raisebox{1ex}[0pt][0pt]{\sffamily{\@title \ \@date}}} 
\fancyhead[R]{\raisebox{1ex}[0pt][0pt]{\sffamily{\@author}}}
\makeatother

\begin{document}
% sezione (data)
\section{Lezione del 06-11-24}

% stili pagina
\thispagestyle{empty}
\pagestyle{fancy}

% testo
\subsection{Potenza su induttori mutuamente accoppiati}
Abbiamo visto che i componenti come i resistori, hanno potenza \textbf{attiva}, cioè contribuiscono alla parte reale della potenza complessa, mentre componenti come condensatori e induttori hanno potenza \textbf{reattiva}, cioè contribuiscono alla parte complessa della potenza complessa.
Vediamo adesso se la potenza su due induttori mutuamente accoppiati è attiva o reattiva.

Iniziamo col dire che la formula che conoscevamo per la mutua induttanza, assunte correnti concordi sui contrassegni, cioè:
$$
\dot(V) = j \omega L_1 \dot{I_1} + j \omega M \dot{I_2}
$$
può anche esprimersi attraverso l'ammettenza $\overline{Z}$, con $\dot{V} = \overline{Z} \dot{I}$, cioè:
$$
\dot{V} = \overline{Z} \dot{I_1} + \overline{Z} \dot{I_2}
$$

Prendiamo quindi una coppia di induttori \textbf{reali} (quindi in serie ad una resistenza) mutuamente accoppiati:
\begin{center}
	\begin{circuitikz}
		\draw (0,0) to[ R, l=$R_1$, i=$i_1$] (2,0)
			to[ inductor , l=$L_1$] (2,-2)
			-- (0, -2);

		\draw (6,0) to[ R, l_=$R_2$, i_=$i_2$] (4,0)
			to[ inductor , l_=$L_2$] (4,-2)
			-- (6, -2);

			\draw (1.9,-0.6) node {$\scriptscriptstyle\bullet$};
			\draw (3.9,-0.6) node {$\scriptscriptstyle\bullet$};

			\draw (0,0) node[anchor=east] {$A$};
			\draw (0,-2) node[anchor=east] {$B$};

			\draw (0,-0.5) node[anchor=east] {$+$};
			\draw (0,-1.5) node[anchor=east] {$-$};

			\draw (6,0) node[anchor=west] {$C$};
			\draw (6,-2) node[anchor=west] {$D$};

			\draw (6,-0.5) node[anchor=west] {$+$};
			\draw (6,-1.5) node[anchor=west] {$-$};

			\draw (0,-1) node[anchor=east] {$v_1$};
			\draw (6,-1) node[anchor=west] {$v_2$};

			\draw (3,-0.5) node[anchor=south] {$M$};
	\end{circuitikz}
\end{center}

Potremo esprimere la potenza complessa come:
$$
\overline{S} = V_1 \dot{I_1}^* + V_2 \dot{I_2}^* = \left( R_1 \dot{I_1} + j \omega L_1 \dot{I_1} + j \omega M \dot{I_2} \right) \cdot \dot{I_1}^* + \left( R_2 \dot{I_2} + j \omega L_2 \dot{I_2} + j \omega M \dot{I_1} \right) \cdot \dot{I_2}^*
$$
$$
= R_1 I_1^2 + j \omega L_1 I_1^2 + j \omega M \dot{I_2} \cdot \dot{I_1}^* + R_2 I_2^2 + j \omega L_2 I_2^2 + j \omega M \dot{I_1} \cdot \dot{I_2}^*
$$
$$
= R_1 I_1^2 + R_2 I_2^2 + j \omega L_1 I_1^2 + j \omega L_2 I_2^2 + j \omega M ( \dot{I_2} \cdot \dot{I_1}^* + \dot{I_2} \cdot \dot{I_1}^* )
$$

Da cui notiamo i primi quattro termini essere reali, ergo l'unico termine con possibile componente complessa è $j \omega M ( \dot{I_2} \cdot \dot{I_1}^* + \dot{I_2} \cdot \dot{I_1}^* )$.
Possiamo dire che:
$$
\dot{I_1} = I_1 e^{j \phi_1}, \quad \dot{I_2} = I_2 e^{j \phi_2}
$$
da cui:
$$
\overline{S} = ... + j \omega M \left( I_2 e^{j \phi_2} \cdot I_1 e^{-j \phi_1} + I_1 e^{j \phi_1} \cdot I_2 e^{-j \phi_2} \right) = ... + j \omega M I_1 I_2 \left( e^{j(\phi_2 - \phi_1)} + e^{j(\phi_1 - \phi_2)} \right)
$$
$$
= ... + j \omega M I_1 I_2 \left( \cos(\phi_2 - \phi_1) + j \sin (\phi_2 - \phi_1) + \cos(\phi_1 - \phi_2) + j \sin(\phi_1 - \phi_2) \right)
$$

E quindi dalle proprietà di seni e coseni di argomento negato:
$$
\overline{S} = R_1 I_1^2 + R_2 I_2 ^2 + j \omega L_1 I_1^2 + j \omega L_2 I_2^2 + 2 j \omega M I_1 I_2 \cos(\phi_2 - \phi_2)
$$

Da cui si ha che potenza attiva e reattiva sono rispettivamente:
$$
P = R_1 I_1^2 R_2 I_2^2, \quad jQ = j \omega L_1 I_1^2 + j \omega L_2 I_2^2 + 2 j \omega M I_1 I_2 \cos(\phi_1 - \phi_2)
$$
e quindi la potenza sulle mutue induttanze reali non ha solo componente reattiva (come nelle comuni induttanze), ma anche attiva. 

\subsection{Circuiti equivalenti a induttori mutuamente accoppiati}
Possiamo creare circuiti equivalenti, usando undattori non mutuamente accoppiati, quando si hanno induttori mutuamente accoppiati con nodi in comune:

\begin{center}
	\begin{circuitikz}
		\draw (0,0) to[ short, i=$i_1$] (1,0)
			to[ inductor , l=$L_1$] (1,-2)
			-- (0, -2);

		\draw (4,0) to[ short, i_=$i_2$] (3,0)
			to[ inductor , l_=$L_2$] (3,-2)
			-- (4, -2);

		\draw  (1, -2) to[short] (3, -2);

			\draw (0.9,-0.6) node {$\scriptscriptstyle\bullet$};
			\draw (2.9,-0.6) node {$\scriptscriptstyle\bullet$};

			\draw (0,0) node[anchor=east] {$A$};
			\draw (0,-2) node[anchor=east] {$B$};

			\draw (0,-0.5) node[anchor=east] {$+$};
			\draw (0,-1.5) node[anchor=east] {$-$};

			\draw (4,0) node[anchor=west] {$C$};
			\draw (4,-2) node[anchor=west] {$D$};

			\draw (4,-0.5) node[anchor=west] {$+$};
			\draw (4,-1.5) node[anchor=west] {$-$};

			\draw (0,-1) node[anchor=east] {$v_1$};
			\draw (4,-1) node[anchor=west] {$v_2$};
			
			\draw (2,-0.5) node[anchor=south] {$M$};
	\end{circuitikz}
\end{center}

In questo caso si può usare l'equivalente a stella, selto un nuovo nodo centrale $O$:
\begin{center}
	\begin{circuitikz}
		\draw (0,0) to[ short, i=$i_1$] (1,0)
			to[ inductor , l_=$\scriptstyle L_1 \mp M$] (2,-1)
			to[ inductor, l=$\scriptstyle \pm M$ ] (2,-2);

		\draw (4,0) to[ short, i_=$i_2$] (3,0)
			to[ inductor , l=$\scriptstyle L_1 \mp M$] (2,-1);

		\draw  (0, -2) to[short] (4, -2);

			\draw (0,0) node[anchor=east] {$A$};
			\draw (0,-2) node[anchor=east] {$B$};

			\draw (0,-0.5) node[anchor=east] {$+$};
			\draw (0,-1.5) node[anchor=east] {$-$};

			\draw (4,0) node[anchor=west] {$C$};
			\draw (4,-2) node[anchor=west] {$D$};

			\draw (4,-0.5) node[anchor=west] {$+$};
			\draw (4,-1.5) node[anchor=west] {$-$};

			\draw (0,-1) node[anchor=east] {$v_1$};
			\draw (4,-1) node[anchor=west] {$v_2$};
	\end{circuitikz}
\end{center}

dove i segni superiori si usano nel caso di contrassegni coincidenti rispetto al nodo centrale, e i segni inferiori si usano nel caso di contrassegni opposti.

Possiamo verificare che il circuito è effettivamente equivalente calcolando le cadute di potenziale, ad esempio sul lato sinistro del circuito:
$$
\dot{V_1} = j \omega (L_1 - M) \dot{I_1} + j \omega M (\dot{I_1} + \dot{I_2}) = j \omega L_1 \dot{I_1} - j \omega M \dot{I_1} + j \omega M \dot{I_1} + j \omega M \dot{I_2} = j \omega L \dot{I_1} + j \omega M \dot{I_2} 
$$

Un montaggio alternativo, ma comunque equivalente, è quello a $\pi$:
\begin{center}
	\begin{circuitikz}
		\draw (0,0) to[ short, i=$i_1$] (1,0)
		to[ inductor , l_=$\frac{\Delta}{L_2 \mp M}$] (1,-2)
			-- (0, -2);

		\draw (4,0) to[ short, i_=$i_2$] (3,0)
		to[ inductor , l=$\frac{\Delta}{L_1 \mp M}$] (3,-2)
			-- (4, -2);

		\draw  (1, -2) to[short] (3, -2);

		\draw (1,0) to [ inductor , l=$\frac{\Delta}{\pm M}$] (3, 0);

			\draw (0,0) node[anchor=east] {$A$};
			\draw (0,-2) node[anchor=east] {$B$};

			\draw (0,-0.5) node[anchor=east] {$+$};
			\draw (0,-1.5) node[anchor=east] {$-$};

			\draw (4,0) node[anchor=west] {$C$};
			\draw (4,-2) node[anchor=west] {$D$};

			\draw (4,-0.5) node[anchor=west] {$+$};
			\draw (4,-1.5) node[anchor=west] {$-$};

	\end{circuitikz}
\end{center}

Dove $\Delta = L_1L_2 - M^2$, assunto $L_1L_2 \neq M^2$ (con $M = \sqrt{L_1 L_2}$, siamo in condizioni di mutua induttanza \textit{ideale}).
Come prima, i segni superiori significano induttanze con contrassegni concordi in direzione del polo comune, e i segni inferiori significano induttanze con contrassegni discordi.

\end{document}


\documentclass[a4paper,11pt]{article}
\usepackage[a4paper, margin=8em]{geometry}

% usa i pacchetti per la scrittura in italiano
\usepackage[french,italian]{babel}
\usepackage[T1]{fontenc}
\usepackage[utf8]{inputenc}
\frenchspacing 

% usa i pacchetti per la formattazione matematica
\usepackage{amsmath, amssymb, amsthm, amsfonts}

% usa altri pacchetti
\usepackage{gensymb}
\usepackage{hyperref}
\usepackage{standalone}

% imposta il titolo
\title{Appunti Elettrotecnica}
\author{Luca Seggiani}
\date{2024}

% imposta lo stile
% usa helvetica
\usepackage[scaled]{helvet}
% usa palatino
\usepackage{palatino}
% usa un font monospazio guardabile
\usepackage{lmodern}

\renewcommand{\rmdefault}{ppl}
\renewcommand{\sfdefault}{phv}
\renewcommand{\ttdefault}{lmtt}

% disponi il titolo
\makeatletter
\renewcommand{\maketitle} {
	\begin{center} 
		\begin{minipage}[t]{.8\textwidth}
			\textsf{\huge\bfseries \@title} 
		\end{minipage}%
		\begin{minipage}[t]{.2\textwidth}
			\raggedleft \vspace{-1.65em}
			\textsf{\small \@author} \vfill
			\textsf{\small \@date}
		\end{minipage}
		\par
	\end{center}

	\thispagestyle{empty}
	\pagestyle{fancy}
}
\makeatother

% disponi teoremi
\usepackage{tcolorbox}
\newtcolorbox[auto counter, number within=section]{theorem}[2][]{%
	colback=blue!10, 
	colframe=blue!40!black, 
	sharp corners=northwest,
	fonttitle=\sffamily\bfseries, 
	title=~\thetcbcounter: #2, 
	#1
}

% disponi definizioni
\newtcolorbox[auto counter, number within=section]{definition}[2][]{%
	colback=red!10,
	colframe=red!40!black,
	sharp corners=northwest,
	fonttitle=\sffamily\bfseries,
	title=~\thetcbcounter: #2,
	#1
}

% U.D.M
\newcommand{\amp}{\ensuremath{\mathrm{A}}}
\newcommand{\volt}{\ensuremath{\mathrm{V}}}
\newcommand{\meter}{\ensuremath{\mathrm{m}}}
\newcommand{\second}{\ensuremath{\mathrm{s}}}
\newcommand{\farad}{\ensuremath{\mathrm{F}}}
\newcommand{\henry}{\ensuremath{\mathrm{H}}}
\newcommand{\siemens}{\ensuremath{\mathrm{S}}}

% circuiti
\usepackage{circuitikz}
\usetikzlibrary{babel}

% disegni
\usepackage{pgfplots}
\pgfplotsset{width=10cm,compat=1.9}

% disponi codice
\usepackage{listings}
\usepackage[table]{xcolor}

\lstdefinestyle{codestyle}{
		backgroundcolor=\color{black!5}, 
		commentstyle=\color{codegreen},
		keywordstyle=\bfseries\color{magenta},
		numberstyle=\sffamily\tiny\color{black!60},
		stringstyle=\color{green!50!black},
		basicstyle=\ttfamily\footnotesize,
		breakatwhitespace=false,         
		breaklines=true,                 
		captionpos=b,                    
		keepspaces=true,                 
		numbers=left,                    
		numbersep=5pt,                  
		showspaces=false,                
		showstringspaces=false,
		showtabs=false,                  
		tabsize=2
}

\lstdefinestyle{shellstyle}{
		backgroundcolor=\color{black!5}, 
		basicstyle=\ttfamily\footnotesize\color{black}, 
		commentstyle=\color{black}, 
		keywordstyle=\color{black},
		numberstyle=\color{black!5},
		stringstyle=\color{black}, 
		showspaces=false,
		showstringspaces=false, 
		showtabs=false, 
		tabsize=2, 
		numbers=none, 
		breaklines=true
}

\lstdefinelanguage{javascript}{
	keywords={typeof, new, true, false, catch, function, return, null, catch, switch, var, if, in, while, do, else, case, break},
	keywordstyle=\color{blue}\bfseries,
	ndkeywords={class, export, boolean, throw, implements, import, this},
	ndkeywordstyle=\color{darkgray}\bfseries,
	identifierstyle=\color{black},
	sensitive=false,
	comment=[l]{//},
	morecomment=[s]{/*}{*/},
	commentstyle=\color{purple}\ttfamily,
	stringstyle=\color{red}\ttfamily,
	morestring=[b]',
	morestring=[b]"
}

% disponi sezioni
\usepackage{titlesec}

\titleformat{\section}
	{\sffamily\Large\bfseries} 
	{\thesection}{1em}{} 
\titleformat{\subsection}
	{\sffamily\large\bfseries}   
	{\thesubsection}{1em}{} 
\titleformat{\subsubsection}
	{\sffamily\normalsize\bfseries} 
	{\thesubsubsection}{1em}{}

% disponi alberi
\usepackage{forest}

\forestset{
	rectstyle/.style={
		for tree={rectangle,draw,font=\large\sffamily}
	},
	roundstyle/.style={
		for tree={circle,draw,font=\large}
	}
}

% disponi algoritmi
\usepackage{algorithm}
\usepackage{algorithmic}
\makeatletter
\renewcommand{\ALG@name}{Algoritmo}
\makeatother

% disponi numeri di pagina
\usepackage{fancyhdr}
\fancyhf{} 
\fancyfoot[L]{\sffamily{\thepage}}

\makeatletter
\fancyhead[L]{\raisebox{1ex}[0pt][0pt]{\sffamily{\@title \ \@date}}} 
\fancyhead[R]{\raisebox{1ex}[0pt][0pt]{\sffamily{\@author}}}
\makeatother

\begin{document}
% sezione (data)
\section{Lezione del 13-11-24}

% stili pagina
\thispagestyle{empty}
\pagestyle{fancy}

% testo
Avevamo visto il concetto di \textbf{bipolo}, cioè un componente circuitale con due \textit{punti di contatto} col resto del circuito (\textbf{morsetti}), su cui passa una certa \textbf{corrente} $I$ e su cui si trova una certa \textbf{tensione}, cioè una \textit{differenza di potenzale} $V$.
Potremmo avere anche un \textbf{tripolo}, cioè un componente con morsetti, su cui passano (propriamente, da cui \textit{escono} o \textit{entrano}), anzichè una, 3 correnti, e su cui individuiamo 3 tensioni ($A$, $B$ e $C$) e 3 \textbf{cadute} di tensione su ogni percorso che attraversa il bipolo.
Una possibile rappresentazione di un tripolo è la seguente:
\begin{center}
	\begin{circuitikz}
		\node[rectangle, draw, minimum width = 1cm, minimum height = 1cm] (a) at (0,0) {};
		\draw (-2, 0) to [ short, i=$I_A$] (a.west);
		\draw (0, -2) to [ short, i=$I_C$] (a.south);
		\draw (2, 0) to [ short, i_=$I_B$] (a.east);
		

		\node[anchor=east] at(-2, 0) {$V_A$};
		\node[anchor=north] at(0, -2) {$V_C$};
		\node[anchor=west] at(2, 0) {$V_B$};
	\end{circuitikz}
\end{center}
le cui equazioni sono:
\[
	\begin{cases}
		I_A + I_B + I_C = 0 \\ 
		V_{AB} = V_A - V_B \\ 
		V_{AC} = V_A - V_C \\ 
		V_{BC} = V_B - V_C
	\end{cases}
\]

Notiamo che, dalle equazioni ai potenziali, si possono ricavare le relazioni (piuttosto scontate):
\[
	\begin{cases}
		V_{AB} + V_{BC} = V_{AC} \\ 
		V_{BA} + V_{AC} = V_{BC} \\
		V_{AC} + V_{CB} = V_{AB}
	\end{cases}
\]
con $V_{BA} = - V_{AB}$ e $V_{CB} = -V_{BC}$ (e anche se non si è usata, $V_{CA} = -V_{AC}$).

\subsection{Porte}
Definiamo una \textbf{porta} come una coppia di poli di un circuito dove la corrente entrante è uguale a quella uscente.
Rappresentiamo una porta come segue:

\begin{center}
	\begin{circuitikz}
		\node[rectangle, draw, minimum width = 2cm, minimum height = 2cm] (a) at (0,0) {};
		\draw (-2, 0.6) to [ short, i=$I$] (-1, 0.6);
		\draw(-1, -0.6) to [ short, i=$I$ ] (-2, -0.6);	
	
		\draw (-2.6, 0.6) node[anchor=west] {$+$};
		\draw (-2.6, 0) node[anchor=west] {$V$};
		\draw (-2.6, -0.6) node[anchor=west] {$-$};
	\end{circuitikz}
\end{center}

Notiamo che per $n$ poli si hanno al massimo $\frac{n}{2}$ porte (ammesso un numero pari di poli).

Ciò che ci è di interesse sono i circuiti a \textbf{due porte} (o equivalentemente a \textit{quattro poli}):

\begin{center}
	\begin{circuitikz}
		\node[rectangle, draw, minimum width = 2cm, minimum height = 2cm] (a) at (0,0) {};
		\draw (-2, 0.6) to [ short, i=$I_1$] (-1, 0.6);
		\draw(-1, -0.6) to [ short, i=$I_1$ ] (-2, -0.6);	
	
		\draw (-2.6, 0.6) node[anchor=west] {$+$};
		\draw (-2.6, 0) node[anchor=west] {$V_1$};
		\draw (-2.6, -0.6) node[anchor=west] {$-$};
		
		\draw (2, 0.6) to [ short, i_=$I_2$] (1, 0.6);
		\draw(1, -0.6) to [ short, i_=$I_2$ ] (2, -0.6);	
	
		\draw (2.6, 0.6) node[anchor=east] {$+$};
		\draw (2.6, 0) node[anchor=east] {$V_2$};
		\draw (2.6, -0.6) node[anchor=east] {$-$};
	\end{circuitikz}
\end{center}

Possiamo immaginare che un segnale \textit{entra} da una porta, viene \textit{elaborato} all'interno del circuito, e \textit{esce} dalla porta opposta.

Per convenzione, scegliamo le due correnti $I_1(t)$e $I_2(t)$ come rivolte nello stesso senso, e le due tensioni $V_1(t)$ e $V_2(t)$ come con la stessa polarità:

\subsection{Circuiti equivalenti di circuiti a due porte}
Ciò che può interessarci quando studiamo circuiti a due porte è ricavare \textbf{circuiti equivalenti}, cioè che si comportano in maniera equivalente agli effetti esterni.
L'idea è, come sempre, quella di prendere circuiti arbitrariamente complessi e ridurli a circuiti equivalenti relativamente semplici.

\subsubsection{Rappresentazione a parametri Z}
Una coppia di \textbf{induttori mutuamente accoppiati} rappresenta effettivamente un circuito a due porte, in quanto la stessa corrente entra e esce da ogni induttore (cioè si formano due porte).
\begin{center}
	\begin{circuitikz}
		\draw (0,0) to[ short, i=$i_1$] (1,0)
			to[ inductor , l=$L_1$] (1,-2)
			-- (0, -2);

		\draw (4,0) to[ short, i_=$i_2$] (3,0)
			to[ inductor , l_=$L_2$] (3,-2)
			-- (4, -2);

			\draw (0.9,-0.6) node {$\scriptscriptstyle\bullet$};
			\draw (2.9,-0.6) node {$\scriptscriptstyle\bullet$};

			\draw (0,0) node[anchor=east] {$A$};
			\draw (0,-2) node[anchor=east] {$B$};

			\draw (0,-0.5) node[anchor=east] {$+$};
			\draw (0,-1.5) node[anchor=east] {$-$};

			\draw (4,0) node[anchor=west] {$C$};
			\draw (4,-2) node[anchor=west] {$D$};

			\draw (4,-0.5) node[anchor=west] {$+$};
			\draw (4,-1.5) node[anchor=west] {$-$};

			\draw (0,-1) node[anchor=east] {$v_1$};
			\draw (4,-1) node[anchor=west] {$v_2$};
	\end{circuitikz}
\end{center}

Avevamo rappresentato questi circuiti come:
\[
	\begin{cases}
		\dot{V}_1 = j \omega L_1 \dot{I}_1 + j \omega M \dot{I}_2 \\	
		\dot{V}_2 = j \omega L_2 \dot{I}_2 + j \omega M \dot{I}_1	
	\end{cases}
\]

Analogamente, decidiamo di rappresentare un circuito a due porte attraverso equazioni che legano la tensione su una porta alla corrente su entrambe le porte:
\[
	\begin{cases}
		\dot{V}_1 = \overline{z_{11}} \dot{I}_1 + \overline{z_{12}} \dot{I}_2 \\ 	
		\dot{V}_2 = \overline{z_{21}} \dot{I}_1 + \overline{z_{22}} \dot{I}_2 	
	\end{cases}
\]

Per esprimere queste relazioni in forma più compatta, possiamo sfruttare il calcolo matriciale:
$$
\dot{V} = \overline{Z} \dot{I}
$$
dove $\dot{V}$ e $\dot{I}$ sono matrici:
$$
\begin{pmatrix}
	\dot{V}_1 \\ \dot{V}_2
\end{pmatrix}
= \overline{Z}
\begin{pmatrix}
	\dot{I}_1 \\ \dot{I}_2
\end{pmatrix}
$$
e $\overline{Z}$ sarà l'\textbf{impedenza} in forma matriciale:
$$
\overline{Z} =
\begin{pmatrix}
	\overline{z_{11}} & \overline{z_{12}} \\ 
	\overline{z_{21}} & \overline{z_{22}} 
\end{pmatrix}
$$

Date le equazioni riportate sopra che legano voltaggio a corrente, possiamo ricavare il valore di ogni componente di $\overline{Z}$ come:
$$
\overline{Z} =
\begin{pmatrix}
	\overline{z_{11}} = \frac{\dot{V}_1}{\dot{I}_1} \Big|_{\dot{I}_2 = 0} & 
	\overline{z_{12}} = \frac{\dot{V}_1}{\dot{I}_2} \Big|_{\dot{I}_1 = 0} \\
	\overline{z_{21}} = \frac{\dot{V}_2}{\dot{I}_1} \Big|_{\dot{I}_2 = 0} &
	\overline{z_{22}} = \frac{\dot{V}_2}{\dot{I}_2} \Big|_{\dot{I}_1 = 0}
\end{pmatrix}
$$
dove la notazione $a \Big|_b$ significa "$a$ quando $b$".

Si ha, attraverso queste relazioni, che basta misurare la tensione sulle porte in due stati ($\dot{I}_1 = 0$ e $\dot{I}_2 = 0$) per ricavare completamente i parametri $\overline{Z}$ del circuito, e ricavare quindi un circuito equivalente del tipo:

\begin{center}
	\begin{circuitikz}
		\draw (-4, 1) -- (-3, 1) 
			to [ european resistor, l=$\overline{z_{11}}$, i=$I_1$] (-1, 1)
			to [ controlled voltage source, v<=$\overline{z_{12}} \dot{I}_2$ ] (-1, -1) 
			to [ short, i=$I_1$ ] (-3, -1)	
			-- (-4, -1);
			
		\draw (-4.6, 1) node[anchor=west] {$+$};
		\draw (-4.6, 0) node[anchor=west] {$V_1$};
		\draw (-4.6, -1) node[anchor=west] {$-$};

		\draw (6,1) -- (5, 1) 
			to [ european resistor, l_=$\overline{z_{12}}$, i_=$I_2$] (3, 1)
			to [ controlled voltage source, v_<=$\overline{z_{21}} \dot{I}_1$ ] (3, -1) 
			to [ short, i_=$I_2$ ] (5, -1)
			-- (6, -1);
	
		\draw (6.6, 1) node[anchor=east] {$+$};
		\draw (6.6, 0) node[anchor=east] {$V_2$};
		\draw (6.6, -1) node[anchor=east] {$-$};
		
		\node[rectangle, draw, minimum width = 8.5cm, minimum height = 4cm] (a) at (1,0) {};
	\end{circuitikz}
\end{center}
dove si inseriscono i termini di impedenza $\overline{z_{11}}$ e $\overline{z_{22}}$ semplicemente come impedenze in serie alle porte 1 e 2, e i termini "associati" $\overline{z_{12}}$ e $\overline{z_{21}}$ come generatori di tensione pilotati (che generano, appunto, cadute di tensione pilotate, rispettivamente in $\dot{I}_2$ per la porta 1 e in $\dot{I}_1$ per la porta 2).

Il metodo naturale di analisi per questo circuito è correnti di maglia, che possiamo applicare alle due porte per poi eguagliare con la matrice delle impedenze:
\[
	\begin{cases}
		\dot{V}_1	= \overline{z_{11}} \dot{I}_1 + \overline{Z_{12}} \dot{I}_2 \\  	
		\dot{V}_2	= \overline{z_{22}} \dot{I}_2 + \overline{Z_{21}} \dot{I}_1 \\  	
	\end{cases}
\]
che combacia con quanto definito sulla rappresentazione in impedenza.

In particolare, nel caso $\overline{z_{12}} = \overline{z_{21}}$ si dice che la rete è \textbf{reciproca} e si può formare il circuito equivalente come:

\begin{center}
	\begin{circuitikz}
		\draw (-4, 1) -- (-3, 1) 
			to [ european resistor, l=$\overline{z_a}$, i=$I_1$] (0, 1)
			to [ european resistor, l=$\overline{z_b}$] (0, -1) 
			to [ short, i=$I_1$ ] (-3, -1)	
			-- (-4, -1);
			
		\draw (-4.6, 1) node[anchor=west] {$+$};
		\draw (-4.6, 0) node[anchor=west] {$V_1$};
		\draw (-4.6, -1) node[anchor=west] {$-$};

		\draw (4, 1) -- (3, 1) 
			to [ european resistor, l_=$\overline{z_c}$, i_=$I_2$] (0, 1);

		\draw (0, -1) to [ short, i_=$I_2$ ] (3, -1)
			-- (4, -1);
	
		\draw (4.6, 1) node[anchor=east] {$+$};
		\draw (4.6, 0) node[anchor=east] {$V_2$};
		\draw (4.6, -1) node[anchor=east] {$-$};
		
		\node[rectangle, draw, minimum width = 6.5cm, minimum height = 4cm] (a) at (0,0) {};
	\end{circuitikz}
\end{center}

Anche qui, applicando correnti di maglia, si ha:
\[
	\begin{cases}
		\dot{V}_1 = \overline{z_a} \dot{I}_1 + \overline{z_b}\left( \dot{I}_1 + \dot{I}_2 \right) = ( \overline{z_a} + \overline{z_b} ) \dot{I}_1 + \overline{z_b} \dot{I}_2 \\	
		\dot{V}_2 = \overline{z_c} \dot{I}_2 + \overline{z_b}\left( \dot{I}_1 + \dot{I}_2 \right) = ( \overline{z_c} + \overline{z_c} ) \dot{I}_2 + \overline{z_b} \dot{I}_1 \\ 	
	\end{cases}
\]
che rappresenta la rete reciproca, ponendo:
\[
	\overline{Z} =
	\begin{pmatrix}
		\overline{z_a} + \overline{z_b} & \overline{z_b} \\ 
		\overline{z_b} & \overline{z_b + z_c}
	\end{pmatrix}
	\Leftrightarrow
	\begin{cases}
		\overline{z_a} = \overline{z_{11}} - \overline{z_{12}} = \overline{z_{11}} - \overline{z_{21}} \\ 
		\overline{z_b} = \overline{z_{12}} = \overline{z_{21}} \\
		\overline{z_c} = \overline{z_{22}} - \overline{z_{12}} = \overline{z_{22}} - \overline{z_{21}}
	\end{cases}
\]

Notiamo che, per circuiti a due porte generici, non è detto che i potenziali dei morsetti di uscita di entrambe le porte siano allo stesso potenziale: per modellizzare questo comportamento si usa una \textit{mutua induttanza ideale}, cioè un \textbf{trasformatore ideale}.


\subsubsection{Rappresentazione a parametri Y}
Nella rappresentazione di un circuito a due porte possiamo parametrizzare, anzichè l'impedenza $\overline{Z}$, l'ammettenza $\overline{Y}$: se avevamo espresso il comportamento del circuito come $
\begin{pmatrix}
	\dot{V}_1 \\ \dot{V}_2
\end{pmatrix}
= \overline{Z}
\begin{pmatrix}
	\dot{I}_1 \\ \dot{I}_2
\end{pmatrix}
$, infatti, possiamo trovare l'inverso:
$$
\begin{pmatrix}
	\dot{I}_1 \\ \dot{I}_2
\end{pmatrix}
= \overline{Z}^{-1}
\begin{pmatrix}
	\dot{V}_1 \\ \dot{V}_2
\end{pmatrix}
$$
dove la matrice $\overline{Z}^{-1} = \overline{Y}$ è effettivamente l'\textbf{ammettenza} in forma matriciale del circuito:
$$
\overline{Y} =
\begin{pmatrix}
	\overline{y_{11}} & \overline{y_{12}} \\ 
	\overline{y_{21}} & \overline{y_{22}}
\end{pmatrix}
$$

Da cui il sistema lineare:
\[
	\begin{cases}
		\dot{I}_1 = \overline{y_{11}} \dot{V}_1 + \overline{y_{12}} \dot{V}_2 \\ 	
		\dot{I}_2 = \overline{y_{21}} \dot{V}_1 + \overline{y_{22}} \dot{V}_2 	
	\end{cases}
\]

Date le equazioni riportate sopra, possiamo ricavare il valore di ogni componente di $\overline{Y}$ come:
$$
\overline{Y} =
\begin{pmatrix}
	\overline{y_{11}} = \frac{\dot{I}_1}{\dot{V}_1} \Big|_{\dot{V}_2 = 0} &
	\overline{y_{12}} = \frac{\dot{I}_1}{\dot{V}_2} \Big|_{\dot{V}_1 = 0} \\
	\overline{y_{21}} = \frac{\dot{I}_2}{\dot{V}_1} \Big|_{\dot{V}_2 = 0} &
	\overline{y_{22}} = \frac{\dot{I}_2}{\dot{V}_2} \Big|_{\dot{V}_1 = 0}
\end{pmatrix}
$$

Possiamo quindi disporre un circuito equivalente come segue:
\begin{center}
	\begin{circuitikz}
		\draw (-4, 1) to [ short, i=$I_1$] (-3, 1) 
			to [ short] (-1, 1);
		\draw (-1, 1) to [ controlled current source, cI>=$\overline{y_{12}} \dot{V}_1$ ] (-1, -1);
		\draw (-1, -1) to [ short] (-3, -1)
		-- (-4, -1);
			
		\draw (-4.6, 1) node[anchor=west] {$+$};
		\draw (-4.6, 0) node[anchor=west] {$V_1$};
		\draw (-4.6, -1) node[anchor=west] {$-$};

		\draw (6,1) to [ short, i=$I_2$] (5, 1) 
			to [ short] (3, 1)
			to [ controlled current source, cI_>=$\overline{y_{21}} \dot{V}_2$ ] (3, -1) 
			to [ short] (5, -1)
			-- (6, -1);
	
		\draw (6.6, 1) node[anchor=east] {$+$};
		\draw (6.6, 0) node[anchor=east] {$V_2$};
		\draw (6.6, -1) node[anchor=east] {$-$};
		
		\draw (-2, 1) to [ european resistor, l_=$\overline{y_{11}}$] (-2, -1);
	\draw (4, 1) to [ european resistor, l=$\overline{y_{22}}$] (4, -1);
		
		\node[rectangle, draw, minimum width = 8.5cm, minimum height = 4cm] (a) at (1,0) {};
	\end{circuitikz}
\end{center}
dove stavolta si inseriscono i termini di ammettenza $\overline{y_{11}}$ e $\overline{y_{22}}$ semplicemente ammettenze in parallelo alle porte 1 e 2, e i termini "associati" $\overline{y_{12}}$ e $\overline{y_{21}}$ come generatori di corrente pilotati.
Possiamo analizzare questo circuito considerando le corrente sui rami impedenza e generatore di entrambe le porte, da cui si ottiene:
\[
	\begin{cases}
		\dot{I}_1	= \overline{y_{11}} \dot{V}_1 + \overline{y_{12}} \dot{V}_2 \\  	
		\dot{I}_2	= \overline{y_{22}} \dot{V}_2 + \overline{y_{21}} \dot{V}_1 \\  	
	\end{cases}
\]
che combacia con quanto definito sulla rappresentazione in ammettenza.

In particolare, vediamo il caso \textbf{reciproco} $\overline{y_{11}} = \overline{y_{21}}$: 
\begin{center}
	\begin{circuitikz}
		\draw (-1, 1) to[ european resistor, l_=$\overline{y_a}$] (-1, -1);
		\draw (1, 1) to[ european resistor, l=$\overline{y_c}$] (1, -1);
		
		\draw (-4, 1) to[ short, i=$I_1$] (-3, 1)
			to[ european resistor, l=$\overline{y_b}$] (3, 1)
			to[ short, i<=$I_2$] (4, 1);		

		\draw (-4, -1) to[ short] (4, -1);

		\draw (-4.6, 1) node[anchor=west] {$+$};
		\draw (-4.6, 0) node[anchor=west] {$V_1$};
		\draw (-4.6, -1) node[anchor=west] {$-$};

		\draw (4.6, 1) node[anchor=east] {$+$};
		\draw (4.6, 0) node[anchor=east] {$V_2$};
		\draw (4.6, -1) node[anchor=east] {$-$};
	
		\node [ground] at (0, -1) {};
		
		\draw (-1,1) node[circ] {};
		\draw (-1,1) node[above] {$V_1$};

		\draw (1,1) node[circ] {};
		\draw (1,1) node[above] {$V_2$};
		
		\node[rectangle, draw, minimum width = 6.5cm, minimum height = 4cm] (a) at (0,0) {};
	\end{circuitikz}
\end{center}

Il metodo naturale di analisi per questo circuito è tensioni di nodo, che possiamo applicare alle due porte per poi eguagliare con la matrice delle ammettenze.
Prendiamo i due nodi in alto come $\dot{V}_1$ e $\dot{V}_2$, il nodo in basso come terra, e scriviamo le equazioni:
\[
	\begin{cases}
		\dot{I}_1 = (\overline{y_a} + \overline{y_b}) \dot{V}_1 - \overline{y_b} \dot{V}_2 \\	
		\dot{I}_2 = (\overline{y_b} + \overline{y_c}) \dot{V}_2 - \overline{y_b} \dot{V}_1
	\end{cases}
\]
che rappresenta la rete reciproca, ponendo:
\[
	\overline{Y} = 
	\begin{pmatrix}
		\overline{y_a} + \overline{y_b} & -\overline{y_b} \\ 
		-\overline{y_b} & \overline{y_b} + \overline{y_c}
	\end{pmatrix}
	\Leftrightarrow
	\begin{cases}
		\overline{y_a} = \overline{y_{11}} + \overline{y_{12}} \\ 
		\overline{y_b} = -\overline{y_{12}} = -\overline{y_{21}} \\
		\overline{y_c} = \overline{y_{22}} + \overline{y_{12}}
	\end{cases}
\]

Notiamo che ancora che il circuito più generale si ottiene disaccoppiando i potenziali sul ramo in basso attraverso un trasformatore ideale.
\end{document}


\documentclass[a4paper,11pt]{article}
\usepackage[a4paper, margin=8em]{geometry}

% usa i pacchetti per la scrittura in italiano
\usepackage[french,italian]{babel}
\usepackage[T1]{fontenc}
\usepackage[utf8]{inputenc}
\frenchspacing 

% usa i pacchetti per la formattazione matematica
\usepackage{amsmath, amssymb, amsthm, amsfonts}

% usa altri pacchetti
\usepackage{gensymb}
\usepackage{hyperref}
\usepackage{standalone}

% imposta il titolo
\title{Appunti Elettrotecnica}
\author{Luca Seggiani}
\date{2024}

% imposta lo stile
% usa helvetica
\usepackage[scaled]{helvet}
% usa palatino
\usepackage{palatino}
% usa un font monospazio guardabile
\usepackage{lmodern}

\renewcommand{\rmdefault}{ppl}
\renewcommand{\sfdefault}{phv}
\renewcommand{\ttdefault}{lmtt}

% disponi il titolo
\makeatletter
\renewcommand{\maketitle} {
	\begin{center} 
		\begin{minipage}[t]{.8\textwidth}
			\textsf{\huge\bfseries \@title} 
		\end{minipage}%
		\begin{minipage}[t]{.2\textwidth}
			\raggedleft \vspace{-1.65em}
			\textsf{\small \@author} \vfill
			\textsf{\small \@date}
		\end{minipage}
		\par
	\end{center}

	\thispagestyle{empty}
	\pagestyle{fancy}
}
\makeatother

% disponi teoremi
\usepackage{tcolorbox}
\newtcolorbox[auto counter, number within=section]{theorem}[2][]{%
	colback=blue!10, 
	colframe=blue!40!black, 
	sharp corners=northwest,
	fonttitle=\sffamily\bfseries, 
	title=~\thetcbcounter: #2, 
	#1
}

% disponi definizioni
\newtcolorbox[auto counter, number within=section]{definition}[2][]{%
	colback=red!10,
	colframe=red!40!black,
	sharp corners=northwest,
	fonttitle=\sffamily\bfseries,
	title=~\thetcbcounter: #2,
	#1
}

% U.D.M
\newcommand{\amp}{\ensuremath{\mathrm{A}}}
\newcommand{\volt}{\ensuremath{\mathrm{V}}}
\newcommand{\meter}{\ensuremath{\mathrm{m}}}
\newcommand{\second}{\ensuremath{\mathrm{s}}}
\newcommand{\farad}{\ensuremath{\mathrm{F}}}
\newcommand{\henry}{\ensuremath{\mathrm{H}}}
\newcommand{\siemens}{\ensuremath{\mathrm{S}}}

% circuiti
\usepackage{circuitikz}
\usetikzlibrary{babel}

% disegni
\usepackage{pgfplots}
\pgfplotsset{width=10cm,compat=1.9}

% disponi codice
\usepackage{listings}
\usepackage[table]{xcolor}

\lstdefinestyle{codestyle}{
		backgroundcolor=\color{black!5}, 
		commentstyle=\color{codegreen},
		keywordstyle=\bfseries\color{magenta},
		numberstyle=\sffamily\tiny\color{black!60},
		stringstyle=\color{green!50!black},
		basicstyle=\ttfamily\footnotesize,
		breakatwhitespace=false,         
		breaklines=true,                 
		captionpos=b,                    
		keepspaces=true,                 
		numbers=left,                    
		numbersep=5pt,                  
		showspaces=false,                
		showstringspaces=false,
		showtabs=false,                  
		tabsize=2
}

\lstdefinestyle{shellstyle}{
		backgroundcolor=\color{black!5}, 
		basicstyle=\ttfamily\footnotesize\color{black}, 
		commentstyle=\color{black}, 
		keywordstyle=\color{black},
		numberstyle=\color{black!5},
		stringstyle=\color{black}, 
		showspaces=false,
		showstringspaces=false, 
		showtabs=false, 
		tabsize=2, 
		numbers=none, 
		breaklines=true
}

\lstdefinelanguage{javascript}{
	keywords={typeof, new, true, false, catch, function, return, null, catch, switch, var, if, in, while, do, else, case, break},
	keywordstyle=\color{blue}\bfseries,
	ndkeywords={class, export, boolean, throw, implements, import, this},
	ndkeywordstyle=\color{darkgray}\bfseries,
	identifierstyle=\color{black},
	sensitive=false,
	comment=[l]{//},
	morecomment=[s]{/*}{*/},
	commentstyle=\color{purple}\ttfamily,
	stringstyle=\color{red}\ttfamily,
	morestring=[b]',
	morestring=[b]"
}

% disponi sezioni
\usepackage{titlesec}

\titleformat{\section}
	{\sffamily\Large\bfseries} 
	{\thesection}{1em}{} 
\titleformat{\subsection}
	{\sffamily\large\bfseries}   
	{\thesubsection}{1em}{} 
\titleformat{\subsubsection}
	{\sffamily\normalsize\bfseries} 
	{\thesubsubsection}{1em}{}

% disponi alberi
\usepackage{forest}

\forestset{
	rectstyle/.style={
		for tree={rectangle,draw,font=\large\sffamily}
	},
	roundstyle/.style={
		for tree={circle,draw,font=\large}
	}
}

% disponi algoritmi
\usepackage{algorithm}
\usepackage{algorithmic}
\makeatletter
\renewcommand{\ALG@name}{Algoritmo}
\makeatother

% disponi numeri di pagina
\usepackage{fancyhdr}
\fancyhf{} 
\fancyfoot[L]{\sffamily{\thepage}}

\makeatletter
\fancyhead[L]{\raisebox{1ex}[0pt][0pt]{\sffamily{\@title \ \@date}}} 
\fancyhead[R]{\raisebox{1ex}[0pt][0pt]{\sffamily{\@author}}}
\makeatother

\begin{document}
% sezione (data)
\section{Lezione del 14-11-24}

% stili pagina
\thispagestyle{empty}
\pagestyle{fancy}

% testo
\subsection{Legge di Ohm per circuiti a due porte}
Abbiamo visto come, su circuiti a due porte, \textit{tensione} e \textit{corrente} sono \textbf{vettori} e \textit{impedenza} e \textit{ammettenza} sono \textbf{matrici}, ergo valgono le equazioni:
\[
	\begin{cases}
		\dot{V} = \overline{Z} \dot{I} \\ 
		\dot{I} = \overline{Y} \dot{V}
	\end{cases}
\]
con $\overline{Y} = \overline{Z}^{-1}$. 

Inoltre, abbiamo visto la rappresentazioni in parametri Z e Y che si possono ricavare da queste due forme.
Esistono altre rappresentazioni, che non si basano direttamente sulle equazioni della legge di Ohm rispetto all'impedenza o all'ammettenza, ma su altre caratteristiche del circuito.
Vediamo infatti una rappresentazione utile a modellizzare parametri significativi di circuiti a due porte, la \textbf{rappresentazione a parametri h}.

\subsection{Sintesi a parametri ibridi}
Finora abbiamo scelto le variabili indipendenti come entrambe le tensioni o entrambe le correnti. 
Nessuno ci nega però di scegliere come variabili indipendenti una tensione e una corrente.
Chiamiamo la parametrizzazione che otteniamo da questa scelta \textbf{sintesi a parametri ibridi}, o \textbf{a parametri h}.

La forma generale di una sintesi a parametri h è:
\[
	\begin{cases}
		\dot{V}_1 = \overline{h_{11}} \dot{I}_1 + \overline{h_{12}} \dot{V}_2 \\ 
		\dot{I}_2 = \overline{h_{21}} \dot{I}_1 + \overline{h_{22}} \dot{V}_2
	\end{cases}
\]
o come matrice:
$$
\begin{pmatrix}
	\dot{V}_1 \\ \dot{I}_2
\end{pmatrix}
= \overline{h}
\begin{pmatrix}
	\dot{I}_1 \\ \dot{V}_2
\end{pmatrix}
$$

Le componenti di $\overline{h}$ si ricavano quindi come:
$$
h:
\begin{pmatrix}
		\overline{h_{11}} = \frac{\dot{V}_1}{\dot{I}_1} \Big|_{\dot{V}_2 = 0} &
		\overline{h_{12}} = \frac{\dot{V}_1}{\dot{V}_2} \Big|_{\dot{I}_1 = 0} \\
		\overline{h_{21}} = \frac{\dot{I}_2}{\dot{I}_1} \Big|_{\dot{V}_2 = 0} &
		\overline{h_{22}} = \frac{\dot{I}_2}{\dot{V}_2} \Big|_{\dot{I}_1 = 0}
\end{pmatrix}
$$

Questo tipo di sintesi ha un significato interessante sui vari parametri:
\begin{itemize}
	\item $\overline{h_{11}}$: impedenza in entrata;
	\item $\overline{h_{12}}$: inverso dell'amplificazione di tensione;
	\item $\overline{h_{21}}$: amplificazione di corrente;
	\item $\overline{h_{22}}$: inverso dell impedenza in uscita (ammettenza in uscita).
\end{itemize}

Il circuito equivalente che possiamo formare da una sintesi a parametri ibridi è il seguente: 
\begin{center}
	\begin{circuitikz}
		\draw (-4, 1) -- (-3, 1) 
			to [ european resistor, l=$\overline{h_{11}}$, i=$I_1$] (-1, 1)
			to [ controlled voltage source, v<=$\overline{h_{12}} \dot{V}_2$ ] (-1, -1) 
			to [ short, i=$I_1$ ] (-3, -1)	
			-- (-4, -1);
			
		\draw (-4.6, 1) node[anchor=west] {$+$};
		\draw (-4.6, 0) node[anchor=west] {$V_1$};
		\draw (-4.6, -1) node[anchor=west] {$-$};

		\draw (6,1) to [ short, i=$I_2$] (5, 1) 
			to [ short] (3, 1)
			to [ controlled current source, cI_>=$\overline{h_{21}} \dot{I}_2$ ] (3, -1) 
			to [ short] (5, -1)
			-- (6, -1);
	
		\draw (6.6, 1) node[anchor=east] {$+$};
		\draw (6.6, 0) node[anchor=east] {$V_2$};
		\draw (6.6, -1) node[anchor=east] {$-$};
		
		\draw (4, 1) to [ european resistor, l=$\overline{h_{22}}$] (4, -1);

		\node[rectangle, draw, minimum width = 8.5cm, minimum height = 4cm] (a) at (1,0) {};
	\end{circuitikz}
\end{center}
da cui l'impedenza $\overline{h_{11}}$ e l'ammettenza $\overline{h_{22}}$ vengono rese come impedenze rispetivamente in serie e in parallelo, l'inverso dell'amplificazione di tensione $\overline{h_{12}}$ come il coefficiente di un generatore di tensione pilotato, e l'amplificazione di corrente $\overline{h_{21}}$ come il coefficiente di un generatore di corrente pilotato.
Si noti che sia $\overline{h_{12}}$ che $\overline{h_{21}}$ sono numeri puri, mentre $\overline{h_{11}}$ e $\overline{h_{22}}$

\subsubsection{Condizioni di reciprocità}
Vediamo quindi quando ci troviamo in condizioni di reciprocità.
Si ha dalle formule della rappresentazione:
\[
	\begin{cases}
		\dot{V}_1 = \overline{h_{11}} \dot{I}_1 + \overline{h_{12}} \dot{V}_2 \\ 
		\dot{I}_2 = \overline{h_{21}} \dot{I}_1 + \overline{h_{22}} \dot{V}_2
	\end{cases}
\]

Dalla seconda equazione possiamo scrivere:
$$
\overline{h_{22}} \dot{V}_2 = - \overline{h_{21}} \dot{I_1} + \dot{I_2} \Rightarrow \dot{V}_2 = -\frac{\overline{h_{21}}}{\overline{h_{22}}} \dot{I}_1 + \frac{1}{\overline{h_{22}}} \dot{I}_2 
$$
dove $-\frac{\overline{h_{21}}}{\overline{h_{22}}} = \overline{z_{21}}$ e $\frac{1}{\overline{h_{22}}} = \overline{z_{22}}$ sono effettivamente i parametri Z del circuito.
Analogamente, dalla prima equazione possiamo scrivere:
$$
\dot{V}_1 = \overline{h_{11}} \dot{I_1} + \overline{h_{12}} \left(  -\frac{\overline{h_{21}}}{\overline{h_{22}}} \dot{I}_1 + \frac{1}{\overline{h_{22}}} \dot{I}_2  \right) \Rightarrow \dot{V}_1 = \left( \overline{h_{11}} - \frac{\overline{h_{12}}\overline{h_{21}}}{\overline{h_{22}}} \right) \dot{I_1} + \frac{\overline{h_{12}}}{\overline{h_{22}}} \dot{I}_2 
$$
dove ancora una volta $\overline{h_{11}} - \frac{\overline{h_{12}}\overline{h_{21}}}{\overline{h_{22}}} = \overline{z_{11}}$ e $\frac{\overline{h_{12}}}{\overline{h_{22}}} = \overline{z_{12}}$ sono i parametri Z del circuito. 

Si ricavano quindi i parametri Z in funzione dei parametri h:
$$
\overline{Z} =
\begin{pmatrix}
	\overline{h_{11}} - \frac{\overline{z_{12}}\overline{z_{21}}}{\overline{h_{22}}} & \frac{\overline{h_{12}}}{\overline{h_{22}}} \\ 
	- \frac{\overline{h_{21}}}{\overline{h_{22}}} & \frac{1}{\overline{h_{22}}}
\end{pmatrix}
$$

Imponendo quindi la condizione di reciprocità $\overline{z_{12}} = \overline{z_{21}}$, si ha:
$$
\frac{\overline{z_{12}}}{\overline{z_{22}}} = -\frac{\overline{z_{21}}}{\overline{z_{22}}} \Rightarrow \overline{z_{12}} = - \overline{z_{21}}
$$
posto $\overline{z_{22}} \neq 0$, che è comunque assunto rispettato (l'impedenza non può essere nulla).

Dalle equazioni si ha che quindi che il circuito è \textbf{reciproco} quando la matrice dei parametri h è "antisimmetrica", cioè ha valori negati lungo la diagonale (si noti che questa definizione non è assolutamente quella data in algebra lineare, cioè $A^T= -A$: questa implicherebbe digonale nulla). 

\end{document}


\documentclass[a4paper,11pt]{article}
\usepackage[a4paper, margin=8em]{geometry}

% usa i pacchetti per la scrittura in italiano
\usepackage[french,italian]{babel}
\usepackage[T1]{fontenc}
\usepackage[utf8]{inputenc}
\frenchspacing 

% usa i pacchetti per la formattazione matematica
\usepackage{amsmath, amssymb, amsthm, amsfonts}

% usa altri pacchetti
\usepackage{gensymb}
\usepackage{hyperref}
\usepackage{standalone}

% imposta il titolo
\title{Appunti Elettrotecnica}
\author{Luca Seggiani}
\date{2024}

% imposta lo stile
% usa helvetica
\usepackage[scaled]{helvet}
% usa palatino
\usepackage{palatino}
% usa un font monospazio guardabile
\usepackage{lmodern}

\renewcommand{\rmdefault}{ppl}
\renewcommand{\sfdefault}{phv}
\renewcommand{\ttdefault}{lmtt}

% disponi il titolo
\makeatletter
\renewcommand{\maketitle} {
	\begin{center} 
		\begin{minipage}[t]{.8\textwidth}
			\textsf{\huge\bfseries \@title} 
		\end{minipage}%
		\begin{minipage}[t]{.2\textwidth}
			\raggedleft \vspace{-1.65em}
			\textsf{\small \@author} \vfill
			\textsf{\small \@date}
		\end{minipage}
		\par
	\end{center}

	\thispagestyle{empty}
	\pagestyle{fancy}
}
\makeatother

% disponi teoremi
\usepackage{tcolorbox}
\newtcolorbox[auto counter, number within=section]{theorem}[2][]{%
	colback=blue!10, 
	colframe=blue!40!black, 
	sharp corners=northwest,
	fonttitle=\sffamily\bfseries, 
	title=~\thetcbcounter: #2, 
	#1
}

% disponi definizioni
\newtcolorbox[auto counter, number within=section]{definition}[2][]{%
	colback=red!10,
	colframe=red!40!black,
	sharp corners=northwest,
	fonttitle=\sffamily\bfseries,
	title=~\thetcbcounter: #2,
	#1
}

% U.D.M
\newcommand{\amp}{\ensuremath{\mathrm{A}}}
\newcommand{\volt}{\ensuremath{\mathrm{V}}}
\newcommand{\meter}{\ensuremath{\mathrm{m}}}
\newcommand{\second}{\ensuremath{\mathrm{s}}}
\newcommand{\farad}{\ensuremath{\mathrm{F}}}
\newcommand{\henry}{\ensuremath{\mathrm{H}}}
\newcommand{\siemens}{\ensuremath{\mathrm{S}}}

% circuiti
\usepackage{circuitikz}
\usetikzlibrary{babel}

% disegni
\usepackage{pgfplots}
\pgfplotsset{width=10cm,compat=1.9}

% disponi codice
\usepackage{listings}
\usepackage[table]{xcolor}

\lstdefinestyle{codestyle}{
		backgroundcolor=\color{black!5}, 
		commentstyle=\color{codegreen},
		keywordstyle=\bfseries\color{magenta},
		numberstyle=\sffamily\tiny\color{black!60},
		stringstyle=\color{green!50!black},
		basicstyle=\ttfamily\footnotesize,
		breakatwhitespace=false,         
		breaklines=true,                 
		captionpos=b,                    
		keepspaces=true,                 
		numbers=left,                    
		numbersep=5pt,                  
		showspaces=false,                
		showstringspaces=false,
		showtabs=false,                  
		tabsize=2
}

\lstdefinestyle{shellstyle}{
		backgroundcolor=\color{black!5}, 
		basicstyle=\ttfamily\footnotesize\color{black}, 
		commentstyle=\color{black}, 
		keywordstyle=\color{black},
		numberstyle=\color{black!5},
		stringstyle=\color{black}, 
		showspaces=false,
		showstringspaces=false, 
		showtabs=false, 
		tabsize=2, 
		numbers=none, 
		breaklines=true
}

\lstdefinelanguage{javascript}{
	keywords={typeof, new, true, false, catch, function, return, null, catch, switch, var, if, in, while, do, else, case, break},
	keywordstyle=\color{blue}\bfseries,
	ndkeywords={class, export, boolean, throw, implements, import, this},
	ndkeywordstyle=\color{darkgray}\bfseries,
	identifierstyle=\color{black},
	sensitive=false,
	comment=[l]{//},
	morecomment=[s]{/*}{*/},
	commentstyle=\color{purple}\ttfamily,
	stringstyle=\color{red}\ttfamily,
	morestring=[b]',
	morestring=[b]"
}

% disponi sezioni
\usepackage{titlesec}

\titleformat{\section}
	{\sffamily\Large\bfseries} 
	{\thesection}{1em}{} 
\titleformat{\subsection}
	{\sffamily\large\bfseries}   
	{\thesubsection}{1em}{} 
\titleformat{\subsubsection}
	{\sffamily\normalsize\bfseries} 
	{\thesubsubsection}{1em}{}

% disponi alberi
\usepackage{forest}

\forestset{
	rectstyle/.style={
		for tree={rectangle,draw,font=\large\sffamily}
	},
	roundstyle/.style={
		for tree={circle,draw,font=\large}
	}
}

% disponi algoritmi
\usepackage{algorithm}
\usepackage{algorithmic}
\makeatletter
\renewcommand{\ALG@name}{Algoritmo}
\makeatother

% disponi numeri di pagina
\usepackage{fancyhdr}
\fancyhf{} 
\fancyfoot[L]{\sffamily{\thepage}}

\makeatletter
\fancyhead[L]{\raisebox{1ex}[0pt][0pt]{\sffamily{\@title \ \@date}}} 
\fancyhead[R]{\raisebox{1ex}[0pt][0pt]{\sffamily{\@author}}}
\makeatother

\begin{document}
% sezione (data)
\section{Lezione del 15-11-24}

% stili pagina
\thispagestyle{empty}
\pagestyle{fancy}

% testo
\subsection{Rappresentazione a parametri T}
Vediamo un ultimo tipo di parametrizzazione, la \textbf{parametrizzazione T}.
Le equazioni di rappresentazione sono:
\[
	\begin{cases}
		\dot{V}_1	= \overline{A} \dot{V}_2 + \overline{B} ( - \dot{I}_2 ) \\ 
		\dot{V}_2	= \overline{C} \dot{V}_2 + \overline{D} ( - \dot{I}_2 ) \\ 
	\end{cases}
\]

Notiamo come le grandezze indipendenti qui sono sempre sia tensioni e correnti, ma non in alternanza come nella parametrizzazione h.
Inoltre, notiamo che il termine $\dot{I}_2$ compare con segno negato.

Potremmo pensare di calcolare i parametri T come segue:
$$
T:
\begin{pmatrix}
	\overline{A} = \frac{\dot{V}_1}{\dot{V}_2} \Big|_{-\dot{I}_2 = 0}	& \overline{B} = -\frac{\dot{V}_1}{\dot{I}_2} \Big|_{\dot{V}_2 = 0} \\
	\overline{C} = \frac{\dot{I}_1}{\dot{V}_2} \Big|_{-\dot{I}_2 = 0}	& \overline{D} = -\frac{\dot{I}_1}{\dot{I}_2} \Big|_{\dot{V}_2 = 0} \\
\end{pmatrix}
$$
ma notiamo che risulta difficile calcolare, ad esempio $\overline{A}$, in quanto si chiede di mettere sia un generatore che un aperto alla porta 2 
Scriviamo quindi una matrice del tipo:
$$
T:
\begin{pmatrix}
	\frac{1}{\overline{A}} = \frac{\dot{V}_2}{\dot{V}_1} \Big|_{- \dot{I}_2 = 0} & \frac{1}{\overline{B}} = -\frac{\dot{I}_2}{\dot{V}_1} \Big|_{- \dot{V}_2 = 0} \\
	\frac{1}{\overline{C}} = \frac{\dot{V}_2}{\dot{I}_1} \Big|_{- \dot{I}_2 = 0} & \frac{1}{\overline{D}} = -\frac{\dot{I}_2}{\dot{I}_1} \Big|_{- \dot{V}_2 = 0} \\
\end{pmatrix}
$$

\subsubsection{Circuito equivalente}
Riscriviamo la seconda equazione di rappresentazione come:
$$
\dot{I}_1 = \overline{C} \dot{V}_2 + \overline{D} (-\dot{I}_2) \Rightarrow -\dot{I}_2 = \frac{1}{\overline{D}} \dot{I}_1 - \frac{\overline{C}}{\overline{D}}\dot{V}_2
$$
Un possibile circuito equivalente di una parametrizzazione T sarà quindi il seguente:
\begin{center}
	\begin{circuitikz}
		\draw (-4, 1) -- (-3, 1) 
		to [ controlled voltage source, v<=$\overline{A}\dot{V}_2$, i>=$I_1$] (-1, 1)
		to [ controlled voltage source, v<=$-\overline{B} \dot{I}_2$ ] (-1, -1) 
			to [ short, i=$I_1$ ] (-3, -1)	
			-- (-4, -1);
			
		\draw (-4.6, 1) node[anchor=west] {$+$};
		\draw (-4.6, 0) node[anchor=west] {$V_1$};
		\draw (-4.6, -1) node[anchor=west] {$-$};

		\draw (6,1) to [ short, i=$I_2$] (5, 1) 
			to [ short] (3, 1)
			to [ controlled current source, cI_<=$\frac{\dot{I}_1}{\overline{D}}$ ] (3, -1) 
			to [ short] (5, -1)
			-- (6, -1);
	
		\draw (6.6, 1) node[anchor=east] {$+$};
		\draw (6.6, 0) node[anchor=east] {$V_2$};
		\draw (6.6, -1) node[anchor=east] {$-$};
		
		\draw (4, 1) to [ european resistor, l=$\frac{\overline{D}}{\overline{C}}$] (4, -1);

		\node[rectangle, draw, minimum width = 8.5cm, minimum height = 5cm] (a) at (1,0) {};
	\end{circuitikz}
\end{center}

\subsubsection{Condizioni di reciprocità}
Troviamo quindi le condizioni di reciprocità.
Come avevamo fatto per i parametri h, riportiamoci in parameri Z, ad esempio partendo dalla seconda equazione:
$$
\dot{V}_2 = \frac{1}{\overline{C}} \dot{I}_1 + \frac{\overline{D}}{\overline{C}} \dot{I}_2
$$
da cui $\frac{1}{\overline{C}} = \overline{z_{21}}$ e $\frac{\overline{D}}{\overline{C}} = \overline{z_{22}}$, e analogamente per la prima:
$$
\dot{V}_1 = \overline{A} \left( \frac{1}{\overline{C}} \dot{I}_1 + \frac{\overline{D}}{\overline{C}} \dot{I}_2 \right) + \overline{B} \left( - \dot{I}_2 \right) = \dot{I}_1 \frac{1}{\overline{C}} + \dot{I}_2 \left( \frac{\overline{D}}{\overline{C}} - \overline{B} \right)
$$
da cui $\overline{z_{11}} = \frac{\overline{A}}{\overline{C}}$ e $\overline{z_{12}} = \frac{\overline{A} \overline{D}}{\overline{C}} - \overline{B}$.
Si ricavano quindi i parametri Z in funzione dei parametri T:
$$
\overline{Z} =
\begin{pmatrix}
	\frac{\overline{A}}{\overline{C}} & \frac{\overline{A} \overline{D}}{\overline{C}} - \overline{B} \\
	\frac{1}{\overline{C}} & \frac{\overline{D}}{\overline{C}}
\end{pmatrix}
$$

Imponendo la condizione di reciprocità $\overline{z_{12}} = \overline{z_{21}}$, si ha:
$$ 
\frac{\overline{A} \overline{D}}{\overline{C}} - \overline{B} = \frac{1}{\overline{C}} \Rightarrow \overline{A}\overline{D} - \overline{B}\overline{C} = 1
$$
cioè, il circuito è \textbf{reciproco} quando la matrice dei parametri T ha determinante $\det(T) = 1$. 

\subsection{Circuiti a due porte in serie}
Poniamo di avere due circuiti a due porte $A$ e $B$, percorsi rispettivamente dalle correnti $I_{1a}$ e $I_{2a}$, e $I_{1b}$ e $I_{2b}$. collegati fra di loro in \textbf{serie}, cioè su cui scorre la \textit{stessa corrente}:

\begin{center}
	\begin{circuitikz}
		\node[rectangle, draw, minimum width = 2cm, minimum height = 2cm] (a) at (0,0) {};
		\draw (-2, 0.6) to [ short, i=$I_{1a}$] (-1, 0.6);
		\draw(-1, -0.6) to [ short, i=$I_{1a}$ ] (-2, -0.6);	
	
		\draw (-2.6, 0.6) node[anchor=west] {$+$};
		\draw (-2.6, 0) node[anchor=west] {$V_{1a}$};
		\draw (-2.6, -0.6) node[anchor=west] {$-$};
		
		\draw (2, 0.6) to [ short, i_=$I_{2a}$] (1, 0.6);
		\draw(1, -0.6) to [ short, i_=$I_{2a}$ ] (2, -0.6);	
	
		\draw (2.6, 0.6) node[anchor=east] {$+$};
		\draw (2.6, 0) node[anchor=east] {$V_{2b}$};
		\draw (2.6, -0.6) node[anchor=east] {$-$};


		\node[rectangle, draw, minimum width = 2cm, minimum height = 2cm] (a) at (0,-3) {};
		\draw (-2, -2.4) to [ short, i=$I_{1b}$] (-1, -2.4);
		\draw(-1, -3.6) to [ short, i=$I_{1b}$ ] (-2, -3.6);	
	
		\draw (-2.6, -2.4) node[anchor=west] {$+$};
		\draw (-2.6, -3) node[anchor=west] {$V_{1b}$};
		\draw (-2.6, -3.6) node[anchor=west] {$-$};
		
		\draw (2, -2.4) to [ short, i_=$I_{2b}$] (1, -2.4);
		\draw(1, -3.6) to [ short, i_=$I_{2b}$ ] (2, -3.6);	
	
		\draw (2.6, -2.4) node[anchor=east] {$+$};
		\draw (2.6, -3) node[anchor=east] {$V_{2b}$};
		\draw (2.6, -3.6) node[anchor=east] {$-$};

		\node at (0, 0) {$Z_A$};
		\node at (0, -3) {$Z_B$};

		\draw (2, -2.4) -- (2, -0.6);
		\draw (-2, -2.4) -- (-2, -0.6);
	\end{circuitikz}
\end{center}

# nota che questa è sempre una porta

Calcoliamo la caduta di potenziale sulle due porte ($1$ e $2$):
$$
\begin{pmatrix}
	\dot{V}_{1s} \\ \dot{V}_{2s}
\end{pmatrix}
=
\begin{pmatrix}
	\dot{V}_{1a} \\ \dot{V}_{2a}
\end{pmatrix}
+
\begin{pmatrix}
	\dot{V}_{1b} \\ \dot{V}_{2b}
\end{pmatrix}
=
\overline{Z_{a}} 
\begin{pmatrix}
	\dot{I}_{1a} \\ \dot{I}_{1b}
\end{pmatrix}
+
\overline{Z_{b}}
\begin{pmatrix}
	\dot{I}_{2a} \\ \dot{I}_{2b}
\end{pmatrix}
=
\left( \overline{Z_a} + \overline{Z_b} \right) 
\begin{pmatrix}
	\dot{I_{1s}} \\ \dot{I_{2s}}
\end{pmatrix}
$$
che è quello che ci aspettavamo: la matrice dei parametri Z di due circuiti a due porte in serie è data dalla \textit{somma} delle matrici dei parametri Z dei singoli circuiti.

\subsubsection{Circuiti a due porte in parallelo}
Poniamo adesso di avere due circuiti a due porte $A$ e $B$, percorsi sempre dalle correnti $I_{1a}$ e $I_{2a}$, e $I_{1b}$ e $I_{2b}$. collegati fra di loro in \textbf{parallelo}, cioè che si trovano allo \textit{stesso potenziale}:

\begin{center}
	\begin{circuitikz}
		\node[rectangle, draw, minimum width = 2cm, minimum height = 2cm] (a) at (0,0) {};
		\draw (-2, 0.6) to [ short, i=$I_{1a}$] (-1, 0.6);
		\draw(-1, -0.6) to [ short, i=$I_{1a}$ ] (-2, -0.6);	
	
		\draw (-2.6, 0.6) node[anchor=west] {$+$};
		\draw (-2.6, 0) node[anchor=west] {$V_{1a}$};
		\draw (-2.6, -0.6) node[anchor=west] {$-$};
		
		\draw (2, 0.6) to [ short, i_=$I_{2a}$] (1, 0.6);
		\draw(1, -0.6) to [ short, i_=$I_{2a}$ ] (2, -0.6);	
	
		\draw (2.6, 0.6) node[anchor=east] {$+$};
		\draw (2.6, 0) node[anchor=east] {$V_{2b}$};
		\draw (2.6, -0.6) node[anchor=east] {$-$};


		\node[rectangle, draw, minimum width = 2cm, minimum height = 2cm] (a) at (0,-3) {};
		\draw (-2, -2.4) to [ short, i=$I_{1b}$] (-1, -2.4);
		\draw(-1, -3.6) to [ short, i=$I_{1b}$ ] (-2, -3.6);	
	
		\draw (-2.6, -2.4) node[anchor=west] {$+$};
		\draw (-2.6, -3) node[anchor=west] {$V_{1b}$};
		\draw (-2.6, -3.6) node[anchor=west] {$-$};
		
		\draw (2, -2.4) to [ short, i_=$I_{2b}$] (1, -2.4);
		\draw(1, -3.6) to [ short, i_=$I_{2b}$ ] (2, -3.6);	
	
		\draw (2.6, -2.4) node[anchor=east] {$+$};
		\draw (2.6, -3) node[anchor=east] {$V_{2b}$};
		\draw (2.6, -3.6) node[anchor=east] {$-$};

		\node at (0, 0) {$Y_A$}; \TODO % non so il perchè delle Y. poi finisci circuito parallelo
		\node at (0, -3) {$Y_B$};

	\end{circuitikz}
\end{center}

# come prima è sempre una porta

Calcoliamo quindi la corrente che attraversa le porte:
$$
\begin{pmatrix}
	\dot{I}_{1s} \\ \dot{I}_{2s} 
\end{pmatrix}
=
\begin{pmatrix}
	\dot{I}_{1a} \\ \dot{I}_{2a}
\end{pmatrix}
+
\begin{pmatrix}
	\dot{I}_{1b} \\ \dot{I}_{2b}
\end{pmatrix}
=
\overline{Y_a} 
\begin{pmatrix}
	\dot{V}_{1a} \\ \dot{V}_{2a}
\end{pmatrix}
+
\overline{Y_b}
\begin{pmatrix}
	\dot{V}_{1b} \\ \dot{V}_{2b}
\end{pmatrix}
=
(\overline{Y_a} + \overline{Y_b})
\begin{pmatrix}
	\dot{I}_{1s} \\ \dot{I}_{2s}
\end{pmatrix}
$$

# riguarda formule, poi ha preso un circuito in serie e l ha risolto coi parametri Y (Y_A^-1 + Y_B^_1)^_1

\subsection{Circuiti a due porte in cascata}
Nel collegamento \textbf{a cascata}, l'uscita di una porta và direttamente in ingresso a una seconda porta, cioè:
\begin{center}
	\begin{circuitikz}
		\node[rectangle, draw, minimum width = 2cm, minimum height = 2cm] (a) at (0,0) {};
		\draw (-2, 0.6) to [ short, i=$I_{1a}$] (-1, 0.6);
		\draw(-1, -0.6) to [ short, i=$I_{1a}$ ] (-2, -0.6);	
	
		\draw (-2.6, 0.6) node[anchor=west] {$+$};
		\draw (-2.6, 0) node[anchor=west] {$V_{1a}$};
		\draw (-2.6, -0.6) node[anchor=west] {$-$};
		
		\draw (2, 0.6) to [ short, i_=$I_{2a}$] (1, 0.6);
		\draw(1, -0.6) to [ short, i_=$I_{2a}$ ] (2, -0.6);	
	
		\draw (2.6, 0.6) node[anchor=east] {$+$};
		\draw (2.6, 0) node[anchor=east] {$V_{2b}$};
		\draw (2.6, -0.6) node[anchor=east] {$-$};


		\node[rectangle, draw, minimum width = 2cm, minimum height = 2cm] (a) at (0,-3) {};
		\draw (-2, -2.4) to [ short, i=$I_{1b}$] (-1, -2.4);
		\draw(-1, -3.6) to [ short, i=$I_{1b}$ ] (-2, -3.6);	
	
		\draw (-2.6, -2.4) node[anchor=west] {$+$};
		\draw (-2.6, -3) node[anchor=west] {$V_{1b}$};
		\draw (-2.6, -3.6) node[anchor=west] {$-$};
		
		\draw (2, -2.4) to [ short, i_=$I_{2b}$] (1, -2.4);
		\draw(1, -3.6) to [ short, i_=$I_{2b}$ ] (2, -3.6);	
	
		\draw (2.6, -2.4) node[anchor=east] {$+$};
		\draw (2.6, -3) node[anchor=east] {$V_{2b}$};
		\draw (2.6, -3.6) node[anchor=east] {$-$};

		\node at (0, 0) {$T_A$}; \TODO % finisci circuito cascata 
		\node at (0, -3) {$T_B$};

	\end{circuitikz}
\end{center}

Non dobbiamo dimostrare che anche questo circuito è una porta, in quanto si prende come ingresso l'ingresso della porta $T_A$ ($I_{1c})$ , $V_{1c}$) e come uscita l'uscita della porta $T_B$ ($I_{2c})$ , $V_{2c}$).

Possiamo quindi esprimere queste relazioni fra le i circuiti interni e le porte esterne come segue:
$$
\begin{pmatrix}
	\dot{V}_{1c} \\ \dot{I}_{1c}
\end{pmatrix}
=
\begin{pmatrix}
	\dot{V}_{1a} \\ \dot{I}_{1a}
\end{pmatrix}
=
\overline{T_a}
\begin{pmatrix}
	\dot{V}_{2a} \\ - \dot{I}_{2a}
\end{pmatrix}
=
\overline{T_a}
\begin{pmatrix}
	\dot{V}_{1b} \\ \dot{I}_{1b}
\end{pmatrix}
=
(\overline{T_a} \overline{T_b}) 
\begin{pmatrix}
	\dot{V}_{2b} \\ -\dot{I}_{2b}
\end{pmatrix}
(\overline{T_a} \overline{T_b})
\begin{pmatrix}
	\dot{V}_{2c} \\ -\dot{I}_{2c}
\end{pmatrix}
$$
da cui:
$$
\begin{pmatrix}
	\dot{V}_{1c} \\ \dot{I}_{1v} 
\end{pmatrix}
=
(\overline{T_a} \overline{T_b})
\begin{pmatrix}
	\dot{V}_{2c} \\ -\dot{I}_{2c}
\end{pmatrix}
$$

# questo sopra boh

\subsection{Collegamento ibrido serie/parallelo}
Attraverso le porte abbiamo a disposizione un ulteriore tipo di collegamento, il cosiddetto collegamento \textbf{ibrido}, cioè dove una coppia di porte viene connessa in serie e l'altra coppia in parallelo:
\begin{center}
	\begin{circuitikz}
		\node[rectangle, draw, minimum width = 2cm, minimum height = 2cm] (a) at (0,0) {};
		\draw (-2, 0.6) to [ short, i=$I_{1a}$] (-1, 0.6);
		\draw(-1, -0.6) to [ short, i=$I_{1a}$ ] (-2, -0.6);	
	
		\draw (-2.6, 0.6) node[anchor=west] {$+$};
		\draw (-2.6, 0) node[anchor=west] {$V_{1a}$};
		\draw (-2.6, -0.6) node[anchor=west] {$-$};
		
		\draw (2, 0.6) to [ short, i_=$I_{2a}$] (1, 0.6);
		\draw(1, -0.6) to [ short, i_=$I_{2a}$ ] (2, -0.6);	
	
		\draw (2.6, 0.6) node[anchor=east] {$+$};
		\draw (2.6, 0) node[anchor=east] {$V_{2b}$};
		\draw (2.6, -0.6) node[anchor=east] {$-$};


		\node[rectangle, draw, minimum width = 2cm, minimum height = 2cm] (a) at (0,-3) {};
		\draw (-2, -2.4) to [ short, i=$I_{1b}$] (-1, -2.4);
		\draw(-1, -3.6) to [ short, i=$I_{1b}$ ] (-2, -3.6);	
	
		\draw (-2.6, -2.4) node[anchor=west] {$+$};
		\draw (-2.6, -3) node[anchor=west] {$V_{1b}$};
		\draw (-2.6, -3.6) node[anchor=west] {$-$};
		
		\draw (2, -2.4) to [ short, i_=$I_{2b}$] (1, -2.4);
		\draw(1, -3.6) to [ short, i_=$I_{2b}$ ] (2, -3.6);	
	
		\draw (2.6, -2.4) node[anchor=east] {$+$};
		\draw (2.6, -3) node[anchor=east] {$V_{2b}$};
		\draw (2.6, -3.6) node[anchor=east] {$-$};

		\node at (0, 0) {$h_A$}; \TODO % finisci circuito ibrido 
		\node at (0, -3) {$h_B$};

	\end{circuitikz}
\end{center}

Per questo tipo di circuiti potremmo dire:
$$
\begin{pmatrix}
	\dot{V}_{1m} \\ \dot{I}_{2m}
\end{pmatrix}
=
\begin{pmatrix}
	\dot{V}_{1a} \\ \dot{I}_{2a}
\end{pmatrix}
+
\begin{pmatrix}
	\dot{V}_{1b} \\ \dot{I}_{2b}
\end{pmatrix}
=
\overline{h_a}
\begin{pmatrix}
	\dot{I}_{1a} \\ \dot{V}_{2a}
\end{pmatrix}
+
\overline{h_b}
\begin{pmatrix}
	\dot{I}_{1b} \\ \dot{V_{2b}}
\end{pmatrix}
=
(\overline{h_a} + \overline{h_b})
\begin{pmatrix}
	\dot{I}_{1m} \\ \dot{V}_{2m}
\end{pmatrix}
$$

\end{document}
# chiudi tutto il discorso che parametrizzazioni diverse aiutano a risolvere circuiti diversi


\documentclass[a4paper,11pt]{article}
\usepackage[a4paper, margin=8em]{geometry}

% usa i pacchetti per la scrittura in italiano
\usepackage[french,italian]{babel}
\usepackage[T1]{fontenc}
\usepackage[utf8]{inputenc}
\frenchspacing 

% usa i pacchetti per la formattazione matematica
\usepackage{amsmath, amssymb, amsthm, amsfonts}

% usa altri pacchetti
\usepackage{gensymb}
\usepackage{hyperref}
\usepackage{standalone}

% imposta il titolo
\title{Appunti Elettrotecnica}
\author{Luca Seggiani}
\date{2024}

% imposta lo stile
% usa helvetica
\usepackage[scaled]{helvet}
% usa palatino
\usepackage{palatino}
% usa un font monospazio guardabile
\usepackage{lmodern}

\renewcommand{\rmdefault}{ppl}
\renewcommand{\sfdefault}{phv}
\renewcommand{\ttdefault}{lmtt}

% disponi il titolo
\makeatletter
\renewcommand{\maketitle} {
	\begin{center} 
		\begin{minipage}[t]{.8\textwidth}
			\textsf{\huge\bfseries \@title} 
		\end{minipage}%
		\begin{minipage}[t]{.2\textwidth}
			\raggedleft \vspace{-1.65em}
			\textsf{\small \@author} \vfill
			\textsf{\small \@date}
		\end{minipage}
		\par
	\end{center}

	\thispagestyle{empty}
	\pagestyle{fancy}
}
\makeatother

% disponi teoremi
\usepackage{tcolorbox}
\newtcolorbox[auto counter, number within=section]{theorem}[2][]{%
	colback=blue!10, 
	colframe=blue!40!black, 
	sharp corners=northwest,
	fonttitle=\sffamily\bfseries, 
	title=~\thetcbcounter: #2, 
	#1
}

% disponi definizioni
\newtcolorbox[auto counter, number within=section]{definition}[2][]{%
	colback=red!10,
	colframe=red!40!black,
	sharp corners=northwest,
	fonttitle=\sffamily\bfseries,
	title=~\thetcbcounter: #2,
	#1
}

% U.D.M
\newcommand{\amp}{\ensuremath{\mathrm{A}}}
\newcommand{\volt}{\ensuremath{\mathrm{V}}}
\newcommand{\meter}{\ensuremath{\mathrm{m}}}
\newcommand{\second}{\ensuremath{\mathrm{s}}}
\newcommand{\farad}{\ensuremath{\mathrm{F}}}
\newcommand{\henry}{\ensuremath{\mathrm{H}}}
\newcommand{\siemens}{\ensuremath{\mathrm{S}}}

% circuiti
\usepackage{circuitikz}
\usetikzlibrary{babel}

% disegni
\usepackage{pgfplots}
\pgfplotsset{width=10cm,compat=1.9}

% disponi codice
\usepackage{listings}
\usepackage[table]{xcolor}

\lstdefinestyle{codestyle}{
		backgroundcolor=\color{black!5}, 
		commentstyle=\color{codegreen},
		keywordstyle=\bfseries\color{magenta},
		numberstyle=\sffamily\tiny\color{black!60},
		stringstyle=\color{green!50!black},
		basicstyle=\ttfamily\footnotesize,
		breakatwhitespace=false,         
		breaklines=true,                 
		captionpos=b,                    
		keepspaces=true,                 
		numbers=left,                    
		numbersep=5pt,                  
		showspaces=false,                
		showstringspaces=false,
		showtabs=false,                  
		tabsize=2
}

\lstdefinestyle{shellstyle}{
		backgroundcolor=\color{black!5}, 
		basicstyle=\ttfamily\footnotesize\color{black}, 
		commentstyle=\color{black}, 
		keywordstyle=\color{black},
		numberstyle=\color{black!5},
		stringstyle=\color{black}, 
		showspaces=false,
		showstringspaces=false, 
		showtabs=false, 
		tabsize=2, 
		numbers=none, 
		breaklines=true
}

\lstdefinelanguage{javascript}{
	keywords={typeof, new, true, false, catch, function, return, null, catch, switch, var, if, in, while, do, else, case, break},
	keywordstyle=\color{blue}\bfseries,
	ndkeywords={class, export, boolean, throw, implements, import, this},
	ndkeywordstyle=\color{darkgray}\bfseries,
	identifierstyle=\color{black},
	sensitive=false,
	comment=[l]{//},
	morecomment=[s]{/*}{*/},
	commentstyle=\color{purple}\ttfamily,
	stringstyle=\color{red}\ttfamily,
	morestring=[b]',
	morestring=[b]"
}

% disponi sezioni
\usepackage{titlesec}

\titleformat{\section}
	{\sffamily\Large\bfseries} 
	{\thesection}{1em}{} 
\titleformat{\subsection}
	{\sffamily\large\bfseries}   
	{\thesubsection}{1em}{} 
\titleformat{\subsubsection}
	{\sffamily\normalsize\bfseries} 
	{\thesubsubsection}{1em}{}

% disponi alberi
\usepackage{forest}

\forestset{
	rectstyle/.style={
		for tree={rectangle,draw,font=\large\sffamily}
	},
	roundstyle/.style={
		for tree={circle,draw,font=\large}
	}
}

% disponi algoritmi
\usepackage{algorithm}
\usepackage{algorithmic}
\makeatletter
\renewcommand{\ALG@name}{Algoritmo}
\makeatother

% disponi numeri di pagina
\usepackage{fancyhdr}
\fancyhf{} 
\fancyfoot[L]{\sffamily{\thepage}}

\makeatletter
\fancyhead[L]{\raisebox{1ex}[0pt][0pt]{\sffamily{\@title \ \@date}}} 
\fancyhead[R]{\raisebox{1ex}[0pt][0pt]{\sffamily{\@author}}}
\makeatother

\begin{document}
% sezione (data)
\section{Lezione del 21-11-24}

% stili pagina
\thispagestyle{empty}
\pagestyle{fancy}

% testo
\subsection{Analisi di circuiti aperiodici}
Finora abbiamo studiato circuiti in \textbf{corrente continua} e in \textbf{regime sinusoidale}.
Adesso vedremo come studiare circuiti dove le forme d'onda dei generatori sono arbitrarie.

\subsubsection{Circuito RL}
Poniamo un circuito formato da un resistore in serie a un induttore, come segue:

\begin{center}
	\begin{tikzpicture}
		\draw (0,0) to [ voltage source,  l=$e(t)$] (0,3)
			to [ resistor, l=$R$] (3,3)
			to [ inductor, l=$L$] (3,0)
			-- (0,0);
	\end{tikzpicture}
\end{center}

Diciamo che il generatore di tensione $e(t)$ ha forma d'onda:
\[
	e(t) =
	\begin{cases}
		E_0, \quad t < 0 \\ 
		2E_0, \quad t \geq 0
	\end{cases}
\]
di cui si riporta un grafico:
\begin{center}
\begin{tikzpicture}
    \begin{axis}[
        xlabel={$t$},
        ylabel={$e(t)$},
        domain=-10:10, % set the x range you want,
				samples=100,
        grid=major, % add a grid
				ytick={1, 2},
				yticklabels={$E_0$, $2 E_0$},
				xtick={0},
				xticklabels={$0$},
				width=15cm,
				height=7cm
    ]
    \addplot[
        red,
        thick,
				domain=-10:0
    ] {1};

    \addplot[
        red,
        thick,
				domain=0:10
    ] {2}; 
    \end{axis}
\end{tikzpicture}
\end{center}

Nell'intervallo negativo possiamo assumere che il circuito sia rimasto a $E_0$ costante per un tempo tale da poterlo studiare nello stato di equilibrio.
Cioè, se cercavamo $i(t)$, avremo semplicemente:
$$
i(t) = \frac{E_0}{R}, \quad t < 0
$$

Allo stesso modo, per tempi $t >> 0$, quindi con $t \rightarrow \infty$, potremo immaginare che il circito si trova nuovamente allo stato di equilibrio, cioè:
$$
i(t) = \frac{2 E_0}{R} \quad t >> 0
$$

La domanda è quindi come \textit{varia} la corrente $i(t)$ nell'intervallo immediatamente $t \geq 0$.
Chiamiamo il comportamento della corrente in questa fase \textbf{transitorio}.

Potremo applicare la prima legge di Kirchoff all'unica maglia del circuito, ricordando la caduta di potenziale sull'induttanza in funzione della corrente $i(t)$:
$$
-2 E_0 + R i(t) + L \frac{d \, i(t)}{dt} = 0
$$

Questa è un'equazione differenziale del primo ordine, che sappiamo ha soluzione omogenea e disomogenea (o generale e particolare \textit{specifica}, insomma qualsiasi nomenclatura viene riportata nel testo di Analisi 1 preferito del lettore):
$$
i(t) = i_o(t) + i_p(t)
$$

Risolvendo per $i_o(t)$:
$$
R i_o(t) + L \frac{d \, i_o(t)}{dt} = 0, \quad i_o(t) = Ae^{\lambda t}
$$
$$
R \lambda^0 + L \lambda^1 = 0 \Rightarrow \lambda = -\frac{R}{L}
$$
da cui:
$$
i_o(t) = Ae^{-\frac{R}{L}t}
$$

Troviamo quindi $i_p(t)$, posto $i_p(t) = I$ e dal regime costante in $t < 0$, $\frac{dI}{dt} = 0$:
$$
-2 E_0 + RI = 0 \Rightarrow I = i_p(t) = 2 \frac{E_0}{R}
$$

Otteniamo quindi la soluzione:
$$
i(t) = i_o(t) + i_p(t) = 2 \frac{E_0}{R} + A e^{-\frac{R}{l}t}
$$

Imponiamo la condizione iniziale, cioè $i(t) = \frac{E_0}{R}$ per $t < 0$:
$$
i(0) = \frac{E_0}{R} = \frac{2 E_0}{R} + A \Rightarrow A = \frac{E_0}{R} - \frac{2 E_0}{R} = -\frac{E_0}{R}
$$
da cui:
$$
i(t) = \frac{2 E_0}{R} - \frac{E_0}{R} e^{-\frac{R}{L}t} = \frac{E_0}{R}\left( 2 - e^{-\frac{R}{L}t} \right)
$$

Da cui la soluzione finale.
Possiamo tracciare un grafico dell'andamento della corrente:

\begin{center}
\begin{tikzpicture}
    \begin{axis}[
        xlabel={$t$},
        ylabel={$i(t)$},
        domain=-10:10, % set the x range you want,
				samples=100,
        grid=major, % add a grid
				ytick={1, 2},
				yticklabels={$\frac{E_0}{R}$, $\frac{2 E_0}{R}$},
				xtick={0},
				xticklabels={$0$},
				width=15cm,
				height=7cm
    ]
    \addplot[
        blue,
        thick,
				domain=-10:0
    ] {1};

    \addplot[
        blue,
        thick,
				domain=0:10
    ] {2-exp(-x)};

		\addplot[
        red,
        dotted,
				domain=0:10
    ] {2}; 
    \end{axis}
\end{tikzpicture}
\end{center}

Questa procedura, sebbene sia completamente generale, è difficilmente applicabile su circuiti più complessi.
Conviene quindi spostarsi in un nuovo dominio, seguendo un procedimento simile a quello che avevamo seguito usando i fasori.

\subsection{Trasformata di Laplace}
Rappresentato un segnale come una funzione $f(t)$, la \textbf{trasformata di Laplace} viene indicata come:
$$
F(s) = \mathcal{L} \left\{ f(t) \right\} = \int_{0^-}^{+\infty} f(t)e^{-st} \, dt
$$
dove $s$ è un complesso $\sigma + i \omega$.

Questa trasformata risulta molto utile in quanto ci permette di ricondurre le funzioni con cui lavoriamo nel cosiddetto spazio $s$, all'interno del cui le operazioni di derivazione e integrazione si riducono a divisioni e moltiplicazioni, e quindi all'algebra.

\subsubsection{Proprietà della trasformata di Laplace}

\begin{itemize}
	\item \textbf{Derivata:} la trasformata di Laplace di $f'(t)$ sarà:
		$$
		\mathcal{L}\left\{ f'(t) \right\} = \int_{0^-}^{+\infty} f'(t) e^{-st} dt	
		$$
		dove possiamo applicare la formula di integrazione per parti:
		$$ 
		\int_{0^-}^{+\infty} f'(t) e^{-st} dt = f(t) e^{-st} \Big|_{0^-}^{+\infty} - \int_{0^-}^{+\infty} f(t) \cdot -s e^{-st} dt 
		$$
		$$
		= s \int_{0^-}^{+\infty} f(t) e^{-st} dt + \lim_{t \rightarrow +\infty} f(t)e^{-st} - f(0^-) = s F(s) - f(0^-)
		$$
		dove compare un termine $f(0^-)$ dovuto alle condizioni iniziali.
\item \textbf{Integrale:} la trasformata di Laplace dell'integrale $g(t) = \int_{-\infty}^t f(\tau) d\tau$ sarà:
		$$
		\mathcal{L}\left\{ \frac{d\,g(t)}{dt} \right\} = \mathcal{L}\left\{ f(t) \right\} = s \mathcal{L}\left\{ g(t) \right\} - g(0^-) =  s \mathcal{L}\left\{ \int_\infty^t f(\tau) d\tau \right\} - \int_{-\infty}^{0^-} f(\tau) d\tau
		$$
		da cui possiamo ricavare la trasformata dell'integrale:
		$$
		\mathcal{L}\left\{ \int_{-\infty}^t f(\tau) d\tau \right\} = \frac{F(s)}{s} + \frac{1}{s} \int_{-\infty}^{0^-} f(\tau) d \tau
		$$
dove nuovamente compare un termine $\frac{1}{s} \int_{-\infty}^{0^-} f(\tau) d \tau$ dovuto alle condizioni iniziali.
\end{itemize}

\end{document}


\documentclass[a4paper,11pt]{article}
\usepackage[a4paper, margin=8em]{geometry}

% usa i pacchetti per la scrittura in italiano
\usepackage[french,italian]{babel}
\usepackage[T1]{fontenc}
\usepackage[utf8]{inputenc}
\frenchspacing 

% usa i pacchetti per la formattazione matematica
\usepackage{amsmath, amssymb, amsthm, amsfonts}

% usa altri pacchetti
\usepackage{gensymb}
\usepackage{hyperref}
\usepackage{standalone}

% imposta il titolo
\title{Appunti Elettrotecnica}
\author{Luca Seggiani}
\date{2024}

% imposta lo stile
% usa helvetica
\usepackage[scaled]{helvet}
% usa palatino
\usepackage{palatino}
% usa un font monospazio guardabile
\usepackage{lmodern}

\renewcommand{\rmdefault}{ppl}
\renewcommand{\sfdefault}{phv}
\renewcommand{\ttdefault}{lmtt}

% disponi il titolo
\makeatletter
\renewcommand{\maketitle} {
	\begin{center} 
		\begin{minipage}[t]{.8\textwidth}
			\textsf{\huge\bfseries \@title} 
		\end{minipage}%
		\begin{minipage}[t]{.2\textwidth}
			\raggedleft \vspace{-1.65em}
			\textsf{\small \@author} \vfill
			\textsf{\small \@date}
		\end{minipage}
		\par
	\end{center}

	\thispagestyle{empty}
	\pagestyle{fancy}
}
\makeatother

% disponi teoremi
\usepackage{tcolorbox}
\newtcolorbox[auto counter, number within=section]{theorem}[2][]{%
	colback=blue!10, 
	colframe=blue!40!black, 
	sharp corners=northwest,
	fonttitle=\sffamily\bfseries, 
	title=~\thetcbcounter: #2, 
	#1
}

% disponi definizioni
\newtcolorbox[auto counter, number within=section]{definition}[2][]{%
	colback=red!10,
	colframe=red!40!black,
	sharp corners=northwest,
	fonttitle=\sffamily\bfseries,
	title=~\thetcbcounter: #2,
	#1
}

% U.D.M
\newcommand{\amp}{\ensuremath{\mathrm{A}}}
\newcommand{\volt}{\ensuremath{\mathrm{V}}}
\newcommand{\meter}{\ensuremath{\mathrm{m}}}
\newcommand{\second}{\ensuremath{\mathrm{s}}}
\newcommand{\farad}{\ensuremath{\mathrm{F}}}
\newcommand{\henry}{\ensuremath{\mathrm{H}}}
\newcommand{\siemens}{\ensuremath{\mathrm{S}}}

% circuiti
\usepackage{circuitikz}
\usetikzlibrary{babel}

% disegni
\usepackage{pgfplots}
\pgfplotsset{width=10cm,compat=1.9}

% disponi codice
\usepackage{listings}
\usepackage[table]{xcolor}

\lstdefinestyle{codestyle}{
		backgroundcolor=\color{black!5}, 
		commentstyle=\color{codegreen},
		keywordstyle=\bfseries\color{magenta},
		numberstyle=\sffamily\tiny\color{black!60},
		stringstyle=\color{green!50!black},
		basicstyle=\ttfamily\footnotesize,
		breakatwhitespace=false,         
		breaklines=true,                 
		captionpos=b,                    
		keepspaces=true,                 
		numbers=left,                    
		numbersep=5pt,                  
		showspaces=false,                
		showstringspaces=false,
		showtabs=false,                  
		tabsize=2
}

\lstdefinestyle{shellstyle}{
		backgroundcolor=\color{black!5}, 
		basicstyle=\ttfamily\footnotesize\color{black}, 
		commentstyle=\color{black}, 
		keywordstyle=\color{black},
		numberstyle=\color{black!5},
		stringstyle=\color{black}, 
		showspaces=false,
		showstringspaces=false, 
		showtabs=false, 
		tabsize=2, 
		numbers=none, 
		breaklines=true
}

\lstdefinelanguage{javascript}{
	keywords={typeof, new, true, false, catch, function, return, null, catch, switch, var, if, in, while, do, else, case, break},
	keywordstyle=\color{blue}\bfseries,
	ndkeywords={class, export, boolean, throw, implements, import, this},
	ndkeywordstyle=\color{darkgray}\bfseries,
	identifierstyle=\color{black},
	sensitive=false,
	comment=[l]{//},
	morecomment=[s]{/*}{*/},
	commentstyle=\color{purple}\ttfamily,
	stringstyle=\color{red}\ttfamily,
	morestring=[b]',
	morestring=[b]"
}

% disponi sezioni
\usepackage{titlesec}

\titleformat{\section}
	{\sffamily\Large\bfseries} 
	{\thesection}{1em}{} 
\titleformat{\subsection}
	{\sffamily\large\bfseries}   
	{\thesubsection}{1em}{} 
\titleformat{\subsubsection}
	{\sffamily\normalsize\bfseries} 
	{\thesubsubsection}{1em}{}

% disponi alberi
\usepackage{forest}

\forestset{
	rectstyle/.style={
		for tree={rectangle,draw,font=\large\sffamily}
	},
	roundstyle/.style={
		for tree={circle,draw,font=\large}
	}
}

% disponi algoritmi
\usepackage{algorithm}
\usepackage{algorithmic}
\makeatletter
\renewcommand{\ALG@name}{Algoritmo}
\makeatother

% disponi numeri di pagina
\usepackage{fancyhdr}
\fancyhf{} 
\fancyfoot[L]{\sffamily{\thepage}}

\makeatletter
\fancyhead[L]{\raisebox{1ex}[0pt][0pt]{\sffamily{\@title \ \@date}}} 
\fancyhead[R]{\raisebox{1ex}[0pt][0pt]{\sffamily{\@author}}}
\makeatother

\begin{document}
% sezione (data)
\section{Lezione del 22-11-24}

% stili pagina
\thispagestyle{empty}
\pagestyle{fancy}

% testo
\subsection{Leggi dei bipoli con la trasformata di Laplace}
Vediamo quindi come possiamo esprimere il legame fra corrente e tensione dei bipoli visti finora attraverso la trasformata di Laplace.
\subsubsection{Resistori}
Poniamo di avere un resistore. Avevamo che:
$$
v_R(t) = R i_R(t)
$$
Visto che non abbiamo derivate o integrali, possiamo passare direttamente al dominio $s$:
$$
V_R(s) = R I_R(s)
$$

\subsubsection{Induttori}
Rispetto agli induttori, avevamo che:
$$
v_L(t) = L\frac{di_L(t)}{dt}
$$

che nella trasformata di Laplace, applicando la legge di derivata, diventerà:
$$
V_L(s) = L \left( s I_L(s) - i_L(0^-) \right) = s L I_L(s) - L i_L (0^-)
$$
dove appare un termine dovuto alle condizioni iniziali, cioè $L i_L (0^-)$. 

Abbiamo quindi che un circuito equivalente all'induttore secondo la trasformata di Laplace è dato da una serie fra un induttore con induttanza generalizzata $sL$ e un generatore di tensione (detto \textbf{generatore di condizioni iniziali}) $L i_L(0^-)$, cioè:

\begin{center}
	\begin{tikzpicture}
		\draw (0,0) to [ inductor, l=$sL$] (2, 0)
			to [ voltage source, v=$L i_L(0^-)$] (4, 0);
	\end{tikzpicture}
\end{center}

\subsubsection{Condensatori}
Rispetto ai condensatori, avevamo che:
$$
v_C(t) = \frac{1}{C} \int_{-\infty}^t i_C(\tau) d\tau
$$

che nella trasformata di Laplace, applicando la legge di integrale, diventerà:
$$
V_C(s) = \frac{1}{C} \left( \frac{I_C(s)}{s} + \frac{1}{s}\int_{-\infty}^{0^-}i_C(t)dt \right) = \frac{I_C(s)}{sC} + \frac{1}{sC} q(0^-)
$$
ricordando che $\int_{-\infty}^{0^-} i(t) dt = q(t)$.
Potrebbe però essere più conveniente applicare la definizione di capacità, da cui $v_C(t) = \frac{q(t)}{C}$ e quindi:
$$
V_C(s) = \frac{I_C(s)}{sC} + \frac{v_C(0^-)}{s}
$$
cioè anche qui appare un termine dovuto alle condizioni iniziali,$ \frac{v_C(0^-)}{s}$.

Abbiamo quindi che un circuito equivalente al condensatore secondo la trasformata di Laplace è dato da una serie fra un condensatore $\frac{1}{sC}$ e un generatore di tensione (il generatore di condizioni iniziali) $\frac{v_C(0^-)}{s}$, cioè:

\begin{center}
	\begin{tikzpicture}
		\draw (0,0) to [ capacitor, l=$\frac{1}{sC}$] (2, 0)
			to [ voltage source, v<=$\frac{v_C(0^-)}{s}$] (4, 0);
	\end{tikzpicture}
\end{center}

\subsubsection{Induttanze mutuamente accoppiate}
Due induttanze mutuamente accoppiate venivano governate dalla legge:
\[
	\begin{cases}			
		v_1(t) = L_1 \frac{d i_1(t)}{dt} \pm M \frac{d i_2(t)}{dt} \\
		v_2(t) = L_2 \frac{d i_2(t)}{dt} \pm M \frac{d i_1(t)}{dt}
	\end{cases}
\]

Nel dominio $s$, queste diventano:
\[
	\begin{cases}
		V_1(s) = L_1 \left( s I_1(s) - i_1(0^-) \right) \pm M \left( s I_2(s) - i_2(0^-) \right) \\
		V_2(s) = L_2 \left( s I_2(s) - i_2(0^-) \right) \pm M \left( s I_1(s) - i_1(0^-) \right)
	\end{cases}
\]
che dà:
\[
	\begin{cases}
		V_1(s) = s L_1 I_1(s) \pm s M I_2(s) - L_1 i_1(0^-) \mp M i_2(0^-) \\
		V_2(s) = s L_2 I_2(s) \pm s M I_1(s) - L_2 i_2(0^-) \mp M i_1(0^-)
	\end{cases}
\]

Abbiamo allora che un circuito equivalente alle induttanze mutuamente accoppiate secondo la trasformata di Laplace è dato da, su ogni ramo, una serie di:
\begin{itemize}
	\item Un generatore di tensione $L_1 i_1(0^-)$ ($L_2 i_2 (0^-))$;
	\item Un generatore di tensione $\pm M i_2(0^-)$ ($\pm M i_1(0^-)$)
\end{itemize}
entrambi con contrassegni concordi alla direzione della corrente:

\begin{center}
	\begin{circuitikz}
		\draw (0,0) to[ short, i=$i_1$] (1,0)
			to[ inductor , l=$sL_1$] (1,-2)
			to[ voltage source, v_=$M i_2(0^-)$] (1, -4)
			to[ voltage source, v_=$L_1 i_1(0^-)$] (1, -6)
			-- (0, -6);

		\draw (4,0) to[ short, i_=$i_2$] (3,0)
			to[ inductor , l_=$sL_2$] (3,-2)
			to[ voltage source, v=$M i_1(0^-)$] (3, -4)
			to[ voltage source, v=$L_2 i_2(0^-)$] (3, -6)
			-- (4, -6);

			\draw (0.9,-0.6) node {$\scriptscriptstyle\bullet$};
			\draw (2.9,-0.6) node {$\scriptscriptstyle\bullet$};
	\end{circuitikz}
\end{center}


\subsection{Analisi circuitale nel dominio s}
Vediamo quindi come usare la trasformata di Laplace per risolvere il circuito (RL) che abbiamo usato per introdurre i circuiti a regime aperiodico:

\begin{center}
	\begin{tikzpicture}
		\draw (0,0) to [ voltage source,  l=$e(t)$] (0,3)
			to [ resistor, l=$R$] (3,3)
			to [ inductor, l=$L$] (3,0)
			-- (0,0);
	\end{tikzpicture}
\end{center}

Dove avevamo che il generatore di tensione $e(t)$ ha forma d'onda:
\[
	e(t) =
	\begin{cases}
		E_0, \quad t < 0 \\ 
		2E_0, \quad t \geq 0
	\end{cases}
\]
di cui si riporta un grafico:
\begin{center}
\begin{tikzpicture}
    \begin{axis}[
        xlabel={$t$},
        ylabel={$e(t)$},
        domain=-10:10, % set the x range you want,
				samples=100,
        grid=major, % add a grid
				ytick={1, 2},
				yticklabels={$E_0$, $2 E_0$},
				xtick={0},
				xticklabels={$0$},
				width=15cm,
				height=7cm
    ]
    \addplot[
        red,
        thick,
				domain=-10:0
    ] {1};

    \addplot[
        red,
        thick,
				domain=0:10
    ] {2}; 
    \end{axis}
\end{tikzpicture}
\end{center}

Iniziamo la risoluzione.
Dovremmo dividere il problema in intervalli temporali:
\begin{itemize}
	\item $t < 0$: qui sarà incognita, oltre a $i(t)$, anche la condizione iniziale $i_L(0^-)$.
		Appare quindi chiaro come mai abbiamo usato $0^-$ nelle formule: le condizioni iniziali che ci interessano sono quelle \textit{un attimo prima} della transizione, e non dopo.

		La corrente $i_L(0^-)$ sull'induttore varrà quindi quanto quella nel circuito a regime costante, ergo $i_L(0^-) = \frac{E_0}{R}$.
	\item $t \geq 0$: ci troviamo sul transitorio, e dobbiamo quindi usare Laplace.
		Trasformiamo l'induttanza nel circuito equivalente col generatore di condizioni iniziali:

\begin{center}
	\begin{tikzpicture}
		\draw (0,0) to [ voltage source,  v=$2 \frac{E_0}{s}$] (0,3)
			to [ resistor, l=$R$] (3,3)
			to [ inductor, l=$sL$] (3,1.5)
			to [ voltage source, v=$L \frac{E_0}{R}$] (3,0)
			-- (0,0);
	\end{tikzpicture}
\end{center}

A questo punto l'incognita sarà la corrente $i_x(t)$ sul circuito, che potremo trovare con la seconda legge di Kirchoff:
$$
-2 \frac{E_0}{s} + R I_x(s) + s L I_x(s) - L \frac{E_0}{R} = 0 \Rightarrow I_x(s) \frac{L \frac{E_0}{R} + 2 \frac{E_0}{s}}{r + s L} = \frac{L E_0 s + 2 R E_0}{R L s^2 + R^2 s}
$$

Infine, vorremo riportare l'espressione ottenuta nel dominio del tempo, in quanto quello che abbiamo trovato è effettivamente:
$$I_x(s) = \mathcal{L}\{ i_x(t) \}$$
Per fare questo abbiamo bisogno dell'\textbf{antitrasformata} di Laplace.
Vediamo come si ricava.
\end{itemize}

\subsection{Antitrasformata di Laplace}
Data una forma $I_x(s) = \mathcal{L}\{ i_x(t) \}$, vediamo come calcolare $ i_x(t) = \mathcal{L}^{-1} \{ I_x(t) \}$.
Facciamo l'esempio con la trasformata:
$$
I(s) = \frac{s^2 + 5}{3s^3 + 9s^2 + 6s}
$$

Innanzitutto vorremo riportare il denominatore in forma \textit{normalizzata}, cioè esprimere il termine di grado più alto senza coefficienti:
$$
I(s) = \frac{\frac{s^2}{3} + \frac{5}{3}}{s^3 + 3s^2 + 2s}
$$

A questo punto vorremo \textbf{fattorizzare} il denominatore:
$$
I(s) = \frac{\frac{s^2}{3} + \frac{5}{3}}{s (s^2 + 3s +1)} = \frac{\frac{s^2}{3} + \frac{5}{3}}{s (s + 1) (s + 2)}
$$

Poniamo quindi:
$$
I(s) = \frac{A_1}{s} + \frac{A_2}{s + 1} + \frac{A_3}{s+2}
$$
che ci permetterà di sfruttare il fatto che $\mathcal{L}^-1\left\{ \frac{1}{s + a} \right\} = e^{-at}$ e $\mathcal{L}^{-1} \left\{ \frac{1}{s} \right\} = u(t)$ (gradino di Heaviside), e quindi dire che:
$$
i(t) = \left( A_1 + A_2 e^{-t} + A_3 e^{-2t} \right) u(t)
$$

Questo equivale a riportare la frazione trovata in \textbf{fratti semplici}, da cui si ottiene:
$$
A_1 = \frac{5}{6}, \quad A_2 = -2, \quad A_3 = \frac{3}{2} 
$$
e quindi:
$$
i(t) = \left( \frac{5}{6} -2e^{-t} + \frac{3}{2}e^{-2t} \right) u(t)
$$

Vediamo un metodo rapido per il calcolo dei fratti semplici.

\subsubsection{Teorema dei residui}
Potremo dire che: # introducilo non l'ho mai sentito
$$
I(s) = \sum_{i = 1}^n \frac{A_i}{s - P_i}
$$
e quindi:
$$
A_i = \lim_{s \rightarrow P_i} (s - P_i) I(s)
$$

Ad esempio, applichiamo sull'esempio precedente:
$$
A_1 = \lim_{s \rightarrow 0} s \cdot \frac{\frac{s^2}{3} + \frac{5}{3}}{s(s+1)(s+2)}
= \frac{\frac{s^2}{3} + \frac{5}{3}}{(s+1)(s+2)} = \frac{\frac{5}{3}}{3} = \frac{5}{6}
$$
e via dicendo.

\par\smallskip

Risolviamo quindi l'ultimo passaggio dell'esercizio precendente.
Avevamo ricavato la trasformata:
$$
I_x(s) = \frac{L E_0 s + 2 R E_0}{R L s^2 + R^2 s}
$$

Normalizzando, si ha:
$$
I_x(s) = \frac{\frac{E_0}{R} s + 2 \frac{E_0}{L}}{s^2 + \frac{R}{L}s} = \frac{\frac{E_0}{R} s + 2 \frac{E_0}{L}}{s \left(s + \frac{R}{L}\right)}
$$

A questo punto possiamo impostare i fratti semplici:
$$
I(s) = \frac{A_1}{s} + \frac{A_2}{s + \frac{R}{L}}
$$
e risolvere, ad esempio col metodo dei residui:

Si ritrova quindi la funzione di $t$:
$$
i(t) = \left( 2\frac{E_0}{R} - \frac{E_0}{R} e^{-\frac{R}{L}t} \right) u(t) = \frac{E_0}{R} \left( 2 - e^{-\frac{R}{L}t} \right) u(t)
$$
che equivale a quanto avevamo trovato risolvendo direttamente l'equazione differenziale, salvo il termine $u(t)$ per il gradino di Heaviside, che però non ci interessa in quanto avremmo preso comunque il transiente da $t = 0$ in poi, cioè combinando le soluzioni:
\[
	i(t) = 
	\begin{cases}
		\frac{E_0}{R}, \quad t < 0 \\ 
		\frac{E_0}{R} \left( 2 - e^{-\frac{R}{L}t} \right), \quad t \geq 0
	\end{cases}
\]

\subsubsection{Verifica dei risultati}
Potremmo voler fare una verifica della validità dei risultati ottenuti dopo la risoluzione di un circuito con la trasformata di Laplace.
Una prima verifica potrebbe essere considerare il limite:
$$
\lim_{t\rightarrow\infty} i(t)
$$
cioè, dopo un tempo $t >> 0$, il circuito tende a un regime stazionario che corrisponde con quello che avremmo prendendo il circuito a corrente continua? 
Nel caso precedente, notiamo come la corrente $i(t)$ tende giustamente a $\frac{2 E_0}{R}$.

Una verifica più sofisticata si può fare considerando le \textbf{variabili di stato} del circuito, cioè quei valori che non possono variare in maniera discontinua. 
Ad esempio, nell'esempio precedente, una variabile di stato è l'energia nell'induttore, cioè:
$$
W_L(t) = \frac{1}{2}L i_L^2(t)
$$

Questa non potrà mai essere discontinua, in quanto implicherebbe derivata infinità sulla discontinuità, e visto che $\frac{d W_L(t)}{dt} = P$ potenza, avremmo dal teorema di Boucherot che da qualche parte nel circuito erogherebbe (seppur istantaneamente) una potenza infinita, che è impossibile.

Verifichiamo quindi la continuità:
$$
\lim_{t \rightarrow 0^-} i_L(t) =^? \lim_{t \rightarrow 0^+} i_L(t)
$$
che nell'esempio è verificata da:
$$
\lim_{t \rightarrow 0^-} i_L(t) = \lim_{t \rightarrow 0^+} i_L(t) = \frac{E_0}{R}
$$

\end{document}


\documentclass[a4paper,11pt]{article}
\usepackage[a4paper, margin=8em]{geometry}

% usa i pacchetti per la scrittura in italiano
\usepackage[french,italian]{babel}
\usepackage[T1]{fontenc}
\usepackage[utf8]{inputenc}
\frenchspacing 

% usa i pacchetti per la formattazione matematica
\usepackage{amsmath, amssymb, amsthm, amsfonts}

% usa altri pacchetti
\usepackage{gensymb}
\usepackage{hyperref}
\usepackage{standalone}

% imposta il titolo
\title{Appunti Elettrotecnica}
\author{Luca Seggiani}
\date{2024}

% imposta lo stile
% usa helvetica
\usepackage[scaled]{helvet}
% usa palatino
\usepackage{palatino}
% usa un font monospazio guardabile
\usepackage{lmodern}

\renewcommand{\rmdefault}{ppl}
\renewcommand{\sfdefault}{phv}
\renewcommand{\ttdefault}{lmtt}

% disponi il titolo
\makeatletter
\renewcommand{\maketitle} {
	\begin{center} 
		\begin{minipage}[t]{.8\textwidth}
			\textsf{\huge\bfseries \@title} 
		\end{minipage}%
		\begin{minipage}[t]{.2\textwidth}
			\raggedleft \vspace{-1.65em}
			\textsf{\small \@author} \vfill
			\textsf{\small \@date}
		\end{minipage}
		\par
	\end{center}

	\thispagestyle{empty}
	\pagestyle{fancy}
}
\makeatother

% disponi teoremi
\usepackage{tcolorbox}
\newtcolorbox[auto counter, number within=section]{theorem}[2][]{%
	colback=blue!10, 
	colframe=blue!40!black, 
	sharp corners=northwest,
	fonttitle=\sffamily\bfseries, 
	title=~\thetcbcounter: #2, 
	#1
}

% disponi definizioni
\newtcolorbox[auto counter, number within=section]{definition}[2][]{%
	colback=red!10,
	colframe=red!40!black,
	sharp corners=northwest,
	fonttitle=\sffamily\bfseries,
	title=~\thetcbcounter: #2,
	#1
}

% U.D.M
\newcommand{\amp}{\ensuremath{\mathrm{A}}}
\newcommand{\volt}{\ensuremath{\mathrm{V}}}
\newcommand{\meter}{\ensuremath{\mathrm{m}}}
\newcommand{\second}{\ensuremath{\mathrm{s}}}
\newcommand{\farad}{\ensuremath{\mathrm{F}}}
\newcommand{\henry}{\ensuremath{\mathrm{H}}}
\newcommand{\siemens}{\ensuremath{\mathrm{S}}}

% circuiti
\usepackage{circuitikz}
\usetikzlibrary{babel}

% disegni
\usepackage{pgfplots}
\pgfplotsset{width=10cm,compat=1.9}

% disponi codice
\usepackage{listings}
\usepackage[table]{xcolor}

\lstdefinestyle{codestyle}{
		backgroundcolor=\color{black!5}, 
		commentstyle=\color{codegreen},
		keywordstyle=\bfseries\color{magenta},
		numberstyle=\sffamily\tiny\color{black!60},
		stringstyle=\color{green!50!black},
		basicstyle=\ttfamily\footnotesize,
		breakatwhitespace=false,         
		breaklines=true,                 
		captionpos=b,                    
		keepspaces=true,                 
		numbers=left,                    
		numbersep=5pt,                  
		showspaces=false,                
		showstringspaces=false,
		showtabs=false,                  
		tabsize=2
}

\lstdefinestyle{shellstyle}{
		backgroundcolor=\color{black!5}, 
		basicstyle=\ttfamily\footnotesize\color{black}, 
		commentstyle=\color{black}, 
		keywordstyle=\color{black},
		numberstyle=\color{black!5},
		stringstyle=\color{black}, 
		showspaces=false,
		showstringspaces=false, 
		showtabs=false, 
		tabsize=2, 
		numbers=none, 
		breaklines=true
}

\lstdefinelanguage{javascript}{
	keywords={typeof, new, true, false, catch, function, return, null, catch, switch, var, if, in, while, do, else, case, break},
	keywordstyle=\color{blue}\bfseries,
	ndkeywords={class, export, boolean, throw, implements, import, this},
	ndkeywordstyle=\color{darkgray}\bfseries,
	identifierstyle=\color{black},
	sensitive=false,
	comment=[l]{//},
	morecomment=[s]{/*}{*/},
	commentstyle=\color{purple}\ttfamily,
	stringstyle=\color{red}\ttfamily,
	morestring=[b]',
	morestring=[b]"
}

% disponi sezioni
\usepackage{titlesec}

\titleformat{\section}
	{\sffamily\Large\bfseries} 
	{\thesection}{1em}{} 
\titleformat{\subsection}
	{\sffamily\large\bfseries}   
	{\thesubsection}{1em}{} 
\titleformat{\subsubsection}
	{\sffamily\normalsize\bfseries} 
	{\thesubsubsection}{1em}{}

% disponi alberi
\usepackage{forest}

\forestset{
	rectstyle/.style={
		for tree={rectangle,draw,font=\large\sffamily}
	},
	roundstyle/.style={
		for tree={circle,draw,font=\large}
	}
}

% disponi algoritmi
\usepackage{algorithm}
\usepackage{algorithmic}
\makeatletter
\renewcommand{\ALG@name}{Algoritmo}
\makeatother

% disponi numeri di pagina
\usepackage{fancyhdr}
\fancyhf{} 
\fancyfoot[L]{\sffamily{\thepage}}

\makeatletter
\fancyhead[L]{\raisebox{1ex}[0pt][0pt]{\sffamily{\@title \ \@date}}} 
\fancyhead[R]{\raisebox{1ex}[0pt][0pt]{\sffamily{\@author}}}
\makeatother

\begin{document}
% sezione (data)
\section{Lezione del 27-11-24}

% stili pagina
\thispagestyle{empty}
\pagestyle{fancy}

% testo
\subsection{Eccezioni al processo di antitrasformata}
Esistono eccezioni al processo visto di antitrasformazione.
Prendiamo ad esempio la forma rapporto di polinomi:
$$
I(s) = \frac{s + 2}{s^2 + 9}
$$
Notiamo come i poli saranno numeri complessi.

Si avrà quindi:
$$
= \frac{s + 2}{(s+3j)(s-3j)} = \frac{A_1}{s+3j} + \frac{A_2}{s-3j}
$$

Applicando il teorema dei residui per trovare $A_1$ e $A_2$ si ottiene:
$$
A_1 = \lim_{s\rightarrow -3j} (s + 3j) \frac{s + 2}{(s+3j)(s-3j)} = \frac{2-3j}{-6j} = \frac{1}{2} + \frac{1}{3}j
$$
$$
A_2 = \lim_{s\rightarrow 3j} (s - 3j) \frac{s + 2}{(s+3j)(s-3j)} = \frac{2+3j}{6j} = \frac{1}{2} - \frac{1}{3}j
$$
da cui l'espressione in funzione di $t$:
$$
i(t) = \left( \left( \frac{1}{2} + \frac{1}{3}j \right) e^{-3j t} + \left( \frac{1}{2} - \frac{1}{3}j \right) e^{3jt} \right) u(t)
$$
dove notiamo che anche i coefficienti, oltre che gli esponenti, risultano numeri complessi.

Abbiamo quindi che incontreremo generalmente soluzioni in forma:
$$
i(t) = \left( M + jN \right) e^{-(\sigma + j \omega)t} + \left( M - jN \right) e^{-(\sigma - j \omega)t}
$$
da cui:
$$
= M e^{-\sigma t}e^{-j \omega t} + j N e^{-\sigma t}e^{-j \omega t} + M e^{-\sigma t}e^{j \omega t} - jN e^{-\sigma t}e^{j \omega t}
$$
$$
= M \left( e^{- \sigma t}e^{-j \omega t} + e^{- \sigma t}e^{j \omega t} \right) + j N  \left( e^{-\sigma t}e^{-j \omega t} - e^{-\sigma t}e^{j \omega t} \right)
$$
$$
= M e^{- \sigma t} \left( e^{-j \omega t} + e^{j \omega t} \right) + j N e^{-\sigma t} \left( e^{-j \omega t} - e^{j \omega t} \right)
$$
visto che:
\[
	\begin{cases}
		e^{j \omega t} = \cos(\omega t) + i \sin(\omega t) \\ 	
		e^{-j \omega t} = \cos(\omega t) - i \sin(\omega t) \\ 	
	\end{cases}
\]
si ha:
$$
i(t) = M e^{-\sigma t} ( \cos(\omega t) + i \sin(\omega t) + \cos(\omega t) - i \sin(\omega t) ) 
$$
$$
+ j N e^{-\sigma t} ( \cos(\omega t) - i \sin(\omega t) - \cos(\omega t) - i \sin(\omega t) )
$$
$$
= 2 M e^{-\sigma t} \cos{\omega t} + 2 N e^{-\sigma t} \sin(\omega t)
$$

Vogliamo riportare questa forma nella più concisa $ke^{-\sigma t} \sin(\omega t + \alpha)$. Usiamo allora le formule di traduzione in forma sinusoidale, di cui una dimostrazione nel caso cosinusoidale si trova a \url{https://github.com/seggiani-luca/appunti-fis/blob/main/master/master.pdf}:
\[
	c_1 \sin{\omega t} + c_2 \cos{\omega t}
	\Leftrightarrow
	k \sin(\omega t + \alpha)
\]
con:
\[
	\begin{cases}
		k = \sqrt{c_1^2 + c_2^2} \\ 
		\alpha = \tan^{-1}(\frac{c1}{c2})
	\end{cases}
\]

da cui diciamo:
$$
i(t) = 2 M e^{-\sigma t} \cos{\omega t} + 2 N e^{-\sigma t} \sin(\omega t)
$$
$$
= 2 e^{-\sigma t} \left( M \cos{\omega t} + N \sin{\omega t} \right)
= ke^{-\sigma t} \sin(\omega t + \alpha)
$$
con:
\[
	\begin{cases}
		k = 2\sqrt{M^2 + N^2} \\ 
		\alpha = \tan^{-1}\left(\frac{M}{N}\right)
	\end{cases}
\]

\end{document}


\documentclass[a4paper,11pt]{article}
\usepackage[a4paper, margin=8em]{geometry}

% usa i pacchetti per la scrittura in italiano
\usepackage[french,italian]{babel}
\usepackage[T1]{fontenc}
\usepackage[utf8]{inputenc}
\frenchspacing 

% usa i pacchetti per la formattazione matematica
\usepackage{amsmath, amssymb, amsthm, amsfonts}

% usa altri pacchetti
\usepackage{gensymb}
\usepackage{hyperref}
\usepackage{standalone}

% imposta il titolo
\title{Appunti Elettrotecnica}
\author{Luca Seggiani}
\date{2024}

% imposta lo stile
% usa helvetica
\usepackage[scaled]{helvet}
% usa palatino
\usepackage{palatino}
% usa un font monospazio guardabile
\usepackage{lmodern}

\renewcommand{\rmdefault}{ppl}
\renewcommand{\sfdefault}{phv}
\renewcommand{\ttdefault}{lmtt}

% disponi il titolo
\makeatletter
\renewcommand{\maketitle} {
	\begin{center} 
		\begin{minipage}[t]{.8\textwidth}
			\textsf{\huge\bfseries \@title} 
		\end{minipage}%
		\begin{minipage}[t]{.2\textwidth}
			\raggedleft \vspace{-1.65em}
			\textsf{\small \@author} \vfill
			\textsf{\small \@date}
		\end{minipage}
		\par
	\end{center}

	\thispagestyle{empty}
	\pagestyle{fancy}
}
\makeatother

% disponi teoremi
\usepackage{tcolorbox}
\newtcolorbox[auto counter, number within=section]{theorem}[2][]{%
	colback=blue!10, 
	colframe=blue!40!black, 
	sharp corners=northwest,
	fonttitle=\sffamily\bfseries, 
	title=~\thetcbcounter: #2, 
	#1
}

% disponi definizioni
\newtcolorbox[auto counter, number within=section]{definition}[2][]{%
	colback=red!10,
	colframe=red!40!black,
	sharp corners=northwest,
	fonttitle=\sffamily\bfseries,
	title=~\thetcbcounter: #2,
	#1
}

% U.D.M
\newcommand{\amp}{\ensuremath{\mathrm{A}}}
\newcommand{\volt}{\ensuremath{\mathrm{V}}}
\newcommand{\meter}{\ensuremath{\mathrm{m}}}
\newcommand{\second}{\ensuremath{\mathrm{s}}}
\newcommand{\farad}{\ensuremath{\mathrm{F}}}
\newcommand{\henry}{\ensuremath{\mathrm{H}}}
\newcommand{\siemens}{\ensuremath{\mathrm{S}}}

% circuiti
\usepackage{circuitikz}
\usetikzlibrary{babel}

% disegni
\usepackage{pgfplots}
\pgfplotsset{width=10cm,compat=1.9}

% disponi codice
\usepackage{listings}
\usepackage[table]{xcolor}

\lstdefinestyle{codestyle}{
		backgroundcolor=\color{black!5}, 
		commentstyle=\color{codegreen},
		keywordstyle=\bfseries\color{magenta},
		numberstyle=\sffamily\tiny\color{black!60},
		stringstyle=\color{green!50!black},
		basicstyle=\ttfamily\footnotesize,
		breakatwhitespace=false,         
		breaklines=true,                 
		captionpos=b,                    
		keepspaces=true,                 
		numbers=left,                    
		numbersep=5pt,                  
		showspaces=false,                
		showstringspaces=false,
		showtabs=false,                  
		tabsize=2
}

\lstdefinestyle{shellstyle}{
		backgroundcolor=\color{black!5}, 
		basicstyle=\ttfamily\footnotesize\color{black}, 
		commentstyle=\color{black}, 
		keywordstyle=\color{black},
		numberstyle=\color{black!5},
		stringstyle=\color{black}, 
		showspaces=false,
		showstringspaces=false, 
		showtabs=false, 
		tabsize=2, 
		numbers=none, 
		breaklines=true
}

\lstdefinelanguage{javascript}{
	keywords={typeof, new, true, false, catch, function, return, null, catch, switch, var, if, in, while, do, else, case, break},
	keywordstyle=\color{blue}\bfseries,
	ndkeywords={class, export, boolean, throw, implements, import, this},
	ndkeywordstyle=\color{darkgray}\bfseries,
	identifierstyle=\color{black},
	sensitive=false,
	comment=[l]{//},
	morecomment=[s]{/*}{*/},
	commentstyle=\color{purple}\ttfamily,
	stringstyle=\color{red}\ttfamily,
	morestring=[b]',
	morestring=[b]"
}

% disponi sezioni
\usepackage{titlesec}

\titleformat{\section}
	{\sffamily\Large\bfseries} 
	{\thesection}{1em}{} 
\titleformat{\subsection}
	{\sffamily\large\bfseries}   
	{\thesubsection}{1em}{} 
\titleformat{\subsubsection}
	{\sffamily\normalsize\bfseries} 
	{\thesubsubsection}{1em}{}

% disponi alberi
\usepackage{forest}

\forestset{
	rectstyle/.style={
		for tree={rectangle,draw,font=\large\sffamily}
	},
	roundstyle/.style={
		for tree={circle,draw,font=\large}
	}
}

% disponi algoritmi
\usepackage{algorithm}
\usepackage{algorithmic}
\makeatletter
\renewcommand{\ALG@name}{Algoritmo}
\makeatother

% disponi numeri di pagina
\usepackage{fancyhdr}
\fancyhf{} 
\fancyfoot[L]{\sffamily{\thepage}}

\makeatletter
\fancyhead[L]{\raisebox{1ex}[0pt][0pt]{\sffamily{\@title \ \@date}}} 
\fancyhead[R]{\raisebox{1ex}[0pt][0pt]{\sffamily{\@author}}}
\makeatother

\begin{document}
% sezione (data)
\section{Lezione del 28-11-24}

% stili pagina
\thispagestyle{empty}
\pagestyle{fancy}

% testo
\subsection{Antitrasformata con poli multipli}
Vediamo un'altra particolarità del processo di antitrasformazione, in particolare nel caso di \textbf{poli multipli}, cioè poli che compaiono con molteplicità $\geq 2$ al denominatore.
Prendiamo ad esempio la forma rapporto di polinomi:
$$
I(s) = \frac{s^2 + 2s + 3}{(s+1)^3}
$$
Avremo che, per esprimere i residui, non basterà il rapporto su $s+1$ al primo grado, ma servirà bensì:
$$
I(s) = \frac{A_1}{s + 1} + \frac{A_2}{(s+1)^2} + \frac{A_3}{(s+1)^2}
$$

Vorremo quindi portare le soluzioni con poli multipli nella forma generale:
$$
I(s) = \frac{A_1}{(s-p)} + \frac{A_2}{(s-p)^2} + ... + \frac{A_n}{(s-p)^n} = \sum_{i=1}^n \frac{A_i}{(s-p)^i}
$$
su cui si userà agevolmente la regola di antitrasformazione:
$$
\mathcal{L}^{-1} \left\{ \frac{A_i}{(s-p)^i} \right\} = \frac{A_i}{(i - 1)!} t^{i-1} e^{pt}
$$
da cui quindi:
$$
\mathcal{L}^{-1}(I(s)) = A_1 e^{pt} + \frac{A_2}{2} t e^{pt} + ... + \frac{A_n}{(n-1)!} t^{n-1} e^{pt} = \sum\limits_{i=1}^n  \frac{A_i}{(i-1)!} t^{i - 1} e^{pt}
$$ 

A questo punto basterà calcolare i residui, cosa che potremo fare applicando il \textbf{teorema dei residui ai poli multipli}:
$$
A_i = \lim_{s \rightarrow p} \frac{1}{(n - 1)!} \frac{\partial^{n-i} \, (s-p)^n \cdot I(s)}{\partial s^{n-i}}
$$

Riprendendo l'esempio:
$$
A_1 = \lim_{s \rightarrow -1} \frac{1}{2!} \frac{\partial^2 (s+1)^3 \frac{s^2+2s+3}{(s+1)^3} }{\partial s^2} = \lim_{s \rightarrow -1} \frac{1}{2} \frac{\partial^2}{\partial s^2} (s^2 + 2s + 3) = \lim_{s \rightarrow -1} \frac{1}{2} 2 = 1
$$
$$
A_2 = \lim_{s \rightarrow -1} \frac{1}{1!} \frac{\partial (s+1)^3 \frac{s^2+2s+3}{(s+1)^3} }{\partial s} = \lim_{s \rightarrow -1} 1 \cdot (2s + 2) = (2s + 2) \Big|_{s = -1} = 0
$$
$$
A_3 = \lim_{s \rightarrow -1} \frac{1}{0!} \frac{\partial^0 (s+1)^3 \frac{s^2+2s+3}{(s+1)^3} }{\partial s^0}
$$
che assumendo $\frac{\partial^0}{\partial s^0} = 1$ significa:
$$
A_3 = \lim_{s \rightarrow -1} = 1 \cdot (s^2 +2s +3) \Big|_{s=-1} = 2
$$

Si ha quindi il risultato finale:
$$
i(t) = \left( e^{-t} + t^2 e^{-t} \right) u(t)
$$


\end{document}

\end{document}