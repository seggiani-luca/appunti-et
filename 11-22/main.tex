
\documentclass[a4paper,11pt]{article}
\usepackage[a4paper, margin=8em]{geometry}

% usa i pacchetti per la scrittura in italiano
\usepackage[french,italian]{babel}
\usepackage[T1]{fontenc}
\usepackage[utf8]{inputenc}
\frenchspacing 

% usa i pacchetti per la formattazione matematica
\usepackage{amsmath, amssymb, amsthm, amsfonts}

% usa altri pacchetti
\usepackage{gensymb}
\usepackage{hyperref}
\usepackage{standalone}

% imposta il titolo
\title{Appunti Elettrotecnica}
\author{Luca Seggiani}
\date{2024}

% imposta lo stile
% usa helvetica
\usepackage[scaled]{helvet}
% usa palatino
\usepackage{palatino}
% usa un font monospazio guardabile
\usepackage{lmodern}

\renewcommand{\rmdefault}{ppl}
\renewcommand{\sfdefault}{phv}
\renewcommand{\ttdefault}{lmtt}

% disponi il titolo
\makeatletter
\renewcommand{\maketitle} {
	\begin{center} 
		\begin{minipage}[t]{.8\textwidth}
			\textsf{\huge\bfseries \@title} 
		\end{minipage}%
		\begin{minipage}[t]{.2\textwidth}
			\raggedleft \vspace{-1.65em}
			\textsf{\small \@author} \vfill
			\textsf{\small \@date}
		\end{minipage}
		\par
	\end{center}

	\thispagestyle{empty}
	\pagestyle{fancy}
}
\makeatother

% disponi teoremi
\usepackage{tcolorbox}
\newtcolorbox[auto counter, number within=section]{theorem}[2][]{%
	colback=blue!10, 
	colframe=blue!40!black, 
	sharp corners=northwest,
	fonttitle=\sffamily\bfseries, 
	title=~\thetcbcounter: #2, 
	#1
}

% disponi definizioni
\newtcolorbox[auto counter, number within=section]{definition}[2][]{%
	colback=red!10,
	colframe=red!40!black,
	sharp corners=northwest,
	fonttitle=\sffamily\bfseries,
	title=~\thetcbcounter: #2,
	#1
}

% U.D.M
\newcommand{\amp}{\ensuremath{\mathrm{A}}}
\newcommand{\volt}{\ensuremath{\mathrm{V}}}
\newcommand{\meter}{\ensuremath{\mathrm{m}}}
\newcommand{\second}{\ensuremath{\mathrm{s}}}
\newcommand{\farad}{\ensuremath{\mathrm{F}}}
\newcommand{\henry}{\ensuremath{\mathrm{H}}}
\newcommand{\siemens}{\ensuremath{\mathrm{S}}}

% circuiti
\usepackage{circuitikz}
\usetikzlibrary{babel}

% disegni
\usepackage{pgfplots}
\pgfplotsset{width=10cm,compat=1.9}

% disponi codice
\usepackage{listings}
\usepackage[table]{xcolor}

\lstdefinestyle{codestyle}{
		backgroundcolor=\color{black!5}, 
		commentstyle=\color{codegreen},
		keywordstyle=\bfseries\color{magenta},
		numberstyle=\sffamily\tiny\color{black!60},
		stringstyle=\color{green!50!black},
		basicstyle=\ttfamily\footnotesize,
		breakatwhitespace=false,         
		breaklines=true,                 
		captionpos=b,                    
		keepspaces=true,                 
		numbers=left,                    
		numbersep=5pt,                  
		showspaces=false,                
		showstringspaces=false,
		showtabs=false,                  
		tabsize=2
}

\lstdefinestyle{shellstyle}{
		backgroundcolor=\color{black!5}, 
		basicstyle=\ttfamily\footnotesize\color{black}, 
		commentstyle=\color{black}, 
		keywordstyle=\color{black},
		numberstyle=\color{black!5},
		stringstyle=\color{black}, 
		showspaces=false,
		showstringspaces=false, 
		showtabs=false, 
		tabsize=2, 
		numbers=none, 
		breaklines=true
}

\lstdefinelanguage{javascript}{
	keywords={typeof, new, true, false, catch, function, return, null, catch, switch, var, if, in, while, do, else, case, break},
	keywordstyle=\color{blue}\bfseries,
	ndkeywords={class, export, boolean, throw, implements, import, this},
	ndkeywordstyle=\color{darkgray}\bfseries,
	identifierstyle=\color{black},
	sensitive=false,
	comment=[l]{//},
	morecomment=[s]{/*}{*/},
	commentstyle=\color{purple}\ttfamily,
	stringstyle=\color{red}\ttfamily,
	morestring=[b]',
	morestring=[b]"
}

% disponi sezioni
\usepackage{titlesec}

\titleformat{\section}
	{\sffamily\Large\bfseries} 
	{\thesection}{1em}{} 
\titleformat{\subsection}
	{\sffamily\large\bfseries}   
	{\thesubsection}{1em}{} 
\titleformat{\subsubsection}
	{\sffamily\normalsize\bfseries} 
	{\thesubsubsection}{1em}{}

% disponi alberi
\usepackage{forest}

\forestset{
	rectstyle/.style={
		for tree={rectangle,draw,font=\large\sffamily}
	},
	roundstyle/.style={
		for tree={circle,draw,font=\large}
	}
}

% disponi algoritmi
\usepackage{algorithm}
\usepackage{algorithmic}
\makeatletter
\renewcommand{\ALG@name}{Algoritmo}
\makeatother

% disponi numeri di pagina
\usepackage{fancyhdr}
\fancyhf{} 
\fancyfoot[L]{\sffamily{\thepage}}

\makeatletter
\fancyhead[L]{\raisebox{1ex}[0pt][0pt]{\sffamily{\@title \ \@date}}} 
\fancyhead[R]{\raisebox{1ex}[0pt][0pt]{\sffamily{\@author}}}
\makeatother

\begin{document}
% sezione (data)
\section{Lezione del 22-11-24}

% stili pagina
\thispagestyle{empty}
\pagestyle{fancy}

% testo
\subsection{Leggi dei bipoli con la trasformata di Laplace}
Vediamo quindi come possiamo esprimere il legame fra corrente e tensione dei bipoli visti finora attraverso la trasformata di Laplace.
\subsubsection{Resistori}
Poniamo di avere un resistore. Avevamo che:
$$
v_R(t) = R i_R(t)
$$
Visto che non abbiamo derivate o integrali, possiamo passare direttamente al dominio $s$:
$$
V_R(s) = R I_R(s)
$$

\subsubsection{Induttori}
Rispetto agli induttori, avevamo che:
$$
v_L(t) = L\frac{di_L(t)}{dt}
$$

che nella trasformata di Laplace, applicando la legge di derivata, diventerà:
$$
V_L(s) = L \left( s I_L(s) - i_L(0^-) \right) = s L I_L(s) - L i_L (0^-)
$$
dove appare un termine dovuto alle condizioni iniziali, cioè $L i_L (0^-)$. 

Abbiamo quindi che un circuito equivalente all'induttore secondo la trasformata di Laplace è dato da una serie fra un induttore con induttanza generalizzata $sL$ e un generatore di tensione (detto \textbf{generatore di condizioni iniziali}) $L i_L(0^-)$, cioè:

\begin{center}
	\begin{tikzpicture}
		\draw (0,0) to [ inductor, l=$sL$] (2, 0)
			to [ voltage source, v=$L i_L(0^-)$] (4, 0);
	\end{tikzpicture}
\end{center}

\subsubsection{Condensatori}
Rispetto ai condensatori, avevamo che:
$$
v_C(t) = \frac{1}{C} \int_{-\infty}^t i_C(\tau) d\tau
$$

che nella trasformata di Laplace, applicando la legge di integrale, diventerà:
$$
V_C(s) = \frac{1}{C} \left( \frac{I_C(s)}{s} + \frac{1}{s}\int_{-\infty}^{0^-}i_C(t)dt \right) = \frac{I_C(s)}{sC} + \frac{1}{sC} q(0^-)
$$
ricordando che $\int_{-\infty}^{0^-} i(t) dt = q(t)$.
Potrebbe però essere più conveniente applicare la definizione di capacità, da cui $v_C(t) = \frac{q(t)}{C}$ e quindi:
$$
V_C(s) = \frac{I_C(s)}{sC} + \frac{v_C(0^-)}{s}
$$
cioè anche qui appare un termine dovuto alle condizioni iniziali,$ \frac{v_C(0^-)}{s}$.

Abbiamo quindi che un circuito equivalente al condensatore secondo la trasformata di Laplace è dato da una serie fra un condensatore $\frac{1}{sC}$ e un generatore di tensione (il generatore di condizioni iniziali) $\frac{v_C(0^-)}{s}$, cioè:

\begin{center}
	\begin{tikzpicture}
		\draw (0,0) to [ capacitor, l=$\frac{1}{sC}$] (2, 0)
			to [ voltage source, v<=$\frac{v_C(0^-)}{s}$] (4, 0);
	\end{tikzpicture}
\end{center}

\subsubsection{Induttanze mutuamente accoppiate}
Due induttanze mutuamente accoppiate venivano governate dalla legge:
\[
	\begin{cases}			
		v_1(t) = L_1 \frac{d i_1(t)}{dt} \pm M \frac{d i_2(t)}{dt} \\
		v_2(t) = L_2 \frac{d i_2(t)}{dt} \pm M \frac{d i_1(t)}{dt}
	\end{cases}
\]

Nel dominio $s$, queste diventano:
\[
	\begin{cases}
		V_1(s) = L_1 \left( s I_1(s) - i_1(0^-) \right) \pm M \left( s I_2(s) - i_2(0^-) \right) \\
		V_2(s) = L_2 \left( s I_2(s) - i_2(0^-) \right) \pm M \left( s I_1(s) - i_1(0^-) \right)
	\end{cases}
\]
che dà:
\[
	\begin{cases}
		V_1(s) = s L_1 I_1(s) \pm s M I_2(s) - L_1 i_1(0^-) \mp M i_2(0^-) \\
		V_2(s) = s L_2 I_2(s) \pm s M I_1(s) - L_2 i_2(0^-) \mp M i_1(0^-)
	\end{cases}
\]

Abbiamo allora che un circuito equivalente alle induttanze mutuamente accoppiate secondo la trasformata di Laplace è dato da, su ogni ramo, una serie di:
\begin{itemize}
	\item Un generatore di tensione $L_1 i_1(0^-)$ ($L_2 i_2 (0^-))$;
	\item Un generatore di tensione $\pm M i_2(0^-)$ ($\pm M i_1(0^-)$)
\end{itemize}
entrambi con contrassegni concordi alla direzione della corrente:

\begin{center}
	\begin{circuitikz}
		\draw (0,0) to[ short, i=$i_1$] (1,0)
			to[ inductor , l=$sL_1$] (1,-2)
			to[ voltage source, v_=$M i_2(0^-)$] (1, -4)
			to[ voltage source, v_=$L_1 i_1(0^-)$] (1, -6)
			-- (0, -6);

		\draw (4,0) to[ short, i_=$i_2$] (3,0)
			to[ inductor , l_=$sL_2$] (3,-2)
			to[ voltage source, v=$M i_1(0^-)$] (3, -4)
			to[ voltage source, v=$L_2 i_2(0^-)$] (3, -6)
			-- (4, -6);

			\draw (0.9,-0.6) node {$\scriptscriptstyle\bullet$};
			\draw (2.9,-0.6) node {$\scriptscriptstyle\bullet$};
	\end{circuitikz}
\end{center}


\subsection{Analisi circuitale nel dominio s}
Vediamo quindi come usare la trasformata di Laplace per risolvere il circuito (RL) che abbiamo usato per introdurre i circuiti a regime aperiodico:

\begin{center}
	\begin{tikzpicture}
		\draw (0,0) to [ voltage source,  l=$e(t)$] (0,3)
			to [ resistor, l=$R$] (3,3)
			to [ inductor, l=$L$] (3,0)
			-- (0,0);
	\end{tikzpicture}
\end{center}

Dove avevamo che il generatore di tensione $e(t)$ ha forma d'onda:
\[
	e(t) =
	\begin{cases}
		E_0, \quad t < 0 \\ 
		2E_0, \quad t \geq 0
	\end{cases}
\]
di cui si riporta un grafico:
\begin{center}
\begin{tikzpicture}
    \begin{axis}[
        xlabel={$t$},
        ylabel={$e(t)$},
        domain=-10:10, % set the x range you want,
				samples=100,
        grid=major, % add a grid
				ytick={1, 2},
				yticklabels={$E_0$, $2 E_0$},
				xtick={0},
				xticklabels={$0$},
				width=15cm,
				height=7cm
    ]
    \addplot[
        red,
        thick,
				domain=-10:0
    ] {1};

    \addplot[
        red,
        thick,
				domain=0:10
    ] {2}; 
    \end{axis}
\end{tikzpicture}
\end{center}

Iniziamo la risoluzione.
Dovremmo dividere il problema in intervalli temporali:
\begin{itemize}
	\item $t < 0$: qui sarà incognita, oltre a $i(t)$, anche la condizione iniziale $i_L(0^-)$.
		Appare quindi chiaro come mai abbiamo usato $0^-$ nelle formule: le condizioni iniziali che ci interessano sono quelle \textit{un attimo prima} della transizione, e non dopo.

		La corrente $i_L(0^-)$ sull'induttore varrà quindi quanto quella nel circuito a regime costante, ergo $i_L(0^-) = \frac{E_0}{R}$.
	\item $t \geq 0$: ci troviamo sul transitorio, e dobbiamo quindi usare Laplace.
		Trasformiamo l'induttanza nel circuito equivalente col generatore di condizioni iniziali:

\begin{center}
	\begin{tikzpicture}
		\draw (0,0) to [ voltage source,  v=$2 \frac{E_0}{s}$] (0,3)
			to [ resistor, l=$R$] (3,3)
			to [ inductor, l=$sL$] (3,1.5)
			to [ voltage source, v=$L \frac{E_0}{R}$] (3,0)
			-- (0,0);
	\end{tikzpicture}
\end{center}

A questo punto l'incognita sarà la corrente $i_x(t)$ sul circuito, che potremo trovare con la seconda legge di Kirchoff:
$$
-2 \frac{E_0}{s} + R I_x(s) + s L I_x(s) - L \frac{E_0}{R} = 0 \Rightarrow I_x(s) \frac{L \frac{E_0}{R} + 2 \frac{E_0}{s}}{r + s L} = \frac{L E_0 s + 2 R E_0}{R L s^2 + R^2 s}
$$

Infine, vorremo riportare l'espressione ottenuta nel dominio del tempo, in quanto quello che abbiamo trovato è effettivamente:
$$I_x(s) = \mathcal{L}\{ i_x(t) \}$$
Per fare questo abbiamo bisogno dell'\textbf{antitrasformata} di Laplace.
Vediamo come si ricava.
\end{itemize}

\subsection{Antitrasformata di Laplace}
Data una forma $I_x(s) = \mathcal{L}\{ i_x(t) \}$, vediamo come calcolare $ i_x(t) = \mathcal{L}^{-1} \{ I_x(t) \}$.
Facciamo l'esempio con la trasformata:
$$
I(s) = \frac{s^2 + 5}{3s^3 + 9s^2 + 6s}
$$

Innanzitutto vorremo riportare il denominatore in forma \textit{normalizzata}, cioè esprimere il termine di grado più alto senza coefficienti:
$$
I(s) = \frac{\frac{s^2}{3} + \frac{5}{3}}{s^3 + 3s^2 + 2s}
$$

A questo punto vorremo \textbf{fattorizzare} il denominatore:
$$
I(s) = \frac{\frac{s^2}{3} + \frac{5}{3}}{s (s^2 + 3s +1)} = \frac{\frac{s^2}{3} + \frac{5}{3}}{s (s + 1) (s + 2)}
$$

Poniamo quindi:
$$
I(s) = \frac{A_1}{s} + \frac{A_2}{s + 1} + \frac{A_3}{s+2}
$$
che ci permetterà di sfruttare il fatto che $\mathcal{L}^-1\left\{ \frac{1}{s + a} \right\} = e^{-at}$ e $\mathcal{L}^{-1} \left\{ \frac{1}{s} \right\} = u(t)$ (gradino di Heaviside), e quindi dire che:
$$
i(t) = \left( A_1 + A_2 e^{-t} + A_3 e^{-2t} \right) u(t)
$$

Questo equivale a riportare la frazione trovata in \textbf{fratti semplici}, da cui si ottiene:
$$
A_1 = \frac{5}{6}, \quad A_2 = -2, \quad A_3 = \frac{3}{2} 
$$
e quindi:
$$
i(t) = \left( \frac{5}{6} -2e^{-t} + \frac{3}{2}e^{-2t} \right) u(t)
$$

Vediamo un metodo rapido per il calcolo dei fratti semplici.

\subsubsection{Teorema dei residui}
Potremo dire che: # introducilo non l'ho mai sentito
$$
I(s) = \sum_{i = 1}^n \frac{A_i}{s - P_i}
$$
e quindi:
$$
A_i = \lim_{s \rightarrow P_i} (s - P_i) I(s)
$$

Ad esempio, applichiamo sull'esempio precedente:
$$
A_1 = \lim_{s \rightarrow 0} s \cdot \frac{\frac{s^2}{3} + \frac{5}{3}}{s(s+1)(s+2)}
= \frac{\frac{s^2}{3} + \frac{5}{3}}{(s+1)(s+2)} = \frac{\frac{5}{3}}{3} = \frac{5}{6}
$$
e via dicendo.

\par\smallskip

Risolviamo quindi l'ultimo passaggio dell'esercizio precendente.
Avevamo ricavato la trasformata:
$$
I_x(s) = \frac{L E_0 s + 2 R E_0}{R L s^2 + R^2 s}
$$

Normalizzando, si ha:
$$
I_x(s) = \frac{\frac{E_0}{R} s + 2 \frac{E_0}{L}}{s^2 + \frac{R}{L}s} = \frac{\frac{E_0}{R} s + 2 \frac{E_0}{L}}{s \left(s + \frac{R}{L}\right)}
$$

A questo punto possiamo impostare i fratti semplici:
$$
I(s) = \frac{A_1}{s} + \frac{A_2}{s + \frac{R}{L}}
$$
e risolvere, ad esempio col metodo dei residui:

Si ritrova quindi la funzione di $t$:
$$
i(t) = \left( 2\frac{E_0}{R} - \frac{E_0}{R} e^{-\frac{R}{L}t} \right) u(t) = \frac{E_0}{R} \left( 2 - e^{-\frac{R}{L}t} \right) u(t)
$$
che equivale a quanto avevamo trovato risolvendo direttamente l'equazione differenziale, salvo il termine $u(t)$ per il gradino di Heaviside, che però non ci interessa in quanto avremmo preso comunque il transiente da $t = 0$ in poi, cioè combinando le soluzioni:
\[
	i(t) = 
	\begin{cases}
		\frac{E_0}{R}, \quad t < 0 \\ 
		\frac{E_0}{R} \left( 2 - e^{-\frac{R}{L}t} \right), \quad t \geq 0
	\end{cases}
\]

\subsubsection{Verifica dei risultati}
Potremmo voler fare una verifica della validità dei risultati ottenuti dopo la risoluzione di un circuito con la trasformata di Laplace.
Una prima verifica potrebbe essere considerare il limite:
$$
\lim_{t\rightarrow\infty} i(t)
$$
cioè, dopo un tempo $t >> 0$, il circuito tende a un regime stazionario che corrisponde con quello che avremmo prendendo il circuito a corrente continua? 
Nel caso precedente, notiamo come la corrente $i(t)$ tende giustamente a $\frac{2 E_0}{R}$.

Una verifica più sofisticata si può fare considerando le \textbf{variabili di stato} del circuito, cioè quei valori che non possono variare in maniera discontinua. 
Ad esempio, nell'esempio precedente, una variabile di stato è l'energia nell'induttore, cioè:
$$
W_L(t) = \frac{1}{2}L i_L^2(t)
$$

Questa non potrà mai essere discontinua, in quanto implicherebbe derivata infinità sulla discontinuità, e visto che $\frac{d W_L(t)}{dt} = P$ potenza, avremmo dal teorema di Boucherot che da qualche parte nel circuito erogherebbe (seppur istantaneamente) una potenza infinita, che è impossibile.

Verifichiamo quindi la continuità:
$$
\lim_{t \rightarrow 0^-} i_L(t) =^? \lim_{t \rightarrow 0^+} i_L(t)
$$
che nell'esempio è verificata da:
$$
\lim_{t \rightarrow 0^-} i_L(t) = \lim_{t \rightarrow 0^+} i_L(t) = \frac{E_0}{R}
$$

\end{document}
