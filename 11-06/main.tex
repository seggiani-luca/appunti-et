
\documentclass[a4paper,11pt]{article}
\usepackage[a4paper, margin=8em]{geometry}

% usa i pacchetti per la scrittura in italiano
\usepackage[french,italian]{babel}
\usepackage[T1]{fontenc}
\usepackage[utf8]{inputenc}
\frenchspacing 

% usa i pacchetti per la formattazione matematica
\usepackage{amsmath, amssymb, amsthm, amsfonts}

% usa altri pacchetti
\usepackage{gensymb}
\usepackage{hyperref}
\usepackage{standalone}

% imposta il titolo
\title{Appunti Elettrotecnica}
\author{Luca Seggiani}
\date{2024}

% imposta lo stile
% usa helvetica
\usepackage[scaled]{helvet}
% usa palatino
\usepackage{palatino}
% usa un font monospazio guardabile
\usepackage{lmodern}

\renewcommand{\rmdefault}{ppl}
\renewcommand{\sfdefault}{phv}
\renewcommand{\ttdefault}{lmtt}

% disponi il titolo
\makeatletter
\renewcommand{\maketitle} {
	\begin{center} 
		\begin{minipage}[t]{.8\textwidth}
			\textsf{\huge\bfseries \@title} 
		\end{minipage}%
		\begin{minipage}[t]{.2\textwidth}
			\raggedleft \vspace{-1.65em}
			\textsf{\small \@author} \vfill
			\textsf{\small \@date}
		\end{minipage}
		\par
	\end{center}

	\thispagestyle{empty}
	\pagestyle{fancy}
}
\makeatother

% disponi teoremi
\usepackage{tcolorbox}
\newtcolorbox[auto counter, number within=section]{theorem}[2][]{%
	colback=blue!10, 
	colframe=blue!40!black, 
	sharp corners=northwest,
	fonttitle=\sffamily\bfseries, 
	title=~\thetcbcounter: #2, 
	#1
}

% disponi definizioni
\newtcolorbox[auto counter, number within=section]{definition}[2][]{%
	colback=red!10,
	colframe=red!40!black,
	sharp corners=northwest,
	fonttitle=\sffamily\bfseries,
	title=~\thetcbcounter: #2,
	#1
}

% U.D.M
\newcommand{\amp}{\ensuremath{\mathrm{A}}}
\newcommand{\volt}{\ensuremath{\mathrm{V}}}
\newcommand{\meter}{\ensuremath{\mathrm{m}}}
\newcommand{\second}{\ensuremath{\mathrm{s}}}
\newcommand{\farad}{\ensuremath{\mathrm{F}}}
\newcommand{\henry}{\ensuremath{\mathrm{H}}}
\newcommand{\siemens}{\ensuremath{\mathrm{S}}}

% circuiti
\usepackage{circuitikz}
\usetikzlibrary{babel}

% disegni
\usepackage{pgfplots}
\pgfplotsset{width=10cm,compat=1.9}

% disponi codice
\usepackage{listings}
\usepackage[table]{xcolor}

\lstdefinestyle{codestyle}{
		backgroundcolor=\color{black!5}, 
		commentstyle=\color{codegreen},
		keywordstyle=\bfseries\color{magenta},
		numberstyle=\sffamily\tiny\color{black!60},
		stringstyle=\color{green!50!black},
		basicstyle=\ttfamily\footnotesize,
		breakatwhitespace=false,         
		breaklines=true,                 
		captionpos=b,                    
		keepspaces=true,                 
		numbers=left,                    
		numbersep=5pt,                  
		showspaces=false,                
		showstringspaces=false,
		showtabs=false,                  
		tabsize=2
}

\lstdefinestyle{shellstyle}{
		backgroundcolor=\color{black!5}, 
		basicstyle=\ttfamily\footnotesize\color{black}, 
		commentstyle=\color{black}, 
		keywordstyle=\color{black},
		numberstyle=\color{black!5},
		stringstyle=\color{black}, 
		showspaces=false,
		showstringspaces=false, 
		showtabs=false, 
		tabsize=2, 
		numbers=none, 
		breaklines=true
}

\lstdefinelanguage{javascript}{
	keywords={typeof, new, true, false, catch, function, return, null, catch, switch, var, if, in, while, do, else, case, break},
	keywordstyle=\color{blue}\bfseries,
	ndkeywords={class, export, boolean, throw, implements, import, this},
	ndkeywordstyle=\color{darkgray}\bfseries,
	identifierstyle=\color{black},
	sensitive=false,
	comment=[l]{//},
	morecomment=[s]{/*}{*/},
	commentstyle=\color{purple}\ttfamily,
	stringstyle=\color{red}\ttfamily,
	morestring=[b]',
	morestring=[b]"
}

% disponi sezioni
\usepackage{titlesec}

\titleformat{\section}
	{\sffamily\Large\bfseries} 
	{\thesection}{1em}{} 
\titleformat{\subsection}
	{\sffamily\large\bfseries}   
	{\thesubsection}{1em}{} 
\titleformat{\subsubsection}
	{\sffamily\normalsize\bfseries} 
	{\thesubsubsection}{1em}{}

% disponi alberi
\usepackage{forest}

\forestset{
	rectstyle/.style={
		for tree={rectangle,draw,font=\large\sffamily}
	},
	roundstyle/.style={
		for tree={circle,draw,font=\large}
	}
}

% disponi algoritmi
\usepackage{algorithm}
\usepackage{algorithmic}
\makeatletter
\renewcommand{\ALG@name}{Algoritmo}
\makeatother

% disponi numeri di pagina
\usepackage{fancyhdr}
\fancyhf{} 
\fancyfoot[L]{\sffamily{\thepage}}

\makeatletter
\fancyhead[L]{\raisebox{1ex}[0pt][0pt]{\sffamily{\@title \ \@date}}} 
\fancyhead[R]{\raisebox{1ex}[0pt][0pt]{\sffamily{\@author}}}
\makeatother

\begin{document}
% sezione (data)
\section{Lezione del 06-11-24}

% stili pagina
\thispagestyle{empty}
\pagestyle{fancy}

% testo
\subsection{Potenza su induttori mutuamente accoppiati}
Abbiamo visto che i componenti come i resistori, hanno potenza \textbf{attiva}, cioè contribuiscono alla parte reale della potenza complessa, mentre componenti come condensatori e induttori hanno potenza \textbf{reattiva}, cioè contribuiscono alla parte complessa della potenza complessa.
Vediamo adesso se la potenza su due induttori mutuamente accoppiati è attiva o reattiva.

Iniziamo col dire che la formula che conoscevamo per la mutua induttanza, assunte correnti concordi sui contrassegni, cioè:
$$
\dot(V) = j \omega L_1 \dot{I_1} + j \omega M \dot{I_2}
$$
può anche esprimersi attraverso l'ammettenza $\overline{Z}$, con $\dot{V} = \overline{Z} \dot{I}$, cioè:
$$
\dot{V} = \overline{Z} \dot{I_1} + \overline{Z} \dot{I_2}
$$

Prendiamo quindi una coppia di induttori \textbf{reali} (quindi in serie ad una resistenza) mutuamente accoppiati:
\begin{center}
	\begin{circuitikz}
		\draw (0,0) to[ R, l=$R_1$, i=$i_1$] (2,0)
			to[ inductor , l=$L_1$] (2,-2)
			-- (0, -2);

		\draw (6,0) to[ R, l_=$R_2$, i_=$i_2$] (4,0)
			to[ inductor , l_=$L_2$] (4,-2)
			-- (6, -2);

			\draw (1.9,-0.6) node {$\scriptscriptstyle\bullet$};
			\draw (3.9,-0.6) node {$\scriptscriptstyle\bullet$};

			\draw (0,0) node[anchor=east] {$A$};
			\draw (0,-2) node[anchor=east] {$B$};

			\draw (0,-0.5) node[anchor=east] {$+$};
			\draw (0,-1.5) node[anchor=east] {$-$};

			\draw (6,0) node[anchor=west] {$C$};
			\draw (6,-2) node[anchor=west] {$D$};

			\draw (6,-0.5) node[anchor=west] {$+$};
			\draw (6,-1.5) node[anchor=west] {$-$};

			\draw (0,-1) node[anchor=east] {$v_1$};
			\draw (6,-1) node[anchor=west] {$v_2$};

			\draw (3,-0.5) node[anchor=south] {$M$};
	\end{circuitikz}
\end{center}

Potremo esprimere la potenza complessa come:
$$
\overline{S} = V_1 \dot{I_1}^* + V_2 \dot{I_2}^* = \left( R_1 \dot{I_1} + j \omega L_1 \dot{I_1} + j \omega M \dot{I_2} \right) \cdot \dot{I_1}^* + \left( R_2 \dot{I_2} + j \omega L_2 \dot{I_2} + j \omega M \dot{I_1} \right) \cdot \dot{I_2}^*
$$
$$
= R_1 I_1^2 + j \omega L_1 I_1^2 + j \omega M \dot{I_2} \cdot \dot{I_1}^* + R_2 I_2^2 + j \omega L_2 I_2^2 + j \omega M \dot{I_1} \cdot \dot{I_2}^*
$$
$$
= R_1 I_1^2 + R_2 I_2^2 + j \omega L_1 I_1^2 + j \omega L_2 I_2^2 + j \omega M ( \dot{I_2} \cdot \dot{I_1}^* + \dot{I_1} \cdot \dot{I_2}^* )
$$

Da cui notiamo i primi quattro termini essere reali, ergo l'unico termine con possibile componente complessa è $j \omega M ( \dot{I_2} \cdot \dot{I_1}^* + \dot{I_2} \cdot \dot{I_1}^* )$.
Possiamo dire che:
$$
\dot{I_1} = I_1 e^{j \phi_1}, \quad \dot{I_2} = I_2 e^{j \phi_2}
$$
da cui:
$$
\overline{S} = ... + j \omega M \left( I_2 e^{j \phi_2} \cdot I_1 e^{-j \phi_1} + I_1 e^{j \phi_1} \cdot I_2 e^{-j \phi_2} \right) = ... + j \omega M I_1 I_2 \left( e^{j(\phi_2 - \phi_1)} + e^{j(\phi_1 - \phi_2)} \right)
$$
$$
= ... + j \omega M I_1 I_2 \left( \cos(\phi_2 - \phi_1) + j \sin (\phi_2 - \phi_1) + \cos(\phi_1 - \phi_2) + j \sin(\phi_1 - \phi_2) \right)
$$

E quindi dalle proprietà di seni e coseni di argomento negato:
$$
\overline{S} = R_1 I_1^2 + R_2 I_2 ^2 + j \omega L_1 I_1^2 + j \omega L_2 I_2^2 + 2 j \omega M I_1 I_2 \cos(\phi_2 - \phi_2)
$$

Da cui si ha che potenza attiva e reattiva sono rispettivamente:
$$
P = R_1 I_1^2 R_2 I_2^2, \quad jQ = j \omega L_1 I_1^2 + j \omega L_2 I_2^2 + 2 j \omega M I_1 I_2 \cos(\phi_1 - \phi_2)
$$
e quindi la potenza sulle mutue induttanze reali non ha solo componente reattiva (come nelle comuni induttanze), ma anche attiva. 

\subsection{Circuiti equivalenti a induttori mutuamente accoppiati}
Possiamo creare circuiti equivalenti, usando undattori non mutuamente accoppiati, quando si hanno induttori mutuamente accoppiati con nodi in comune:

\begin{center}
	\begin{circuitikz}
		\draw (0,0) to[ short, i=$i_1$] (1,0)
			to[ inductor , l=$L_1$] (1,-2)
			-- (0, -2);

		\draw (4,0) to[ short, i_=$i_2$] (3,0)
			to[ inductor , l_=$L_2$] (3,-2)
			-- (4, -2);

		\draw  (1, -2) to[short] (3, -2);

			\draw (0.9,-0.6) node {$\scriptscriptstyle\bullet$};
			\draw (2.9,-0.6) node {$\scriptscriptstyle\bullet$};

			\draw (0,0) node[anchor=east] {$A$};
			\draw (0,-2) node[anchor=east] {$B$};

			\draw (0,-0.5) node[anchor=east] {$+$};
			\draw (0,-1.5) node[anchor=east] {$-$};

			\draw (4,0) node[anchor=west] {$C$};
			\draw (4,-2) node[anchor=west] {$D$};

			\draw (4,-0.5) node[anchor=west] {$+$};
			\draw (4,-1.5) node[anchor=west] {$-$};

			\draw (0,-1) node[anchor=east] {$v_1$};
			\draw (4,-1) node[anchor=west] {$v_2$};
			
			\draw (2,-0.5) node[anchor=south] {$M$};
	\end{circuitikz}
\end{center}

In questo caso si può usare l'equivalente a stella, selto un nuovo nodo centrale $O$:
\begin{center}
	\begin{circuitikz}
		\draw (0,0) to[ short, i=$i_1$] (1,0)
			to[ inductor , l_=$\scriptstyle L_1 \mp M$] (2,-1)
			to[ inductor, l=$\scriptstyle \pm M$ ] (2,-2);

		\draw (4,0) to[ short, i_=$i_2$] (3,0)
			to[ inductor , l=$\scriptstyle L_1 \mp M$] (2,-1);

		\draw  (0, -2) to[short] (4, -2);

			\draw (0,0) node[anchor=east] {$A$};
			\draw (0,-2) node[anchor=east] {$B$};

			\draw (0,-0.5) node[anchor=east] {$+$};
			\draw (0,-1.5) node[anchor=east] {$-$};

			\draw (4,0) node[anchor=west] {$C$};
			\draw (4,-2) node[anchor=west] {$D$};

			\draw (4,-0.5) node[anchor=west] {$+$};
			\draw (4,-1.5) node[anchor=west] {$-$};

			\draw (0,-1) node[anchor=east] {$v_1$};
			\draw (4,-1) node[anchor=west] {$v_2$};
	\end{circuitikz}
\end{center}

dove i segni superiori si usano nel caso di contrassegni coincidenti rispetto al nodo centrale, e i segni inferiori si usano nel caso di contrassegni opposti.

Possiamo verificare che il circuito è effettivamente equivalente calcolando le cadute di potenziale, ad esempio sul lato sinistro del circuito:
$$
\dot{V_1} = j \omega (L_1 - M) \dot{I_1} + j \omega M (\dot{I_1} + \dot{I_2}) = j \omega L_1 \dot{I_1} - j \omega M \dot{I_1} + j \omega M \dot{I_1} + j \omega M \dot{I_2} = j \omega L \dot{I_1} + j \omega M \dot{I_2} 
$$

Un montaggio alternativo, ma comunque equivalente, è quello a $\pi$:
\begin{center}
	\begin{circuitikz}
		\draw (0,0) to[ short, i=$i_1$] (1,0)
		to[ inductor , l_=$\frac{\Delta}{L_2 \mp M}$] (1,-2)
			-- (0, -2);

		\draw (4,0) to[ short, i_=$i_2$] (3,0)
		to[ inductor , l=$\frac{\Delta}{L_1 \mp M}$] (3,-2)
			-- (4, -2);

		\draw  (1, -2) to[short] (3, -2);

		\draw (1,0) to [ inductor , l=$\frac{\Delta}{\pm M}$] (3, 0);

			\draw (0,0) node[anchor=east] {$A$};
			\draw (0,-2) node[anchor=east] {$B$};

			\draw (0,-0.5) node[anchor=east] {$+$};
			\draw (0,-1.5) node[anchor=east] {$-$};

			\draw (4,0) node[anchor=west] {$C$};
			\draw (4,-2) node[anchor=west] {$D$};

			\draw (4,-0.5) node[anchor=west] {$+$};
			\draw (4,-1.5) node[anchor=west] {$-$};

	\end{circuitikz}
\end{center}

Dove $\Delta = L_1L_2 - M^2$, assunto $L_1L_2 \neq M^2$ (con $M = \sqrt{L_1 L_2}$, siamo in condizioni di mutua induttanza \textit{ideale}).
Come prima, i segni superiori significano induttanze con contrassegni concordi in direzione del polo comune, e i segni inferiori significano induttanze con contrassegni discordi.

\end{document}
