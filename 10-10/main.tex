
\documentclass[a4paper,11pt]{article}
\usepackage[a4paper, margin=8em]{geometry}

% usa i pacchetti per la scrittura in italiano
\usepackage[french,italian]{babel}
\usepackage[T1]{fontenc}
\usepackage[utf8]{inputenc}
\frenchspacing 

% usa i pacchetti per la formattazione matematica
\usepackage{amsmath, amssymb, amsthm, amsfonts}

% usa altri pacchetti
\usepackage{gensymb}
\usepackage{hyperref}
\usepackage{standalone}

% imposta il titolo
\title{Appunti Elettrotecnica}
\author{Luca Seggiani}
\date{2024}

% imposta lo stile
% usa helvetica
\usepackage[scaled]{helvet}
% usa palatino
\usepackage{palatino}
% usa un font monospazio guardabile
\usepackage{lmodern}

\renewcommand{\rmdefault}{ppl}
\renewcommand{\sfdefault}{phv}
\renewcommand{\ttdefault}{lmtt}

% disponi il titolo
\makeatletter
\renewcommand{\maketitle} {
	\begin{center} 
		\begin{minipage}[t]{.8\textwidth}
			\textsf{\huge\bfseries \@title} 
		\end{minipage}%
		\begin{minipage}[t]{.2\textwidth}
			\raggedleft \vspace{-1.65em}
			\textsf{\small \@author} \vfill
			\textsf{\small \@date}
		\end{minipage}
		\par
	\end{center}

	\thispagestyle{empty}
	\pagestyle{fancy}
}
\makeatother

% disponi teoremi
\usepackage{tcolorbox}
\newtcolorbox[auto counter, number within=section]{theorem}[2][]{%
	colback=blue!10, 
	colframe=blue!40!black, 
	sharp corners=northwest,
	fonttitle=\sffamily\bfseries, 
	title=~\thetcbcounter: #2, 
	#1
}

% disponi definizioni
\newtcolorbox[auto counter, number within=section]{definition}[2][]{%
	colback=red!10,
	colframe=red!40!black,
	sharp corners=northwest,
	fonttitle=\sffamily\bfseries,
	title=~\thetcbcounter: #2,
	#1
}

% U.D.M
\newcommand{\amp}{\ensuremath{\mathrm{A}}}
\newcommand{\volt}{\ensuremath{\mathrm{V}}}
\newcommand{\meter}{\ensuremath{\mathrm{m}}}
\newcommand{\second}{\ensuremath{\mathrm{s}}}
\newcommand{\farad}{\ensuremath{\mathrm{F}}}
\newcommand{\henry}{\ensuremath{\mathrm{H}}}
\newcommand{\siemens}{\ensuremath{\mathrm{S}}}

% circuiti
\usepackage{circuitikz}
\usetikzlibrary{babel}

% disegni
\usepackage{pgfplots}
\pgfplotsset{width=10cm,compat=1.9}

% disponi codice
\usepackage{listings}
\usepackage[table]{xcolor}

\lstdefinestyle{codestyle}{
		backgroundcolor=\color{black!5}, 
		commentstyle=\color{codegreen},
		keywordstyle=\bfseries\color{magenta},
		numberstyle=\sffamily\tiny\color{black!60},
		stringstyle=\color{green!50!black},
		basicstyle=\ttfamily\footnotesize,
		breakatwhitespace=false,         
		breaklines=true,                 
		captionpos=b,                    
		keepspaces=true,                 
		numbers=left,                    
		numbersep=5pt,                  
		showspaces=false,                
		showstringspaces=false,
		showtabs=false,                  
		tabsize=2
}

\lstdefinestyle{shellstyle}{
		backgroundcolor=\color{black!5}, 
		basicstyle=\ttfamily\footnotesize\color{black}, 
		commentstyle=\color{black}, 
		keywordstyle=\color{black},
		numberstyle=\color{black!5},
		stringstyle=\color{black}, 
		showspaces=false,
		showstringspaces=false, 
		showtabs=false, 
		tabsize=2, 
		numbers=none, 
		breaklines=true
}

\lstdefinelanguage{javascript}{
	keywords={typeof, new, true, false, catch, function, return, null, catch, switch, var, if, in, while, do, else, case, break},
	keywordstyle=\color{blue}\bfseries,
	ndkeywords={class, export, boolean, throw, implements, import, this},
	ndkeywordstyle=\color{darkgray}\bfseries,
	identifierstyle=\color{black},
	sensitive=false,
	comment=[l]{//},
	morecomment=[s]{/*}{*/},
	commentstyle=\color{purple}\ttfamily,
	stringstyle=\color{red}\ttfamily,
	morestring=[b]',
	morestring=[b]"
}

% disponi sezioni
\usepackage{titlesec}

\titleformat{\section}
	{\sffamily\Large\bfseries} 
	{\thesection}{1em}{} 
\titleformat{\subsection}
	{\sffamily\large\bfseries}   
	{\thesubsection}{1em}{} 
\titleformat{\subsubsection}
	{\sffamily\normalsize\bfseries} 
	{\thesubsubsection}{1em}{}

% disponi alberi
\usepackage{forest}

\forestset{
	rectstyle/.style={
		for tree={rectangle,draw,font=\large\sffamily}
	},
	roundstyle/.style={
		for tree={circle,draw,font=\large}
	}
}

% disponi algoritmi
\usepackage{algorithm}
\usepackage{algorithmic}
\makeatletter
\renewcommand{\ALG@name}{Algoritmo}
\makeatother

% disponi numeri di pagina
\usepackage{fancyhdr}
\fancyhf{} 
\fancyfoot[L]{\sffamily{\thepage}}

\makeatletter
\fancyhead[L]{\raisebox{1ex}[0pt][0pt]{\sffamily{\@title \ \@date}}} 
\fancyhead[R]{\raisebox{1ex}[0pt][0pt]{\sffamily{\@author}}}
\makeatother

\begin{document}
% sezione (data)
\section{Lezione del 10-10-24}

% stili pagina
\thispagestyle{empty}
\pagestyle{fancy}

% testo
\subsection{Dimostrazione di Thevenin}
Avevamo detto che, attraverso Thevenin, si può trasformare qualsiasi rete come vista da due morsetti in una rete equivalente formata da un resistore e un generatore di tensione in serie.
Dimostriamo questo risultato.

Supponiamo di avere un circuito elettrico che andiamo a dividere in due sottoreti.
Poniamo di voler semplificare una di queste sottoreti (magari quella di sinistra), presi due morsetti, $1$ e $2$, che la collegano all'altra.
Avremo che da questi morsetti passa una corrente $i(t)$ e che essi si trovano ad una differenza di potenziale $v(t)$.

\begin{center}
	\begin{circuitikz}
    % First rectangle (component)
    \draw (0,0) rectangle (3,3) node[midway] {};
    
    % Second rectangle (component)
    \draw (7,0) rectangle (10,3) node[midway] {};

		\node at(5,2) {$+$};
		\node at(5,1.5) {$v(t)$};
		\node at(5,1) {$-$};

		\node at(3.2,2.8) {$1$};
		\node at(3.2,0.8) {$2$};
		
		\draw (3,0.5) to[ short ] (7,0.5);
		\draw (3,2.5) to[ short, i = $i(t)$ ] (7,2.5);
	\end{circuitikz}
\end{center}

Possiamo quindi pensare di separare le due sottoreti, e di sostituire ai morsetti dei generatori di corrente che replicano la corrente $i(t)$ vista prima.
Se la tensione ai capi dei generatori resta $v(t)$, allora i circuiti non sono cambiati.

\begin{center}
	\begin{circuitikz}
    % First rectangle (component)
    \draw (0,0) rectangle (3,3) node[midway] {};
    
    % Second rectangle (component)
    \draw (7,0) rectangle (10,3) node[midway] {};

		\node at(5,2) {$+$};
		\node at(5,1.5) {$v(t)$};
		\node at(5,1) {$-$};

		\node at(3.2,2.8) {$1$};
		\node at(3.2,0.8) {$2$};
		
		\draw (7,0.5) to[ short ] (6,0.5)
			to [ short, I = $i(t)$ ] (6, 2.5)
			to [ short ] (7, 2.5);
		\draw (3,2.5) to[ short ] (4,2.5)
			to[ short, I = $i(t)$] (4,0.5)
			to [short] (3, 0.5);
	\end{circuitikz}
\end{center}

Questo prende il nome di \textbf{principio di sostituzione}.

A questo punto possiamo dire che all'interno delle sottoreti troveremo dipoli attivi (genratori di tensione e di corrente), e passivi (resistori e generatori pilotati di tensione e di corrente):

\begin{center}
	\begin{circuitikz}
    % First rectangle (component)
    \draw (0,0) rectangle (3,3) node[midway] {};
    
		\node at(5,2) {$+$};
		\node at(5,1.5) {$v(t)$};
		\node at(5,1) {$-$};

		\node at(3.2,2.8) {$1$};
		\node at(3.2,0.8) {$2$};
		
		\draw (3,2.5) to[ short ] (4,2.5)
			to[ short, I = $i(t)$] (4,0.5)
			to [short] (3, 0.5);

		\draw (0,0.5) to[ short ] (-1,0.5)
			to[ short, I = $I$] (-1,2.5)
			to [short] (0, 2.5);

		\draw (0.5,0) to[ short ] (0.5,-1)
			to[ short, V_ = $E$] (2.5,-1)
			to [short] (2.5, 0);

		\draw (0.4,0.2) to [ R ] (0.4, 2.8);
		\draw (1.3,0.2) to [ controlled voltage source ] (1.3, 2.8);
		\draw (2.4,0.2) to [ controlled current source ] (2.4, 2.8);
	\end{circuitikz}
\end{center}

Si decide di applicare il principio di sovrapposizione, ponendo il voltaggio dato dai generatori interni come $v'(t)$ e quello dato dal generatore esterno come $v''(t)$, da cui:
$$
v(t) = v'(t) + v''(t)
$$

Staccate dal resto del circuito si ha che le sottoreti portano i morsetti a tensione $v'(t) = v_0(t)$.
Incluso il generatore di corrente di prima, possiamo pensare di disattivare i generatori dipendenti propri della sottorete, per ricavare un'altra tensione $v''(t)$.
Visto che il circuito è formato effettivamente da componenti passivi, possiamo calcolare una resistenza vista, che chiamiamo $R_{12}$.
Abbiamo allora che, incluso il generatore di corrente di prova $i(t)$:
$$
v''(t) = -R_{12} \cdot i(t)
$$
dove il segno meno è dato dal fatto che abbiamo preso il generatore esterno come un riferimento non associato (spinge carica dal potenziale positivo a quello negativo), e quindi la tensione totale è:
$$
v(t) = v_0(t) - R_{12} \cdot i(t)
$$

Questo equivale a rappresentare il circuito come un generatore di tensione $v_{TH} = v_0(t)$ in serie ad una resistenza $v_{TH} = R_{12}$.

\subsection{Metodo delle correnti di maglia}
Vediamo un metodo alternativo per la risoluzione dei circuiti, simile a quello delle correnti di ramo ma che genera meno variabili.
Questo metodo presuppone di prendere, anziché le correnti su ogni ramo, un numero minore di correnti, prese su intere maglie.
Le maglie vengono scelte secondo lo stesso metodo del tableau: dopo aver usato una maglia, se ne taglia un ramo.

Prendiamo in esempio il circuito:

\begin{center}
\begin{circuitikz}
	\draw (0,0)
		-- (0,1.5)
		to[R, l=$R_1$] (5,1.5)
		-- (5, 0);
	\draw (0,0)
		to[R, l=$R_2$] (2.5, 0)
		to[R, l=$R_3$] (5, 0);
	\draw (0,0)
		to[R, l=$R_4$] (0, -3);
	\draw (2.5,0)
		to[R, l=$R_5$] (2.5, -3);
	\draw (5,-3)
		to[ european voltage source, v=$E_2$] (5, 0);
	\draw (5, -3)
		to[R, l=$R_6$] (2.5, -3)
		to[ european voltage source, v=$E_1$] (0, -3);
\end{circuitikz}
\end{center}

Avevamo già risolto questa rete usando il metodo del tableau, che potremmo dire \textit{"metodo delle correnti di ramo"}, trovando un sistema lineare dalla prima e seconda legge di Kirchoff.
Il problema di tale metodo era la grande quantità di variabili: una per ogni ramo.
Applichiamo adesso il metodo delle correnti di maglia.
Scegliamo ogni maglia a sé stante: in basso a sinistra, in basso a destra e in alto.
Diamo un nome alla corrente su ognuna di queste maglie, rispettivamente $I_a$, $I_b$ e $I_c$.
Le direzioni delle correnti sono come in figura (scelte ad arbitrio):

\begin{center}
\begin{circuitikz}
	\draw (0,0)
		-- (0,1.5)
		to[R, l=$R_1$] (5,1.5)
		-- (5, 0);
	\draw (0,0)
		to[R, l=$R_2$] (2.5, 0)
		to[R, l=$R_3$] (5, 0);
	\draw (0,0)
		to[R, l=$R_4$] (0, -3);
	\draw (2.5,0)
		to[R, l=$R_5$] (2.5, -3);
	\draw (5,-3)
		to[ european voltage source, v_=$E_2$] (5, 0);
	\draw (5, -3)
		to[R, l=$R_6$] (2.5, -3) 
		to[ european voltage source, v=$E_1$] (0, -3);

	\draw[->, thick] (3,0.6) arc[start angle=0, end angle=-270, radius=0.4cm]; % Loop direction
	\node at (3, 0.8) {$I_c$}; % Adjust the position (0.5, 0.5) to place the label where desired

	\draw[->, thick] (1.8,-1.6) arc[start angle=0, end angle=-270, radius=0.4cm]; % Loop direction
	\node at (1.8, -1.2) {$I_a$}; % Adjust the position (0.5, 0.5) to place the label where desired

	\draw[->, thick] (4.2,-1.6) arc[start angle=0, end angle=270, radius=0.4cm]; % Loop direction
	\node at (4.2, -1.9) {$I_b$}; % Adjust the position (0.5, 0.5) to place the label where desired
\end{circuitikz}
\end{center}

Iniziamo con la maglia in basso a sinistra: percorsa in senso orario, avremo che la seconda legge di Kirchoff dà:
$$
E_1 - R_4 I_a + R_2 ( I_c - I_a ) + R_5 ( -I_a - I_b) = 0
$$
dove si nota che per dipoli percorsi da più correnti di maglia, si prende come corrente la somma algebrica delle correnti sulla base del loro segno.
Applichiamo quindi lo stesso metodo alle altre due maglie, percorrendo quella in basso a destra in senso antiorario e quella in alto in senso orario:
$$
E_2 + R_3(-I_b - I_c) + R_5 (-I_b - I_a) - R_6 I_b = 0
$$
$$
-R_1 I_c + R_3 (-I_c - I_b) + R_2 (I_a - I_c) = 0
$$

Da cui si ricava il sistema lineare:
\[
	\begin{cases}
			E_1 - R_4 I_a + R_2 ( I_c - I_a ) + R_5 ( -I_a - I_b) = 0 \\
			E_2 + R_3(-I_b - I_c) + R_5 (-I_b - I_a) - R_6 I_b = 0 \\
			-R_1 I_c + R_3 (-I_c - I_b) + R_2 (I_a - I_c) = 0
	\end{cases}
\]
in forma matriciale:
$$
\begin{pmatrix}
	-R_4-R_2-R_5 & -R_5 & R_2 \\ 
	-R_5 & -R_3-R_5-R_6 -R_3 \\ 
	R_2 & -R_3 & -R_1-R_3-R_2
\end{pmatrix}
\begin{pmatrix}
I_a \\ I_b \\ I_v
\end{pmatrix}
=
\begin{pmatrix}
-E_1 \\ -E_2 \\ 0
\end{pmatrix}
$$

Come sempre, questo sistema si risolve con qualsiasi metodo di risoluzione dei sistemi lineari.

\subsubsection{Circuiti con generatori di corrente}
Notiamo un caso particolare per l'applicazione di questo problema: nel caso il circuito contenga generatori di corrente, bisogna evitare di introdurre tensioni incognite, come nel metodo del tabeau.
Si sceglie quindi una maglia contenente il generatore di corrente per prima, e poi si taglia quel ramo.
Prendiamo ad esempio:

\begin{center}
\begin{circuitikz}
	\draw (0,0)
		to[R, l=$R_2$] (2.5, 0)
		to[R, l=$R_3$] (5, 0);
	\draw (0,-3)
		to[ european voltage source, v=$E$] (0, 0);
	\draw (2.5,-3)
		to[ european current source, I=$J$] (2.5, 0);
	\draw (5,-3)
		to[ short ] (5, 0);
	\draw (5, -3)
		to[ short ] (2.5, -3)
		to[ short ] (0, -3);
\end{circuitikz}
\end{center}

Qui il generatore di corrente $J$ genera una differenza di potenziale $V_J$, che non vogliamo come incognita.
Prenderemo quindi per prima una maglia che lo contiene (in questo caso qualsiasi tranne la più esterna), e poi taglieremo il suo ramo.
Ad esempio, prendendo prima la maglia a destra, si ha al primo passo:
$$
-E + R I_x + 2R(I_x + J) = 0
$$
dove si è potuto chiamare l'unica corrente incognita $I_x$.

Non ci è effettivamente negato scegliere di scegliere le maglie in quest'ordine, ma se lo facciamo, allora non dobbiamo usare la corrente $J$ nel calcolo, ma introdurre una corrente incognita per ogni maglia.
Ad esempio, se volessimo considerare le due maglie a sinistra ($I_s$) e a destra ($I_d$) separatamente:
$$ 
R I_s + E - V_j = 0 
$$
$$
J = I_x + I_b
$$

\end{document}
