
\documentclass[a4paper,11pt]{article}
\usepackage[a4paper, margin=8em]{geometry}

% usa i pacchetti per la scrittura in italiano
\usepackage[french,italian]{babel}
\usepackage[T1]{fontenc}
\usepackage[utf8]{inputenc}
\frenchspacing 

% usa i pacchetti per la formattazione matematica
\usepackage{amsmath, amssymb, amsthm, amsfonts}

% usa altri pacchetti
\usepackage{gensymb}
\usepackage{hyperref}
\usepackage{standalone}

% imposta il titolo
\title{Appunti Elettrotecnica}
\author{Luca Seggiani}
\date{2024}

% imposta lo stile
% usa helvetica
\usepackage[scaled]{helvet}
% usa palatino
\usepackage{palatino}
% usa un font monospazio guardabile
\usepackage{lmodern}

\renewcommand{\rmdefault}{ppl}
\renewcommand{\sfdefault}{phv}
\renewcommand{\ttdefault}{lmtt}

% disponi il titolo
\makeatletter
\renewcommand{\maketitle} {
	\begin{center} 
		\begin{minipage}[t]{.8\textwidth}
			\textsf{\huge\bfseries \@title} 
		\end{minipage}%
		\begin{minipage}[t]{.2\textwidth}
			\raggedleft \vspace{-1.65em}
			\textsf{\small \@author} \vfill
			\textsf{\small \@date}
		\end{minipage}
		\par
	\end{center}

	\thispagestyle{empty}
	\pagestyle{fancy}
}
\makeatother

% disponi teoremi
\usepackage{tcolorbox}
\newtcolorbox[auto counter, number within=section]{theorem}[2][]{%
	colback=blue!10, 
	colframe=blue!40!black, 
	sharp corners=northwest,
	fonttitle=\sffamily\bfseries, 
	title=~\thetcbcounter: #2, 
	#1
}

% disponi definizioni
\newtcolorbox[auto counter, number within=section]{definition}[2][]{%
	colback=red!10,
	colframe=red!40!black,
	sharp corners=northwest,
	fonttitle=\sffamily\bfseries,
	title=~\thetcbcounter: #2,
	#1
}

% U.D.M
\newcommand{\amp}{\ensuremath{\mathrm{A}}}
\newcommand{\volt}{\ensuremath{\mathrm{V}}}
\newcommand{\meter}{\ensuremath{\mathrm{m}}}
\newcommand{\second}{\ensuremath{\mathrm{s}}}
\newcommand{\farad}{\ensuremath{\mathrm{F}}}
\newcommand{\henry}{\ensuremath{\mathrm{H}}}
\newcommand{\siemens}{\ensuremath{\mathrm{S}}}

% circuiti
\usepackage{circuitikz}
\usetikzlibrary{babel}

% disegni
\usepackage{pgfplots}
\pgfplotsset{width=10cm,compat=1.9}

% disponi codice
\usepackage{listings}
\usepackage[table]{xcolor}

\lstdefinestyle{codestyle}{
		backgroundcolor=\color{black!5}, 
		commentstyle=\color{codegreen},
		keywordstyle=\bfseries\color{magenta},
		numberstyle=\sffamily\tiny\color{black!60},
		stringstyle=\color{green!50!black},
		basicstyle=\ttfamily\footnotesize,
		breakatwhitespace=false,         
		breaklines=true,                 
		captionpos=b,                    
		keepspaces=true,                 
		numbers=left,                    
		numbersep=5pt,                  
		showspaces=false,                
		showstringspaces=false,
		showtabs=false,                  
		tabsize=2
}

\lstdefinestyle{shellstyle}{
		backgroundcolor=\color{black!5}, 
		basicstyle=\ttfamily\footnotesize\color{black}, 
		commentstyle=\color{black}, 
		keywordstyle=\color{black},
		numberstyle=\color{black!5},
		stringstyle=\color{black}, 
		showspaces=false,
		showstringspaces=false, 
		showtabs=false, 
		tabsize=2, 
		numbers=none, 
		breaklines=true
}

\lstdefinelanguage{javascript}{
	keywords={typeof, new, true, false, catch, function, return, null, catch, switch, var, if, in, while, do, else, case, break},
	keywordstyle=\color{blue}\bfseries,
	ndkeywords={class, export, boolean, throw, implements, import, this},
	ndkeywordstyle=\color{darkgray}\bfseries,
	identifierstyle=\color{black},
	sensitive=false,
	comment=[l]{//},
	morecomment=[s]{/*}{*/},
	commentstyle=\color{purple}\ttfamily,
	stringstyle=\color{red}\ttfamily,
	morestring=[b]',
	morestring=[b]"
}

% disponi sezioni
\usepackage{titlesec}

\titleformat{\section}
	{\sffamily\Large\bfseries} 
	{\thesection}{1em}{} 
\titleformat{\subsection}
	{\sffamily\large\bfseries}   
	{\thesubsection}{1em}{} 
\titleformat{\subsubsection}
	{\sffamily\normalsize\bfseries} 
	{\thesubsubsection}{1em}{}

% disponi alberi
\usepackage{forest}

\forestset{
	rectstyle/.style={
		for tree={rectangle,draw,font=\large\sffamily}
	},
	roundstyle/.style={
		for tree={circle,draw,font=\large}
	}
}

% disponi algoritmi
\usepackage{algorithm}
\usepackage{algorithmic}
\makeatletter
\renewcommand{\ALG@name}{Algoritmo}
\makeatother

% disponi numeri di pagina
\usepackage{fancyhdr}
\fancyhf{} 
\fancyfoot[L]{\sffamily{\thepage}}

\makeatletter
\fancyhead[L]{\raisebox{1ex}[0pt][0pt]{\sffamily{\@title \ \@date}}} 
\fancyhead[R]{\raisebox{1ex}[0pt][0pt]{\sffamily{\@author}}}
\makeatother

\begin{document}
% sezione (data)
\section{Lezione del 09-10-24}

% stili pagina
\thispagestyle{empty}
\pagestyle{fancy}

% testo
\subsection{Teorema di Thevenin}
Poniamo di avere un certo bipolo, inteso come una sottorete con due morsetti.
All'interno del bipolo avremo resistori, generatori di corrente e generatori di tensioni.
Quello che vogliamo fare è rappresentare la sottorete con un circuito semplificato, come avevamo fatto con circuiti di solo resistori.

Pensiamo di mettere un generatore di corrente fra i morsetti del bipolo, in modo da imporre una certa corrente, e calcolare la tensione risultante.
Facendo misure diverse a correnti diverse, dovremmo essere in grado di trovare un legame fra tensione $v(t)$ e corrente $i(t)$ attraverso, ad esempio, la legge di Ohm.
A questo punto sapremmo sostituire l'intera rete con un singolo dispositivo che rappresenti la sua risposta interna a qualsiasi stimolo esterno.

Questo è quello che ci permette di fare il teorema di Thevenin:
\begin{theorem}{Teorema di Thevenin}
	Qualsiasi circuito lineare, visto da due morsetti, è equivalente a un generatore di tensione in serie a un resistore.

\begin{itemize}
	\item La \textbf{resistenza di Thevenin} $R_{th}$, equivale alla resistenza vista da dai morsetti calcolata dopo aver disattivato tutti i generatori indipendenti della sottorete;
	\item La \textbf{tensione di Thevenin} $V_{th}$, è pari al valore della tensione a vuoto misurata fra i morsetti, dove per tensione a vuoto si intende la tensione della sottorete isolata da altri circuiti. Questo equivale a \textbf{tagliare} fuori i rami del circuito che vanno ai morsetti, compresi eventuali bipoli presenti su quei rami (non vi scorrerà alcuna corrente).
\end{itemize}
\end{theorem}

Poste $R_{th}$ e $V_{th}$, si può dire che la correlazione fra voltaggio e corrente del circuito è:
$$
v(t) = V_{th} - R_{th} i(t)
$$
dove il segno meno alla corrente è lì perchè, all'aumentare della corrente fatta passare attraverso i morsetti (considerando, per il principio di sovrapposizione, i generatori interni disattivati), aumenta la differenza di potenziale fra di loro ($v_{load} = R_{th} i(t)$), e quindi diminuisce quella sul generatore equivalente di Thevenin $v(t)$.
Infatti, $V_{th}$ era stata definita come la differenza di potenziale a vuoto fra i morsetti, e quindi dovrà essere:
$$
v_{load} + v(t) = V_{th}
$$

\subsection{Teorema di Norton}
Il teorema di Thevenin ammette un duale: come possiamo usare un generatore di tensione, possiamo infatti usare un generatore di corrente per rappresentare un'intera sottorete.
Formuliamo quindi:
\begin{theorem}{Teorema di Norton}
	Qualsiasi circuito lineare, visto da due morsetti, è equivalente a un generatore di corrente in parallelo a un resistore.

\begin{itemize}
	\item La \textbf{resistenza di Norton} $R_{no}$, equivale alla resistenza di Thevenin, cioè alla resistenza vista da dai morsetti calcolata dopo aver disattivato tutti i generatori indipendenti della sottorete;
	\item La \textbf{corrente di Norton} $I_{no}$, è pari al valore della corrente misurata su un ramo (posto da noi per effettuare la misura) che unisce i morsetti.\end{itemize}
\end{theorem}

Con Norton, anzichè la differenza di potenziale a vuoto $V_{th}$ ricavata per Thevenin, si considera la corrente che scorrerebbe fra i morsetti se fossero collegati fra di loro.
Avremo quindi che la corrente al variaggio della differenza di potenziale fra le maglie è:
$$
i(t) = I_{no} - \frac{v(t)}{R_{no}}
$$
Questo perché, come prima, dovrà essere che la corrente fra i morsetti, disattivati i generatori interni, $i_load$ e la corrente sul circuito equivalente di Norton $i(t)$ sono legate da:
$$
i_{load} + i(t) = I_{no}
$$

\par\medskip 

Possiamo stabilire una relazione fra la resistenza e la tensione di Thevenin, $R_{th}$ e $V_{th}$, e la resistenza e la corrente di Norton, $R_{no}$ e $I_{no}$:
\[
	\begin{cases}
		R_{th} = R_{no} \\ 
		V_{th} = I_{no} \cdot R_{no}
	\end{cases}
\]

In ogni caso, vale $R_{th} = R_{no} = R_{eq}$ del circuito.

Si può quindi ricavare il corollario:
\begin{theorem}{Corollario da Thevenin e Norton}
	Un generatore di corrente in parallelo a una resistenza equivale ad un generatore di tensione in serie alla stessa resistenza (salvo casi limite).
\end{theorem}

\par\medskip 

\subsubsection{Generatori dipendenti}
Nel caso si incontrino generatori di corrente o di tensione dipendenti, bisogna considerare che potrebbe essere impossibile applicare Thevenin o Norton su una parte di circuito, poiché la grandezza pilota del generatore potrebbe trovarsi nell'altra parte del circuito.

Nel caso questo sia possibile, invece, si adotta un \textbf{generatore di prova} che può essere arbitrariamente di tensione o di corrente.
Potremo infatti ricavare l'una dall'altra come:
$$
R_{eq} = \frac{V_p}{I_p}
$$
dalla legge di Ohm.


\end{document}
